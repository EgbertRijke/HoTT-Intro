\chapter{Descent}\label{chap:descent}

\section{The flattening lemma}

\begin{defn}
Consider a span $\mathcal{S}\jdeq (S,f,g)$ from $A$ to $B$. Then we define the type of \define{descent data} for the pushout of $\mathcal{S}$ to be
\begin{equation*}
\mathsf{Desc}_{\mathcal{S}} \defeq \sm{P_A : A\to \UU}{P_B : B\to \UU}\prd{x:S}\eqv{P_A(f(x))}{P_B(g(x))}.
\end{equation*}
Moreover, we define the map
\begin{equation*}
\mathsf{desc}_{\mathcal{S}} : (A \sqcup^{\mathcal{S}} B \to \UU) \to \mathsf{Desc}_{\mathcal{S}}
\end{equation*}
by $\mathsf{desc}_{\mathcal{S}}(P)\defeq (P\circ\inl,P\circ\inr,\mathsf{tr}_P(\glue(\blank)))$.
\end{defn}

\begin{thm}
The map
\begin{equation*}
\mathsf{desc}_{\mathcal{S}} : (A \sqcup^{\mathcal{S}} B \to \UU) \to \mathsf{Desc}_{\mathcal{S}}
\end{equation*}
is an equivalence for any span $\mathcal{S}$.
\end{thm}

\begin{thm}
Consider a cocartesian square
\begin{equation*}
\begin{tikzcd}
S \arrow[r,"g"] \arrow[d,swap,"f"] & B \arrow[d,"j"] \\
A \arrow[r,swap,"i"] & C.
\end{tikzcd}
\end{equation*}
Furthermore, suppose we have families
\begin{align*}
P_A : A \to \UU \\
P_B : B \to \UU,
\end{align*}
equipped with equivalences $e_x : \eqv{P_A(f(x))}{P_B(g(x))}$ for every $x:S$.
Then there is a family $Q:C\to \UU$
\end{thm}

\section{The descent property for pushouts}

\begin{defn}
A commuting cube
\begin{equation*}
\begin{tikzcd}
& A_{111} \arrow[dl] \arrow[dr] \arrow[d] \\
A_{110} \arrow[d] & A_{101} \arrow[dl] \arrow[dr] & A_{011} \arrow[dl,crossing over] \arrow[d] \\
A_{100} \arrow[dr] & A_{010} \arrow[d] \arrow[from=ul,crossing over] & A_{001} \arrow[dl] \\
& A_{000},
\end{tikzcd}
\end{equation*}
consists of 
\begin{enumerate}
\item types
\begin{equation*}
A_{111},A_{110},A_{101},A_{011},A_{100},A_{010},A_{001},A_{000},
\end{equation*}
\item maps
\begin{align*}
f_{11\check{1}} & : A_{111}\to A_{110} & f_{\check{1}01} & : A_{101}\to A_{001} \\
f_{1\check{1}1} & : A_{111}\to A_{101} & f_{01\check{1}} & : A_{011}\to A_{010} \\
f_{\check{1}11} & : A_{111}\to A_{011} & f_{0\check{1}1} & : A_{011}\to A_{001} \\
f_{1\check{1}0} & : A_{110}\to A_{100} & f_{\check{1}00} & : A_{100}\to A_{000} \\
f_{\check{1}10} & : A_{110}\to A_{010} & f_{0\check{1}0} & : A_{010}\to A_{000} \\
f_{10\check{1}} & : A_{101}\to A_{100} & f_{00\check{1}} & : A_{001}\to A_{000},
\end{align*}
\item homotopies
\begin{align*}
H_{1\check{1}\check{1}} & : f_{1\check{1}0}\circ f_{11\check{1}} \htpy f_{10\check{1}}\circ f_{1\check{1}1} & H_{0\check{1}\check{1}} & : f_{0\check{1}0}\circ f_{01\check{1}} \htpy f_{00\check{1}}\circ f_{0\check{1}1} \\
H_{\check{1}1\check{1}} & : f_{\check{1}10}\circ f_{11\check{1}} \htpy f_{01\check{1}}\circ f_{\check{1}11} & H_{\check{1}0\check{1}} & : f_{\check{1}00}\circ f_{10\check{1}} \htpy f_{00\check{1}}\circ f_{\check{1}01} \\
H_{\check{1}\check{1}1} & : f_{\check{1}01}\circ f_{1\check{1}1} \htpy f_{0\check{1}1}\circ f_{\check{1}11} & H_{\check{1}\check{1}0} & : f_{\check{1}00}\circ f_{1\check{1}0} \htpy f_{0\check{1}0}\circ f_{\check{1}10},
\end{align*}
\item and a homotopy 
\begin{align*}
C & : \ct{(f_{\check{1}00}\cdot H_{1\check{1}\check{1}})}{(\ct{(H_{\check{1}0\check{1}}\cdot f_{1\check{1}1})}{(f_{00\check{1}}\cdot H_{\check{1}\check{1}1})})} \\
& \qquad \htpy \ct{(H_{\check{1}\check{1}0}\cdot f_{11\check{1}})}{(\ct{(f_{0\check{1}0}\cdot H_{\check{1}1\check{1}})}{(H_{0\check{1}\check{1}}\cdot f_{\check{1}11})})}
\end{align*}
filling the cube.
\end{enumerate}
\end{defn}

\begin{thm}
Consider the commuting diagram
\begin{equation*}
\begin{tikzcd}[column sep=small,row sep=tiny]
A_{00} \arrow[rr] \arrow[dr] \arrow[dd] & & A_{10} \arrow[dd] \arrow[dr] \\
& A_{01} \arrow[rr,crossing over] & & A_{11} \arrow[dd] \\
B_{00} \arrow[rr] \arrow[dr] & & B_{10} \arrow[dr] \\
& B_{01} \arrow[from=uu,crossing over] \arrow[rr] & & B_{11}.
\end{tikzcd}
\end{equation*}
If each of the vertical squares is a pullback, and the bottom square  is a pushout, then the top square is a pushout.
\end{thm}

\begin{cor}
For any map $f:A\sqcup^S B\to X$, and any $x:X$, the square
\begin{equation*}
\begin{tikzcd}
\fib{f_S}{x} \arrow[r] \arrow[d] & \fib{f_B}{x} \arrow[d] \\
\fib{f_A}{x} \arrow[r] & \fib{f}{x}
\end{tikzcd}
\end{equation*}
is a pushout square.
\end{cor}

\begin{thm}
Consider a commuting cube of types 
\begin{equation}\label{eq:cube}
\begin{tikzcd}
& A_{11} \arrow[dl] \arrow[dr] \arrow[d] \\
A_{10} \arrow[d] & B_{11} \arrow[dl] \arrow[dr] & A_{01} \arrow[dl,crossing over] \arrow[d] \\
B_{10} \arrow[dr] & A_{00} \arrow[d] \arrow[from=ul,crossing over] & B_{01} \arrow[dl] \\
& B_{00},
\end{tikzcd}
\end{equation}
and suppose the vertical squares are pullback squares. Then the commuting square
\begin{equation*}
\begin{tikzcd}
A_{10}\sqcup^{A_{11}} A_{01} \arrow[r] \arrow[d] & A_{00} \arrow[d] \\
B_{10}\sqcup^{B_{11}} B_{01} \arrow[r] & B_{00}
\end{tikzcd}
\end{equation*}
is a pullback square.
\end{thm}

\begin{proof}
It suffices to show that the pullback 
\begin{equation*}
(B_{10}\sqcup^{B_{11}} B_{01})\times_{B_{00}}A_{00}
\end{equation*}
has the universal property of the pushout. This follows by the descent theorem, since by the pasting lemma for pullbacks we also have that the vertical squares in the cube
\begin{equation*}
\begin{tikzcd}
& A_{11} \arrow[dl] \arrow[dr] \arrow[d] \\
A_{10} \arrow[d] & B_{11} \arrow[dl] \arrow[dr] & A_{01} \arrow[dl,crossing over] \arrow[d] \\
B_{10} \arrow[dr] & (B_{10}\sqcup^{B_{11}} B_{01})\times_{B_{00}}A_{00} \arrow[d] \arrow[from=ul,crossing over] & B_{01} \arrow[dl] \\
& B_{10}\sqcup^{B_{11}} B_{01}
\end{tikzcd}
\end{equation*}
are pullback squares.
\end{proof}

\begin{exercises}
\item Let $f:A\to B$ be a map. The \define{codiagonal} $\nabla_f$ of $f$ is the map obtained from the universal property of the pushout, as indicated in the diagram
\begin{equation*}
\begin{tikzcd}
A \arrow[d,swap,"f"] \arrow[r,"f"] \arrow[dr, phantom, "\ulcorner", very near end] & B \arrow[d,"\inr"] \arrow[ddr,bend left=15,"{\idfunc[B]}"] \\
A \arrow[r,"\inl"] \arrow[drr,bend right=15,swap,"{\idfunc[B]}"] & B\sqcup^{A} B \arrow[dr,densely dotted,near start,swap,"\nabla_f"] \\
& & B
\end{tikzcd}
\end{equation*}
Show that $\fib{\nabla_f}{b}\eqvsym \susp(\fib{f}{b})$ for any $b:B$.
\item Consider two maps $f:A\to X$ and $g:B\to X$. The \define{join} $\join{f}{g}$ is defined by the universal property of the pushout as the unique map rendering the diagram
\begin{equation*}
\begin{tikzcd}
A\times_X B \arrow[d,"\pi_1"] \arrow[r,"\pi_2"] \arrow[dr, phantom, "\ulcorner", very near end] & B \arrow[d,"\inr"] \arrow[ddr,bend left=15,"g"] \\
A \arrow[r,"\inl"] \arrow[drr,bend right=15,swap,"f"] & \join[X]{A}{B} \arrow[dr,densely dotted,near start,swap,"\join{f}{g}"] \\
& & X
\end{tikzcd}
\end{equation*}
commutative, where $\join[X]{A}{B}$ is defined as a pushout, as indicated.
Construct an equivalence
\begin{equation*}
\eqv{\fib{\join{f}{g}}{x}}{\join{\fib{f}{x}}{\fib{g}{x}}}
\end{equation*}
for any $x:X$. 
\item Consider two maps $f:A\to B$ and $g:C\to D$.
The \define{pushout-product}
\begin{equation*}
f\square g : (A\times D)\sqcup^{A\times B} (B\times C)\to B\times D
\end{equation*}
of $f$ and $g$ is defined by the universal property of the pushout as the unique map rendering the diagram
\begin{equation*}
\begin{tikzcd}
A\times C \arrow[r,"{f\times \idfunc[C]}"] \arrow[d,swap,"{\idfunc[A]\times g}"] & B\times C \arrow[d,"\inr"] \arrow[ddr,bend left=15,"{\idfunc[B]\times g}"] \\
A\times D \arrow[r,"\inl"] \arrow[drr,bend right=15,swap,"{f\times\idfunc[D]}"] & (A\times D)\sqcup^{A\times B} (B\times C) \arrow[dr,densely dotted,swap,near start,"f\square g"] \\
& & B\times D
\end{tikzcd}
\end{equation*}
commutative. Construct an equivalence
\begin{equation*}
\eqv{\fib{f\square g}{b,d}}{\join{\fib{f}{b}}{\fib{g}{d}}}
\end{equation*}
for all $b:B$ and $d:D$.
\item Show that the fiber of the wedge inclusion $A\vee B\to A\times B$ is equivalent to $\join{\loopspace{A}}{\loopspace{B}}$.
\end{exercises}
