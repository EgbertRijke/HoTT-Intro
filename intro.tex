\chapter{Introduction}

\emph{These are notes and exercise sets for the course Introduction to Homotopy Type Theory, taught at Carnegie Mellon University in the spring semester of 2018.} 

\bigskip
\noindent Homotopy Type Theory (HoTT) is an emerging field of mathematics and computer science that extends Martin-Löf's dependent type theory by the addition of the univalence axiom and higher inductive types. In HoTT we think of types as spaces, dependent types as fibrations, and of the identity types as path spaces.

\begin{figure}
\makebox[\textwidth][c]{%
\begin{tikzpicture}
  [small mindmap,
  every node/.style={concept, execute at begin node=\hskip0pt},
  concept color = black!20,
  grow cyclic,
  level 1/.append style = {level distance = 5cm, sibling angle = 90}, 
  level 2/.append style={level distance = 2.5cm, sibling angle = 45}]
  \node [root concept] {\scshape Homotopy Type Theory}
    child { node {\scshape Computer proof assistants}
      child { node {\tiny Automath, Coq, Agda, Lean, \ldots}}
      child { node {Four color theorem}}
      child { node {Feit-Thompson theorem}}
      child { node {Kepler conjecture}}
      child { node {Formal abstracts project}}}
    child { node {\scshape Dependent type theories}
      child { node {$\lambda$-calculus}}
      child { node {\tiny Calculus of inductive constructions}}
      child { node {Cubical type theory}}
      child { node {Cohesive type theory}}}
    child { node {\scshape Category theory}
      child { node {The groupoid model}}
      child { node {Categorical semantics}}
      child { node {Realizability}}
      child { node {Toposes}}
      child { node {$n$-categories}}
      child { node {Higher topos theory}}}
    child { node {\scshape Homotopy Theory}
      child { node {Quillen model categories}}
      child { node {Homotopy groups}}
      child { node {Homology and cohomology}}
      child { node {Spectral sequences}}
      child { node {Stable homotopy theory}}
      child { node {The simplicial model}}};
\end{tikzpicture}}

\end{figure}

From a syntactic point of view, type theory is a just a deductive system, or a language with enough structure to encode (most) mathematical practice. It can be used as a constructive logic, a constructive set theory, a programming language, or indeed for `synthetic homotopy theory'. These are four points of view, where types are treated as propositions, sets, data types, or $\omega$-groupoids, respectively. In this course we will mostly focus on the latter point of view, but it is worth stressing that each of the four points of view is of fundamental importance, and will occur in the course.

In the present lecture we do not commit to either point of view, and simply present a deductive system of dependent type theory. It must be noted that \emph{types} and \emph{terms} are primitive notions in this system. Therefore, we will only learn what a type is by learning in what ways types may be used. The semantics of type theory is currently an active field of research (part of homotopy type theory), and could be the topic of an entire other course.

We start at the very beginning, describing the deductive system of dependent type theory is in \autoref{ch:dtt} without any type forming operations. Then we gradually introduce the standard type forming operations until we give Martin-L\"of's inductive definition of identity types. 

We cover the basics --- including equivalences, the univalence axiom, and higher inductive types --- and we introduce the student to the subfield of homotopy type theory that is sometimes called \emph{synthetic homotopy theory}. Early on, students will get acquainted with the basic techniques that are used in homotopy type theory to characterize the identity types of various classes of types, and once higher inductive types are introduced students will get acquainted with the descent property that can be used to construct type families over higher inductive types in order to prove properties about them.
