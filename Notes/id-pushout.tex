\chapter{The identity types of pushouts}

\begin{thm}
  Consider a pushout square
  \begin{equation*}
    \begin{tikzcd}
      S \arrow[r,"g"] \arrow[d,swap,"f"] & B \arrow[d,"j"] \\
      A \arrow[r,swap,"i"] & X
    \end{tikzcd}
  \end{equation*}
  with $H:i\circ f \htpy j\circ g$, and let $a:A$. Furthermore consider $P\jdeq(P_A,P_B,r,e)$ consisting of type families $P_A$ over $A$ and $P_B$ over $B$, equipped with a point $r:P_A(a)$, and a family of equivalences
  \begin{equation*}
    e:\prd{s:S} P_A(f(s)) \simeq P_B(g(s)).
  \end{equation*}
  If $P$ is the initial such object, then we have two families of equivalences
  \begin{align*}
    \alpha & : \prd{x:A}P_A(x)\simeq (i(a)=i(x)) \\
    \beta & : \prd{y:B} P_B(y)\simeq (i(a)=j(b)) 
  \end{align*}
  such that $\alpha(a,r)=\refl{i(a)}$, and the square
  \begin{equation*}
    \begin{tikzcd}[column sep=huge]
      P_A(f(s)) \arrow[r,"e(s)"] \arrow[d,swap,"\alpha(f(s))"] & P_B(g(s)) \arrow[d,"\beta(g(s))"] \\
      (i(a)=i(f(s))) \arrow[r,swap,"\lam{p}\ct{p}{H(s)}"] & (i(a)=g(s))
    \end{tikzcd}
  \end{equation*}
  commutes for each $s:S$
\end{thm}

\begin{thm}
  Let $X$ be a pointed type with base point $x_0:X$. Then the loop space of $\susp{X}$ is the initial type $Y$ equipped with a base point $y_0:Y$, and a pointed map
  \begin{equation*}
    X \to_\ast (Y\simeq Y).
  \end{equation*}
\end{thm}

\begin{cor}
  The loop space of $\sphere{2}$ is the initial type $X$ equipped with a point $x_0:X$ and a homotopy $H:\idfunc\htpy\idfunc$.
\end{cor}

\begin{exercises}
\item Show that if $X$ has decidable equality, then $\susp{X}$ is a $1$-type.
\item Consider a pushout square
  \begin{equation*}
    \begin{tikzcd}
      A \arrow[r] \arrow[d,swap,"f"] & \unit \arrow[d,"j"] \\
      B \arrow[r,swap,"i"] & X
    \end{tikzcd}
  \end{equation*}
  where $f:A\to B$ is an embedding.
  \begin{subexenum}
  \item Show that there are equivalences
  \begin{align*}
    (i(b)=i(y)) & \simeq (b=y)\ast \fib{f}{b} \\
    (i(b)=j(\ttt)) & \simeq \fib{f}{b}
  \end{align*}
  for any $b,y:B$.
  \item Use \cref{ex:trunc-join-with-prop} to show that if $B$ is a $k$-type, then so is $X$, for any $k\geq 0$.
  \end{subexenum}
\end{exercises}
