\chapter{Cubical diagrams}

\section{Commuting cubes}
\begin{defn}\label{defn:cube}
A \define{commuting cube}\index{commuting cube|textbf}
\begin{equation*}
\begin{tikzcd}
& A_{111} \arrow[dl] \arrow[dr] \arrow[d] \\
A_{110} \arrow[d] & A_{101} \arrow[dl] \arrow[dr] & A_{011} \arrow[dl,crossing over] \arrow[d] \\
A_{100} \arrow[dr] & A_{010} \arrow[d] \arrow[from=ul,crossing over] & A_{001} \arrow[dl] \\
& A_{000},
\end{tikzcd}
\end{equation*}
consists of 
\begin{enumerate}
\item types
\begin{equation*}
A_{111},A_{110},A_{101},A_{011},A_{100},A_{010},A_{001},A_{000},
\end{equation*}
\item \begin{samepage}%
maps
\begin{align*}
f_{11\check{1}} & : A_{111}\to A_{110} & f_{\check{1}01} & : A_{101}\to A_{001} \\
f_{1\check{1}1} & : A_{111}\to A_{101} & f_{01\check{1}} & : A_{011}\to A_{010} \\
f_{\check{1}11} & : A_{111}\to A_{011} & f_{0\check{1}1} & : A_{011}\to A_{001} \\
f_{1\check{1}0} & : A_{110}\to A_{100} & f_{\check{1}00} & : A_{100}\to A_{000} \\
f_{\check{1}10} & : A_{110}\to A_{010} & f_{0\check{1}0} & : A_{010}\to A_{000} \\
f_{10\check{1}} & : A_{101}\to A_{100} & f_{00\check{1}} & : A_{001}\to A_{000},
\end{align*}
\end{samepage}%
\item homotopies
\begin{align*}
H_{1\check{1}\check{1}} & : f_{1\check{1}0}\circ f_{11\check{1}} \htpy f_{10\check{1}}\circ f_{1\check{1}1} & H_{0\check{1}\check{1}} & : f_{0\check{1}0}\circ f_{01\check{1}} \htpy f_{00\check{1}}\circ f_{0\check{1}1} \\
H_{\check{1}1\check{1}} & : f_{\check{1}10}\circ f_{11\check{1}} \htpy f_{01\check{1}}\circ f_{\check{1}11} & H_{\check{1}0\check{1}} & : f_{\check{1}00}\circ f_{10\check{1}} \htpy f_{00\check{1}}\circ f_{\check{1}01} \\
H_{\check{1}\check{1}1} & : f_{\check{1}01}\circ f_{1\check{1}1} \htpy f_{0\check{1}1}\circ f_{\check{1}11} & H_{\check{1}\check{1}0} & : f_{\check{1}00}\circ f_{1\check{1}0} \htpy f_{0\check{1}0}\circ f_{\check{1}10},
\end{align*}
\item and a homotopy 
\begin{align*}
C & : \ct{(f_{\check{1}00}\cdot H_{1\check{1}\check{1}})}{(\ct{(H_{\check{1}0\check{1}}\cdot f_{1\check{1}1})}{(f_{00\check{1}}\cdot H_{\check{1}\check{1}1})})} \\
& \qquad \htpy \ct{(H_{\check{1}\check{1}0}\cdot f_{11\check{1}})}{(\ct{(f_{0\check{1}0}\cdot H_{\check{1}1\check{1}})}{(H_{0\check{1}\check{1}}\cdot f_{\check{1}11})})}
\end{align*}
filling the cube.
\end{enumerate}
\end{defn}

\begin{lem}
Given a commuting cube as in \cref{defn:cube} we obtain a commuting square
\begin{equation*}
\begin{tikzcd}
\fib{f_{1\check{1}1}}{x} \arrow[r] \arrow[d] & \fib{f_{0\check{1}1}}{f_{\check{1}01}(x)} \arrow[d] \\
\fib{f_{1\check{1}0}}{f_{10\check{1}}(x)} \arrow[r] & \fib{f_{0\check{1}0}}{f_{00\check{1}}(x)}
\end{tikzcd}
\end{equation*}
for any $x:A_{101}$. 
\end{lem}

\begin{lem}
Consider a commuting cube
\begin{equation*}
\begin{tikzcd}
& A_{111} \arrow[dl] \arrow[dr] \arrow[d] \\
A_{110} \arrow[d] & A_{101} \arrow[dl] \arrow[dr] & A_{011} \arrow[dl,crossing over] \arrow[d] \\
A_{100} \arrow[dr] & A_{010} \arrow[d] \arrow[from=ul,crossing over] & A_{001} \arrow[dl] \\
& A_{000}.
\end{tikzcd}
\end{equation*}
If the bottom and front right squares are pullback squares, then the back left square is a pullback if and only if the top square is.
\end{lem}

\begin{rmk}
By rotating the cube we also obtain:
\begin{enumerate}
\item If the bottom and front left squares are pullback squares, then the back right square is a pullback if and only if the top square is.
\item If the front left and front right squares are pullback, then the back left square is a pullback if and only if the back right square is.
\end{enumerate}
By combining these statements it also follows that if the front left, front right, and bottom squares are pullback squares, then if any of the remaining three squares are pullback squares, all of them are. Cubes that consist entirely of pullback squares are sometimes called \define{strongly cartesian}\index{strongly cartesian cube}.
\end{rmk}

\section{Fiberwise pullbacks}

\begin{lem}
Consider a pullback square\index{pullback!Sigma-type of pullbacks@{$\Sigma$-type of pullbacks}}
  \begin{equation*}
    \begin{tikzcd}
      C \arrow[r,"q"] \arrow[d,swap,"p"] & B \arrow[d,"g"] \\
      A \arrow[r,swap,"f"] & X
    \end{tikzcd}
  \end{equation*}
  with $H : f \circ p \htpy g \circ h$. Furthermore, consider type families $P_X$, $P_A$, $P_B$, and $P_C$ over $X$, $A$, $B$, and $C$ respectively, equipped with families of maps
  \begin{align*}
    f' & : \prd{a:A} P_A(a) \to P_X(f(a)) \\
    g' & : \prd{b:B} P_B(b) \to P_X(g(b)) \\
    p' & : \prd{c:C} P_C(c) \to P_A(p(c)) \\
    q' & : \prd{c:C} P_C(c) \to P_B(q(c)),
  \end{align*}
  and for each $c:C$ a homotopy $H'_c$ witnessing that the square
  \begin{equation}\label{eq:family-squares-pullback}
    \begin{tikzcd}
      P_C(c) \arrow[rr,"{q'_c}"] \arrow[d,swap,"{p'_c}"] & &[3em] P_B(q(c)) \arrow[d,"{g'_{q(c)}}"] \\
      P_A(p(c)) \arrow[r,swap,"{f'_{p(c)}}"] & P_X(f(p(c))) \arrow[r,swap,"{\tr_{P_X}(H(c))}"] & P_X(g(q(c)))
    \end{tikzcd}
  \end{equation}
  commutes. Then the following are equivalent:
  \begin{enumerate}
  \item For each $c:C$ the square in \cref{eq:family-squares-pullback} is a pullback square.
  \item The square
    \begin{equation}\label{eq:total-square-pullback}
      \begin{tikzcd}[column sep=huge]
        \sm{c:C}P_C(c)
        \arrow[r,"{\total[q]{q'}}"] \arrow[d,swap,"{\total[p]{p'}}"] &
        \sm{b:B}P_B(b) \arrow[d,"{\total[g]{g'}}"] \\
        \sm{a:A}P_A(a) \arrow[r,swap,"{\total[f]{f'}}"] & \sm{x:X}P_X(x)
      \end{tikzcd}
    \end{equation}
    is a pullback square.
  \end{enumerate}
\end{lem}


\begin{cor}
Consider a pullback square
\begin{equation*}
\begin{tikzcd}
C \arrow[r,"q"] \arrow[d,swap,"p"] & B \arrow[d,"g"] \\
A \arrow[r,swap,"f"] & X,
\end{tikzcd}
\end{equation*}
with $H:f\circ p\htpy g\circ q$, and let $c_1,c_2:C$. Then the square
\begin{equation*}
\begin{tikzcd}[column sep=8em]
(c_1=c_2) \arrow[r,"\apfunc{q}"] \arrow[d,swap,"\apfunc{p}"] & (q(c_1)=q(c_2)) \arrow[d,"\lam{\beta}\ct{H(c_1)}{\ap{g}{\beta}}"] \\
(p(c_1)=p(c_2)) \arrow[r,swap,"\lam{\alpha}\ct{\ap{f}{\alpha}}{H(c_2)}"] & f(p(c_1))=g(q(c_2)),
\end{tikzcd}
\end{equation*}
commutes and is a pullback square.
\end{cor}


\begin{thm}
  Consider a commuting cube
  \begin{equation*}
    \begin{tikzcd}
      & C' \arrow[dl] \arrow[dr] \arrow[d] \\
      A' \arrow[d] & C \arrow[dl] \arrow[dr] & B' \arrow[crossing over,dl] \arrow[d] \\
      A \arrow[dr] & X' \arrow[d] \arrow[from=ul,crossing over] & B \arrow[dl] \\
      & X
    \end{tikzcd}
  \end{equation*}
  in which the bottom square is a pullback square. Then the following are equivalent:
  \begin{enumerate}
  \item The top square is a pullback square.
  \item The square
    \begin{equation*}
      \begin{tikzcd}
        \fib{\gamma}{c} \arrow[d] \arrow[r] & \fib{\beta}{q(c)} \arrow[d] \\
        \fib{\alpha}{p(c)} \arrow[r] & \fib{\varphi}{f(p(c))}
      \end{tikzcd}
    \end{equation*}
    is a pullback square for each $c:C$.
  \end{enumerate}
\end{thm}


\section{The 3-by-3-properties for pullbacks and pushouts}

\begin{thm}
  Consider a commuting diagram of the form
  \begin{equation*}
    \begin{tikzcd}
      A_0 \arrow[r] \arrow[d] & B_0 \arrow[d] & C_0 \arrow[l] \arrow[d] \\
      A_1 \arrow[r] & B_1 & C_1 \arrow[l] \\
      A_2 \arrow[u] \arrow[r] & B_2 \arrow[u] & C_2 \arrow[u] \arrow[l]
    \end{tikzcd}
  \end{equation*}
  with homotopies filling the (small) squares. Furthermore, consider
  pullback squares
  \begin{equation*}
    \begin{tikzcd}
      D_0 \arrow[r] \arrow[d] & C_0 \arrow[d] & D_1 \arrow[r] \arrow[d] & C_1 \arrow[d] & D_2 \arrow[r] \arrow[d] & C_2 \arrow[d] \\
      A_0 \arrow[r] & B_0 & A_1 \arrow[r] & B_1 & A_2 \arrow[r] & B_2
    \end{tikzcd}
  \end{equation*}
  \begin{equation*}
    \begin{tikzcd}
      A_3 \arrow[r] \arrow[d] & A_2 \arrow[d] & B_3 \arrow[r] \arrow[d] & B_2 \arrow[d] & C_3 \arrow[r] \arrow[d] & C_2 \arrow[d] \\
      A_0 \arrow[r] & A_1 & B_0 \arrow[r] & B_1 & C_0 \arrow[r] & C_1.
    \end{tikzcd}
  \end{equation*}
  Finally, consider a commuting square
  \begin{equation*}
    \begin{tikzcd}
      D_3 \arrow[r] \arrow[d] & D_2 \arrow[d] \\
      D_0 \arrow[r] & D_1.
    \end{tikzcd}
  \end{equation*}
  Then the following are equivalent:
  \begin{enumerate}
  \item This square is a pullback square.
  \item The induced square
    \begin{equation*}
      \begin{tikzcd}
        D_3 \arrow[r] \arrow[d] & C_3 \arrow[d] \\
        A_3 \arrow[r] & B_3
      \end{tikzcd}
    \end{equation*}
    is a pullback square.
  \end{enumerate}
\end{thm}


\begin{exercises}
\item Some exercises.
\end{exercises}
