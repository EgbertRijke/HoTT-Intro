% arara: makechapters: {items: [syllabus, dtt, pi, inductive, identity, equivalences, contractible, fundamental, hierarchy, funext, pullback, univalence]}

\documentclass[11pt]{memoir} %[ebook,10pt,oneside]

\title{Introduction to homotopy type theory}
\author{Egbert Rijke}
\date{Carnegie Mellon University\\Pittsburgh PA\\Spring 2018}%\\Version: \today}
%\address{Carnegie Mellon University}
%\email{erijke@andrew.cmu.edu}

\pretitle{\begin{center}\textsc\bgroup\LARGE}
\posttitle{\egroup\end{center}\vspace{2cm}}
\preauthor{\begin{center}\textsc\bgroup\Large}\postauthor{\egroup\end{center}\vfill}
\predate{\begin{center}\textsc\bgroup}{\postdate{\egroup\end{center}}

\usepackage{hott}

\usetikzlibrary{mindmap}

\makeatletter
\renewcommand{\@chapapp}{Lecture}
\makeatother

\crefname{table}{Table}{Tables}
\crefname{chapter}{Lecture}{Lectures}

\newlist{exenum}{enumerate}{1}
\setlist[exenum]{noitemsep,label=\thechapter.\arabic*}%,ref=\thechapter.\arabic*}
%\crefalias{exenumi}{Exercise} 
\crefname{exenumi}{Exercise}{Exercises}

\newlist{subexenum}{enumerate}{1}
\setlist[subexenum]{noitemsep,label=(\alph*),ref=\theexenumi.\alph*}
\crefname{subexenumi}{Exercise}{Exercises}

\newenvironment{exercises}%
{%
\section*{Exercises}\addcontentsline{toc}{section}{Exercises}\sectionmark{Exercises}
\begin{exenum}}
{%
\end{exenum}}

\addbibresource{bibliography.bib}

\makeindex

\begin{document}

\frontmatter

\begin{titlingpage}
\maketitle 
\end{titlingpage}

\tableofcontents

%\chapter{Introduction}
%\addcontentsline{toc}{chapter}{Introduction}
These are notes for the course Introduction to Homotopy Type Theory, taught at Carnegie Mellon University in the spring semester of 2018. Homotopy Type Theory (HoTT) is an emerging field of mathematics and computer science that extends Martin-Löf's dependent type theory by the addition of the univalence axiom and higher inductive types. In HoTT we think of types as spaces, dependent types as fibrations, and of the identity types as path spaces. We will see that many spaces that are familiar to topologists can be represented as higher inductive types, and we will develop the basic theorems and constructions in HoTT to reason about them.

We cover the basics --- including equivalences, the univalence axiom, and higher inductive types --- and we introduce the student to the subfield of homotopy type theory that is sometimes called \emph{synthetic homotopy theory}. Early on, students will get acquainted with the basic techniques that are used in homotopy type theory to characterize the identity types of various classes of types, and once higher inductive types are introduced students will get acquainted with the descent property that can be used to construct type families over higher inductive types in order to prove properties about them.

We do not cover the univalence axiom or the function extensionality axiom until we're about a third through the course. 
There are several reasons for this. 
First of all, dependent type theory with the inductive types, and among them the identity types, deserve to be covered thoroughly. 
Second, the univalence axiom is pretty powerful, I believe it is good to go through some `agnostic' type theory first to be able to present concisely what the univalence axiom is really about, and to feel its power once we have it. 
The final reason is that actually quite a lot can and has to be established without those axioms anyway. 
These results include the fact that a map is an equivalence if and only if its fibers are contractible, the fundamental theorem of identity types, and theorems that help us identify certain classes of types as sets.
All of these results belong properly to \emph{homotopy} type theory.

\section{The Curry-Howard correspondence}
From a logical point of view, type theory can be seen as a deductive system for constructive logic, in which types are propositions of which the constituents are precisely its proofs. In the view of Heyting, `to know the meaning of a proposition is to know which constructions can be considered as proofs of that proposition'. For instance, a proof of the proposition $A\to B$ is an algorithm that transforms proofs of $A$ into proofs of $B$.
\begin{table}
\caption{The Curry-Howard correspondence}
\begin{center}
\begin{tabular}{lll}
\toprule
\emph{First order logic} & \emph{Set theory} & \emph{Type theory}\\
\midrule
Propositions & Sets & Types\\
Predicates & Families of sets & Dependent types\\
Proofs & Elements & Terms \\
$\top$ & $\{\emptyset\}$ & $\unit$\\
$\bot$ & $\emptyset$ & $\emptyt$ \\
$P \land Q$ & $A \times B$ & $A \times B$ \\
$P \vee Q$ & $A \sqcup B$ & $A + B$ \\
$\exists x.P(x)$ & $\coprod_{i\in I}A_i$ & $\sm{x:A}B(x)$ \\
$\forall x.P(x)$ & $\prod_{i\in I}A_i$ & $\prd{x:A}B(x)$\\
\bottomrule
\end{tabular}
\end{center}
\end{table}


\chapter{Syllabus}

\section{Essential course information}
\begin{center}
\begin{tabular}{ll}
\emph{Course title} & Introduction to Homotopy Type Theory \\
\emph{Instructor} & Egbert Rijke \\
& Department of Philosophy \\
& Carnegie Mellon University \\
\emph{Course number} & 80-518, 80-818 \\
\emph{Semester} & Spring 2018 \\
\emph{Website} & \url{http://www.andrew.cmu.edu/user/erijke/hott/} \\
\emph{Lecture room} & Baker Hall 150 \\
\emph{Meeting time} & Tue/Thu 12:00 - 1:20 \\
\emph{Email} & \href{mailto:erijke@andrew.cmu.edu}{erijke@andrew.cmu.edu} \\
\emph{Instructor's office} & Baker Hall 148 \\
\emph{Office Hours} & Mon/Wed 5:00 - 6:00, or by appointment
\end{tabular}
\end{center}

\section{Course description}
Homotopy Type Theory (HoTT) is an emerging field of mathematics and computer science that extends Martin-Löf's dependent type theory by the addition of the univalence axiom and higher inductive types. In HoTT we think of types as spaces, dependent types as fibrations, and of the identity types as path spaces. We start the course by introducing type theory as a deductive system, and once the basic ingredients of homotopy type theory are in place we will mainly focus on \emph{synthetic homotopy theory}, i.e.~the development of homotopy theory in type theory.

\section{Course material}

We will roughly follow the book \emph{Homotopy Type Theory: Univalent foundation of mathematics} \cite{hottbook}, of which a PDF is freely available.

Some of the later results of synthetic homotopy theory can only be found in recent research papers. We will also use the PhD thesis of Guillaume Brunerie \cite{BruneriePhD} as a resource.

\section{Organization}

Each session will consist of two parts: a 50 minute lecture and 30 minutes in which students present solutions to exercises provided with the previous lecture. These presentations are intended to be short (roughly 5 minutes) and focused to the problem at hand. Problem sets will be posted below with the lecture synopses.

Students are expected to:
\begin{enumerate}
\item Present a solution when they are asked to do so (usually a week in advance). Graduate students will be asked to present more often than undergraduate students.
\item Per lecture, either correct a somewhat substantial mistake made by the instructor, or hand in a written solution for one exercise of their choice. Written solutions are to be handed in at the start of the next lecture for an A, or at the start of the next lecture after that for a B. Collaborations are encouraged, but solutions must be handed in individually. Presenting students hand in a written solution for the exercise they are asked to present.
\end{enumerate}

Hints for the exercises will be presented by the instructor during office hours, a day before they have to be handed in. 


\chapter{Introduction}

To include introduction:
\begin{enumerate}
\item Informal introduction to homotopy type theory
\item What are types in mathematics
\item Why univalent foundations
\item Constructive nature of homotopy type theory
\item What this course is about
\item How to use this book
\item Formal type theory versus informal type theory
\item Mention the formalization
\end{enumerate}

\begin{rmk}
  One difference between set theory and type theory is that every well-formed term is specified along with its type and with its context. One way of looking at this is that there are three sorts in type theory: contexts, types, and terms. On the other hand, there is only one sort in set theory: sets. Sets are governed by the elementhood relation: the formula $x\in y$ is a well-formed formula of set theory for any two sets $x$ and $y$. In particular, for a given set $x$, the formula $x\in y$ can be true for many sets $y$, which is very different to the situation in type theory, where every term is assigned a unique type.
  
  Another important difference between set theory and type theory is that set theory is formulated in the language of first order logic, whereas type theory is its own deductive system, not making use of any ambient logic. We will see in the present chapter and in the next few chapters what this deductive system looks like.
\end{rmk}

\mainmatter

\chapter{Dependent type theory}
\label{ch:dtt}

In this lecture we describe the deductive system of dependent type theory, without introducing yet any ways of actually forming types. In other words, we just give the rules of type dependency.

\section{The primitive judgments of type theory}

The theory of type dependency is formulated as a deductive system in which derivations establish that a given construction is well-formed. In any dependent type theory there are four \define{primitive judgments}\index{primitive judgment}:
\begin{enumerate}
%\item `\emph{$\Gamma$ is a well-formed context.}'
\item `\emph{$A$ is a (well-formed) \define{type}\index{well-formed type}\index{type} in context $\Gamma$.}'\index{primitive judgment!type in context}
\item `\emph{$A$ and $B$ are \define{judgmentally equal types} in context $\Gamma$.}'\index{primitive judgment!equal types in context}\index{judgmental equality!of types}
\item `\emph{$a$ is a (well-formed) \define{term}\index{well-formed term}\index{term} of type $A$ in context $\Gamma$.}'\index{primitive judgment!term of a type in context}
\item `\emph{$a$ and $b$ are \define{judgmentally equal terms} of type $A$ in context $\Gamma$.}'\index{primitive judgment!equal terms of a type in context}\index{judgmental equality!of terms}
\end{enumerate}
\begin{samepage}
The symbolic expressions for these four primitive judgments are as follows:
\begin{align*}
%& \vdash \Gamma~\mathrm{ctx} & & \vdash \Gamma\jdeq \Gamma'~\mathrm{ctx} \\
\Gamma & \vdash A~\textrm{type} & \Gamma & \vdash A\jdeq B~\textrm{type}\\
\Gamma & \vdash a:A & \Gamma & \vdash a\jdeq b:A.
\end{align*}
\end{samepage}
In these judgments, well-formedness of a type $A$ in context $\Gamma$ just means that the type $A$ has been formed in accordance with the rules of type theory, which we are about to describe.

A well-formed \define{context}\index{context|textbf} is an expression of the form
\begin{equation*}
x_1:A_1,~x_2:A_2(x_1),~\ldots,~x_n:A_n(x_1,\ldots,x_{n-1}),
\end{equation*}
which we often simply write as $x_1:A_1,~x_2:A_2,~\ldots,~x_n:A_n$,
satisfying the condition that for each $1\leq k\leq n$ we have that $A_k$ is a well-formed type in context $x_1:A_1,x_2:A_2,\ldots,x_{k-1}:A_{k-1}$, i.e.
\begin{equation*}
x_1:A_1,x_2:A_2,\ldots,x_{k-1}:A_{k-1} \vdash A_k~\textrm{type}.
\end{equation*}
We say that a context $x_1:A_1,~\ldots,~x_n:A_n$ \define{declares the variables}\index{variable declaration} $x_1,\ldots,x_n$. In other words, a type $A(x_1,\ldots,x_n)$ in context $\Gamma$ is well-formed if all its variables $x_1,\ldots,x_n$ are assigned well-formed types in the context $\Gamma$.

We may use variable names other than $x_1,\ldots,x_n$, as long as \emph{no variable is declared more than once.} For example, we will often use the variable names $x$, $y$, and $z$ when they are assigned a general type, variables $f$, $g$, and $h$ for function types, and so on.
%For example we used the variable names $A,\mu,u_l,p,u_r,q$ when we displayed the context of \autoref{lem:unit}.

In the special case where $n=0$, the list $x_1:A_1,x_2:A_2,\ldots,x_n:A_n$ is empty, which satisfies the well-formedness condition vacuously. In other words, the \define{empty context}\index{context!empty context|textbf}\index{empty context|textbf} is well-formed. A well-formed type in the empty context is also called a \define{closed type}\index{closed type|textbf}, and a well-formed term of a closed type is called a \define{closed term}\index{closed term|textbf}.

When $B$ is a type in context $\Gamma,x:A$, we also say that $B$ is a \define{family of types}\index{family!of types|textbf} over $A$ (in context $\Gamma$).

\section{Renaming variables}
In some situations one might want to change the name of a variable in a context. This is allowed, provided that the new variable does not occur anywhere else in the context, so that also after renaming no variable is declared more than once. 
The inference rules that rename a variable are sometimes called \define{$\alpha$-conversion}\index{alpha-conversion@{$\alpha$-conversion}}\index{conversion rules!alpha-conversion@{$\alpha$-conversion}}\index{rule!alpha conversion@{$\alpha$-conversion}} rules. 

If we are given a type $A$ in context $\Gamma$, then for any type $B$ in context $\Gamma,x:A,\Delta$ we can form the type $B[x'/x]$ in context $\Gamma,x':A,\Delta[x'/x]$, where $B[x'/x]$ is an abbreviation for
\begin{equation*}
B(x_1,\ldots,x_{n-1},x',x_{n+1},\ldots,x_{n+m-1})
\end{equation*}
This definition of \define{renaming}\index{variable renaming|textbf} the variable $x$ by $x'$ is understood to be recursive over the length of $\Delta$. The first variable renaming rule\index{rule!variable renaming} postulates that the renaming of a variable preserves well-formedness of types:
\begin{prooftree}
\AxiomC{$\Gamma,x:A,\Delta\vdash B~\mathrm{type}$}
\RightLabel{$x'/x$}
\UnaryInfC{$\Gamma,x':A,\Delta[x'/x]\vdash B[x'/x]~\mathrm{type}$}
\end{prooftree}

Similarly we obtain for any term $b:B$ in context $\Gamma,x:A,\Delta$ a term $b[x'/x]:B[x'/x]$, and there is a variable renaming rule postulating that the renaming of a variable preserves the well-formedness of terms.
In fact, there is variable renaming rule for each of the primitive judgments. To avoid having to state essentially the same rule four times in a row, we postulate the four variable renaming rules all at once using a \emph{generic judgment}\index{generic judgment} $\mathcal{J}$. 
\begin{prooftree}
\AxiomC{$\Gamma,x:A,\Delta\vdash \mathcal{J}$}
\RightLabel{$x'/x$}
\UnaryInfC{$\Gamma,x':A,\Delta[x'/x]\vdash \mathcal{J}[x'/x]$}
\end{prooftree}
where $\mathcal{J}$ may be a typing judgment, a judgment of equality of types, a term judgment, or a judgment of equality of terms.
We will use generic judgments extensively to postulate the rest of the rules of dependent type theory.

\section{Inference rules governing judgmental equality}

\begin{samepage}
Both on types and on terms, we postulate that judgmental equality is an equivalence relation. That is, we provide inference rules for the reflexivity, symmetry and transitivity of both kinds of judgmental equality\index{judgmental equality!equivalence relation}:
\begin{center}
\begin{small}
\begin{minipage}{.2\textwidth}
\begin{prooftree}
\AxiomC{$\Gamma\vdash A~\textrm{type}$}
\UnaryInfC{$\Gamma\vdash A\jdeq A~\textrm{type}$}
\end{prooftree}
\end{minipage}
\begin{minipage}{.25\textwidth}
\begin{prooftree}
\AxiomC{$\Gamma\vdash A\jdeq A'~\textrm{type}$}
\UnaryInfC{$\Gamma\vdash A'\jdeq A~\textrm{type}$}
\end{prooftree}
\end{minipage}
\begin{minipage}{.5\textwidth}
\begin{prooftree}
\AxiomC{$\Gamma\vdash A\jdeq A'~\textrm{type}$}
\AxiomC{$\Gamma\vdash A'\jdeq A''~\textrm{type}$}
\BinaryInfC{$\Gamma\vdash A\jdeq A''~\textrm{type}$}
\end{prooftree}
\end{minipage}
\\
\bigskip
\begin{minipage}{.2\textwidth}
\begin{prooftree}
\AxiomC{$\Gamma\vdash a:A$}
\UnaryInfC{$\Gamma\vdash a\jdeq a : A$}
\end{prooftree}
\end{minipage}
\begin{minipage}{.25\textwidth}
\begin{prooftree}
\AxiomC{$\Gamma\vdash a\jdeq a':A$}
\UnaryInfC{$\Gamma\vdash a'\jdeq a: A$}
\end{prooftree}
\end{minipage}
\begin{minipage}{.5\textwidth}
\begin{prooftree}
\AxiomC{$\Gamma\vdash a\jdeq a' : A$}
\AxiomC{$\Gamma\vdash a'\jdeq a'': A$}
\BinaryInfC{$\Gamma\vdash a\jdeq a'': A$}
\end{prooftree}
\end{minipage}
\end{small}
\end{center}
\end{samepage}

Apart from the rules postulating that judgmental equality is an equivalence relation, there are also \define{variable conversion rules}\index{judgmental equality!conversion rules}\index{variable conversion rules}\index{conversion rule!variable}\index{rule!variable conversion}.
Informally, these are rules stating that if $A$ and $A'$ are judgmentally equal types in context $\Gamma$, then any valid judgment in context $\Gamma,x:A$ is also a valid judgment in context $\Gamma,x:A'$. In other words: we can convert the type of a variable to a judgmentally equal type. We state this with a generic judgment $\mathcal{J}$
\begin{prooftree}
\AxiomC{$\Gamma\vdash A\jdeq A'~\textrm{type}$}
\AxiomC{$\Gamma,x:A,\Delta\vdash \mathcal{J}$}
\RightLabel{$A'/A$}
\BinaryInfC{$\Gamma,x:A',\Delta\vdash \mathcal{J}$}
\end{prooftree}
An analogous \emph{term conversion rule}\index{term conversion rule}\index{conversion rule!term}\index{rule!term conversion}, stated in \cref{ex:term_conversion}, converting the type of a term to a judgmentally equal type, is derivable using the `structural rules' of type theory described in the next section.


\section{Structural rules of type theory}

We complete the specification of dependent type theory by postulating rules for \emph{weakening} and \emph{substitution}, and the \emph{variable rule}:
\begin{enumerate}
\item If we are given a type $A$ in context $\Gamma$, then any judgment made in a longer context $\Gamma,\Delta$ can also be made in the context $\Gamma,x:A,\Delta$, for a fresh variable $x$. The \define{weakening rule}\index{weakening}\index{rule!weakening} asserts that weakening by a type $A$ in context preserves well-formedness and judgmental equality of types and terms.
\begin{prooftree}
\AxiomC{$\Gamma\vdash A~\textrm{type}$}
\AxiomC{$\Gamma,\Delta\vdash \mathcal{J}$}
\RightLabel{$W_A$}
\BinaryInfC{$\Gamma,x:A,\Delta \vdash \mathcal{J}$}
\end{prooftree}
This process of expanding the context by a fresh variable of type $A$ is called \define{weakening (by $A$)}.

For example, when we have two types $A$ and $B$ in context $\Gamma$, we can weaken $B$ by $A$ as follows
\begin{prooftree}
  \AxiomC{$\Gamma\vdash A~\textrm{type}$}
  \AxiomC{$\Gamma\vdash B~\textrm{type}$}
  \RightLabel{$W_A$}
  \BinaryInfC{$\Gamma,x:A\vdash B~\mathrm{type}$}
\end{prooftree}
in order to form the type $B$ in context $\Gamma,x:A$. The type $B$ in context $\Gamma,x:A$ is also called the \define{constant family}\index{family!constant family}\index{constant family} $B$, or the \define{trivial family}\index{family!trivial family}\index{trivial family} $B$.
\item If we are given a type $A$ in context $\Gamma$, then $x$ is a well-formed term of type $A$ in context $\Gamma,x:A$.
\begin{prooftree}
\AxiomC{$\Gamma\vdash A~\textrm{type}$}
\RightLabel{$\delta_A$}
\UnaryInfC{$\Gamma,x:A\vdash x:A$}
\end{prooftree}
This is called the \define{variable rule}\index{variable rule}\index{rule!variable rule|textbf}. It provides an \emph{identity function}\index{identity function} on the type $A$ in context $\Gamma$.
\item If we are given a term $a:A$ in context $\Gamma$, then for any type $B$ in context $\Gamma,x:A,\Delta$ we can form the type $B[a/x]$ in context $\Gamma,\Delta[a/x]$, where $B[a/x]$ is an abbreviation for
\begin{equation*}
B(x_1,\ldots,x_{n-1},a(x_1,\ldots,x_{n-1}),x_{n+1},\ldots,x_{n+m-1})
\end{equation*}
This definition of substituting $a$ for $x$ is understood to be recursive over the length of $\Delta$. Similarly we obtain for any term $b:B$ in context $\Gamma,x:A,\Delta$ a term $b[a/x]:B[a/x]$. The \define{substitution rule}\index{substitution}\index{rule!substitution} asserts that substitution preserves well-formedness and judgmental equality of types and terms:
\begin{prooftree}
\AxiomC{$\Gamma\vdash a:A$}
\AxiomC{$\Gamma,x:A,\Delta\vdash \mathcal{J}$}
\RightLabel{$S_a$}
\BinaryInfC{$\Gamma,\Delta[a/x]\vdash \mathcal{J}[a/x]$}
\end{prooftree}
Furthermore, we postulate that substitution by judgmentally equal terms results in judgmentally equal types
\begin{prooftree}
\AxiomC{$\Gamma\vdash a\jdeq a':A$}
\AxiomC{$\Gamma,x:A,\Delta\vdash B~\mathrm{type}$}
\BinaryInfC{$\Gamma,\Delta[a/x]\vdash B[a/x]\jdeq B[a'/x]~\mathrm{type}$}
\end{prooftree}
and it also results in judgmentally equal terms
\begin{prooftree}
\AxiomC{$\Gamma\vdash a\jdeq a':A$}
\AxiomC{$\Gamma,x:A,\Delta\vdash b:B$}
\BinaryInfC{$\Gamma,\Delta[a/x]\vdash b[a/x]\jdeq b[a'/x]:B[a/x]$}
\end{prooftree}
When $B$ is a family of types over $A$ and $a:A$, we also say that $B[a/x]$ is the \define{fiber}\index{family!fiber of}\index{fiber!of a family} of $B$ at $a$. Often we write $B(a)$ for $B[a/x]$.
\end{enumerate}




\begin{comment}
\bigskip
\begin{minipage}{.45\textwidth}
\begin{prooftree}
\AxiomC{$\Gamma\vdash A~\textrm{type}$}
\AxiomC{$\Gamma,\Delta\vdash B~\textrm{type}$}
\RightLabel{$W_A$}
\BinaryInfC{$\Gamma,x:A,\Delta \vdash B~\textrm{type}$}
\end{prooftree}
\end{minipage}\hfill
\begin{minipage}{.45\textwidth}
\begin{prooftree}
\AxiomC{$\Gamma\vdash A~\textrm{type}$}
\AxiomC{$\Gamma,\Delta\vdash b:B$}
\RightLabel{$W_A$}
\BinaryInfC{$\Gamma,x:A,\Delta \vdash b:B$}
\end{prooftree}
\end{minipage}

\noindent
\begin{prooftree}
\AxiomC{$\Gamma\vdash A~\textrm{type}$}
\RightLabel{$\delta_A$}
\UnaryInfC{$\Gamma,x:A\vdash x:A$}
\end{prooftree}

\noindent
\begin{minipage}{.5\textwidth}
\begin{prooftree}
\AxiomC{$\Gamma\vdash a:A$}
\AxiomC{$\Gamma,x:A,\Delta\vdash B~\textrm{type}$}
\RightLabel{$S_a$}
\BinaryInfC{$\Gamma,\Delta[a/x]\vdash B[a/x]~\textrm{type}$}
\end{prooftree}
\end{minipage}\hfill
\begin{minipage}{.5\textwidth}
\begin{prooftree}
\AxiomC{$\Gamma\vdash a:A$}
\AxiomC{$\Gamma,x:A,\Delta\vdash b:B$}
\RightLabel{$S_a$}
\BinaryInfC{$\Gamma,\Delta[a/x]\vdash b[a/x] : B[a/x]$}
\end{prooftree}
\end{minipage}

\bigskip
\end{comment}


\begin{eg}
To give an example of how the deductive system works, we give a deduction for the \define{interchange rule}\index{rule!interchange}\index{interchange rule}
\begin{prooftree}
\AxiomC{$\Gamma\vdash B~\textrm{type}$}
\AxiomC{$\Gamma,x:A,y:B,\Delta\vdash \mathcal{J}$}
\BinaryInfC{$\Gamma,y:B,x:A,\Delta\vdash \mathcal{J}$}
\end{prooftree}
In other words, if we have two types $A$ and $B$ in context $\Gamma$, and we make a judgment in context $\Gamma,x:A,y:B$, then we can make that same judgment in context $\Gamma,y:B,x:A$.
The derivation is as follows:
\begin{small}
\begin{prooftree}
\AxiomC{$\Gamma\vdash B~\textrm{type}$}
\RightLabel{$\delta_B$}
\UnaryInfC{$\Gamma,y:B\vdash y:B$}
\RightLabel{$W_{W_B(A)}$}
\UnaryInfC{$\Gamma,y:B,x:A\vdash y:B$}
\AxiomC{$\Gamma,x:A,y:B,\Delta\vdash \mathcal{J}$}
\RightLabel{$y'/y$}
\UnaryInfC{$\Gamma,x:A,y':B,\Delta[y'/y]\vdash \mathcal{J}[y'/y]$}
\RightLabel{$W_B$}
\UnaryInfC{$\Gamma,y:B,x:A,y':B,\Delta[y'/y]\vdash \mathcal{J}[y'/y]$}
\RightLabel{$S_{W_A(y)}$}
\BinaryInfC{$\Gamma,y:B,x:A,\Delta\vdash \mathcal{J}$}
\end{prooftree}
\end{small}
\end{eg}


\begin{comment}
For $A\in T_n$ we define $T_{n+k+1}(A):= \{B\in T_{n+k}\mid \mathrm{ft}^{k+1}(B)=A\}$. Similarly we define $\tilde{T}_{n+k+1}(A):=\{b\in\tilde{T}_{n+k+1}\mid\mathrm{ft}^{k+1}(\partial(b))=A\}$. For any closed type $A$ we maintain the convention that $T_{k}(\mathrm{ft}(A)):= T_k$.
\begin{enumerate}
\item For all $A\in T_n$, and all $k\in\N$, \define{weakening} operations
\begin{align*}
W_A & : T_{n+k}(\mathrm{ft}(A)) \to T_{n+k+1}(A) \\
\tilde{W}_A & : \tilde{T}_{n+k}(\mathrm{ft}(A))\to \tilde{T}_{n+k+1}(A)
\end{align*}
subject to the conditions $\mathrm{ft}(W_A(B))=W_A(\mathrm{ft}(B))$ if $B\in T_{n+k}$ with $k\geq 1$, and $\partial(\tilde{W}_A(b))=W_A(\partial(b))$.
\item For all $A\in T_n$ a term $\delta_A\in \tilde{T}_{n+1}$ satisfying $\partial(\delta_A)=W_A(A)$. 
\item For all $a\in \tilde{T}_n$ satisfying $\partial(a)=A$, and all $k\in\N$, \define{substitution} operations
\begin{align*}
S_a & : \{B\in T_{n+k+1}\mid \mathrm{ft}^{k+1}(B)= A\}\to T_k \\
\tilde{S}_a & : \{b\in \tilde{T}_{n+k+1}\mid \mathrm{ft}^{k+1}(\partial(b))=A\}\to \tilde{T}_{n+k}
\end{align*}
subject to the conditions $\mathrm{ft}(S_a(B))=\mathrm{ft}(A)$ if $B\in T_{n+1}$, $\mathrm{ft}(S_a(B))=S_a(\mathrm{ft}(B))$ if $B\in T_{n+k+1}$ with $k\geq 1$, and $\partial(\tilde{S}_a(b))=S_a(\partial(b))$.
\end{enumerate}
\end{comment}

%\section{Axioms for weakening, substitution, and the diagonal}
\begin{comment}
\begin{prooftree}
\AxiomC{$\Gamma\vdash A~\textrm{type}$}
  \AxiomC{$\Gamma,x:A,\Delta\vdash B~\textrm{type}$}
    \AxiomC{$\Gamma,x:A,\Delta,y:B,E\vdash C~\textrm{type}$}
\TrinaryInfC{$\Gamma,\Delta[a/x],E[b/y][a/x]\vdash C[b/y][a/x]\jdeq C[a/x][b[a/x]/y']~\textrm{type}$}
\end{prooftree}
\begin{prooftree}
\AxiomC{$\Gamma\vdash A~\textrm{type}$}
  \AxiomC{$\Gamma,x:A,\Delta\vdash B~\textrm{type}$}
    \AxiomC{$\Gamma,x:A,\Delta,y:B,E\vdash c:C$}
\TrinaryInfC{$\Gamma,\Delta[a/x],E[b/y][a/x]\vdash c[b/y][a/x]\jdeq c[a/x][b[a/x]/y']:C[b/y][a/x]$}
\end{prooftree}
\begin{prooftree}
\AxiomC{$\Gamma\vdash a:A$}
  \AxiomC{$\Gamma,\Delta\vdash B~\textrm{type}$}
\RightLabel{($x$ not free in $B$)}
\BinaryInfC{$\Gamma,\Delta\vdash B[a/x]\jdeq B~\textrm{type}$}
\end{prooftree}
\end{comment}


\begin{comment}
\section{An algebraic presentation of dependent type theory}

%Let us write $T_n$ for the set of well-formed contexts of length $n$, for $n>1$. Then any well-formed context of length $n+1$ is of the form $\Gamma,x:A$, where $\Gamma$ is a well-formed context of length $n$. Thus we see that there are maps $\eft:T_{n+1}\to T_n$ for $n>1$. Similarly, if we write $\tilde{T}_n$ for the set of all terms of a type in a context of length $n-1$, we see that there is a map $\tilde{T}_n\to T_n$. The following picture emerges:
%\begin{equation*}
%\begin{tikzcd}
%\tilde{T}_3 \arrow[dr,"\partial"] & \vdots \arrow[d,"\mathrm{ft}"] \\
%\tilde{T}_2 \arrow[dr,"\partial"] & T_3 \arrow[d,"\mathrm{ft}"] \\
%\tilde{T}_1 \arrow[dr,"\partial"] & T_2 \arrow[d,"\mathrm{ft}"] \\
%& T_1
%\end{tikzcd}
%\end{equation*}

Observe that given a type $A$ in context $\Gamma$ and a type $B$ in context $\Gamma,\Delta$ we can weaken twice by first weakening by $B$ and then by $A$, as indicated in the following derivation:
\begin{prooftree}
\AxiomC{$\Gamma\vdash A~\textrm{type}$}
\AxiomC{$\Gamma,\Delta\vdash B~\textrm{type}$}
  \AxiomC{$\Gamma,\Delta,\mathrm{E}\vdash \mathcal{J}$}
\BinaryInfC{$\Gamma,\Delta,y:B,\mathrm{E}\vdash \mathcal{J}$}
\BinaryInfC{$\Gamma,x:A,\Delta,y:B,\mathrm{E}\vdash \mathcal{J}$}
\end{prooftree}
However, we can also first weaken by $A$, and then by `$B$ weakened by $A$', as indicated in the following derivation:
\begin{prooftree}
\AxiomC{$\Gamma\vdash A~\textrm{type}$}
  \AxiomC{$\Gamma,\Delta\vdash B~\textrm{type}$}
\BinaryInfC{$\Gamma,x:A,\Delta\vdash B~\textrm{type}$}
  \AxiomC{$\Gamma\vdash A~\textrm{type}$}
    \AxiomC{$\Gamma,\Delta,\mathrm{E}\vdash \mathcal{J}$}
  \BinaryInfC{$\Gamma,x:A,\Delta,\mathrm{E}\vdash \mathcal{J}$}
\BinaryInfC{$\Gamma,x:A,\Delta,y:B,\mathrm{E}\vdash \mathcal{J}$}
\end{prooftree}
For the end result it does not matter in what order the two weakenings are performed. We can express this conveniently as an equation:
\begin{equation*}
W_A(W_B(\mathcal{J}))\jdeq W_{W_A(B)}(W_A(\mathcal{J})).
\end{equation*}
The complete list of rules (in alphabetic order) is
\begin{align*}
S_a(\delta_B) & \jdeq \delta_{S_a(B)} \\
S_a(\delta_A) & \jdeq a \\
S_a(S_b(\mathcal{J})) & \jdeq S_{S_a(b)}(S_a(\mathcal{J})) \\
S_a(W_A(\mathcal{J})) & \jdeq \mathcal{J} \\
S_a(W_B(\mathcal{J})) & \jdeq W_{S_a(B)}(S_a(\mathcal{J})) \\
S_{\delta_A}(W_{W_A}(\mathcal{J})) & \jdeq \mathcal{J} \\
W_A(\delta_B) & \jdeq \delta_{W_A(B)} \\
W_A(S_b(\mathcal{J})) & \jdeq S_{W_A(b)}(W_A(\mathcal{J})) \\
W_A(W_B(\mathcal{J})) & \jdeq W_{W_A(B)}(W_A(\mathcal{J}))
\end{align*}
Here $A$ is a type in context $\Gamma$ and $a$ is a term of type $A$, $B$ is a type in context $\Gamma,x:A,\Delta$ and $b$ is a term of type $B$.
\end{comment}

%\begin{rmk}
%To avoid overly long proof trees maintain as a convention that every derivation with hypotheses $\mathcal{H}_1,\ldots,\mathcal{H}_n$ and conclusion $\mathcal{C}$ can be used as a rule
%\begin{prooftree}
%\AxiomC{$\mathcal{H}_1$}
%\AxiomC{$\cdots$}
%\AxiomC{$\mathcal{H}_n$}
%\TrinaryInfC{$\mathcal{C}$}
%\end{prooftree}
%in later derivations.
%\end{rmk}

\begin{exercises}
\item \label{ex:term_conversion}Give a derivation for the following conversion rule\index{term conversion rule}\index{term conversion rule}\index{rule!term conversion}\index{term conversion rule}\index{conversion rule!term}:
\begin{prooftree}
\AxiomC{$\Gamma\vdash A\jdeq A'~\textrm{type}$}
\AxiomC{$\Gamma\vdash a:A$}
\BinaryInfC{$\Gamma\vdash a:A'$}
\end{prooftree}
\begin{comment}
\item Consider a type $A$ in context $\Gamma$. In this exercise we establish a connection between types in context $\Gamma,x:A$, and uniform choices of types $B_a$, where $a$ ranges over terms of $A$ in a uniform way. A similar connection is made for terms.
\begin{subexenum}
\item We define a \define{uniform family} over $A$ to consist of a type
\begin{equation*}
\Delta,\Gamma\vdash B_a~\mathrm{type}
\end{equation*}
for every context $\Delta$, and every term $\Delta,\Gamma\vdash a:A$, subject to the condition that one can derive
\begin{prooftree}
\AxiomC{$\Delta\vdash d:D$}
\AxiomC{$\Delta,y:D,\Gamma\vdash a:A$}
\BinaryInfC{$\Delta,\Gamma\vdash B_a[d/y]\jdeq B_{a[d/y]}~\mathrm{type}$}
\end{prooftree}
Define a bijection between types in context $\Gamma,x:A$ and uniform families over $A$. 
\item Consider a type $\Gamma,x:A\vdash B$. We define a \define{uniform term} of $B$ over $A$ to consist of a type
\begin{equation*}
\Delta,\Gamma\vdash b_a:B[a/x]~\mathrm{type}
\end{equation*}
for every context $\Delta$, and every term $\Delta,\Gamma\vdash a:A$, subject to the condition that one can derive
\begin{prooftree}
\AxiomC{$\Delta\vdash d:D$}
\AxiomC{$\Delta,y:D,\Gamma\vdash a:A$}
\BinaryInfC{$\Delta,\Gamma\vdash b_a[d/y]\jdeq b_{a[d/y]}:B[a/x][d/y]$}
\end{prooftree}
Define a bijection between terms of $B$ in context $\Gamma,x:A$ and uniform terms of $B$ over $A$. 
\end{subexenum}
\end{comment}
\end{exercises}


\section{Dependent function types and the natural numbers}
\sectionmark{\texorpdfstring{$\Pi$}{Π}-types and the natural numbers}

In this lecture we introduce types of functions of which the output can depend on the input. We call such functions \emph{dependent functions}. We will see that ordinary function types are a special case of dependent function types.

We then introduce the type of natural numbers, which is the single most important type. Dependent function types are used to formulate a type theoretic analogue of the induction principle for the natural numbers.

\subsection{Dependent function types}
Consider a section $b$ of a family $B$ over $A$ in context $\Gamma$, i.e.,
\begin{equation*}
  \Gamma,x:A\vdash b(x):B(x).
\end{equation*}
From one point of view, such a section $b$ is an operation, or a program, that takes as input $x:A$ and produces a term $b(x):B(x)$. From a more mathematical point of view we see $b$ as a function of which the type of the output may vary over its input. We call such functions \define{dependent functions}.

In this section we postulate rules for the \emph{type} of all such dependent functions: whenever $B$ is a family over $A$ in context $\Gamma$, there is a type
\begin{equation*}
  \prd{x:A}B(x)
\end{equation*}
in context $\Gamma$, consisting of all the dependent functions of which the output at $x:A$ has type $B(x)$. The rules for dependent function types are organized in four groups:
\begin{enumerate}
\item The formation rule, which tells us how we may form dependent function types. Along with the formation rules we will also postulate conversion rules of judgmental type-equality for dependent function types.
\item The introduction rule, which tells us how to introduce new terms of dependent function types, and the elimination rule tells us how to use a term of a dependent function type. Along with the introduction rule we postulate conversion rules that assert that our way of introducing functions behaves well with respect to judgmental equality.
\item The elimination rule, which tells us how to use arbitrary terms of dependent function types. We also postulate conversion rules that assert that elimination of functions behaves well with respect to judgmental equality.
\item The computation rules, which tell us how the introduction and elimination rules interact, and indeed that every term of a dependent function type behaves as expected: a function.
\end{enumerate}

\subsubsection{The $\Pi$-formation rule}
\define{Dependent function types}\index{dependent function type}\index{pi-type@{$\Pi$-type}} are formed by the following \define{$\Pi$-formation rule}\index{rule!pi-formation@{$\Pi$-formation}}:
\begin{prooftree}
\AxiomC{$\Gamma,x:A\vdash B(x)~\textrm{type}$}
\RightLabel{$\Pi$.}
\UnaryInfC{$\Gamma\vdash \prd{x:A}B(x)~\mathrm{type}$}
\end{prooftree}
With the following conversion rule we postulate that formation of dependent function types respects judgmental equality:
\begin{prooftree}
\AxiomC{$\Gamma\vdash A\jdeq A'~\mathrm{type}$}
\AxiomC{$\Gamma,x:A\vdash B(x)\jdeq B'(x)~\textrm{type}$}
\RightLabel{$\Pi$-eq.}
\BinaryInfC{$\Gamma\vdash \prd{x:A}B(x)\jdeq\prd{x:A'}B'(x)~\mathrm{type}$}
\end{prooftree}
Furthermore, when $x'$ is a fresh variable, i.e., which does not occur in the context $\Gamma,x:A$, we also postulate that
\begin{prooftree}
\AxiomC{$\Gamma,x:A\vdash B(x)~\textrm{type}$}
\RightLabel{$\Pi$-$x'/x$.}
\UnaryInfC{$\Gamma\vdash \prd{x:A}B(x)\jdeq \prd{x':A}B(x')~\mathrm{type}$}
\end{prooftree}

\subsubsection{The $\Pi$-introduction rule}
The introduction rule for dependent function types is also called the $\lambda$-abstraction rule. Recall that dependent functions are formed from terms $b(x)$ of type $B(x)$ in context $\Gamma,x:A$. Therefore \define{$\lambda$-abstraction rule}\index{lambda-abstraction@{$\lambda$-abstraction}}\index{rule!lambda-abstraction@{$\lambda$-abstraction}} is as follows:
\begin{prooftree}
  \AxiomC{$\Gamma,x:A \vdash b(x) : B(x)$}
  \RightLabel{$\lambda$}
  \UnaryInfC{$\Gamma\vdash \lam{x}b(x) : \prd{x:A}B(x)$}
\end{prooftree}

Just like ordinary mathematicians, we will sometimes write $x\mapsto f(x)$ for a function $f$. The map $n\mapsto n^2$ is an example. The $\lambda$-abstraction is also required to respect judgmental equality. Therefore we postulate that
\begin{prooftree}
  \AxiomC{$\Gamma,x:A \vdash b(x)\jdeq b'(x) : B(x)$}
  \RightLabel{$\lambda$-eq}
  \UnaryInfC{$\Gamma\vdash \lam{x}b(x)\jdeq \lam{x}b'(x) : \prd{x:A}B(x)$.}
\end{prooftree}

\subsubsection{The $\Pi$-elimination rule}

The elimination rule for dependent function types provides us with a way to \emph{use} dependent functions. The way to use a dependent function is to apply it to an argument of the domain type. The $\Pi$-elimination rule is therefore also called the \define{evaluation rule}\index{evaluation}\index{rule!evaluation}. It asserts that given a dependent function $f:\prd{x:A}B(x)$ in context $\Gamma$ we obtain a term $f(x)$ of type $B(x)$ in context $\Gamma,x:A$. More formally:
\begin{prooftree}
\AxiomC{$\Gamma\vdash f:\prd{x:A}B(x)$}
\RightLabel{$ev$}
\UnaryInfC{$\Gamma,x:A\vdash f(x) : B(x)$}
\end{prooftree}
Again we require that evaluation respects judgmental equality:
\begin{prooftree}
  \AxiomC{$\Gamma\vdash f\jdeq f':\prd{x:A}B(x)$}
  \UnaryInfC{$\Gamma,x:A\vdash f(x)\jdeq f'(x):B(x)$}
\end{prooftree}

\subsubsection{The $\Pi$-computation rules}
The computation rules for dependent function types postulate that $\lambda$-abstraction rule and the evaluation rule are mutual inverses. Thus we have two computation rules.

First we postulate the \define{$\beta$-rule}\index{beta-rule@{$\beta$-rule}}\index{rule!beta-rule@{$\beta$-rule}}
\begin{prooftree}
\AxiomC{$\Gamma,x:A \vdash b(x) : B(x)$}
\RightLabel{$\beta$}
\UnaryInfC{$\Gamma,x:A \vdash (\lambda y.b(y))(x)\jdeq b(x) : B(x)$.}
\end{prooftree}
Second, we postulate the \define{$\eta$-rule}\index{eta-rule@{$\eta$-rule}}\index{rule!eta-rule@{$\eta$-rule}}
\begin{prooftree}
\AxiomC{$\Gamma\vdash f:\prd{x:A}B(x)$}
\RightLabel{$\eta$}
\UnaryInfC{$\Gamma \vdash \lam{x}f(x) \jdeq f : \prd{x:A}B(x)$}
\end{prooftree}
This completes the specification of dependent function types.

\subsection{Ordinary function types}
In the case where both $A$ and $B$ are types in context $\Gamma$, we may first weaken $B$ by $A$, and then apply the formation rule for the dependent function type:
\begin{prooftree}
\AxiomC{$\Gamma\vdash A~\textrm{type}$}
\AxiomC{$\Gamma\vdash B~\textrm{type}$}
\BinaryInfC{$\Gamma,x:A\vdash B~\textrm{type}$}
\UnaryInfC{$\Gamma\vdash \prd{x:A}B~\textrm{type}$}
\end{prooftree}
The result is the type of functions that take an argument of type $A$, and return a term of type $B$. In other words, terms of the type $\prd{x:A}B$ are \emph{ordinary} functions from $A$ to $B$. We write $A\to B$ for the \define{type of functions}\index{function type} from $A$ to $B$. Sometimes we will also write $B^A$ for the type $A\to B$.

We give a brief summary of the rules specifying ordinary function types, omitting the rules that the asserted operations respect judgmental equality. All of these rules can be derived from the corresponding rules for $\Pi$-types.
\begin{prooftree}
\AxiomC{$\Gamma\vdash A~\textrm{type}$}
\AxiomC{$\Gamma\vdash B~\textrm{type}$}
\RightLabel{$\to$\index{arrow-formation@{$\to$-formation}}\index{rule!arrow-formation@{$\to$-formation}}}
\BinaryInfC{$\Gamma\vdash A\to B~\textrm{type}$}
\end{prooftree}%
\begin{prooftree}
\AxiomC{$\Gamma\vdash B~\textrm{type}$}
\AxiomC{$\Gamma,x:A\vdash b(x):B$}
\RightLabel{$\lambda$\index{lambda-abstraction@{$\lambda$-abstraction}}\index{rule!lambda-abstraction@{$\lambda$-abstraction}}}
\BinaryInfC{$\Gamma\vdash \lam{x}b(x):A\to B$}
\end{prooftree}%
\begin{prooftree}
\AxiomC{$\Gamma\vdash f:A\to B$}
\RightLabel{$ev$\index{rule!evaluation}\index{evaluation}}
\UnaryInfC{$\Gamma,x:A\vdash f(x):B$}
\end{prooftree}%
\begin{prooftree}
\AxiomC{$\Gamma\vdash B~\textrm{type}$}
\AxiomC{$\Gamma,x:A\vdash b(x):B$}
\RightLabel{$\beta$\index{rule!beta-rule@{$\beta$-rule}}\index{beta-rule@{$\beta$-rule}}}
\BinaryInfC{$\Gamma,x:A\vdash(\lam{y}b(y))(x)\jdeq b(x):B$}
\end{prooftree}%
\begin{prooftree}
\AxiomC{$\Gamma\vdash f:A\to B$}
\RightLabel{$\eta$\index{rule!eta-rule@{$\eta$-rule}}\index{eta-rule@{$\eta$-rule}}}
\UnaryInfC{$\Gamma\vdash\lam{x} f(x)\jdeq f:A\to B$}
\end{prooftree}

\begin{comment}
\begin{rmk}
Similar to \cref{rmk:ev_var}, we can derive
\begin{prooftree}
\AxiomC{$\Gamma\vdash A~\mathrm{type}$}
\AxiomC{$\Gamma\vdash B~\mathrm{type}$}
\BinaryInfC{$\Gamma,f:B^A,x:A\vdash f(x):B$}
\end{prooftree}
\end{rmk}
\end{comment}

\subsection{The identity function, composition, and their laws}
\begin{defn}
For any type $A$ in context $\Gamma$, we define the \define{identity function}\index{identity function|textbf} $\idfunc[A]:A\to A$ using the variable rule\index{variable rule}\index{rule!variable rule}:
\begin{prooftree}
\AxiomC{$\Gamma\vdash A~\textrm{type}$}
\UnaryInfC{$\Gamma,x:A\vdash x:A$}
\UnaryInfC{$\Gamma\vdash \idfunc[A]\defeq \lam{x}x:A\to A$}
\end{prooftree}
\end{defn}

A judgment of the form $\Gamma\vdash a\defeq b:A$ should be read as "$b$ is a well-defined term of type $A$ in context $\Gamma$, and we will refer to it as $a$".

\begin{defn}
For any three types $A$, $B$, and $C$ in context $\Gamma$, there is a \define{composition}\index{composition!of functions|textbf} operation
\begin{equation*}
\mathsf{comp}:(B\to C)\to ((A\to B)\to (A\to C)),
\end{equation*}
i.e., we can derive
\begin{prooftree}
\AxiomC{$\Gamma\vdash A~\textrm{type}$}
\AxiomC{$\Gamma\vdash B~\textrm{type}$}
\AxiomC{$\Gamma\vdash C~\textrm{type}$}
\TrinaryInfC{$\Gamma\vdash\mathsf{comp}:(B\to C)\to ((A\to B)\to (A\to C))$}
\end{prooftree}
We will write $g\circ f$ for $\mathsf{comp}(g,f)$.
\end{defn}

\begin{constr}
  The idea of the definition is to define $\mathsf{comp}(g,f)$ to be the function $\lam{x}g(f(x))$. The derivation we use to construct $\mathsf{comp}$ is as follows:
  \begin{small}
    \begin{prooftree}
      \AxiomC{$\Gamma\vdash A~\mathrm{type}$}
      \AxiomC{$\Gamma\vdash B~\mathrm{type}$}
      \BinaryInfC{$\Gamma,f:B^A,x:A\vdash f(x):B$}
      \UnaryInfC{$\Gamma,g:C^B,f:B^A,x:A\vdash f(x):B$}
      \AxiomC{$\Gamma\vdash B~\mathrm{type}$}
      \AxiomC{$\Gamma\vdash C~\mathrm{type}$}
      \BinaryInfC{$\Gamma,g:C^B,y:B\vdash g(y):C$}
      \UnaryInfC{$\Gamma,g:C^B,f:B^A,y:B\vdash g(y):C$}
      \UnaryInfC{$\Gamma,g:C^B,f:B^A,x:A,y:B\vdash g(y):C$}
      \BinaryInfC{$\Gamma,g:C^B,f:B^A,x:A\vdash g(f(x)) : C$}
      \UnaryInfC{$\Gamma,g:C^B,f:B^A\vdash \lam{x}g(f(x)):C^A$}
      \UnaryInfC{$\Gamma,g:B\to C\vdash \lam{f}{x}g(f(x)):B^A\to C^A$}
      \UnaryInfC{$\Gamma\vdash\mathsf{comp}\defeq \lam{g}{f}{x}g(f(x)):C^B\to (B^A\to C^A)$}
    \end{prooftree}
  \end{small}
\end{constr}

\begin{lem}
Composition of functions is associative\index{associativity!of function composition}\index{composition!of functions!associativity}, i.e., we can derive
\begin{prooftree}
\AxiomC{$\Gamma\vdash f:A\to B$}
\AxiomC{$\Gamma\vdash g:B\to C$}
\AxiomC{$\Gamma\vdash h:C\to D$}
\TrinaryInfC{$\Gamma \vdash (h\circ g)\circ f\jdeq h\circ(g\circ f):A\to D$}
\end{prooftree}
\end{lem}

\begin{proof}
  The main idea of the proof is that both $((h\circ g)\circ f)(x)$ and $(h\circ (g\circ f))(x)$ evaluate to $h(g(f(x))$, and therefore $(h\circ g)\circ f$ and $h\circ(g\circ f)$ must be judgmentally equal.

  This idea is made formal in the following derivation:
  \begin{prooftree}
    \AxiomC{$\Gamma\vdash f:A\to B$}
    \UnaryInfC{$\Gamma,x:A\vdash f(x):B$}
    \AxiomC{$\Gamma\vdash g:B\to C$}
    \UnaryInfC{$\Gamma,y:B\vdash g(y):C$}
    \BinaryInfC{$\Gamma,x:A\vdash g(f(x)):C$}
    \AxiomC{$\Gamma\vdash h:C\to D$}
    \UnaryInfC{$\Gamma,z:C\vdash h(z):D$}
    \BinaryInfC{$\Gamma,x:A\vdash h(g(f(x))):D$}
    \UnaryInfC{$\Gamma,x:A\vdash h(g(f(x)))\jdeq h(g(f(x))):D$}
    \UnaryInfC{$\Gamma,x:A\vdash (h\circ g)(f(x))\jdeq h((g\circ f)(x)):D$}
    \UnaryInfC{$\Gamma,x:A\vdash ((h\circ g)\circ f)(x)\jdeq (h\circ (g \circ f))(x):D$}
    \UnaryInfC{$\Gamma\vdash (h\circ g)\circ f\jdeq h\circ(g\circ f):A\to D$.}
  \end{prooftree}
\end{proof}

\begin{lem}\label{lem:fun_unit}
Composition of functions satisfies the left and right unit laws\index{left unit law|see {unit laws}}\index{right unit law|see {unit laws}}\index{unit laws!of function composition}, i.e., we can derive
\begin{prooftree}
\AxiomC{$\Gamma\vdash f:A\to B$}
\UnaryInfC{$\Gamma\vdash \idfunc[B]\circ f\jdeq f:A\to B$}
\end{prooftree}
and
\begin{prooftree}
\AxiomC{$\Gamma\vdash f:A\to B$}
\UnaryInfC{$\Gamma\vdash f\circ\idfunc[A]\jdeq f:A\to B$}
\end{prooftree}
\end{lem}

\begin{proof}
The derivation for the left unit law is
\begin{prooftree}
\AxiomC{$\Gamma\vdash f:A\to B$}
\UnaryInfC{$\Gamma,x:A\vdash f(x):B$}
\AxiomC{$\Gamma\vdash B~\mathrm{type}$}
\UnaryInfC{$\Gamma,y:B\vdash \idfunc[B](y)\jdeq y:B$}
\UnaryInfC{$\Gamma,x:A,y:B\vdash \idfunc[B](y)\jdeq y:B$}
\BinaryInfC{$\Gamma,x:A\vdash \idfunc[B](f(x))\jdeq f(x):B$}
\UnaryInfC{$\Gamma,x:A\vdash (\idfunc[B]\circ f)(x)\jdeq f(x):B$}
\UnaryInfC{$\Gamma\vdash \idfunc[B]\circ f\jdeq f:A\to B$}
\end{prooftree}
The right unit law is left as \cref{ex:fun_right_unit}.
\end{proof}

\subsection{The natural numbers}
The archetypal example of an inductive type is the type $\N$ of \emph{natural numbers}.
The type of \define{natural numbers}\index{natural numbers|see N@{$\N$}} is defined to be a closed type $\nat$\index{N@{$\N$}} equipped with closed terms for a \define{zero term}\index{zero term} $\zeroN:\N$ and a \define{successor function}\index{successor function!of N@{of $\N$}}\index{function!successor on N@{successor on $\N$}} $\succN:\N\to\N$. The rules we postulate for the type of natural numbers again come in four sets:
\begin{enumerate}
\item The formation rule, which asserts that the type $\N$ can be formed.
\item The introduction rules, which provide the zero element and the successor function.
\item The elimination rule. This rule is the type theoretic analogue of the induction principle for $\N$.
\item The computation rules, which assert that any application of the elimination rule behaves as expected on the constructors $\zeroN$ and $\succN$ of $\N$.
\end{enumerate}
\begin{rmk}
  We annotate the terms $\zeroN$ and $\succN$ of type $\N$ with their type in the subscript, as a reminder that $\zeroN$ and $\succN$ are declared to be terms of type $\N$, and not of any other type. In the next chapter we will introduce the type $\Z$ of the integers, on which we can also define a zero term $\zeroZ$, and a successor function $\succZ$. These should be distinguished from the terms $\zeroN$ and $\succN$. In general, we will make sure that every term is given a unique name. In libraries of mathematics formalized in a computer proof assistant it is also the case that every type must be given a unique name.
\end{rmk}

\subsubsection{The formation rule of $\N$}
The type $\N$ is formed by the $\N$-formation rule
\begin{prooftree}
  \AxiomC{}
  \RightLabel{$\N$-form}
  \UnaryInfC{$\vdash \N~\mathrm{type}$.}
\end{prooftree}

\subsubsection{The introduction rules of $\N$}
The introduction rules for $\N$ introduce the zero term and the successor function

\bigskip
\begin{minipage}{.45\textwidth}
  \begin{prooftree}
    \AxiomC{}
    \UnaryInfC{$\vdash \zeroN:\N$}
  \end{prooftree}
\end{minipage}
\begin{minipage}{.45\textwidth}
  \begin{prooftree}
    \AxiomC{}
    \UnaryInfC{$\vdash \succN:\N\to\N$}
  \end{prooftree}
\end{minipage}

\subsubsection{The elimination rule of $\N$}
To prove properties about the natural numbers, we postulate an \emph{induction principle}\index{induction principle!of N@{of $\N$}} for $\N$. In dependent type theory, however, the induction principle for the natural numbers provides a way to construct \emph{dependent functions} of types depending on the natural numbers.

The induction principle for $\N$ states what one has to do in order to construct a dependent function of type $\prd{n:\N}P(n)$, for any given type family $P$ over $\N$. Just like for the usual induction principle of the natural numbers, there are two things to be constructed: first one has to construct $p_0:P(\zeroN)$, and the second task is to construct a function of type $P(n)\to P(\succN(n))$ for all $n:\N$. 

Therefore the induction principle for $\N$ is as follows:
\begin{prooftree}
  \def\fCenter{\Gamma}
  \Axiom$\fCenter, n:\N\vdash P(n)~\mathrm{type}$
  \noLine
  \UnaryInf$\fCenter\ \vdash p_0:P(\zeroN)$
  \noLine
  \UnaryInf$\fCenter\ \vdash p_S:\prd{n:\N}P(n)\to P(\succN(n))$
  \RightLabel{$\N{-}\mathrm{Ind}$}
  \UnaryInf$\fCenter\ \vdash \ind{\N}(p_0,p_S):\prd{n:\N} P(n)$
\end{prooftree}

\subsubsection{The computation rules of $\N$}
Furthermore we require that the dependent function $\ind{\N}(P,p_0,p_S)$ behaves as expected when it is applied to $\zeroN$ or a successor, i.e., with the same hypotheses as for the induction principle we postulate the \define{computation rules}\index{computation rules!of N@{of $\N$}} for $\N$
\begin{prooftree}
\AxiomC{$\cdots$}
%\RightLabel{$\N{-}\mathrm{Comp}(\zeroN)$}
\UnaryInfC{$\Gamma \vdash \ind{\N}(p_0,p_S,\zeroN)\jdeq p_0 : P(\zeroN)$}
\end{prooftree}
\begin{prooftree}
\AxiomC{$\cdots$}
%\RightLabel{$\N{-}\mathrm{Comp}(\succN)$}
\UnaryInfC{$\Gamma, n:\N \vdash  \ind{\N}(p_0,p_S,\succN(n))\jdeq p_S(n,\ind{\N}(p_0,p_S,n)) : P(\succN(n))$}
\end{prooftree}
This completes the formal specification of $\N$.

\subsubsection{The definition of addition on $\N$}
Using the induction principle of $\N$ we can perform many familiar constructions. 
For instance, we can define the \define{addition operation}\index{addition!on N@{on $\N$}}\index{function!addition on N@{addition on $\N$}} by induction on $\N$.

\begin{defn}
  We define a function
  \begin{equation*}
    \addN:\N\to (\N\to\N)
  \end{equation*}
  satisfying $\addN(\zeroN,n)\jdeq n$ and $\addN(\succN(m),n)\jdeq\succN(\addN(m,n))$. Usually we will write $n+m$ for $\addN(n,m)$.
\end{defn}

\begin{proof}[Informal construction]
Informally, the definition of addition is as follows. By induction it suffices to construct a function $\mathsf{add\usc{}}\zeroN : \N\to\N$, and a function
\begin{align*}
\mathsf{add\usc{}}\succN(n,f):\N\to\N,
\end{align*}
for every $n:\N$ and every $f:\N\to\N$.

The function $\mathsf{add\usc{}}\zeroN:\N\to\N$ is of course taken to be $\idfunc[\N]$, since the result of adding $0$ to $n$ should be $n$.

Given $n:\N$ and a function $f:\N\to\N$ we define $\mathsf{add\usc{}}\succN(n,f)\defeq \succN\circ f$. The idea is that if $f$ represents adding $m$, then $\mathsf{add\usc{}}\succN(n,f)$ should be adding one more than $f$ did.
\end{proof}

\begin{proof}[Formal derivation]
The derivation for the construction of $\mathsf{add\usc{}}\succN$ looks as follows:
\begin{prooftree}
  \AxiomC{}
  \UnaryInfC{$\succN:\N^\N$}
  \AxiomC{}
  \UnaryInfC{$\vdash\N~\mathrm{type}$}
  \AxiomC{}
  \UnaryInfC{$\vdash\N~\mathrm{type}$}
  \AxiomC{}
  \UnaryInfC{$\vdash\N~\mathrm{type}$}
  \TrinaryInfC{$\vdash \mathsf{comp}:\N^\N\to (\N^\N\to \N^\N)$}
  \UnaryInfC{$g:\N\to\N\vdash \mathsf{comp}(g):\N^\N\to\N^\N$}
  \BinaryInfC{$\vdash \mathsf{comp}(\succN):\N^\N\to\N^\N$}
  \UnaryInfC{$n:\N\vdash \mathsf{comp}(\succN):\N^\N\to\N^\N$}
  \UnaryInfC{$\vdash \mathsf{add\usc{}}\succN\defeq \lam{n}\mathsf{comp}(\succN):\N\to (\N^\N \to \N^\N)$}
%\BinaryInfC{$\vdash\addN:\ind{\N}(add_0,add_S):\N\to \N^\N$}
\end{prooftree}
We combine this derivation with the induction principle of $\N$ to complete the construction of addition:
\begin{prooftree}
  \AxiomC{$\vdots$}
  \UnaryInfC{$n:\N\vdash \N^\N~\mathrm{type}$}
  \AxiomC{$\vdots$}
  \UnaryInfC{$\vdash \mathsf{add\usc{}}\zeroN\defeq \idfunc[\N]:\N^\N$}
  \AxiomC{$\vdots$}
  \UnaryInfC{$\vdash \mathsf{add\usc{}}\succN:\N\to (\N^\N \to \N^\N)$}
  \TrinaryInfC{$\vdash\addN\jdeq\ind{\N}(\mathsf{add\usc{}}\zeroN,\mathsf{add\usc{}}\succN):\N\to \N^\N$}
\end{prooftree}
The asserted judgmental equalities then hold by the computation rules for $\N$.
\end{proof}

\begin{rmk}
The rules that we provided so far are not sufficient to also conclude that $n+\zeroN\jdeq n$ and $n+ \succN(m)\jdeq \succN(n+m)$. However, once we have introduced the \emph{identity type} in \cref{chap:identity} we will nevertheless be able to \emph{identify} $n+\zeroN$ with $n$, and $n+ \succN(m)$ with $\succN(n+m)$. See \cref{ex:semi-ring-laws-N}. 
\end{rmk}

\begin{exercises}
\item \label{ex:fun_right_unit}Give a derivation for the right unit law of \cref{lem:fun_unit}.\index{unit laws!for function composition}
\item Show that the rule
\begin{prooftree}
\AxiomC{$\Gamma,x:A \vdash b(x) : B(x)$}
\RightLabel{$\lambda$-$x'/x$}
\UnaryInfC{$\Gamma\vdash \lam{x}b(x)\jdeq \lam{x'}b(x') : \prd{x:A}B(x)$}
\end{prooftree}
is admissible for any variable $x'$ that does not occur in the context $\Gamma,x:A$.
\item 
  \begin{subexenum}
  \item Construct the \define{constant function}\index{constant function}\index{function!constant function}
    \begin{prooftree}
      \AxiomC{$\Gamma\vdash A~\textrm{type}$}
      \UnaryInfC{$\Gamma,y:B\vdash \mathsf{const}_y:A\to B$}
    \end{prooftree}
  \item Show that
    \begin{prooftree}
      \AxiomC{$\Gamma\vdash f:A\to B$}
      \UnaryInfC{$\Gamma,z:C\vdash \mathsf{const}_z\circ f\jdeq\mathsf{const}_z : A\to C$}
    \end{prooftree}
  \item Show that
    \begin{prooftree}
      \AxiomC{$\Gamma\vdash A~\textrm{type}$}
      \AxiomC{$\Gamma\vdash g:B\to C$}
      \BinaryInfC{$\Gamma,y:B\vdash g\circ\mathsf{const}_y\jdeq \mathsf{const}_{g(y)}:A\to C$}
    \end{prooftree}
  \end{subexenum}
\item In this exercise we construct some standard functions on the natural numbers.
  \begin{subexenum}
  \item Define the binary \define{min} and \define{max} functions $\min_\N,\max_\N:\N\to(\N\to\N)$.\index{minimum function}\index{maximum function}\index{function!min}\index{function!max}
  \item Define the \define{multiplication}\index{multiplication!on N@{on $\N$}}\index{function!multiplication on N@{multiplication on $\N$}} operation $\mathsf{mul}_\N :\N\to(\N\to\N)$.
  \item Define the \define{power}\index{power function on N@{power function on $\N$}}\index{function!power function on N@{power function on $\N$}} operation $n,m\mapsto m^n$ of type $\N\to (\N\to \N)$.
  \item Define the \define{factorial}\index{factorial function}\index{function!factorial function} function $n\mapsto n!$.
  \item Define the \define{binomial coefficient}\index{binomial coefficients} $\binom{n}{k}$ for any $n,k:\N$, making sure that $\binom{n}{k}\jdeq 0$ when $n<k$.
  \item Define the \define{Fibonacci sequence}\index{Fibonacci sequence} $0,1,1,2,3,5,8,13,\ldots$ as a function $F:\N\to\N$.
  \end{subexenum}
\item In this exercise we generalize the composition operation of non-dependent function types\index{composition!of dependent functions}:
\begin{subexenum}
\item Define a composition operation for dependent function types
\begin{prooftree}
\AxiomC{$\Gamma\vdash f:\prd{x:A}B(x)$}
\AxiomC{$\Gamma\vdash g:\prd{x:A}{y:B(x)} C(x,y)$}
\BinaryInfC{$\Gamma\vdash g\circ' f:\prd{x:A} C(x,f(x))$}
\end{prooftree}
and show that this operation agrees with ordinary composition when it is specialized to non-dependent function types.
\item Show that composition of dependent functions agrees with ordinary composition of functions:
  \begin{prooftree}
    \AxiomC{$\Gamma\vdash f:A\to B$}
    \AxiomC{$\Gamma\vdash g:B\to C$}
    \BinaryInfC{$\Gamma\vdash (\lam{x}g)\circ' f\jdeq g\circ f:A \to C$}
  \end{prooftree}
\item Show that composition of dependent functions is associative.\index{associativity!of dependent function composition}\index{composition!of dependent functions!associativity}
\item Show that composition of dependent functions satisfies the right unit law\index{unit laws!dependent function composition}:
\begin{prooftree}
\AxiomC{$\Gamma\vdash f:\prd{x:A}B(x)$}
\UnaryInfC{$\Gamma\vdash (\lam{x}f)\circ'\idfunc[A]\jdeq f :\prd{x:A}B(x)$}
\end{prooftree}
\item Show that composition of dependent functions satisfies the left unit law\index{unit laws!dependent function composition}:
\begin{prooftree}
\AxiomC{$\Gamma\vdash f:\prd{x:A}B(x)$}
\UnaryInfC{$\Gamma\vdash (\lam{x}\idfunc[B(x)])\circ' f\jdeq f:\prd{x:A}B(x)$}
\end{prooftree}
\end{subexenum}
\item \label{ex:swap}
\begin{subexenum}
\item Given two types $A$ and $B$ in context $\Gamma$, and a type $C$ in context $\Gamma,x:A,y:B$, define the \define{swap function}\index{function!swap}\index{swap function}
\begin{equation*}
\Gamma\vdash \sigma:\Big(\prd{x:A}{y:B}C(x,y)\Big)\to\Big(\prd{y:B}{x:A}C(x,y)\Big)
\end{equation*}
that swaps the order of the arguments.
\item Show that
\begin{equation*}
\Gamma\vdash \sigma\circ\sigma\jdeq\idfunc:\Big(\prd{x:A}{y:B}C(x,y)\Big)\to \Big(\prd{x:A}{y:B}C(x,y)\Big).
\end{equation*}
\end{subexenum}
\end{exercises}


\chapter{Inductive types}

\section{The idea of inductive types}

Many other types can also be specified as inductive types, similar to the natural numbers. The unit type, the empty type, and the booleans are the simplest examples of this way of defining types. Just like the type of natural numbers, other inductive types are also specified by their \emph{constructors}, an \emph{induction principle}, and their \emph{computation rules}: 
\begin{enumerate}
\item The constructors tell what structure the inductive type comes equipped with. There may be multiple constructors, or no constructors at all in the specification of an inductive type. 
\item The induction principle specifies the data that should be provided in order to construct a section of an arbitrary dependent type over the inductive type. 
\item The computation rules assert that the inductively defined section agrees on the constructors with the data that was used to define the section. Thus, there is a computation rule for every constructor.
\end{enumerate}
For a more general treatment of inductive types, we refer to Chapter 5 of \cite{hottbook}.
\begin{defn}
We define the \define{unit type} to be a type $\unit$ equipped with
\begin{equation*}
\ttt:\unit,
\end{equation*}
satisfying the induction principle that for any type family $P:\unit\to\type$, there is a term
\begin{equation*}
\ind{\unit} : P(\ttt)\to\prd{x:\unit}P(x)
\end{equation*}
for which the computation rule
\begin{equation*}
\ind{\unit}(p_{\ttt},\ttt) \jdeq p_\ttt
\end{equation*}
holds.
\end{defn}

The empty type is a degenerate example of an inductive type. It does \emph{not} come equipped with any constructors, and therefore there are also no computation rules. The induction principle merely asserts that any type family has a section.
\begin{defn}
We define the \define{empty type} to be a type $\emptyt$ satisfying the induction principle that for any type family $P:\emptyt\to\type$, there is a term
\begin{equation*}
\ind{\emptyt} : \prd{x:\emptyt}P(x).
\end{equation*}
\end{defn}

\begin{defn}
We define the \define{booleans} to be a type $\bool$ that comes equipped with
\begin{align*}
\bfalse & : \bool \\
\btrue & : \bool
\end{align*}
satisfying the induction principle that for any type family $P:\bool\to\type$, there is a term
\begin{equation*}
\ind{\bool} : P(\bfalse)\to P(\btrue)\to \prd{x:\bool}P(x)
\end{equation*}
for which the computation rules
\begin{align*}
\ind{\bool}(p_0,p_1,\bfalse) & \jdeq p_0 \\
\ind{\bool}(p_0,p_1,\btrue) & \jdeq p_1
\end{align*}
hold.
\end{defn}

\begin{defn}
Let $A$ and $B$ be types. We define the \define{disjoint sum} $A+B$ to be a type that comes equipped with
\begin{align*}
\inl & : A \to A+B \\
\inr & : B \to A+B
\end{align*}
satisfying the induction principle that for any type family $P:(A+B)\to\type$, there is a term
\begin{equation*}
\ind{A+B} : \Big(\prd{x:A}P(\inl(x))\Big)\to\Big(\prd{y:B}P(\inr(y))\Big)\to\prd{z:A+B}P(z)
\end{equation*}
for which the computation rules
\begin{align*}
\ind{A+B}(f,g,\inl(x)) & \jdeq f(x) \\
\inr{A+B}(f,g,\inr(y)) & \jdeq g(y)
\end{align*}
hold.
\end{defn}

\section{The type of integers}
\begin{defn}
We define the \define{integers} to be the type $\Z\defeq\nat_{\geq 1}+\unit+\nat_{\geq 1}$.
\end{defn}

\begin{defn}
We construct a successor function $S:\Z\to\Z$.
\end{defn}

\begin{constr}

\end{constr}

\section{Dependent pair types}

\begin{defn}
Let $A$ be a type, and let $P:A\to\type$ be a type family over $A$.
The \define{dependent pair type} $\sm{x:A}P(x)$ is defined to be a type equipped with a \define{pairing function}
\begin{equation*}
\pairr{\blank,\blank}:\prd{x:A} P(x)\to \Big(\sm{y:A}P(y)\Big)
\end{equation*}
and \define{projection maps}
\begin{align*}
\proj 1 & : \Big(\sm{x:A}P(x)\Big)\to A \\
\proj 2 & : \prd{w:\sm{x:A}P(x)}P(\proj 1(w))
\end{align*}
satisfying the equations
\begin{align*}
\proj 1(\pairr{a,p}) & \jdeq a \tag{$\beta_1$}\\
\proj 2(\pairr{a,p}) & \jdeq p \tag{$\beta_2$}\\
\pairr{\proj 1(w),\proj 2(w)} & \jdeq w. \tag{$\eta$}
\end{align*}
\end{defn}

\begin{exercises}
\item For any type $A$, show that $(A+\neg A)\to(\neg\neg A\to A)$. 
\item \label{ex:int_group_ops}Define operations $k,l\mapsto k+l:\Z\to\Z\to\Z$ and $k\mapsto -k:\Z\to \Z$.
\item \label{ex:int_order}Define the relations $\leq$ and $<$ on and $\Z$.
\end{exercises}


\chapter{Identity types}
From the perspective of types as proof-relevant propositions, how should we think of \emph{equality} in type theory? Given a type $A$, and two terms $x,y:A$, the equality $\id{x}{y}$ should again be a type. Indeed, we want to \emph{use} type theory to prove equalities. \emph{Dependent} type theory provides us with a convenient setting for this: the equality type $\id{x}{y}$ is dependent on $x,y:A$. 

Then, if $\id{x}{y}$ is to be a type, how should we think of the terms of $\id{x}{y}$. A term $p:\id{x}{y}$ witnesses that $x$ and $y$ are equal terms of type $A$. In other words $p:\id{x}{y}$ is an \emph{identification} of $x$ and $y$. In a proof-relevant world, there might be many terms of type $\id{x}{y}$. I.e.~there might be many identifications of $x$ and $y$. And, since $\id{x}{y}$ is itself a type, we can form the type $\id{p}{q}$ for any two identifications $p,q:\id{x}{y}$. That is, since $\id{x}{y}$ is a type, we may also use the type theory to prove things \emph{about} identifications (for instance, that two given such identifications can themselves be identified), and we may use the type theory to perform constructions with them. As we will see shortly, we can give every type a groupoid-like structure.

Clearly, the equality type should not just be any type dependent on $x,y:A$. Then how do we form the equality type, and what ways are there to use identifications in constructions in type theory? The answer to both these questions is that we will form the identity type as an \emph{inductive} type, generated by just a reflexivity term providing an identification of $x$ to itself. The induction principle then provides us with a way of performing constructions with identifications, such as concatenating them, inverting them, and so on. Thus, the identity type is equipped with a reflexivity term, and further possesses the structure that are generated by its induction principle and by the type theory. This inductive construction of the identity type is elegant, beautifully simple, but far from trivial!

The situation where two terms can be identified in possibly more than one way is analogous to the situation in \emph{homotopy theory}, where two points of a space can be connected by possibly more than one \emph{path}. Indeed, for any two points $x,y$ in a space, there is a \emph{space of paths} from $x$ to $y$. Moreover, between any two paths from $x$ to $y$ there is a space of \emph{homotopies} between them, and so on. This analogy has been made precise by the construction of \emph{homotopical models} of type theory, and it has led to the fruitful research area of \emph{synthetic homotopy theory}, the subfield of \emph{homotopy type theory} that is the topic of this course.

\section{The inductive definition of identity types}

Let $A$ be a type in context $\Gamma$. The \define{identity type} of $A$ at $a:A$ is the inductive type family 
\begin{equation*}
\Gamma,x:A,y:A\vdash x =_A y~\mathrm{type}
\end{equation*}
with constructor
\begin{equation*}
\Gamma,x:A\vdash \refl{x} : x=_A x.
\end{equation*}
The induction principle that for any type family
\begin{equation*}
\Gamma,x:A,y:A,\alpha: x=_A y\vdash P(x,y,\alpha)~\mathrm{type}
\end{equation*}
%we can derive
%\begin{prooftree}
%\AxiomC{$\Gamma,\Delta[a/x,\refl{a}/p]\vdash \alpha : P(a,\refl{a})$}
%\UnaryInfC{$\Gamma,x:A,p:\idtypevar{A}(a,x),\Delta\vdash \ind{a=}(\alpha,x,p):P(x,p)$}
%\end{prooftree}
there is a term
\begin{equation*}
\ind{x=} : P(x,x,\refl{x})\to \prd{y:A}{\alpha:x=_A y}P(x,y,\alpha)
\end{equation*}
in context $\Gamma,x:A$, satisfying the computation rule
\begin{equation*}
\ind{x=}(p,x,\refl{x})\jdeq p.
\end{equation*}
A term of type $x=_A y$ is also called an \define{identification} of $x$ with $y$, and sometimes it is called a \define{path} from $x$ to $y$.
The induction principle for identity types is sometimes called \define{identification elimination} or \define{path induction}. We also write $\idtypevar{A}$ for the identity type on $A$. 

We also assume that the universe $\UU$ is closed under identity types, i.e. that there is a map
\begin{equation*}
\check{\mathsf{Id}}:\prd{A:\UU}\mathrm{El}(A)\to\mathrm{El}(A)\to\UU
\end{equation*}
satisfying
\begin{equation*}
\mathrm{El}(\check{\mathsf{Id}}(A,x,y))\jdeq x=_{\mathrm{El}(A)} y.
\end{equation*}

In the following lemma we show that the identity type on $A$ is contained in any reflexive relation on $A$.

\begin{lem}
Let $\Gamma,x:A,y:A\vdash R(x,y)~\mathrm{type}$, and suppose that $R$ is reflexive in the sense that there is a term
\begin{equation*}
\rho:\prd{x:A}R(x,x)
\end{equation*}
Then there is a term of type
\begin{equation*}
\prd{y:A} (x=_A y)\to R(x,y)
\end{equation*}
in context $\Gamma,x:A$.
\end{lem}

\begin{constr}
By weakening the reflexive relation $R$ we obtain
\begin{equation*}
\Gamma,x:A,y:A,\alpha:x=_A y\vdash R(x,y)~\mathrm{type},
\end{equation*}
on which the induction principle is applicable.
Thus we see that by the induction principle for identity types we have a term
\begin{equation*}
\ind{x=} : R(x,x)\to \prd{y:A}(x=_A y)\to R(x,y)
\end{equation*}
so it suffices to construct a term of type $R(x,x)$, which we have by reflexivity of $R$.
\end{constr}

\section{The groupoid structure of types}\label{sec:groupoid}
We show that identifications can be \emph{concatenated} and \emph{inverted}, which corresponds to the transitivity and symmetry of the identity type. 

Furthermore, we observe that we can iteratively take identity types, i.e.~we can take identity types of identity types, 
\begin{equation*}
p =_{(x=_Ay)} q,
\end{equation*}
and so on. In other words, for any two identifications $p,q:x=_A y$, there is a type of identifications of $p$ with $y$. One way to think about this is that the identifications $p,q:x=_A y$ are paths in the type (space) $A$, and an identification of $p$ with $q$ is a \emph{higher path} from $p$ to $q$, i.e.~a \emph{homotopy}.

Using the observation that identity types can be iterated we show that concatenation is \emph{associative}, satisfies the left and right \emph{unit laws}, and satisfies the left and right \emph{inverse laws}. These are the \define{groupoid operations} on the identity type.

\begin{defn}\label{defn:id_concat}
Let $A$ be a type. We define the \define{concatenation} operation
\begin{equation*}
\mathsf{concat} : \prd{x,y,z:A} (\id{x}{y})\to(\id{y}{z})\to (\id{x}{z}).
\end{equation*}
We will write $\ct{p}{q}$ for $\mathsf{concat}(p,q)$. Also, we will \emph{associate to the right}, i.e.~by $\ct{p}{q}{r}$ we mean $\ct{p}{(\ct{q}{r})}$.
\end{defn}

\begin{constr}
We construct the concatenation operation by path induction. It suffices to construct
\begin{equation*}
\mathsf{concat}(\refl{x}):\prd{z:A} (x=z)\to(x=z).
\end{equation*}
Here we take $\mathsf{concat}(\refl{x})_z \jdeq \idfunc[(x=z)]$. 
Explicitly, the term we have constructed is
\begin{equation*}
\lam{x}\rec{x=}(\lam{z}\idfunc[(\id{x}{z})]):\prd{x,y:A} (x=y)\to \prd{z:A} (y=z)\to (x=z).
\end{equation*}
To obtain a term of the asserted type we need to swap the order of the arguments $p:x=y$ and $z:A$, using \autoref{ex:swap}.
\end{constr}

\begin{defn}\label{defn:id_inv}
Let $A$ be a type. We define the \define{inverse operation} 
\begin{equation*}
\mathsf{inv}:\prd{x,y:A} (x=y)\to (y=x).
\end{equation*}
Most of the time we will write $p^{-1}$ for $\mathsf{inv}(p)$.
\end{defn}

\begin{constr}
We construct the inverse operation by path induction. It suffices to construct
\begin{equation*}
\mathsf{inv}(\refl{x}): x=x,
\end{equation*}
for any $x:A$. Here we take $\mathsf{inv}(\refl{x})\defeq \refl{x}$.
\end{constr}

\begin{defn}\label{defn:id_assoc}
Let $A$ be a type. We define the \define{associativity operation}, which assigns to each $p:x=y$, $q:y=z$, and $r:z=w$ the \define{associator}
\begin{equation*}
\mathsf{assoc}(p,q,r) : \ct{(\ct{p}{q})}{r}=\ct{p}{(\ct{q}{r})}.
\end{equation*}
\end{defn}

\begin{constr}
By identification elimination it suffices to show that
\begin{equation*}
\prd{z:A}{q:x=z}{z':A}{r:z=w} \ct{(\ct{\refl{x}}{q})}{r}= \ct{\refl{x}}{(\ct{q}{r})}.
\end{equation*}
Let $q:x=z$ and $r:z=w$. Note that by the computation rule $\ct{\refl{x}}{q}\jdeq q$, so $\ct{(\ct{\refl{x}}{q})}{r}\jdeq \ct{q}{r}$. Similarly we have $\ct{\refl{x}}{(\ct{q}{r})}\jdeq \ct{q}{r}$. Therefore we can simply take $\refl{\ct{q}{r}}$.
\end{constr}

\begin{defn}\label{defn:id_unit}
Let $A$ be a type. We define the left and right \define{unit operations}, which assigns to each $p:x=y$ the terms
\begin{align*}
\mathsf{left\usc{}unit}(p) & : \ct{\refl{x}}{p}=p \\
\mathsf{right\usc{}unit}(p) & : \ct{p}{\refl{y}}=p,
\end{align*}
respectively.
\end{defn}

\begin{constr}
By identification elimination it suffices to construct
\begin{align*}
\mathsf{left\usc{}unit}(\refl{x}) & : \ct{\refl{x}}{\refl{x}} = \refl{x} \\
\mathsf{right\usc{}unit}(\refl{x}) & : \ct{\refl{x}}{\refl{x}} = \refl{x}.
\end{align*}
In both cases we take $\refl{\refl{x}}$.
\end{constr}

\begin{defn}\label{defn:id_invlaw}
Let $A$ be a type. We define left and right \define{inverse operations}
\begin{align*}
\mathsf{left\usc{}inv}(p) & : \ct{p^{-1}}{p} = \refl{y} \\
\mathsf{right\usc{}inv}(p) & : \ct{p}{p^{-1}} = \refl{x}.
\end{align*}
\end{defn}

\begin{constr}
By identification elimination it suffices to construct
\begin{align*}
\mathsf{left\usc{}inv}(\refl{x}) & : \ct{\refl{x}^{-1}}{\refl{x}} = \refl{x} \\
\mathsf{right\usc{}inv}(\refl{x}) & : \ct{\refl{x}}{\refl{x}^{-1}} = \refl{x}.
\end{align*}
Using the computation rules we see that
\begin{equation*}
\ct{\refl{x}^{-1}}{\refl{x}}\jdeq \ct{\refl{x}}{\refl{x}}\jdeq\refl{x},
\end{equation*}
so we define $\mathsf{left\usc{}inv}(\refl{x})\defeq \refl{\refl{x}}$. Similarly it follows from the computation rules that
\begin{equation*}
\ct{\refl{x}}{\refl{x}^{-1}} \jdeq \refl{x}^{-1}\jdeq \refl{x}
\end{equation*}
so we again define $\mathsf{right\usc{}inv}(\refl{x})\defeq\refl{\refl{x}}$. 
\end{constr}

\section{The action on paths of functions}

Using the induction principle of the identity type we can show that every function preserves identifications.
In other words, every function sends identified terms to identified terms.
Note that this is a form of continuity for functions in type theory: if there is a path that identifies two points $x$ and $y$ of a type $A$, then there also is a path that identifies the values $f(x)$ and $f(y)$ in the codomain of $f$. 

\begin{defn}\label{defn:ap}
Let $f:A\to B$ be a map. We define the \define{action on paths} of $f$ as an operation
\begin{equation*}
\apfunc{f} : \prd*{x,y:A} (\id{x}{y})\to(\id{f(x)}{f(y)}).
\end{equation*}
Moreover, there are operations
\begin{align*}
\mathsf{ap.idfun}_A & : \prd*{x,y:A}{p:\id{x}{y}} \id{p}{\ap{\idfunc[A]}{p}} \\
\mathsf{ap.comp}(f,g) & : \prd*{x,y:A}{p:\id{x}{y}} \id{\ap{g}{\ap{f}{p}}}{\ap{g\circ f}{p}}.
\end{align*}
\end{defn}

\begin{constr}
First we define $\apfunc{f}$ by identity elimination, taking
\begin{equation*}
\apfunc{f}(\refl{x})\defeq \refl{f(x)}.
\end{equation*}
Next, we construct $\mathsf{ap.idfun}_A$ by identity elimination, taking
\begin{equation*}
\mathsf{ap.idfun}_A(\refl{x}) \defeq \refl{\refl{x}}.
\end{equation*}
Finally, we construct $\mathsf{ap.comp}(f,g)$ by identity elimination, taking
\begin{equation*}
\mathsf{ap.comp}(f,g,\refl{x}) \defeq \refl{g(f(x))}.\qedhere
\end{equation*}
\end{constr}

\begin{defn}
Let $f:A\to B$ be a map. Then there are identifications
\begin{align*}
\mathsf{ap.refl}(f,x) & : \id{\ap{f}{\refl{x}}}{\refl{f}(x)} \\
\mathsf{ap.inv}(f,p) & : \id{\ap{f}{p^{-1}}}{\ap{f}{p}^{-1}} \\
\mathsf{ap.concat}(f,p,q) & : \id{\ap{f}{\ct{p}{q}}}{\ct{\ap{f}{p}}{\ap{f}{q}}}
\end{align*}
for every $p:\id{x}{y}$ and $q:\id{x}{y}$.
\end{defn}

\begin{constr}
To construct $\mathsf{ap.refl}(f,x)$ we simply observe that ${\ap{f}{\refl{x}}}\jdeq {\refl{f}(x)}$, so we take
\begin{equation*}
\mathsf{ap.refl}(f,x)\defeq\refl{\refl{f(x)}}.
\end{equation*}
We construct $\mathsf{ap.inv}(f,p)$ by identification elimination on $p$, taking
\begin{equation*}
\mathsf{ap.inv}(f,\refl{x}) \defeq \refl{\ap{f}{\refl{x}}}.
\end{equation*}
Finally we construct $\mathsf{ap.concat}(f,p,q)$ by identification elimination on $p$, taking
\begin{equation*}
\mathsf{ap.concat}(f,\refl{x},q)  \defeq \refl{\ap{f}{q}}.\qedhere
\end{equation*}
\end{constr}

\section{Transport}

\begin{table}
\begin{center}
\caption{The homotopy interpretation}
\begin{tabular}{ll}
\toprule
\emph{Type theory} &  \emph{Homotopy theory} \\
\midrule
Types  & Spaces \\
Dependent types & Fibrations \\
Terms & Points \\
Dependent pair type & Total space \\
Identity type & Path fibration\\
\bottomrule
\end{tabular}
\end{center}
\end{table}

Dependent types also come with an action on paths: the \emph{transport} functions.
Given an identification $p:\id{x}{y}$ in the base type $A$, we can transport any term $b:B(x)$ to the fiber $B(y)$.
The transport functions have many applications, which we will encounter throughout this course.

\begin{defn}
Let $A$ be a type, and let $B$ be a type family over $A$.
We will construct a \define{transport} operation
\begin{equation*}
\mathsf{tr}_B:\prd*{x,y:A} (\id{x}{y})\to (B(x)\to B(y)).
\end{equation*}
We will write $\trans{p}{b}$ for $\mathsf{tr}_B(p,b)$.
\end{defn}

\begin{constr}
We construct $\mathsf{tr}_B(p)$ by induction on $p:x=_A y$, taking
\begin{equation*}
\mathsf{tr}_B(\refl{x}) \defeq \idfunc[B(x)].\qedhere
\end{equation*}
\end{constr}

Thus we see that type theory cannot distinguish between identified terms $x$ and $y$, because for any type family $B$ over $A$ one gets a term of $B(y)$ as soon as $B(x)$ has a term.

As an application of the transport function we construct the \emph{dependent} action on paths of a dependent function $f:\prd{x:A}B(x)$. Note that for such a dependent function $f$, and an identification $p:\id[A]{x}{y}$, it does not make sense to directly compare $f(x)$ and $f(y)$, since the type of $f(x)$ is $B(x)$ whereas the type of $f(y)$ is $B(y)$, which might not be exactly the same type. However, we can first \emph{transport} $f(x)$ along $p$, so that we obtain the term $\mathsf{tr}_B(p,f(x))$ which is of type $B(y)$. Now we can ask whether it is the case that $\mathsf{tr}_B(p,f(x))=f(y)$. The dependent action on paths of $f$ establishes this identification.

\begin{defn}\label{defn:apd}
Given a dependent function $f:\prd{a:A}B(a)$ and a path $p:\id{x}{y}$ in $A$, we construct a path
\begin{equation*}
\apd{f}{p} : \id{\mathsf{tr}_B(p,f(x))}{f(y)}.
\end{equation*}
\end{defn}

\begin{constr}
The path $\apd{f}{p}$ is constructed by path induction on $p$. Thus, it suffices to construct a path
\begin{equation*}
\apd{f}{\refl{x}}:\id{\mathsf{tr}_B(\refl{x},f(x))}{f(x)}.
\end{equation*}
Since transporting along $\refl{x}$ is the identity function on $B(x)$, we simply take $\apd{f}{\refl{x}}\defeq\refl{f(x)}$. 
\end{constr}

%\begin{defn}\label{defn:path_lifting}
%Let $A$ be a type, and let $B:A\to\type$ be a type family over $A$.
%We will construct a \define{path lifting} operation
%\begin{equation*}
%\mathsf{lift}^B : \prd*{x,y:A}{p:\id{x}{y}}{b:B(x)} \id{\pairr{x,b}}{\pairr{y,\trans{p}{b}}}.
%\end{equation*}
%\end{defn}
%
%\autoref{defn:path_lifting} gives a way to lift a path $p:x=y$ in the base type of a type family, to a path in the $\Sigma$-type. This, along with the basic groupoid operations developed in \autoref{sec:groupoid}, inspired the \emph{homotopy interpretation} of type theory.

\begin{exercises}
\item \label{ex:trans_concat}Let $B$ be a family over a type $A$. Construct for any two identifications $p:x=_A y$ and $q:y=_A z$, and any $b:B(x)$ an identification
\begin{equation*}
\mathsf{tr}_B(q,\mathsf{tr}_B(p,x))=\mathsf{tr}_B(\ct{p}{q},x).
\end{equation*}
\item \label{ex:inv_assoc}Let $p:\id{x}{y}$ and $q:\id{y}{z}$. Construct an identification
\begin{align*}
\mathsf{inv\usc{}assoc}(p,q):\id{(\ct{p}{q})^{-1}}{\ct{q^{-1}}{p^{-1}}}.
\end{align*}
\item \label{ex:trans_triv}Consider two types $A$ and $B$, and let $p:x=y$ in $A$, and $b:B$. Construct an identification
\begin{align*}
\mathsf{tr\usc{}triv}(p,b):\mathsf{tr}_{W_A(B)}(p,b)=b
\end{align*}
where $W_A(B)$ is the family $B$ weakened by $A$.
%\item In this exercise we show that the action on paths of a function preserves the groupoid-structure of a type.
%\begin{subexenum}
%\item Construct an identification
%\begin{equation*}
%\mathsf{ap.assoc}(f,p,q,r)
%\end{equation*}
%witnessing that the diagram
%\begin{equation*}
%\begin{tikzcd}[column sep=large]
%\ap{f}{\ct{(\ct{p}{q})}{r}} \arrow[r,equals,"\ap{\apfunc{f}}{\mathsf{assoc}(p,q,r)}"] \arrow[d,swap,equals,"{\mathsf{ap.ct}(f,%\ct{p}{q},r)}"] & \ap{f}{\ct{p}{(\ct{q}{r})}} \arrow[d,equals,"{\mathsf{ap.ct}(f,p,\ct{q}{r})}"] \\ 
%\ct{\ap{f}{\ct{p}{q}}}{\ap{f}{r}} \arrow[dd,equals,near start,"{\mathsf{whisk\usc{}r}(\mathsf{ap.ct}(f,p,q),\ap{f}{r})}"]   & %\ct{\ap{f}{p}}{\ap{f}{\ct{q}{r}}} \arrow[dd,equals,swap,near end,"{\mathsf{whisk\usc{}l}(\ap{f}{p},\mathsf{ap.ct}(f,q,r))}"]  %\\
%\\
%\ct{(\ct{\ap{f}{p}}{\ap{f}{q}})}{\ap{f}{r}} \arrow[r,equals,swap,"{\mathsf{assoc}(\ap{f}{p},\ap{f}{q},\ap{f}{r})}"yshift=-1em] & \ct{\ap{f}{p}}{(\ct{\ap{f}{q}}{\ap{f}{r}})}
%\end{tikzcd}
%\end{equation*}
%commutes.
%\end{subexenum}
\item \label{ex:trans_ap}Let $f:A\to B$ be a map, and consider $p:x=y$ in $A$. 
\begin{subexenum}
\item Construct for any $q:f(x)=b$ in $B$ an identification
\begin{equation*}
\mathsf{tr\usc{}ap}(p,q):\id{\trans{p}{q}}{\ct{\ap{f}{p}^{-1}}{q}}.
\end{equation*}
\item Similarly, construct for any $q':b=f(x)$ in $B$ an identification
\begin{equation*}
\mathsf{tr\usc{}ap'}(p,q'):\id{\trans{p}{q}}{\ct{q}{\ap{f}{p}}}.
\end{equation*}
\end{subexenum}
\item \label{ex:inv_con}For any $p:x=y$, $q:y=z$, and $r:x=z$, construct maps
\begin{align*}
\mathsf{inv\usc{}con}(p,q,r) & : (\ct{p}{q}=r)\to (q=\ct{p^{-1}}{r}) \\
\mathsf{con\usc{}inv}(p,q,r) & : (\ct{p}{q}=r)\to (p=\ct{r}{q^{-1}}).
\end{align*}
\item Let $A$ be a type, and let $B:A\to\type$ be a type family over $A$.
Construct the \define{path lifting} operation
\begin{equation*}
\mathsf{lift}_B : \prd*{x,y:A}{p:\id{x}{y}}{b:B(x)} \id{\pairr{x,b}}{\pairr{y,\trans{p}{b}}}.
\end{equation*}
In other words, a path in the \emph{base type} $A$ lifts to a path in the total space $\sm{x:A}B(x)$ for every term over the domain.
\item Show that the operations of addition and multiplication on the natural numbers satisfy the following laws:
\begin{align*}
m+(n+k) & =(m+n)+k & m\cdot (n\cdot k) & = (m\cdot n)\cdot k \\
m+0 & = m & m\cdot 1 & = m \\
0+m & = m & 1\cdot m & = m \\
m+n & = n+m & m\cdot n & = n\cdot m\\
& & m\cdot (n+k) & = m\cdot n + m\cdot k.
\end{align*}
\end{exercises}


% !TEX root = hott_intro.tex

\section{Equivalences}

\subsection{Homotopies}
In homotopy type theory, a homotopy is just a pointwise equality between two functions $f$ and $g$.

\begin{defn}
Let $f,g:\prd{x:A}P(x)$ be two dependent functions. The type of \define{homotopies}\index{homotopy|textbf} from $f$ to $g$ is defined as
\begin{equation*}
f\htpy g \defeq \prd{x:A} f(x)=g(x).
\end{equation*}
\end{defn}

Since we formulated homotopies using dependent functions, we may also consider homotopies \emph{between}\index{homotopy!iterated} homotopies, and further homotopies between those higher homotopies. 
Explicitly, if $H,K:f\htpy g$, then the type $H\htpy K$ of homotopies is just the type
\begin{equation*}
\prd{x:A} H(x)=K(x).
\end{equation*}

In the following definition we define the groupoid-like structure of homotopies. Note that we implement the groupoid-laws as \emph{homotopies} rather than as identifications.

\begin{defn}\label{defn:htpy_groupoid}\index{groupoid laws!of homotopies|textbf}
For any dependent type $B:A\to\type$ there are operations
\begin{align*}
& \mathsf{htpy\usc{}refl} & & : \prd{f:\prd{x:A}B(x)}f\htpy f \\
& \mathsf{htpy\usc{}inv} & & : \prd*{f,g:\prd{x:A}B(x)} (f\htpy g)\to(g\htpy f) \\
& \mathsf{htpy\usc{}concat} & & : \prd*{f,g,h:\prd{x:A}B(x)} (f\htpy g)\to ((g\htpy h)\to (f\htpy h)).
\end{align*}
We will write $H^{-1}$ for $\mathsf{htpy\usc{}inv}(H)$, and $\ct{H}{K}$ for $\mathsf{htpy\usc{}concat}(H,K)$. 

Furthermore, we define
\begin{align*}
& \mathsf{htpy\usc{}assoc}(H,K,L) & & : \ct{(\ct{H}{K})}{L}\htpy\ct{H}{(\ct{K}{L})} \\
& \mathsf{htpy\usc{}left\usc{}unit}(H) & & : \ct{\mathsf{htpy\usc{}refl}_f}{H}\htpy H \\
& \mathsf{htpy\usc{}right\usc{}unit}(H) & & : \ct{H}{\mathsf{htpy\usc{}refl}_g}\htpy H \\
& \mathsf{htpy\usc{}left\usc{}inv}(H) & & : \ct{H^{-1}}{H} \htpy \mathsf{htpy\usc{}refl}_g \\
& \mathsf{htpy\usc{}right\usc{}inv}(H) & & : \ct{H}{H^{-1}} \htpy \mathsf{htpy\usc{}refl}_f
\end{align*}
for any $H:f\htpy g$, $K:g\htpy h$ and $L:h\htpy i$, where $f,g,h,i:\prd{x:A}B(x)$.
\end{defn}

\begin{constr}
We define
\begin{align*}
\mathsf{htpy\usc{}refl}(f) & \defeq \lam{x} \refl{f(x)} \\
\mathsf{htpy\usc{}inv}(H) & \defeq \lam{x} H(x)^{-1} \\
\mathsf{htpy\usc{}concat}(H,K) & \defeq \lam{x}\ct{H(x)}{K(x)},
\end{align*}
where $H:f\htpy g$ and $K:g\htpy h$ are homotopies. Furthermore, we define
\begin{align*}
\mathsf{htpy\usc{}assoc}(H,K,L) & \defeq \lam{x}\mathsf{assoc}(H(x),K(x),L(x)) \\
\mathsf{htpy\usc{}left\usc{}unit}(H) & \defeq \lam{x}\mathsf{left\usc{}unit}(H(x)) \\
\mathsf{htpy\usc{}right\usc{}unit}(H) & \defeq \lam{x}\mathsf{right\usc{}unit}(H(x)) \\
\mathsf{htpy\usc{}left\usc{}inv}(H) & \defeq \lam{x}\mathsf{left\usc{}inv}(H(x)) \\
\mathsf{htpy\usc{}right\usc{}inv}(H) & \defeq \lam{x}\mathsf{right\usc{}inv}(H(x)).\qedhere
\end{align*}
\end{constr}


Apart from the groupoid operations and their laws, we will occasionally need \emph{whiskering} operations.

\begin{defn}
We define the following \define{whiskering}\index{homotopy!whiskering operations|textbf}\index{whiskering operations!of homotopies|textbf} operations on homotopies:
\begin{enumerate}
\item Suppose $H:f\htpy g$ for two functions $f,g:A\to B$, and let $h:B\to C$. We define
\begin{equation*}
h\cdot H\defeq \lam{x}\ap{h}{H(x)}:h\circ f\htpy h\circ g.
\end{equation*}
\item Suppose $f:A\to B$ and $H:g\htpy h$ for two functions $g,h:B\to C$. We define
\begin{equation*}
H\cdot f\defeq\lam{x}H(f(x)):h\circ f\htpy g\circ f.
\end{equation*}
\end{enumerate}
\end{defn}

We also use homotopies to express the commutativity of diagrams. For example, we say that a triangle
\begin{equation*}
  \begin{tikzcd}[column sep=tiny]
    A \arrow[rr,"h"] \arrow[dr,swap,"f"] & & B \arrow[dl,"g"] \\
    & X
  \end{tikzcd}
\end{equation*}
commutes if it comes equipped with a homotopy $H:f\htpy g\circ h$, and we say that a square
\begin{equation*}
  \begin{tikzcd}
    A \arrow[r,"g"] \arrow[d,swap,"f"] & A' \arrow[d,"{f'}"] \\
    B \arrow[r,swap,"h"] & B'
  \end{tikzcd}
\end{equation*}
if it comes equipped with a homotopy $h \circ f~g\circ f'$.

\subsection{Bi-invertible maps}
\begin{defn}
Let $f:A\to B$ be a function. We say that $f$ has a \define{section}\index{section!of a map|textbf} if there is a term of type\index{sec(f)@{$\mathsf{sec}(f)$}|textbf}
\begin{equation*}
\mathsf{sec}(f) \defeq \sm{g:B\to A} f\circ g\htpy \idfunc[B].
\end{equation*}
Dually, we say that $f$ has a \define{retraction}\index{retraction} if there is a term of type\index{retr(f)@{$\mathsf{retr}(f)$}|textbf}
\begin{equation*}
\mathsf{retr}(f) \defeq \sm{h:B\to A} h\circ f\htpy \idfunc[A].
\end{equation*}
If a map $f:A \to B$ has a retraction, we also say that $A$ is a \define{retract}\index{retract!of a type} of $B$.
We say that a function $f:A\to B$ is an \define{equivalence}\index{equivalence|textbf}\index{bi-invertible map|see {equivalence}} if it has both a section and a retraction, i.e., if it comes equipped with a term of type\index{is_equiv@{$\isequiv$}|textbf}
\begin{equation*}
\isequiv(f)\defeq\mathsf{sec}(f)\times\mathsf{retr}(f).
\end{equation*}
We will write $\eqv{A}{B}$\index{equiv@{$\eqv{A}{B}$}|textbf} for the type $\sm{f:A\to B}\isequiv(f)$.
\end{defn}

\begin{rmk}
An equivalence, as we defined it here, can be thought of as a \emph{bi-invertible} map, since it comes equipped with a separate left and right inverse. Explicitly, if $f$ is an equivalence, then there are
\begin{align*}
g & : B\to A & h & : B\to A \\
G & : f\circ g \htpy \idfunc[B] & H & : h\circ f \htpy \idfunc[A].
\end{align*}
Clearly, if $f$ is \define{invertible}\index{invertible map} in the sense that it comes equipped with a function $g:B\to A$ such that $f\circ g\htpy\idfunc[B]$ and $g\circ f\htpy\idfunc[A]$, then $f$ is an equivalence. We write\index{is_invertible@{$\mathsf{is\usc{}invertible}$}|textbf}
\begin{equation*}
\mathsf{has\usc{}inverse}(f)\defeq\sm{g:B\to A} (f\circ g\htpy \idfunc[B])\times (g\circ f\htpy\idfunc[A]).
\end{equation*}
\end{rmk}

\begin{defn}\label{defn:inv_equiv}
Any equivalence can be given the structure of an invertible map.\index{equivalence!invertibility of}
\end{defn}

\begin{constr}
First we construct for any equivalence $f$ with right inverse $g$ and left inverse $h$ a homotopy $K:g\htpy h$. For any $y:B$, we have 
\begin{equation*}
\begin{tikzcd}[column sep=huge]
g(y) \arrow[r,equals,"H(g(y))^{-1}"] & hfg(y) \arrow[r,equals,"\ap{h}{G(y)}"] & h(y).
\end{tikzcd}
\end{equation*} 
Therefore we define a homotopy $K:g\htpy h$ by $K\defeq \ct{(H\cdot g)^{-1}}{h\cdot G}$.
Using the homotopy $K$ we are able to show that $g$ is also a left inverse of $f$. For $x:A$ we have the identification
\begin{equation*}
\begin{tikzcd}[column sep=large]
gf(x) \arrow[r,equals,"K(f(x))"] & hf(x) \arrow[r,equals,"H(x)"] & x.
\end{tikzcd}\qedhere
\end{equation*}
\end{constr}

\begin{cor}
The inverse of an equivalence is again an equivalence.
\end{cor}

\begin{proof}
Let $f:A\to B$ be an equivalence. By \cref{defn:inv_equiv} it follows that the section of $f$ is also a retraction. Therefore it follows that the section is itself an invertible map, with inverse $f$. Hence it is an equivalence.
\end{proof}

\begin{thm}\label{thm:id_equiv}
For any type $A$, the identity function $\idfunc[A]$ is an equivalence.\index{identity function!is an equivalence|textit}
\end{thm}

\begin{proof}
The identity function is trivially its own section and its own retraction.
\end{proof}

\begin{eg}
  For any type $C(x,y)$ indexed by $x:A$ and $y:B$, the function
\begin{equation*}
\sigma:\Big(\prd{x:A}{y:B}C(x,y)\Big)\to\Big(\prd{y:B}{x:A}C(x,y)\Big)
\end{equation*}
that swaps the order of the arguments $x$ and $y$ is an equivalence by \cref{ex:swap}.\index{swap function!is an equivalence|textit}
\end{eg}

\subsection{The identity type of a \texorpdfstring{$\Sigma$-}{dependent pair }type}

In this section we characterize the identity type of a $\Sigma$-type as a $\Sigma$-type of identity types. In this course we will be characterizing the identity types of many types, so we will follow the general outline of how such a characterization goes:
\begin{enumerate}
\item First we define a binary relation $R:A\to A\to \UU$ on the type $A$ that we are interested in. This binary relation is intended to be equivalent to its identity type.
\item Then we will show that this binary relation is reflexive, by constructing a term of type
  \begin{equation*}
    \prd{x:A}R(x,x)
  \end{equation*}
\item Using the reflexivity we will show that there is a canonical map
  \begin{equation*}
    (x=y)\to R(x,y)
  \end{equation*}
  for every $x,y:A$. This map is just constructed by path induction, using the reflexivity of $R$.
\item Finally, it has to be shown that the map
  \begin{equation*}
    (x=y)\to R(x,y)
  \end{equation*}
  is an equivalence for each $x,y:A$. 
\end{enumerate}
The last step is usually the most difficult, and we will refine our methods for this step in \cref{chap:fundamental}, where we establish the Fundamental Theorem of Identity Types.

In this section we consider a type family $B$ over $A$. Given two pairs
\begin{equation*}
  (x,y),(x',y'):\sm{x:A}B(x),
\end{equation*}
if we have a path $\alpha:x=x'$ then we can compare $y:B(x)$ to $y':B(x')$ by first transporting $y$ along $\alpha$, i.e., we consider the identity type
\begin{equation*}
  \mathsf{tr}_B(\alpha,y)=y'.
\end{equation*}
Thus it makes sense to think of $(x,y)$ to be identical to $(x',y')$ if there is an identification $\alpha:x=x'$ and an identification $\beta:\mathsf{tr}_B(\alpha,y)=y'$. In the following definition we turn this idea into a binary relation on the $\Sigma$-type.

\begin{defn}
  We will define a relation
  \begin{equation*}
    \mathsf{Eq}_{\Sigma} : \Big(\sm{x:A}B(x)\Big)\to\Big(\sm{x:A}B(x)\Big)\to\UU
  \end{equation*}
  by defining
  \begin{equation*}
    \mathsf{Eq}_{\Sigma}(s,t)\defeq\sm{\alpha:\proj 1(s)=\proj 1(t)}\mathsf{tr}_B(\alpha,\proj 2(s))=\proj 2 (t).
  \end{equation*}
\end{defn}

\begin{lem}
  The relation $\mathsf{Eq}_{\Sigma}$ is reflexive, i.e., there is a term
  \begin{equation*}
    \mathsf{reflexive\usc{}Eq}_{\Sigma}:\prd{s:\sm{x:A}B(x)}\mathsf{Eq}_{\Sigma}(s,s).
  \end{equation*}
\end{lem}

\begin{constr}
  This term is constructed by $\Sigma$-induction on $s:\sm{x:A}B(x)$. Thus, it suffices to construct a term of type
  \begin{equation*}
    \prd{x:A}{y:B(x)}\sm{\alpha:x=x}\mathsf{tr}_B(\alpha,y)=y.
  \end{equation*}
  Here we take $\lam{x}{y}(\refl{x},\refl{y})$.
\end{constr}

\begin{defn}
  Consider a type family $B$ over $A$. Then for any $s,t:\sm{x:A}B(x)$ we define a map\index{pair_eq@{$\mathsf{pair\usc{}eq}$}|textbf}
  \begin{equation*}
    \mathsf{pair\usc{}eq}: (s=t)\to \mathsf{Eq}_\Sigma(s,t)
  \end{equation*}
  by path induction, taking $\mathsf{pair\usc{}eq}(\refl{s})\defeq\mathsf{reflexive\usc{}Eq}_\Sigma(s)$.
\end{defn}

\begin{thm}\label{thm:eq_sigma}
  Let $B$ be a type family over $A$. Then the map
  \begin{equation*}
    \mathsf{pair\usc{}eq}: (s=t)\to \mathsf{Eq}_\Sigma(s,t)
  \end{equation*}
  is an equivalence for every $s,t:\sm{x:A}B(x)$.\index{Sigma type@{$\Sigma$-type}!identity types of|textit}\index{identity type!of a Sigma-type@{of a $\Sigma$-type}|textit}
\end{thm}

\begin{proof}
The maps in the converse direction\index{eq_pair@{$\mathsf{eq\usc{}pair}$}}
\begin{equation*}
\mathsf{eq\usc{}pair} : \mathsf{Eq}_\Sigma(s,t)\to(\id{s}{t})
\end{equation*}
are defined by repeated $\Sigma$-induction. By $\Sigma$-induction on $s$ and $t$  we see that it suffices to define a map
\begin{equation*}
\mathsf{eq\usc{}pair} : \Big(\sm{p:x=x'}\id{\mathsf{tr}_B(p,y)}{y'}\Big)\to(\id{(x,y)}{(x',y')}).
\end{equation*}
A map of this type is again defined by $\Sigma$-induction. Thus it suffices to define a dependent function of type
\begin{equation*}
\prd{p:x=x'} (\id{\mathsf{tr}_B(p,y)}{y'}) \to (\id{(x,y)}{(x',y')}).
\end{equation*}
Such a dependent function is defined by double path induction by sending $\pairr{\refl{x},\refl{y}}$ to $\refl{\pairr{x,y}}$. This completes the definition of the function $\mathsf{eq\usc{}pair}$.

Next, we must show that $\mathsf{eq\usc{}pair}$ is a section of $\mathsf{pair\usc{}eq}$. In other words, we must construct an identification
\begin{equation*}
\mathsf{pair\usc{}eq}(\mathsf{eq\usc{}pair}(\alpha,\beta))=\pairr{\alpha,\beta}
\end{equation*}
for each $\pairr{\alpha,\beta}:\sm{\alpha:x=x'}\id{\mathsf{tr}_B(\alpha,y)}{y'}$. We proceed by path induction on $\alpha$, followed by path induction on $\beta$. Then our goal becomes to construct a term of type
\begin{equation*}
\mathsf{pair\usc{}eq}(\mathsf{eq\usc{}pair}\pairr{\refl{x},\refl{y}})=\pairr{\refl{x},\refl{y}}
\end{equation*}
By the definition of $\mathsf{eq\usc{}pair}$ we have $\mathsf{eq\usc{}pair}\pairr{\refl{x},\refl{y}}\jdeq \refl{\pairr{x,y}}$, and by the definition of $\mathsf{pair\usc{}eq}$ we have $\mathsf{pair\usc{}eq}(\refl{\pairr{x,y}})\jdeq\pairr{\refl{x},\refl{y}}$. Thus we may take $\refl{\pairr{\refl{x},\refl{y}}}$ to complete the construction of the homotopy $\mathsf{pair\usc{}eq}\circ\mathsf{eq\usc{}pair}\htpy\idfunc$.

To complete the proof, we must show that $\mathsf{eq\usc{}pair}$ is a retraction of $\mathsf{pair\usc{}eq}$. In other words, we must construct an identification
\begin{equation*}
\mathsf{eq\usc{}pair}(\mathsf{pair\usc{}eq}(p))=p
\end{equation*}
for each $p:s=t$. We proceed by path induction on $p:s=t$, so it suffices to construct an identification 
\begin{equation*}
\mathsf{eq\usc{}pair}\pairr{\refl{\proj 1(s)},\refl{\proj 2(s)}}=\refl{s}.
\end{equation*}
Now we proceed by $\Sigma$-induction on $s:\sm{x:A}B(x)$, so it suffices to construct an identification
\begin{equation*}
\mathsf{eq\usc{}pair}\pairr{\refl{x},\refl{y}}=\refl{(x,y)}.
\end{equation*}
Since $\mathsf{eq\usc{}pair}\pairr{\refl{x},\refl{y}}$ computes to $\refl{(x,y)}$, we may simply take $\refl{\refl{(x,y)}}$.
\end{proof}

\begin{exercises}
%  \item Show that for any term $a:A$ the functions
%    \begin{align*}
%      \ind{\unit}(a) & : \unit \to A \\
%      \mathsf{const}_a & : \unit \to A
%    \end{align*}
%    are homotopic.
%  \item Let $A$ and $B$ be types, and consider the constant map $\mathsf{const}_b:A\to B$ for some $b:B$. Construct a homotopy
%    \begin{equation*}
%      \mathsf{ap}_{\mathsf{const}_b}(x,y)\htpy \mathsf{const}_{\refl{b}}
%    \end{equation*}
%    for any $x,y:A$.
\item \label{ex:equiv_grpd_ops}Show that the functions
  \begin{align*}
    \mathsf{inv} & :(\id{x}{y})\to(\id{y}{x}) \\
    \mathsf{concat}(p) & : (\id{y}{z})\to(\id{x}{z}) \\
    \mathsf{concat'}(q) & : (\id{x}{y}) \to (\id{x}{z}) \\
    \mathsf{tr}_B(p) & :B(x)\to B(y)
  \end{align*}
  are equivalences, where $\mathsf{concat'}(q,p)\defeq \ct{p}{q}$\index{concat'@{$\mathsf{concat'}$}}. Give their inverses explicitly.
\item Show that the maps
  \begin{align*}
    \inl & : X \to X+\emptyt &     \proj 1 & : \emptyt \times X \to \emptyt \\
    \inr & : X \to \emptyt+X &    \proj 2 & : X \times \emptyt \to \emptyt
  \end{align*}
  are equivalences.
\item
  \begin{subexenum}
  \item \label{ex:htpy_equiv}\index{equivalence!homotopic maps} Consider two functions $f,g:A\to B$ and a homotopy $H:f\htpy g$. Then
    \begin{equation*}
      \isequiv(f)\leftrightarrow\isequiv(g).
    \end{equation*}
  \item Show that for any two homotopic equivalences $e,e':\eqv{A}{B}$, their inverses are also homotopic.
  \end{subexenum}
\item \label{ex:3_for_2}\index{equivalence!three@{3-for-2 property}}\index{3-for-2 property!of equivalences}
  Consider a commuting triangle
  \begin{equation*}
    \begin{tikzcd}[column sep=tiny]
      A \arrow[rr,"h"] \arrow[dr,swap,"f"] & & B \arrow[dl,"g"] \\
      & X.
    \end{tikzcd}
  \end{equation*}
  with $H:f\htpy g\circ h$.
  \begin{subexenum}
  \item Suppose that the map $h$ has a section $s:B \to A$. Show that the triangle
    \begin{equation*}
      \begin{tikzcd}[column sep=tiny]
        B \arrow[rr,"s"] \arrow[dr,swap,"g"] & & A \arrow[dl,"f"] \\
        & X.
      \end{tikzcd}
    \end{equation*}
    commutes, and that $f$ has a section if and only if $g$ has a section.
  \item Suppose that the map $g$ has a retraction $r:X\to B$. Show that the triangle
    \begin{equation*}
      \begin{tikzcd}[column sep=tiny]
        A \arrow[rr,"f"] \arrow[dr,swap,"h"] & & X \arrow[dl,"r"] \\
        & B.
      \end{tikzcd}
    \end{equation*}
    commutes, and that $f$ has a retraction if and only if $h$ has a retraction.
  \item (The \define{3-for-2 property} for equivalences.) Show that if any two of the functions
    \begin{equation*}
      f,\qquad g,\qquad h
    \end{equation*}
    are equivalences, then so is the third.
  \end{subexenum}
\item \label{ex:neg_equiv} 
  \begin{subexenum}
  \item Define the negation function on the booleans, and show that it is an equivalence.\index{negation function!is an equivalence}
  \item Use the observational equality on the booleans, defined in \cref{ex:obs_bool}, to show that $\btrue\neq\bfalse$.
  \item Show that for any $b:\bool$, the constant function $\mathsf{const}_b$ is not an equivalence.
  \end{subexenum}
\item \label{ex:succ_equiv} Show that the successor function on the integers is an equivalence.\index{successor function!on Z@{on $\Z$}!is an equivalence}
\item \label{ex:comm_prod}Construct a equivalences $\eqv{A+B}{B+A}$ and $\eqv{A\times B}{B\times A}$.\index{coproduct!is symmetric}
\item \label{ex:retr_id} Consider a section-retraction pair
  \begin{equation*}
    \begin{tikzcd}
      A \arrow[r,"i"] & B \arrow[r,"r"] & A,
    \end{tikzcd}
  \end{equation*}
  with $H:r\circ i\htpy \idfunc$. Show that $\id{x}{y}$ is a retract of $\id{i(x)}{i(y)}$.\index{retract!identity types of}
\item \label{ex:sigma_assoc}\index{Sigma type@{$\Sigma$-type}!associativity of}Let $B$ be a family of types over $A$, and let $C$ be a family of types indexed by $x:A,y:B(x)$. Construct an equivalence
  \begin{equation*}
    \Sigma\mathsf{\usc{}assoc}:\eqv{\Big(\sm{p:\sm{x:A}B(x)}C(\proj 1(p),\proj 2(p))\Big)}{\Big(\sm{x:A}\sm{y:B(x)}C(x,y)\Big)}.
  \end{equation*}
\item \label{ex:sigma_swap}Let $A$ and $B$ be types, and let $C$ be a family over $x:A,y:B$. Construct an equivalence
  \begin{equation*}
    \Sigma\mathsf{\usc{}swap}:\eqv{\Big(\sm{x:A}{y:B}C(x,y)\Big)}{\Big(\sm{y:B}{x:A}C(x,y)\Big)}.
  \end{equation*}
  %\item \label{ex:sigma_base_equiv}Consider an equivalence $e:A'\eqv A$ and a type family $B$ over $A$. Show that the map
  %\begin{equation*}
  %\lam{(x',y)}(e(x'),y) : \Big(\sm{x':A'}B(e(x'))\Big)\to\Big(\sm{x:A}B(x)\Big)
  %\end{equation*}
  %is an equivalence.
\item \label{ex:int_group_laws}\index{Z@{$\Z$}!group laws} In this exercise we will show that the laws for abelian groups hold for addition on the integers. Note: these are obvious facts, but the proof terms that show \emph{how} the group laws hold are nevertheless fairly involved. This exercise is perfect for a formalization project. 
  \begin{subexenum}
  \item Show that addition satisfies the left and right unit laws, i.e., construct terms
    \begin{align*}
      \mathsf{left\usc{}unit\usc{}law\usc{}add\usc{}}\Z  & : \prd{x:\Z} 0 + x = x \\
      \mathsf{right\usc{}unit\usc{}law\usc{}add\usc{}}\Z  & : \prd{x : \Z} x + 0 = x.
    \end{align*}
  \item Show that addition respects predecessors and successor on both sides, i.e., construct terms
    \begin{align*}
      \mathsf{left\usc{}predecessor\usc{}law\usc{}add\usc{}}\Z & : \prd{x,y:\Z} \mathsf{pred}_{\mathbb{Z}}(x)+y = \mathsf{pred}_{\mathbb{Z}}(x+y) \\
      \mathsf{right\usc{}predecessor\usc{}law\usc{}add\usc{}}\Z & : \prd{x,y:\Z} x+\mathsf{pred}_{\mathbb{Z}}(y) = \mathsf{pred}_{\mathbb{Z}}(x+y) \\
      \mathsf{left\usc{}successor\usc{}law\usc{}add\usc{}}\Z & : \prd{x,y:\Z} \mathsf{succ}_{\mathbb{Z}}(x)+y = \mathsf{succ}_{\mathbb{Z}}(x+y) \\
      \mathsf{right\usc{}successor\usc{}law\usc{}add\usc{}}\Z & : \prd{x,y:\Z} x+\mathsf{succ}_{\mathbb{Z}}(y) = \mathsf{succ}_{\mathbb{Z}}(x+y).
    \end{align*}
    Hint: to avoid an excessive number of cases, use induction on $x$ but not on $y$. You may need to use the homotopies $\mathsf{succ}_{\mathbb{Z}}\circ \mathsf{pred}_{\mathbb{Z}}\htpy \idfunc$ and $\mathsf{pred}_{\mathbb{Z}}\circ\mathsf{succ}_{\mathbb{Z}}$ constructed in exercise \cref{ex:succ_equiv}.
  \item Use part (b) to show that addition on the integers is associative and commutative, i.e., construct terms
    \begin{align*}
      \mathsf{assoc\usc{}add\usc{}}\Z & : \prd{x,y,z:\Z} (x+y)+z = x + (y+z) \\
      \mathsf{comm\usc{}add\usc{}}\Z & : \prd{x,y:\Z} x+y = y+x.
    \end{align*}
    Hint: Especially in the construction of the associator there is a risk of running into an unwieldy amount of cases if you use $\Z$-induction on all arguments. Avoid induction on $y$ and $z$.
  \item Show that addition satisfies the left and right inverse laws:
    \begin{align*}
      \mathsf{left\usc{}inverse\usc{}law\usc{}add\usc{}}\Z & : \prd{x : \Z} (-x)+x=0 \\
      \mathsf{right\usc{}inverse\usc{}law\usc{}add\usc{}}\Z & : \prd{x : \Z} x + (-x)=0.
    \end{align*}
    Conclude that the functions $y \mapsto x + y$ and $x\mapsto x + y$ are equivalences for any $x:\Z$ and $y:\Z$, respectively.
  \end{subexenum}
\item \label{ex:coproduct_functor}In this exercise we will construct the \define{functorial action} of coproducts.
  \begin{subexenum}
  \item Construct for any two maps $f:A \to A'$ and $g:B \to B'$, a map
    \begin{equation*}
      f+g:A+B \to A'+B'.
    \end{equation*}
  \item Show that if $H:f\htpy f'$ and $K:g\htpy g'$, then there is a homotopy
    \begin{equation*}
      H+K:(f+g)\htpy (f'+g').
    \end{equation*}
  \item Show that $\idfunc[A]+\idfunc[B]\htpy \idfunc[A+B]$.
  \item Show that for any $f:A\to A'$, $f':A'\to A''$, $g:B\to B'$, and $g':B'\to B''$ there is a homotopy
    \begin{equation*}
      (f'\circ f)+(g'\circ g) \htpy (f'+g')\circ (f\circ g).
    \end{equation*}
  \item \label{ex:coproduct_functor_equivalence}Show that if $f$ and $g$ are equivalences, then so is $f+g$. (The converse of this statement also holds, see \cref{ex:is-equiv-is-equiv-functor-coprod}.)
  \end{subexenum}
\item Construct equivalences
  \begin{align*}
    \mathsf{Fin}(m+n) & \simeq \mathsf{Fin}(m)+\mathsf{Fin}(n) \\
    \mathsf{Fin}(mn) & \simeq \mathsf{Fin}(m)\times\mathsf{Fin}(n).
  \end{align*}
\end{exercises}


\chapter{Contractible types and contractible maps}
\chaptermark{Contractible types and maps}

\section{Contractible types}

\begin{thm}\label{thm:contractible}
Let $A$ be a type. The following are equivalent:
\begin{enumerate}
\item $A$ is \define{contractible}: there is a term of type
\begin{equation*}
\iscontr(A) \defeq \sm{c:A}\prd{x:A}c=x.
\end{equation*}
Given a term $(c,C):\iscontr(A)$, we call $c:A$ the \define{center of contraction} of $A$, and we call $C:\prd{x:A}a=x$ the \define{contraction} of $A$.
\item $A$ comes equipped with a term $a:A$, and satisfies \define{singleton induction}: for every $B:A\to\mathsf{Type}$ the map
\begin{equation*}
\Big(\prd{x:A}B(x)\Big)\to B(a)
\end{equation*}
given by $f\mapsto f(a)$ has a section.
\end{enumerate}
\end{thm}

\begin{rmk}
Suppose $A$ is a contractible type with center of contraction $c$ and contraction $C$. Then the type of $C$ is (judgmentally) equal to the type
\begin{equation*}
\mathsf{const}_c\htpy\idfunc[A].
\end{equation*}
In other words, the contraction $C$ is a \emph{homotopy} from the constant function to the identity function.
\end{rmk}

\begin{proof}[Proof of \autoref{thm:contractible}]
Suppose $A$ is contractible with center of contraction $c$ and contraction $C$. Without loss of generality we may assume that $C(c)=\refl{c}$, since for any contraction $C$ we can define a new contraction $C'$ satisfying this property by setting $C'(x)\defeq\ct{C(c)^{-1}}{C(x)}$. To show that $A$ satisfies singleton induction let $B:A\to\mathsf{Type}$ be a type family over $A$ equipped with $b:B(a)$. We define $\mathsf{sing\usc{}ind}(b):\prd{x:A}B(x)$ by $\lam{x}\mathsf{tr}^B(C(x),b)$. To see that $\mathsf{sing\usc{}ind}(c)=b$ note that we have
\begin{equation*}
\mathsf{tr}^B(C(c),b)=\mathsf{tr}^B(\refl{c},b)=b.
\end{equation*}
This completes the proof that $A$ satisfies singleton induction.

For the converse, if $A$ comes equipped with a center of contraction $a:A$ and satisfies singleton induction, then we can use singleton induction to construct the contraction using the family $B(x)\defeq a=x$ with the base point $\refl{a}:B(a)$. 
\end{proof}

\begin{eg}
By definition the unit type $\unit$ satisfies singleton induction, so it is contractible.
\end{eg}

\section{Contractible maps}
\begin{defn}
Let $f:A\to B$ be a function, and let $b:B$. The \define{fiber} of $f$ at $b$ is defined to be the type
\begin{equation*}
\fib{f}{b}\defeq\sm{a:A}f(a)=b.
\end{equation*}
\end{defn}

\begin{defn}
We say that a function $f:A\to B$ is \define{contractible} if there is a term of type
\begin{equation*}
\iscontr(f)\defeq\prd{b:B}\iscontr(\fib{f}{b}).
\end{equation*}
\end{defn}

\begin{thm}\label{thm:equiv_contr}
Any contractible map is an equivalence.
\end{thm}

\begin{proof}
Let $f:A\to B$ be a contractible map. Using the center of contraction of each $\fib{f}{y}$, we obtain a term of type
\begin{align*}
\lam{y}\pairr{g(y),G(y)}:\prd{y:B}\fib{f}{y}.
\end{align*}
Thus, we get map $g:B\to A$, and a homotopy $G:\prd{y:B} f(g(y))=y$. In other words, we get a section of $f$.

It remains to construct a retraction of $f$. Taking $g$ as our retraction, we have to show that $\prd{x:A} g(f(x))=x$. Note that we get an identification $p:f(g(f(x)))=f(x)$ since $g$ is a section of $f$. Moreover, since $\fib{f}{f(x)}$ is contractible we get an identification $q:\pairr{g(f(x)),p}=\pairr{x,\refl{f(x)}}$. The base path of this identification is an identification of type $g(f(x))=x$, as desired.
\end{proof}

\section{Equivalences are contractible maps}

In this section we will show the converse to \autoref{thm:equiv_contr}, that equivalences are contractible maps. Before we do so, we will establish some useful constructions on homotopies and section-retraction pairs.

\begin{defn}\label{defn:htpy_nat}
Let $f,g:A\to B$ be functions, and consider $H:f\htpy g$ and $p:x=y$ in $A$. We will construct an identification
\begin{align*}
\mathsf{htpy\usc{}nat}(H,p) & :\ct{H(x)}{\ap{g}{p}}=\ct{\ap{f}{p}}{H(y)}
\end{align*}
witnessing that the square
\begin{equation*}
\begin{tikzcd}
f(x) \arrow[r,equals,"H(x)"] \arrow[d,equals,swap,"\ap{f}{p}"] & g(x) \arrow[d,equals,"\ap{g}{p}"] \\
f(y) \arrow[r,equals,swap,"H(y)"] & g(y)
\end{tikzcd}
\end{equation*}
commutes.
\end{defn}

\begin{defn}\label{defn:retraction_swap}
Consider $f:A\to A$ and $H: f\htpy \idfunc[A]$. We construct an identification $H(f(x))=\ap{f}{H(x)}$, for any $x:A$.
\end{defn}

\begin{constr}
By the naturality of homotopies with respect to identifications the square
\begin{equation*}
\begin{tikzcd}[column sep=large]
ff(x) \arrow[d,swap,equals,"\ap{f}{H(x)}"] \arrow[r,equals,"H(f(x))"] & gf(x) \arrow[d,equals,"H(x)"] \\
f(x) \arrow[r,swap,equals,"H(x)"] & x
\end{tikzcd}
\end{equation*}
commutes. This gives the desired identification $H(f(x))=\ap{f}{H(x)}$.
\end{constr}

\begin{thm}\label{thm:contr_equiv}
Any equivalence is a contractible map.
\end{thm}

\begin{proof}
Since every equivalence has the structure of an invertible map by \autoref{defn:inv_equiv}, it suffices to show that any invertible map is contractible.

Let $f:A\to B$ be a map, with $g:B\to A$, $G:f\circ g\htpy\idfunc[B]$, and $H:h\circ f\htpy \idfunc[A]$.
We have for any $y:B$ the term $\pairr{g(y),G(y)}:\fib{f}{y}$. However, as our center of contraction we take
$\pairr{g(y),\epsilon(y)}$, where
\begin{equation*}
\varepsilon(y) \defeq \ct{\ap{fg}{G(y)}^{-1}}{\ap{f}{H(g(y))}}{G(y)}.
\end{equation*}
Now it remains to construct the contraction. Let $x:A$, and let $p:f(x)=y$.
Since $p:f(x)=y$ has a free endpoint, we can apply path induction on it. 
Our goal is now to construct an identification
\begin{equation*}
\pairr{g(f(x)),\varepsilon(f(x))}=\pairr{x,\refl{f(x)}}.
\end{equation*}
We will construct an identification of the form $\mathsf{eq\usc{}pair}(H(x),\nameless)$,
so it remains to construct an identification of type
\begin{equation*}
\trans{H(x)}{\varepsilon(f(x))}=\refl{f(x)}.
\end{equation*}
Using \autoref{ex:trans_ap} we see that it suffices to show that the square
\begin{equation*}
\begin{tikzcd}[column sep=8em]
fgfgf(x) \arrow[r,equals,"\ap{fg}{G(f(x))}"] \arrow[d,equals,swap,"\ap{f}{H(gf(x))}"] & fgf(x) \arrow[d,equals,"\ap{f}{H(x)}"] \\
fgf(x) \arrow[r,equals,swap,"G(f(x))"] & f(x)
\end{tikzcd}
\end{equation*}
commutes. Recall from \autoref{defn:retraction_swap} that we have $H(gf(x))=\ap{gf}{H(x)}$ and $\ap{fg}{G(y)}=G(fg(y))$. Using these two identifications and \autoref{ex:ap_ap}, we see that it suffices to show that the square
\begin{equation*}
\begin{tikzcd}[column sep=8em]
fgfgf(x) \arrow[r,equals,"G(fgf(x))"] \arrow[d,equals,swap,"\ap{fgf}{H(x)}"] & fgf(x) \arrow[d,equals,"\ap{f}{H(x)}"] \\
fgf(x) \arrow[r,equals,swap,"G(f(x))"] & f(x)
\end{tikzcd}
\end{equation*}
commutes. However, this is just a naturality square of homotopies, which commutes by \autoref{defn:htpy_nat}.
\end{proof}

\begin{cor}\label{cor:contr_path}
Let $A$ be a type, and let $a:A$. Then the type
\begin{equation*}
\sm{x:A}x=a
\end{equation*}
is contractible.
\end{cor}

\begin{proof}
By \autoref{thm:id_equiv}, the identity function is an equivalence. Therefore, the fibers of the identity function are contractible by \autoref{thm:contr_equiv}. Note that $\sm{x:A}x=a$ is exactly the fiber of $\idfunc[A]$ at $a:A$.
\end{proof}

\begin{comment}
\begin{proof}
We have the term $(a,\refl{a}):\sm{x:A}a=x$, which we take for the center of contraction. To construct the contraction, we have to show that
\begin{equation*}
\prd{p:\sm{x:A}a=x} (a,\refl{a})=p.
\end{equation*}
By the induction principle for dependent pair types it suffices to construct a term of type
\begin{equation*}
\prd{x:A}{p:a=x} (a,\refl{a})=(x,p)
\end{equation*}
Note that we may proceed here by path induction on $p$. That is, it suffices to consider the case $p\jdeq\refl{a}$, and show that $(a,\refl{a})=(a,\refl{a})$. Here we choose $\refl{(a,\refl{a})}$.
\end{proof}
\end{comment}

\begin{exercises}
\item Construct an equivalence 
\begin{equation*}
\eqv{\big(\sm{x:A}f(x)=y\big)}{\big(\sm{x:A}y=f(x)\big)}.
\end{equation*}
Conclude that $\sm{x:A}a=x$ is contractible for any $a:A$.
\item \label{ex:contr_retr}Suppose that $A$ is a retract of $B$. Show that
\begin{equation*}
\iscontr(B)\to\iscontr(A).
\end{equation*}
\item \label{ex:contr_equiv}
\begin{subexenum}
\item Show that for any type $A$, the map $\mathsf{const}_\ttt : A\to \unit$ is an equivalence if and only if $A$ is contractible. 
\item Show that for any map $f:A\to B$, if any two of the three assertions
\begin{enumerate}
\item $A$ is contractible
\item $B$ is contractible
\item $f$ is an equivalence
\end{enumerate}
hold, then so does the third.
\end{subexenum}
\item \label{ex:contr_ind} Let $C$ be a contractible type with center of contraction $c:C$. Furthermore, let $B:C\to\type$ be a type family. Show that the map $b\mapsto\pairr{c,b}:B(c)\to\sm{x:C}B(x)$ is an equivalence.
\item \label{ex:coh_intro}Consider a type $A$ with base point $a:A$, and let $B$ be a type family on $A$ that implies the identity type, i.e.~there is a term
\begin{equation*}
\alpha : \prd{x:A} B(x)\to (a=x).
\end{equation*}
Construct an equivalence
\begin{equation*}
\eqv{\Big(\sm{x:A}B(x)\Big)}{\Big(\sm{y:B(a)}\alpha(a,y)=\refl{a}\Big)}.
\end{equation*}
\end{exercises}


% !TEX root = hott_intro.tex

\section{The fundamental theorem of identity types}\label{chap:fundamental}
\sectionmark{The fundamental theorem}

\index{fundamental theorem of identity types|(}
\index{characterization of identity type!fundamental theorem of identity types|(}
For many types it is useful to have a characterization of their identity types. For example, we have used a characterization of the identity types of the fibers of a map in order to conclude that any equivalence is a contractible map. The fundamental theorem of identity types is our main tool to carry out such characterizations, and with the fundamental theorem it becomes a routine task to characterize an identity type whenever that is of interest.

In our first application of the fundamental theorem of identity types we show that any equivalence is an embedding. Embeddings are maps that induce equivalences on identity types, i.e., they are the homotopical analogue of injective maps. In our second application we characterize the identity types of coproducts.

Throughout the rest of this book we will encounter many more occasions to characterize identity types. For example, we will show in \cref{thm:eq_nat} that the identity type of the natural numbers is equivalent to its observational equality, and we will show in \cref{thm:eq-circle} that the loop space of the circle is equivalent to $\Z$.

In order to prove the fundamental theorem of identity types, we first prove the basic fact that a family of maps is a family of equivalences if and only if it induces an equivalence on total spaces. 

\subsection{Families of equivalences}

\index{family of equivalences|(}
\begin{defn}
Consider a family of maps
\begin{equation*}
f : \prd{x:A}B(x)\to C(x).
\end{equation*}
We define the map\index{total(f)@{$\tot{f}$}}
\begin{equation*}
\tot{f}:\sm{x:A}B(x)\to\sm{x:A}C(x)
\end{equation*}
by $\lam{(x,y)}(x,f(x,y))$.
\end{defn}

\begin{lem}\label{lem:fib_total}
  For any family of maps $f:\prd{x:A}B(x)\to C(x)$ and any $t:\sm{x:A}C(x)$,
  there is an equivalence\index{fiber!of total(f)@{of $\tot{f}$}}\index{total(f)@{$\tot{f}$}!fiber}
  \begin{equation*}
    \eqv{\fib{\tot{f}}{t}}{\fib{f(\proj 1(t))}{\proj 2(t)}}.
  \end{equation*}
\end{lem}

\begin{proof}
  For any $p:\fib{\tot{f}}{t}$ we define $\varphi(t,p):\fib{\proj 1(t)}{\proj 2(t)}$ by $\Sigma$-induction on $p$. Therefore it suffices to define $\varphi(t,(s,\alpha)):\fib{\proj 1(t)}{\proj 2 (t)}$ for any $s:\sm{x:A}B(x)$ and $\alpha:\tot{f}(s)=t$. Now we proceed by path induction on $\alpha$, so it suffices to define $\varphi(\tot{f}(s),(s,\refl{})):\fib{f(\proj 1(\tot{f}(s)))}{\proj 2(\tot{f}(s))}$. Finally, we use $\Sigma$-induction on $s$ once more, so it suffices to define
  \begin{equation*}
    \varphi((x,f(x,y)),((x,y),\refl{})):\fib{f(x)}{f(x,y)}.
  \end{equation*}
  Now we take as our definition
  \begin{equation*}
    \varphi((x,f(x,y)),((x,y),\refl{}))\defeq(y,\refl{}).
  \end{equation*}

  For the proof that this map is an equivalence we construct a map
  \begin{equation*}
    \psi(t) : \fib{f(\proj 1(t))}{\proj 2(t)}\to\fib{\tot{f}}{t}
  \end{equation*}
  equipped with homotopies $G(t):\varphi(t)\circ\psi(t)\htpy\idfunc$ and $H(t):\psi(t)\circ\varphi(t)\htpy\idfunc$. In each of these definitions we use $\Sigma$-induction and path induction all the way through, until an obvious choice of definition becomes apparent. We define $\psi(t)$, $G(t)$, and $H(t)$ as follows:
  \begin{align*}
    \psi((x,f(x,y)),(y,\refl{})) & \defeq ((x,y),\refl{}) \\
    G((x,f(x,y)),(y,\refl{})) & \defeq \refl{} \\
    H((x,f(x,y)),((x,y),\refl{})) & \defeq \refl{}.\qedhere
  \end{align*}
\end{proof}

\begin{thm}\label{thm:fib_equiv}
  Let $f:\prd{x:A}B(x)\to C(x)$ be a family of maps. The following are equivalent:
  \index{is an equivalence!total(f) of family of equivalences@{$\tot{f}$ of family of equivalences}}
  \index{total(f)@{$\tot{f}$}!of family of equivalences is an equivalence}\index{is family of equivalences!if total(f) is an equivalence@{iff $\tot{f}$ is an equivalence}}
\begin{enumerate}
\item For each $x:A$, the map $f(x)$ is an equivalence. In this case we say that $f$ is a \define{family of equivalences}.
\item The map $\tot{f}:\sm{x:A}B(x)\to\sm{x:A}C(x)$ is an equivalence.
\end{enumerate}
\end{thm}

\begin{proof}
By \cref{thm:equiv_contr,thm:contr_equiv} it suffices to show that $f(x)$ is a contractible map for each $x:A$, if and only if $\tot{f}$ is a contractible map. Thus, we will show that $\fib{f(x)}{c}$ is contractible if and only if $\fib{\tot{f}}{x,c}$ is contractible, for each $x:A$ and $c:C(x)$. However, by \cref{lem:fib_total} these types are equivalent, so the result follows by \cref{ex:contr_equiv}.
\end{proof}

Now consider the situation where we have a map $f:A\to B$, and a family $C$ over $B$. Then we have the map
\begin{equation*}
  \lam{(x,z)}(f(x),z):\sm{x:A}C(f(x))\to\sm{y:B}C(y).
\end{equation*}
We claim that this map is an equivalence when $f$ is an equivalence. The technique to prove this claim is the same as the technique we used in \cref{thm:fib_equiv}: first we note that the fibers are equivalent to the fibers of $f$, and then we use the fact that a map is an equivalence if and only if its fibers are contractible to finish the proof.

The converse of the following lemma does not hold. Why not?

\begin{lem}\label{lem:total-equiv-base-equiv}
  Consider an equivalence $e:A\simeq B$, and let $C$ be a type family over $B$. Then the map
  \begin{equation*}
    \sigma_f(C) \defeq\lam{(x,z)}(f(x),z):\sm{x:A}C(f(x))\to\sm{y:B}C(y)
  \end{equation*}
  is an equivalence.
\end{lem}

\begin{proof}
  We claim that for each $t:\sm{y:B}C(y)$ there is an equivalence
  \begin{equation*}
    \fib{\sigma_f(C)}{t}\simeq \fib{f}{\proj 1(t)}.
  \end{equation*}
  We obtain such an equivalence by constructing the following functions and homotopies:
  \begin{align*}
    \varphi(t) & : \fib{\sigma_f(C)}{t}\to\fib{f}{\proj 1 (t)} & \varphi((f(x),z),((x,z),\refl{})) & \defeq (x,\refl{}) \\
    \psi(t) & : \fib{f}{\proj 1(t)} \to\fib{\sigma_f(C)}{t} & \psi((f(x),z),(x,\refl{})) & \defeq ((x,z),\refl{}) \\
    G(t) & : \varphi(t)\circ\psi(t)\htpy\idfunc & G((f(x),z),(x,\refl{})) & \defeq \refl{} \\
    H(t) & : \psi(t)\circ\varphi(t)\htpy\idfunc & H((f(x),z),((x,z),\refl{})) & \defeq \refl{}.
  \end{align*}
  Now the claim follows, since we see that $\varphi$ is a contractible map if and only if $f$ is a contractible map.
\end{proof}

We now combine \cref{thm:fib_equiv,lem:total-equiv-base-equiv}.

\begin{defn}
  Consider a map $f:A\to B$ and a family of maps
  \begin{equation*}
    g:\prd{x:A}C(x)\to D(f(x)),
  \end{equation*}
  where $C$ is a type family over $A$, and $D$ is a type family over $B$. In this situation we also say that $g$ is a \define{family of maps over $f$}. Then we define\index{total f(g)@{$\tot[f]{g}$}}
  \begin{equation*}
    \tot[f]{g}:\sm{x:A}C(x)\to\sm{y:B}D(y)
  \end{equation*}
  by $\tot[f]{g}(x,z)\defeq (f(x),g(x,z))$.
\end{defn}

\begin{thm}\label{thm:equiv-toto}
  Suppose that $g$ is a family of maps over $f$, and suppose that $f$ is an equivalence. Then the following are equivalent:
  \begin{enumerate}
  \item The family of maps $g$ over $f$ is a family of equivalences.
  \item The map $\tot[f]{g}$ is an equivalence.
  \end{enumerate}
\end{thm}

\begin{proof}
  Note that we have a commuting triangle
  \begin{equation*}
    \begin{tikzcd}[column sep=0]
      \sm{x:A}C(x) \arrow[rr,"{\tot[f]{g}}"] \arrow[dr,swap,"\tot{g}"]& & \sm{y:B}D(y) \\
      & \sm{x:A}D(f(x)) \arrow[ur,swap,"{\lam{(x,z)}(f(x),z)}"]
    \end{tikzcd}
  \end{equation*}
  By the assumption that $f$ is an equivalence, it follows that the map $\sm{x:A}D(f(x))\to \sm{y:B}D(y)$ is an equivalence. Therefore it follows that $\tot[f]{g}$ is an equivalence if and only if $\tot{g}$ is an equivalence. Now the claim follows, since $\tot{g}$ is an equivalence if and only if $g$ if a family of equivalences.
\end{proof}
\index{family of equivalences|)}

\subsection{The fundamental theorem}

\index{identity system|(}
Many types come equipped with a reflexive relation that possesses a similar
structure as the identity type. The observational equality on the natural
numbers is such an example. We have see that it is a reflexive, symmetric, and
transitive relation, and moreover it is contained in any other reflexive
relation. Thus, it is natural to ask whether observational equality on the natural numbers is equivalent to the identity type.

The fundamental theorem of identity types (\cref{thm:id_fundamental}) is a general theorem that can be used to answer such questions. It describes a necessary and sufficient condition on a type family $B$ over a type $A$ equipped with a point $a:A$, for there to be a family of equivalences $\prd{x:A}(a=x)\simeq B(x)$. In other words, it tells us when a family $B$ is a characterization of the identity type of $A$.

Before we state the fundamental theorem of identity types we introduce the notion of \emph{identity systems}. Those are families $B$ over a $A$ that satisfy an induction principle that is similar to the path induction principle, where the `computation rule' is stated with an identification.

\begin{defn}
  Let $A$ be a type equipped with a term $a:A$. A \define{(unary) identity system} on $A$ at $a$ consists of a type family $B$ over $A$ equipped with $b:B(a)$, such that for any family of types $P(x,y)$ indexed by $x:A$ and $y:B(x)$,
  the function
  \begin{equation*}
    h\mapsto h(a,b):\Big(\prd{x:A}\prd{y:B(x)}P(x,y)\Big)\to P(a,b)
  \end{equation*}
  has a section.
\end{defn}

The most important implication in the fundamental theorem is that (ii) implies (i). Occasionally we will also use the third equivalent statement. We note that the fundamental theorem also appears as Theorem 5.8.4 in \cite{hottbook}.

\begin{thm}\label{thm:id_fundamental}
Let $A$ be a type with $a:A$, and let $B$ be be a type family over $A$ with $b:B(a)$.
Then  the following are logically equivalent for any family of maps
\begin{equation*}
  f:\prd{x:A}(a=x)\to B(x).
\end{equation*}
\begin{enumerate}
\item The family of maps $f$ is a family of equivalences.
\item The total space\index{is contractible!total space of an identity system}
\begin{equation*}
\sm{x:A}B(x)
\end{equation*}
is contractible.
\item The family $B$ is an identity system.
\end{enumerate}
In particular the canonical family of maps
\begin{equation*}
\pathind_a(b):\prd{x:A} (a=x)\to B(x)
\end{equation*}
is a family of equivalences if and only if $\sm{x:A}B(x)$ is contractible.
\end{thm}

\begin{proof}
  First we show that (i) and (ii) are equivalent.
  By \cref{thm:fib_equiv} it follows that the family of maps $f$ is a family of equivalences if and only if it induces an equivalence
  \begin{equation*}
    \eqv{\Big(\sm{x:A}a=x\Big)}{\Big(\sm{x:A}B(x)\Big)}
  \end{equation*}
  on total spaces. We have that $\sm{x:A}a=x$ is contractible. Now it follows by \cref{ex:contr_equiv}, applied in the case
  \begin{equation*}
    \begin{tikzcd}[column sep=3em]
      \sm{x:A}a=x \arrow[rr,"\tot{f}"] \arrow[dr,swap,"\eqvsym"] & & \sm{x:A}B(x) \arrow[dl] \\
      & \unit & \phantom{\sm{x:A}a=x}
    \end{tikzcd}
  \end{equation*}
  that $\tot{f}$ is an equivalence if and only if $\sm{x:A}B(x)$ is contractible.

  Now we show that (ii) and (iii) are equivalent. Note that we have the following commuting triangle
  \begin{equation*}
    \begin{tikzcd}[column sep=0]
      \prd{t:\sm{x:A}B(x)}P(t) \arrow[rr,"\evpair"] \arrow[dr,swap,"{\evpt(a,b)}"] & & \prd{x:A}\prd{y:B(x)}P(x,y) \arrow[dl,"{\lam{h}h(a,b)}"] \\
      \phantom{\prd{x:A}\prd{y:B(x)}P(x,y)} & P(a,b)
    \end{tikzcd}
  \end{equation*}
  In this diagram the top map has a section. Therefore it follows by \cref{ex:3_for_2} that the left map has a section if and only if the right map has a section. Notice that the left map has a section for all $P$ if and only if $\sm{x:A}B(x)$ satisfies singleton induction, which is by \cref{thm:contractible} equivalent to $\sm{x:A}B(x)$ being contractible.
\end{proof}
\index{identity system|)}

\subsection{Embeddings}
\index{embedding|(}
As an application of the fundamental theorem we show that equivalences are embeddings. The notion of embedding is the homotopical analogue of the set theoretic notion of injective map.

\begin{defn}
An \define{embedding} is a map $f:A\to B$\index{is an embedding} satisfying the property that\index{is an equivalence!action on paths of an embedding}
\begin{equation*}
\apfunc{f}:(\id{x}{y})\to(\id{f(x)}{f(y)})
\end{equation*}
is an equivalence for every $x,y:A$. We write $\isemb(f)$\index{is-emb(f)@{$\isemb(f)$}} for the type of witnesses that $f$ is an embedding.
\end{defn}

Another way of phrasing the following statement is that equivalent types have equivalent identity types.

\begin{thm}
\label{cor:emb_equiv} 
Any equivalence is an embedding.\index{is an embedding!equivalence}\index{equivalence!is an embedding}
\end{thm}

\begin{proof}
Let $e:\eqv{A}{B}$ be an equivalence, and let $x:A$. Our goal is to show that
\begin{equation*}
\apfunc{e} : (\id{x}{y})\to (\id{e(x)}{e(y)})
\end{equation*}
is an equivalence for every $y:A$. By \cref{thm:id_fundamental} it suffices to show that 
\begin{equation*}
\sm{y:A}e(x)=e(y)
\end{equation*}
is contractible for every $y:A$. Now observe that there is an equivalence
\begin{samepage}
\begin{align*}
\sm{y:A}e(x)=e(y) & \eqvsym \sm{y:A}e(y)=e(x) \\
& \jdeq \fib{e}{e(x)}
\end{align*}
\end{samepage}
by \cref{thm:fib_equiv}, since for each $y:A$ the map
\begin{equation*}
\invfunc : (e(x)=e(y))\to (e(y)= e(x))
\end{equation*}
is an equivalence by \cref{ex:equiv_grpd_ops}.
The fiber $\fib{e}{e(x)}$ is contractible by \cref{thm:contr_equiv}, so it follows by \cref{ex:contr_equiv} that the type $\sm{y:A}e(x)=e(y)$ is indeed contractible.
\end{proof}
\index{embedding|)}

\subsection{Disjointness of coproducts}

\index{disjointness of coproducts|(}
\index{characterization of identity type!coproduct|(}
\index{identity type!coproduct|(}
\index{coproduct!identity type|(}
\index{coproduct!disjointness|(}
To give a second application of the fundamental theorem of identity types, we characterize the identity types of coproducts. Our goal in this section is to prove the following theorem.

\begin{thm}\label{thm:id-coprod-compute}
Let $A$ and $B$ be types. Then there are equivalences
\begin{align*}
(\inl(x)=\inl(x')) & \eqvsym (x = x')\\
(\inl(x)=\inr(y')) & \eqvsym \emptyt \\
(\inr(y)=\inl(x')) & \eqvsym \emptyt \\
(\inr(y)=\inr(y')) & \eqvsym (y=y')
\end{align*}
for any $x,x':A$ and $y,y':B$.
\end{thm}

In order to prove \cref{thm:id-coprod-compute}, we first define
a binary relation $\Eqcoprod_{A,B}$ on the coproduct $A+B$.

\begin{defn}
Let $A$ and $B$ be types. We define 
\begin{equation*}
\Eqcoprod_{A,B} : (A+B)\to (A+B)\to\UU
\end{equation*}
by double induction on the coproduct, postulating
\begin{align*}
\Eqcoprod_{A,B}(\inl(x),\inl(x')) & \defeq (x=x') \\
\Eqcoprod_{A,B}(\inl(x),\inr(y')) & \defeq \emptyt \\
\Eqcoprod_{A,B}(\inr(y),\inl(x')) & \defeq \emptyt \\
\Eqcoprod_{A,B}(\inr(y),\inr(y')) & \defeq (y=y')
\end{align*}
The relation $\Eqcoprod_{A,B}$ is also called the \define{observational equality of coproducts}\index{observational equality!of coproducts}.
\end{defn}

\begin{lem}
The observational equality relation $\Eqcoprod_{A,B}$ on $A+B$ is reflexive, and therefore there is a map
\begin{equation*}
\Eqcoprodeq:\prd{s,t:A+B} (s=t)\to \Eqcoprod_{A,B}(s,t)
\end{equation*}
\end{lem}

\begin{constr}
The reflexivity term $\rho$ is constructed by induction on $t:A+B$, using
\begin{align*}
\rho(\inl(x))\defeq \refl{\inl(x)}  & : \Eqcoprod_{A,B}(\inl(x)) \\
\rho(\inr(y))\defeq \refl{\inr(y)} & : \Eqcoprod_{A,B}(\inr(y)).\qedhere
\end{align*}
\end{constr}

To show that $\Eqcoprodeq$ is a family of equivalences, we will use the fundamental theorem, \cref{thm:id_fundamental}. Moreover, we will use the functoriality of coproducts (established in \cref{ex:coproduct_functor}), and the fact that any total space over a coproduct is again a coproduct:
\begin{align*}
\sm{t:A+B}P(t) & \eqvsym \Big(\sm{x:A}P(\inl(x))\Big)+\Big(\sm{y:B}P(\inr(y))\Big)
\end{align*}
All of these equivalences are straightforward to construct, so we leave them as an exercise to the reader. 

\begin{lem}\label{lem:is-contr-total-eq-coprod}
For any $s:A+B$ the total space
\begin{equation*}
\sm{t:A+B}\Eqcoprod_{A,B}(s,t)
\end{equation*}
is contractible.
\end{lem}

\begin{proof}
We will do the proof by induction on $s$. The two cases are similar, so we only show that the total space
\begin{equation*}
\sm{t:A+B}\Eqcoprod_{A,B}(\inl(x),t)
\end{equation*}
is contractible. Note that we have equivalences
\begin{samepage}
\begin{align*}
& \sm{t:A+B}\Eqcoprod_{A,B}(\inl(x),t) \\
& \eqvsym \Big(\sm{x':A}\Eqcoprod_{A,B}(\inl(x),\inl(x'))\Big)+\Big(\sm{y':B}\Eqcoprod_{A,B}(\inl(x),\inr(y'))\Big) \\
& \eqvsym \Big(\sm{x':A}x=x'\Big)+\Big(\sm{y':B}\emptyt\Big) \\
& \eqvsym \Big(\sm{x':A}x=x'\Big)+\emptyt \\
& \eqvsym \sm{x':A}x=x'.
\end{align*}%
\end{samepage}%
In the last two equivalences we used \cref{ex:unit-laws-coprod}. This shows that the total space is contractible, since the latter type is contractible by \cref{thm:total_path}.
\end{proof}

\begin{proof}[Proof of \cref{thm:id-coprod-compute}]
The proof is now concluded with an application of \cref{thm:id_fundamental}, using \cref{lem:is-contr-total-eq-coprod}.
\end{proof}
\index{disjointness of coproducts|)}
\index{characterization of identity type!coproduct|)}
\index{identity type!coproduct|)}
\index{coproduct!identity type|)}
\index{coproduct!disjointness|)}

\begin{exercises}
  \exercise
  \begin{subexenum}
  \item \label{ex:is-emb-empty}Show that the map $\emptyt\to A$ is an embedding for every type $A$.\index{is an embedding!0 to A@{$\emptyt\to A$}}
  \item \label{ex:is-emb-inl-inr}Show that $\inl:A\to A+B$ and $\inr:B\to A+B$ are embeddings for any two types $A$ and $B$.
    \index{is an embedding!inl (for coproducts)@{$\inl$ (for coproducts)}}
    \index{is an embedding!inr (for coproducts)@{$\inr$ (for coproducts)}}
    \index{inl@{$\inl$}!is an embedding}
    \index{inr@{$\inr$}!is an embedding}
  \end{subexenum}
  \exercise Consider an equivalence $e:A\simeq B$. Construct an equivalence
  \begin{equation*}
    (e(x)=y)\simeq(x=e^{-1}(y))
  \end{equation*}
  for every $x:A$ and $y:B$.
  \exercise Show that\index{embedding!closed under homotopies}
  \begin{equation*}
    (f\htpy g)\to (\isemb(f)\leftrightarrow\isemb(g))
  \end{equation*}
  for any $f,g:A\to B$.
  \exercise \label{ex:emb_triangle}Consider a commuting triangle
  \begin{equation*}
    \begin{tikzcd}[column sep=tiny]
      A \arrow[rr,"h"] \arrow[dr,swap,"f"] & & B \arrow[dl,"g"] \\
      & X
    \end{tikzcd}
  \end{equation*}
  with $H:f\htpy g\circ h$. 
  \begin{subexenum}
  \item Suppose that $g$ is an embedding. Show that $f$ is an embedding if and only if $h$ is an embedding.\index{is an embedding!composite of embeddings}\index{is an embedding!right factor of embedding if left factor is an embedding}
  \item Suppose that $h$ is an equivalence. Show that $f$ is an embedding if and only if $g$ is an embedding.\index{is an embedding!left factor of embedding if right factor is an equivalence}
  \end{subexenum}
  \exercise \label{ex:is-equiv-is-equiv-functor-coprod}Consider two maps $f:A\to A'$ and $g:B \to B'$.
  \begin{subexenum}
  \item Show that if the map
    \begin{equation*}
      f+g:(A+B)\to (A'+B')
    \end{equation*}
    is an equivalence, then so are both $f$ and $g$ (this is the converse of \cref{ex:coproduct_functor_equivalence}).
  \item \label{ex:is-emb-coprod}Show that $f+g$ is an embedding if and only if both $f$ and $g$ are embeddings.
  \end{subexenum}
  \exercise \label{ex:htpy_total} 
  \begin{subexenum}
  \item Let $f,g:\prd{x:A}B(x)\to C(x)$ be two families of maps. Show that
    \begin{equation*}
      \Big(\prd{x:A}f(x)\htpy g(x)\Big)\to \Big(\tot{f}\htpy \tot{g}\Big). 
    \end{equation*}
  \item Let $f:\prd{x:A}B(x)\to C(x)$ and let $g:\prd{x:A}C(x)\to D(x)$. Show that
    \begin{equation*}
      \tot{\lam{x}g(x)\circ f(x)}\htpy \tot{g}\circ\tot{f}.
    \end{equation*}
  \item For any family $B$ over $A$, show that
    \begin{equation*}
      \tot{\lam{x}\idfunc[B(x)]}\htpy\idfunc.
    \end{equation*}
  \end{subexenum}
  \exercise \label{ex:id_fundamental_retr}Let $a:A$, and let $B$ be a type family over $A$. 
  \begin{subexenum}
  \item Use \cref{ex:htpy_total,ex:contr_retr} to show that if each $B(x)$ is a retract of $\id{a}{x}$, then $B(x)$ is equivalent to $\id{a}{x}$ for every $x:A$.
    \index{fundamental theorem of identity types!formulation with retractions}
  \item Conclude that for any family of maps
    \index{fundamental theorem of identity types!formulation with sections}
    \begin{equation*}
      f : \prd{x:A} (a=x) \to B(x),
    \end{equation*}
    if each $f(x)$ has a section, then $f$ is a family of equivalences.
  \end{subexenum}
  \exercise Use \cref{ex:id_fundamental_retr} to show that for any map $f:A\to B$, if
  \begin{equation*}
    \apfunc{f} : (x=y) \to (f(x)=f(y))
  \end{equation*}
  has a section for each $x,y:A$, then $f$ is an embedding.\index{is an embedding!if the action on paths have sections}
  \exercise \label{ex:path-split}We say that a map $f:A\to B$ is \define{path-split}\index{path-split} if $f$ has a section, and for each $x,y:A$ the map
  \begin{equation*}
    \apfunc{f}(x,y):(x=y)\to (f(x)=f(y))
  \end{equation*}
  also has a section. We write $\pathsplit(f)$\index{path-split(f)@{$\pathsplit(f)$}} for the type
  \begin{equation*}
    \sections(f)\times\prd{x,y:A}\sections(\apfunc{f}(x,y)).
  \end{equation*}
  Show that for any map $f:A\to B$ the following are equivalent:
  \begin{enumerate}
  \item The map $f$ is an equivalence.
  \item The map $f$ is path-split.
  \end{enumerate}
  \exercise \label{ex:fiber_trans}Consider a triangle
  \begin{equation*}
    \begin{tikzcd}[column sep=small]
      A \arrow[rr,"h"] \arrow[dr,swap,"f"] & & B \arrow[dl,"g"] \\
      & X
    \end{tikzcd}
  \end{equation*}
  with a homotopy $H:f\htpy g\circ h$ witnessing that the triangle commutes. 
  \begin{subexenum}
  \item Construct a family of maps
    \begin{equation*}
      \fibtriangle(h,H):\prd{x:X}\fib{f}{x}\to\fib{g}{x},
    \end{equation*}
    for which the square
    \begin{equation*}
      \begin{tikzcd}[column sep=8em]
        \sm{x:X}\fib{f}{x} \arrow[r,"\tot{\fibtriangle(h,H)}"] \arrow[d] & \sm{x:X}\fib{g}{x} \arrow[d] \\
        A \arrow[r,swap,"h"] & B
      \end{tikzcd}
    \end{equation*}
    commutes, where the vertical maps are as constructed in \cref{ex:fib_replacement}.
  \item Show that $h$ is an equivalence if and only if $\fibtriangle(h,H)$ is a family of equivalences.
  \end{subexenum}
\end{exercises}
\index{fundamental theorem of identity types|)}
\index{characterization of identity type!fundamental theorem of identity types|)}

\endinput

  \begin{comment}
    \exercise \label{ex:eqv_sigma_mv}Consider a map
    \begin{equation*}
      f:A \to \sm{y:B}C(y).
    \end{equation*}
    \begin{subexenum}
    \item Construct a family of maps
      \begin{equation*}
        f':\prd{y:B} \fib{\proj 1\circ f}{y}\to C(y).
      \end{equation*}
    \item Construct an equivalence
      \begin{equation*}
        \eqv{\fib{f'(b)}{c}}{\fib{f}{(b,c)}}
      \end{equation*}
      for every $(b,c):\sm{y:B}C(y)$.
    \item Conclude that the following are equivalent:
      \begin{enumerate}
      \item $f$ is an equivalence.
      \item $f'$ is a family of equivalences.
      \end{enumerate}
    \end{subexenum}
    \exercise \label{ex:coh_intro}Consider a type $A$ with base point $a:A$, and let $B$ be a type family on $A$ that implies the identity type, i.e., there is a term
    \begin{equation*}
      \alpha : \prd{x:A} B(x)\to (a=x).
    \end{equation*}
    Show that the \define{coherence reduction map}
    \begin{equation*}
      \cohreduction : \Big(\sm{y:B(a)}\alpha(a,y)=\refl{a}\Big) \to \Big(\sm{x:A}B(x)\Big)
    \end{equation*}
    defined by $\lam{(y,q)}(a,y)$ is an equivalence.
  \end{comment}


\chapter{The hierarchy of homotopical complexity}
\chaptermark{Homotopical complexity}
%Not all types have interesting higher groupoid structure. For example, we will see below that two natural numbers can only be equal in at most one way. Voevodsky articulated a useful notion to detect the homotopical complexity of types, which allows us to distinguish between contractible types (also called \emph{$(-2)$-types}), \emph{propositions} (also called \emph{$(-1)$-types}), \emph{sets} (\emph{$0$-types}), and \emph{$k$-types} for higher $k$.

%We will see [later] that there are types that are not $k$-types for any $k$.

\section{Propositions and subtypes}

\begin{defn}
A type $A$ is said to be a \define{proposition} if there is a term of type
\begin{equation*}
\isprop(A)\defeq\prd{x,y:A}\iscontr(x=y).
\end{equation*}
\end{defn}

In the following lemma we prove that in order to show that a type $A$ is a proposition, it suffices to show that any two terms of $A$ are equal.

\begin{lem}\label{lem:isprop_eq}
Let $A$ be a type. Then we have
\begin{equation*}
\isprop(A)\leftrightarrow \Big(\prd{x,y:A}x=y\Big).
\end{equation*}
\end{lem}

\begin{proof}
Suppose $A$ is a proposition. By taking the center of contraction of $\id{x}{y}$ for each $x,y:A$ we obtain a term of type $\prd{x,y:A}\id{x}{y}$.

Now suppose that $A$ is a type equipped with $H:\prd{x,y:A}\id{x}{y}$. Then we take $\ct{H(x,x)^{-1}}{H(x,y)}$ as the center of contraction of $\id{x}{y}$. To construct the contraction
\begin{equation*}
\prd{p:\id{x}{y}} \ct{H(x,x)^{-1}}{H(x,y)}=p
\end{equation*}
we proceed by path induction. Our goal is to show that
\begin{equation*}
\ct{H(x,x)^{-1}}{H(x,x)}=\refl{x}.\qedhere
\end{equation*}
\end{proof}

\begin{cor}
For any proposition $P$ we have $P\to\iscontr(P)$.
\end{cor}

\begin{thm}
Let $A$ be a type and let $P:A\to\type$ be a family of propositions, in the sense that each $P(x)$ is a proposition. Furthermore, consider $\pairr{x,p},\pairr{y,q}:\sm{x:A}P(x)$. Then the map
\begin{equation*}
\mathsf{subtype\usc{}eq}(x,y) : (\id{\pairr{x,p}}{\pairr{y,q}})\to(\id{x}{y})
\end{equation*}
defined by path induction by sending $\refl{\pairr{x,p}}$ to $\refl{x}$, is an equivalence.
\end{thm}

\begin{proof}
By \autoref{thm:id_fundamental} it suffices to show that
\begin{equation*}
\sm{y:A}P(y)\times(\id{x}{y})
\end{equation*}
is contractible, for any $x:A$ and $p:P(x)$. The center of contraction is taken to be $\pairr{x,\pairr{p,\refl{x}}}$. To construct the contraction, observe that for any $y:A$, $q:P(y)$ and $r:\id{x}{y}$, the type $P(y)$ is contractible, so we have an equivalence $\eqv{P(y)\times(\id{x}{y})}{(\id{x}{y})}$. Therefore it suffices to show that $\sm{y:A}\id{x}{y}$ is contractible, which it is.
\end{proof}

Since the equality of $\sm{x:A}P(x)$ corresponds to the equality of $A$, we call $\sm{x:A}P(x)$ a \define{subtype} of $A$.

\section{Sets}

\begin{defn}
A type $A$ is said to be a \define{set} if there is a term of type
\begin{equation*}
\isset(A)\defeq \prd{x,y:A}\isprop(\id{x}{y}).
\end{equation*}
\end{defn}

\begin{lem}
A type $A$ is a set if and only if it satisfies \define{axiom K}, which asserts that
\begin{equation*}
\prd{x:A}{p:\id{x}{x}}\id{\refl{x}}{p}.
\end{equation*}
\end{lem}

\begin{proof}
If $A$ is a set, then $\id{x}{x}$ is a proposition, so any two of its elements are equal. 
This implies axiom $K$. 

For the converse, if $A$ satisfies axiom $K$, then for any $p,q:\id{x}{y}$ we have $\id{\ct{p}{q^{-1}}}{\refl{x}}$, and hence $\id{p}{q}$. This shows that $A$ is a proposition.
\end{proof}

\begin{lem}\label{lem:prop_to_id}
Let $A$ be a type, and let $R:A\to A\to\UU$ be a binary relation on $A$ satisfying
\begin{enumerate}
\item Each $R(x,y)$ is a proposition,
\item $R$ is reflexive, as witnessed by $\rho:\prd{x:A}R(x,x)$.
\end{enumerate}
Then any fiberwise map
\begin{equation*}
\prd{x,y:A}R(x,y)\to (\id{x}{y})
\end{equation*}
is a fiberwise equivalence. Consequently, if there is such a fiberwise map, then $A$ is a set.
\end{lem}

\begin{proof}
Let $f:\prd{x,y:A}R(x,y)\to(\id{x}{y})$. 
Since $R$ is assumed to be reflexive, we also have a fiberwise transformation
\begin{equation*}
\rec{x=}(\rho(x)):\prd{y:A}(\id{x}{y})\to R(x,y).
\end{equation*}
Since each $R(x,y)$ is assumed to be a proposition, it therefore follows that each $R(x,y)$ is a retract of $\id{x}{y}$. We conclude by \autoref{ex:id_fundamental_retr} that for each $x,y:A$, the map $f(x,y):R(x,y)\to(\id{x}{y})$ must be an equivalence. 
\end{proof}

\begin{thm}
The type of natural numbers is a set.
\end{thm}

\begin{proof}
Let $E:\nat\to\nat\to\type$ be the binary relation given by
\begin{align*}
E(0,0) & \defeq \unit & E(S(m),0) & \defeq \emptyt \\
E(0,S(n)) & \defeq \emptyt & E(S(m),S(n)) & \defeq E(m,n).
\end{align*}
Note that this relation is reflexive, and that $E(m,n)$ is a proposition for all $m,n:\nat$, since $\emptyt$ and $\unit$ are propositions.

Thus, it remains to show that
\begin{equation*}
\prd{m,n:\nat}E(m,n)\to (\id{m}{n}).
\end{equation*}
We proceed by induction om $m$. For the base case, we have to show that
\begin{equation*}
\prd{n:\nat}E(0,n)\to (\id{m}{n}).
\end{equation*}
We proceed by induction on $n:\nat$. In the base case we take $\lam{t}\refl{0}$. For the inductive step we just take the recursor $\rec{\emptyt}$ on the empty type. This completes the construction in the case $m\jdeq 0$.

For the induction step, the induction hypothesis is
\begin{equation*}
e:\prd{n:\nat}E(m,n)\to (\id{m}{n}).
\end{equation*}
Our goal is to show that
\begin{equation*}
\prd{n:\nat}E(S(m),n)\to (\id{S(m)}{n}).
\end{equation*}
We proceed by induction on $n:\nat$. In the base case we take the recursor $\rec{\emptyt}$. 
For the inductive step our goal is to show that $E(S(m),S(n))\to (\id{S(m)}{S(n)})$. Let $p:E(S(m),S(n))$. Since $E(S(m),S(n))\jdeq E(m,n)$, we obtain a term of type $\id{m}{n}$. Now we simply apply $\mapfunc{S}:(\id{m}{n})\to(\id{S(m)}{S(n)})$.
\end{proof}

\begin{comment}
\begin{thm}[Hedberg]\label{thm:dec_eq}
Any type with decidable equality is a set.
\end{thm}

\begin{proof}
Let $A$ be a type, and let $d:\prd{x,y:A}(\id{x}{y})+\neg(\id{x}{y})$ be the witness that $A$ has decidable equality.
We first construct a reflexive binary relation $E:A\to A\to\type$ such that each $E(x,y)$ is a proposition.
For every $x,y:A$, we first define a type family $E'(x,y):((\id{x}{y})+\neg(\id{x}{y}))\to\type$ by
\begin{align*}
E'(x,y,\inl(p)) & \defeq \unit \\
E'(x,y,\inr(p)) & \defeq \emptyt.
\end{align*}
Note that $E'(x,y,q)$ is a proposition for each $x,y:A$ and $q:(\id{x}{y})+\neg(\id{x}{y})$. 
Now we set $E(x,y)\defeq E'(x,y,d(x,y))$. Then $E$ is clearly reflexive, and a family of propositions.
Therefore it remains to show that $E$ implies identity. 

Since $E$ is defined as an instance of $E'$, it suffices to construct a term of type
\begin{equation*}
\prd{x,y:A}{q:(\id{x}{y})+\neg(\id{x}{y})} E'(q)\to (\id{x}{y}). 
\end{equation*}
By induction of disjoint sums, it suffices to construct terms of types
\begin{align*}
& \prd{x,y:A}{p:\id{x}{y}} \unit\to (\id{x}{y}) \\
& \prd{x,y:A}{p:\neg(\id{x}{y})} \emptyt\to (\id{x}{y}).
\end{align*}
In the first case, we take $\lam{x}{y}{p}{t}p$, and the second case is by induction on the empty type.
\end{proof}
\end{comment}

\section{General truncation levels}
\begin{defn}
We define $\istrunc{} : \Z_{\geq-2}\to\UU\to\UU$ by induction on $k:\Z_{\geq -2}$, taking
\begin{align*}
\istrunc{-2}(A) & \defeq \iscontr(A) \\
\istrunc{k+1}(A) & \defeq \prd{x,y:A}\istrunc{k}(\id{x}{y}).\qedhere
\end{align*}
For any type $A$, we say that $A$ is \define{$k$-truncated}, or a \define{$k$-type}, if there is a term of type $\istrunc{k}(A)$. We say that a map $f:A\to B$ is $k$-truncated if its fibers are $k$-truncated.
\end{defn}

%For the rest of this section, let $k:\Z_{\geq-2}$.

\begin{thm}
If $A$ is a $k$-type, then $A$ is also a $(k+1)$-type.
\end{thm}

\begin{proof}
If $A$ is contractible with center of contraction $c$, and contraction $C$, then we have
\begin{equation*}
\lam{x}{y} \ct{C(x)}{C(y)^{-1}} : \prd{x,y:A}\id{x}{y}.
\end{equation*}
By \autoref{lem:isprop_eq} this shows that $A$ is a proposition. If any $k$-type is also a $(k+1)$-type, then any $(k+1)$-type is a $(k+2)$-type, since its identity types are $k$-types and therefore $(k+1)$-types.
\end{proof}

\begin{thm}\label{thm:ntype_eqv}
If $e:\eqv{A}{B}$ is an equivalence, and $B$ is a $k$-type, then so is $A$.
\end{thm}

\begin{proof}
We have seen in \autoref{ex:contr_equiv} that if $B$ is contractible and $e:\eqv{A}{B}$ is an equivalence, then $A$ is also contractible. This proves the base case.

For the inductive step, assume that the $k$-types are stable under equivalences, and consider $e:\eqv{A}{B}$ where $B$ is a $(k+1)$-type. In \autoref{ex:emb_equiv} we have seen that
\begin{equation*}
\apfunc{e}:(\id{x}{y})\to(\id{e(x)}{e(y)})
\end{equation*}
is an equivalence for any $x,y$. Note that $\id{e(x)}{e(y)}$ is a $k$-type, so by the induction hypothesis it follows that $\id{x}{y}$ is a $k$-type. This proves that $A$ is a $(k+1)$-type.
\end{proof}

\begin{comment}
\begin{proof}
By \autoref{ex:contr_retr} it follows that if $A$ is a retract of a contractible type, then $A$ is contractible.
For the inductive step, suppose that the $k$-types are closed under retracts, and consider a section-retraction pair
\begin{equation*}
\begin{tikzcd}
A \arrow[r,"i"] & B \arrow[r,"r"] & A,
\end{tikzcd}
\end{equation*}
with $H:r\circ i\htpy \idfunc$, where $B$ is a $(k+1)$-type.
By the induction hypothesis it suffices to show that for any $x,y:A$, the function $\apfunc{i}:(\id{x}{y})\to (\id{i(x)}{i(y)})$ has a retraction.
The retraction $\varphi:(\id{i(x)}{i(y)})\to(\id{x}{y})$ is defined as
\begin{equation*}
\varphi \defeq \lam{q} \ct{H(x)^{-1}}{\ap{r}{q}}{H(y)}
\end{equation*}
To see that $\varphi(\ap{i}{p})=p$, we have to show that the square
\begin{equation*}
\begin{tikzcd}
r(i(x)) \arrow[d,equals,swap,"\ap{r}{q}"] \arrow[r,equals,"H(x)"] & x \arrow[d,equals,"p"] \\
r(i(y)) \arrow[r,equals,swap,"H(y)"] & y
\end{tikzcd}
\end{equation*}
commutes. This square commutes by the naturality of homotopies, proven in \autoref{ex:htpy_nat}.
\end{proof}
\end{comment}

\begin{exercises}
\item For any proposition $P$, show that $P\to\iscontr(P)$.
\item Let $P$ and $Q$ be propositions. Show that
\begin{equation*}
\eqv{(P\leftrightarrow Q)}{(\eqv{P}{Q})}.
\end{equation*}
Conclude that any two contractible types are equivalent.
\item Let $A$ be a type, and let $\delta_A:A\to A\times A$ be given by $\lam{x}(x,x)$. 
\begin{subexenum}
\item Show that
\begin{equation*}
\eqv{\isequiv(\delta_A)}{\isprop(A)}.
\end{equation*}
\item Show that $A$ is $(k+1)$-truncated if and only if $\delta_A:A\to A\times A$ is $k$-truncated.
\end{subexenum}
\item Let $P:A\to\type$ be a type family. Show that for any $n\geq-2$, if $A$ is an $n$-type, and $P(x)$ is an $n$-type for each $x:A$, then so is $\sm{x:A}P(x)$. 
\item Show that $\bool$ is a set.
\item Show that for any two sets $A$ and $B$, the disjoint sum $A+B$ is again a set.
\item \label{ex:hedberg}(Hedberg's theorem) A type $A$ is said to have \define{decidable equality} if there is a term of type
\begin{equation*}
\prd{x,y:A} (\id{x}{y})+\neg(\id{x}{y}).
\end{equation*}
For any type $A$, and every $x,y:A$, consider the type family $D(x,y):((\id{x}{y})+\neg(\id{x}{y}))\to\type$ given by
\begin{align*}
D(x,y,\inl(p)) & \defeq \unit \\
D(x,y,\inr(p)) & \defeq \emptyt.
\end{align*}
Use $D$ to show that any type with decidable equality is a set.\footnote{By following this suggestion one avoids the use of function extensionality, which is used in Theorem 7.2.5 of \cite{hottbook} to conclude that $\neg\neg(\id{x}{y})$ is a proposition.}
\item Show that $\nat$ and $\bool$ have decidable equality, as defined in \autoref{ex:hedberg}.
\item Show that if $A$ and $B$ have decidable equality, then so do $A+B$ and $A\times B$.
\item 
\begin{subexenum}
\item Let $B:A\to\type$ be a type family over $A$. Show that iff $A$ is a $k$-type, and if $B(x)$ is a $k$-type for each $x:A$, then so is $\sm{x:A}B(x)$.
\item Show that for $k\geq -1$, any subtype of a $k$-type is again a $k$-type.
\item Show that for any map $f:A\to B$ from a $k$-type $A$ to a $(k+1)$-type $B$, the fibers of $f$ are also $k$-types.
\end{subexenum}
\item Use \autoref{ex:contr_retr,ex:retr_id} to show that if $A$ is a retract of a $k$-type $B$, then $A$ is also a $k$-type.
\end{exercises}


\section{Elementary number theory}

One of the things type theory is great for, is for the formalization of mathematics in a computer proof assistant. Those are programs that can compile any type theoretical construction to check that this construction indeed has the type it was claimed it has.

At this point in our development of type theory there are two areas of mathematics that would be natural to try to do in type theory: discrete mathematics and elementary number theory. Indeed, how does one define in type theory the greatest common divisor of two natural numbers, or how does one show that there are infinitely many primes? How does one even formalize that every non-empty subset of the natural numbers has a least element?

To answer these questions we will run into questions of decidability. How do we write a term that decides wheter a number is prime or not? Or indeed, is it even true that every non-empty subset of the natural numbers has a least element? What about the subset of $\N$ that contains $1$, and it contains $0$ if and only if Goldbach's conjecture holds? Finding the least element of this subset is equivalent to settling the conjecture!

Therefore, we will prove the well-foundedness of the natural numbers for decidable subsets of $\N$. In fact, we will show it for decidable families, because sometimes we don't know in advance whether a family of types is in fact a subtype. A consequence of involving decidability in the well-foundedness of the natural numbers is that for many properties one has to prove that they are decidable. Luckily this is the case: many of the familiar properties that one encounters in number theory are indeed decidable.

\subsection{Decidability}

A common way of reasoning in mathematics is via a proof by contradiction: ``in order to show that $P$ holds we show that it cannot be the case that $P$ doesn't hold". There are no inference rules in type theory that allow us to obtain a term of type $P$ from a term of type $\neg\neg P$. However, for some propositions $P$ one can construct a function $\neg\neg P \to P$. The \emph{decidable propositions} from a class of such propositions $P$ for which we can show $\neg\neg P \to P$.

The following definition of decidability is made for general types, even though we will mostly be interested in the decidabilyt of proposition. The reason will become aparent in a moment, when we show that types with decidable equality are sets. This useful theorem would become trivial if we restricted the notion of decidability to propositions.

\begin{defn}
  A type $A$ is said to be decidable if it comes equipped with a term of type
  \begin{equation*}
    \isdecidable(A)\defeq A+\neg A.
  \end{equation*}
  Decidable propositions are called \define{classical}. We will write
  \begin{equation*}
    \classicalprop_\UU \defeq \sm{P:\prop_\UU}\isdecidable(P)
  \end{equation*}
  for the type of all classical propositions (with respect to a universe $\UU$).
\end{defn}

\begin{eg}\label{eg:classical-prop}
  The types $\unit$ and $\emptyt$ are decidable. Indeed, we have
  \begin{align*}
    \decunit & \defeq \inl(\ttt) & & :\isdecidable(\unit) \\
    \decemptyt & \defeq \inr(\idfunc) & & : \isdecidable(\emptyt).\qedhere
  \end{align*}
  Any type $A$ equipped with a point $a:A$ is decidable.
\end{eg}

Since $P$ and $\neg P$ are mutually exclusive cases, it follows that $\isdecidable(P)$ is a proposition. Therefore we see that the type of decidable propositions in a universe $\UU$ form a subtype of the type of all propositions in $\UU$.

\begin{lem}\label{lem:isprop-isdecidable}
  For any proposition $P$, the type $\isdecidable(P)$ is a proposition.
\end{lem}

\begin{proof}
  By \cref{lem:isprop_eq} it suffices to show that
  \begin{equation*}
    \prd{t,t':\isdecidable(P)}t=t.
  \end{equation*}
  We proceed by case analysis on $t$ and $t'$. We have four cases to consider:
  \begin{align*}
    \inl(p) & =\inl(p') & \inr(f) & =\inl(p') \\
    \inl(p) & =\inr(f') & \inr(f) & =\inr(f').
  \end{align*}
  We construct these four identifications as follows:
  \begin{enumerate}
  \item First, we want to show that $\inl(p)=\inl(p')$ for any $p,p':P$. We obtain this identification from the fact that $p=p'$, which we have because $P$ is assumed to be a proposition.
  \item Next, we want to show that $\inl(p)=\inr(f')$ for any $p:P$ and $f':\neg P$. Since we have contradictory assumptions, we obtain $f'(p):\emptyt$. We now obtain the desired identification by applying the function $\emptyt \to (\inl(p)=\inr(f')$.
  \item The construction of an identification $\inr(f)=\inl(p')$ for $f:\neg P$ and $p':P$ is similar. We have $f(p'):\emptyt$, which gives the desired identification via the function $\emptyt\to (\inr(f)=\inl(p'))$.
  \item Finally, we want to show that $\inr(f)=\inr(f')$ for $f,f':\neg P$. The type $\neg P$ is a proposition, so we have an identification $f=f'$ from which we obtain $\inr(f)=\inr(f')$.\qedhere
  \end{enumerate}
\end{proof}

We have seen in \cref{thm:propositional-extensionality} that the univalence axiom implies propositional extensionality. Recall that propositional extensionality is the property that the map
\begin{equation*}
  (P=Q)\to (P\leftrightarrow Q)
\end{equation*}
is an equivalence. We will use this fact here to conclude that $\classicalprop_\UU$ is equivalent to $\bool$.

\begin{prp}
  The type of classical propositions in any universe $\UU$ is equivalent to $\bool$.%
  \index{classical-Prop_U@{$\classicalprop_\UU$}!classical-Prop_U bool@{$\classicalprop_\UU\eqvsym\bool$}}
\end{prp}

\begin{proof}
  Since the empty type and the unit type are decidable propositions, we have a map $\varphi:\bool\to\classicalprop_\UU$ defined by
  \begin{align*}
    \varphi(\btrue) & \defeq (\unit,\decunit) \\
    \varphi(\bfalse) & \defeq (\emptyt,\decemptyt).
  \end{align*}
  Next, we define a map $\psi:\classicalprop_\UU\to\bool$. Let $P$ be a proposition that comes equipped with a term $t:P+\neg P$. To define a boolean, we proceed by case analysis on $t$. The map $\psi$ is thus defined by
  \begin{align*}
    \psi(P,\inl(p)) & \defeq \btrue \\
    \psi(P,\inr(f)) & \defeq \bfalse.
  \end{align*}
  To see that $\psi$ is an inverse of $\varphi$, note that
  \begin{align*}
    \varphi(\psi(P,\inl(p))) & \jdeq (\unit,\decunit) & \psi(\varphi(\btrue)) & \jdeq \btrue \\
    \varphi(\psi(P,\inr(f))) & \jdeq (\emptyt,\decemptyt) & \psi(\varphi(\bfalse)) & \jdeq \bfalse.
  \end{align*}
  It is therefore immediate that $\psi$ is a retract of $\varphi$. However, in order to show that $\psi$ is a section of $\varphi$ we still need to show that
  \begin{align*}
    (\unit,\decunit) & = (P,\inl(p)) \\
    (\emptyt,\decemptyt) & = (P,\inr(f)).
  \end{align*}
  Since $\isdecidable(P)$ is shown to be a proposition in \cref{lem:isprop-isdecidable}, it suffices to show that
  \begin{align*}
    \unit & = P & & \text{if we have }p:P \\
    \emptyt & = P & & \text{if we have }f:\neg P.
  \end{align*}
  In both cases we proceed by propositional extensionality. Therefore we obtain the desired identifications by observing that
  \begin{align*}
    \unit & \leftrightarrow P & & \text{if we have }p:P \\
    \emptyt & \leftrightarrow P & & \text{if we have }f:\neg P.\qedhere
  \end{align*}
\end{proof}

We will now study the concept of decidable equality.

\begin{defn}
  We say that a type $A$ has \define{decidable equality} if the identity type $x=y$ is decidable for every $x,y:A$. Types with decidable equality are also called \define{discrete}.
\end{defn}

\begin{lem}
  For each $m,n:\N$, the types $\EqN(m,n)$, $m\leq n$ and $m<n$ are decidable.
\end{lem}

\begin{proof}
  The proofs in each of the three cases is similar, so we only show that $\EqN(m,n)$ is decidable for each $m,n:\N$. This is done by induction on $m$ and $n$. Note that the types
  \begin{align*}
    \EqN(\zeroN,\zeroN) & \jdeq \unit \\
    \EqN(\zeroN,\succN(n)) & \jdeq \emptyt \\
    \EqN(\succN(m),\zeroN) & \jdeq \emptyt 
  \end{align*}
  are all decidable. Moreover, the type $\EqN(\succN(m),\succN(n))\jdeq \EqN(m,n)$ is decidable by the inductive hypothesis.
\end{proof}

\begin{cor}
  Equality on the natural numbers is decidable.
\end{cor}

\begin{proof}
  Recall from the proof of \cref{thm:eq_nat} that the canonical map
  \begin{equation*}
    (m=n)\simeq \EqN(m,n)
  \end{equation*}
  is an equivalence. Thus we obtain that $(m=n)$ is decidable from the fact that $\EqN(m,n)$ is decidable.
\end{proof}

\begin{comment}
\begin{lem}
  Suppose that $A$ and $B$ are types with decidable equality. Then the coproduct $A+B$ also has decidable equality.
\end{lem}

\begin{proof}
  Our goal is to construct a dependent function
  \begin{equation*}
    d_{A+B} : \prd{z,z':A+B}\isdecidable(z=z').
  \end{equation*}
  This function is constructed by coproduct induction on both $z$ and $z'$, so we have four cases to consider. Recall from \cref{thm:id-coprod-compute} that we have equivalences
  \begin{align*}
    (\inl(x)=\inl(x')) & \simeq (x=x') \\
    (\inl(x)=\inr(y')) & \simeq \emptyt \\
    (\inr(y)=\inl(x')) & \simeq \emptyt \\
    (\inr(y)=\inr(y')) & \simeq (y=y').
  \end{align*}
  Therefore the type $z=z'$ is equivalent to a decidable type in each of the four cases.
\end{proof}

\begin{cor}
  The type $\Z$ has decidable equality.
\end{cor}

\begin{cor}
  For any $n:\N$ the type $\Fin(n)$ has decidable equality. 
\end{cor}
\end{comment}

We have already shown in \cref{thm:eq_nat} that the type of natural numbers is a set. In fact, any type with decidable equality is a set. This fact is known as Hedberg's theorem.

\begin{thm}[Hedberg]
  Any type with decidable equality is a set.
\end{thm}

\begin{proof}
  Let $A$ be a type, and let
  \begin{equation*}
    d:\prd{x,y:A}(x=y)+\neg(x=y).
  \end{equation*}
  Recall from \cref{ex:dne-dec} that $(A+\neg A)\to (\neg\neg A\to A)$ for any type $A$, so we obtain that
  \begin{equation*}
    \prd{x,y:A}\neg\neg(x=y)\to (x=y).
  \end{equation*}
  Now observe that $\neg\neg(x=y)$ is a proposition for each $x,y:A$, and that the relation $x,y\mapsto\neg\neg(x=y)$ is reflexive. Therefore we are in position to apply \cref{lem:prop_to_id} and we conclude that $A$ is a set.
\end{proof}

\subsection{The well-ordering principle for decidable families over \texorpdfstring{$\N$}{ℕ}}

\begin{defn}
  A family $P$ over a type $A$ is said to be decidable if $P(x)$ is decidable for every $x:A$. A \define{decidable subset} of a type $A$ is a map
  \begin{equation*}
    P:A\to\classicalprop.
  \end{equation*}
\end{defn}

\begin{defn}
  Let $P$ be a decidable family over $\N$, and let $n:\N$ be a natural number equipped with $p:P(n)$. We say that $n$ is a \define{minimal $P$-element} if it comes equipped with a term of type
  \begin{equation*}
    \isminimal_P(n,p)\defeq \Big(\prd{m:\N}P(m)\to (n\leq m)\Big)
  \end{equation*}
\end{defn}

Note that the type $\isminimal_P(n,p)$ doesn't depend on $p$. However, it doesn't make much sense that $n$ is a minimal element of $P$ unless we already know that $n$ is in $P$. Indeed, if we would omit the hypothesis that $n$ is in $P$, it would be more accurate to say that $n$ is a \emph{lower bound} of $P$. The following theorem is the well-ordering principle of $\N$. 

\begin{thm}
  Let $P$ be a decidable family over $\N$. Then there is a function
  \begin{equation*}
    \Big(\sm{n:\N}P(n)\Big)\to\Big(\sm{m:\N}{p:P(m)}\isminimal_P(m,p)\Big).
  \end{equation*}
\end{thm}

\begin{proof}
  Consider a universe $\UU$ that contains $P$. We show by induction on $n:\N$ that there is a function
  \begin{equation*}
    Q(n)\to \Big(\sm{m:\N}{p:Q(m)}\isminimal_Q(m,p)\Big) 
  \end{equation*}
  for every decidable family $Q:\N\to\UU$. Note that we performed a swap in the order of quantification, using the universe that contains $P$. This slightly strengthens the inductive hypothesis, which we will be able to exploit.

  The base case is trivial, since $\zeroN$ is the least natural number. For the inductive step, suppose that $Q(\succN(n))$ holds. Note that $Q(\zeroN)$ is assumed to be decidable, so we proceed by case analysis on $Q(\zeroN)+\neg Q(\zeroN)$. Given $q:Q(\zeroN)$, it follows immediately that $\zeroN$ must be minimal. In the case where $\neg Q(\zeroN)$, we consider the decidable subset $Q'$ of $\N$ given by
  \begin{equation*}
    Q'(n)\defeq Q(\succN(n)).
  \end{equation*}
  Since we have $q:Q'(n)$, we obtain a minimal element in $Q'$ by the inductive hypothesis. Of course, by the assumption that $Q(\zeroN)$ doesn't hold, the minimal element of $Q'$ is also the minimal element of $Q$.
\end{proof}

\subsection{The strong induction principle of \texorpdfstring{$\N$}{N}}

\begin{thm}
  For any type family $P$ over $\N$ there an operation
  \begin{equation*}
    \strongindN : P(\zeroN)\to\Big(\prd{n:\N}\Big(\prd{m:\N}(m\leq n)\to P(m)\Big)\to P(n+1)\Big)\to \Big(\prd{n:\N}P(n)\Big).
  \end{equation*}
  Moreover, the operation $\strongindN$ comes equipped with identifications
  \begin{align*}
    \strongindN(p_0,p_S,\zeroN) & = p_0 \\
    \strongindN(p_0,p_S,n+1) & = p_S(n,(\lam{m}\lam{p}\strongindN(p_0,p_S,m))),
  \end{align*}
  for any $p_0:P(\zeroN)$ and $p_S:\prd{n:\N}\Big(\prd{m:\N}(m\leq n)\to P(m)\Big)\to P(n+1)$.
\end{thm}

\begin{proof}
  Consider
  \begin{align*}
    p_0 & : P(\zeroN) \\
    p_S & : \prd{n:\N}\Big(\prd{m:\N}(m\leq n)\to P(m)\Big)\to P(n+1)
  \end{align*}
  
  First, we claim that there is a function
  \begin{equation*}
    \tilde{p}_0 : \prd{m:\N}(m\leq\zeroN)\to P(m)
  \end{equation*}
  that comes equipped with an identification
  \begin{equation*}
    \tilde{p}_0(\zeroN,p)=p_0
  \end{equation*}
  for any $p:\zeroN\leq\zeroN$. The fact that we have such a dependent function $\tilde{p}_0$ follows immediately by induction on $m$ and $p:m\leq \zeroN$.

  Next, we claim that there is a function
  \begin{equation*}
    \tilde{p}_S : \prd{n:\N}\Big(\prd{m:\N}(m\leq n) \to P(m)\Big)\to \Big(\prd{m:\N}(m\leq n+1)\to P(m)\Big)
  \end{equation*}
  equipped with a homotopy
  \begin{equation*}
    \prd{m:\N}\prd{q:m\leq n}{p:m\leq n+1} \tilde{p}_S(n,H,m,p) = H(m,q)
  \end{equation*}
  and an identification
  \begin{equation*}
    \tilde{p}_S(n,H,n+1,p)=p_S(n,H)
  \end{equation*}
  for every $p:n+1\leq n+1$.

  Using $\tilde{p}_0$ and $\tilde{p}_S$, we obtain by induction on $n$ a function
  \begin{equation*}
    \tilde{s}:\prd{n:\N}\prd{m:\N} (m\leq n)\to P(m)
  \end{equation*}
  satisfying the computation rules
  \begin{align*}
    \tilde{s}(\zeroN) & \jdeq \tilde{p}_0 \\
    \tilde{s}(n+1) & \jdeq \tilde{p}_S(n,\tilde{s}(n)).
  \end{align*}
  Now we define
  \begin{equation*}
    \strongindN(p_0,p_S,n) \defeq \tilde{s}(n,n,\reflleqN(n)),
  \end{equation*}
  where $\reflleqN(n):n\leq n$ is the proof of reflexivity of $\leq$.

  It remains to show that $\strongindN$ satisfies the identifications claimed in the statement of the theorem. The identification that computes $\strongindN$ at $\zeroN$ is easy to obtain:
  \begin{align*}
    \strongindN(p_0,p_S,\zeroN) & \jdeq \tilde{s}(\zeroN,\zeroN,\reflleqN(\zeroN)) \\
                                & \jdeq \tilde{p}_{0}(\zeroN,\reflleqN) \\
                                & = p_0.
  \end{align*}
  To construct the identification that computes $\strongindN$ at a successor, we start with a similar computation:
  \begin{align*}
    \strongindN(p_0,p_S,n+1) & \jdeq \tilde{s}(n+1,n+1,\reflleqN(n+1)) \\
                                   & \jdeq \tilde{p}_S(n,\tilde{s}(n),n+1,\reflleqN(n+1)) \\
    & = p_S(n,\tilde{s}(n))
  \end{align*}
  Thus we see that, in order to show that
  \begin{equation*}
    p_S(n,\tilde{s}(n))=p_S(n,(\lam{m}\lam{p}\tilde{s}(m,m,\reflleqN(m)))),
  \end{equation*}
  we need to prove that
  \begin{equation*}
    \tilde{s}(n)=\lam{m}\lam{p}\tilde{s}(m,m,\reflleqN(m)).
  \end{equation*}
  Here we apply function extensionality, so it suffices to show that
  \begin{equation*}
    \tilde{s}(n,m,p)=\tilde{s}(m,m,\reflleqN(m))
  \end{equation*}
  for every $m:\N$ and $p:m\leq n$. We proceed by induction on $n:\N$. The base case is trivial. For the inductive step, we note that
  \begin{align*}
    \tilde{s}(n+1,m,p)=\tilde{p}_S(n,\tilde{s}(n),m,p)=\begin{cases}\tilde{s}(n,m,p) & \text{if }m\leq n \\
    p_S(n,\tilde{s}(n)) & \text{if }m=n+1.\end{cases}
  \end{align*}
  Therefore it follows by the inductive hypothesis that
  \begin{equation*}
    \tilde{s}(n+1,m,p)=\tilde{s}(m,m,\reflleqN(m))
  \end{equation*}
  if $m\leq n$ holds. In the remaining case, where $m=n+1$, note that we have
  \begin{align*}
    \tilde{s}(\succN,\succN,\reflleqN(\succN)) & = \tilde{p}(n,\tilde{s}(n),n+1,\reflleqN(n+1)) \\
    & = p_S(n,\tilde{s}(n)).
  \end{align*}
  Therefore we see that we also have an identification
  \begin{equation*}
    \tilde{s}(n+1,m,p)=\tilde{s}(m,m,\reflleqN(m))
  \end{equation*}
  when $m=n+1$. This completes the proof of the strong induction principle for $\N$.
\end{proof}

\subsection{Defining the greatest common divisor}

\begin{lem}
  For any $d,n:\N$, the type $d\mid n$ is decidable.
\end{lem}

\begin{proof}
  We give the proof by case analysis on $(d=\zeroN)+(d\neq\zeroN)$. If $d=\zeroN$, then $d\mid n$ holds if and only if $\zeroN=n$, which is decidable.

  If $d\neq\zeroN$, then it follows that $n\leq nd$. Therefore we obtain by the well-ordering principle of the natural numbers a minimal $m:\N$ that satisfies the decidable property $n\leq md$. Now we observe that $d\mid n$ holds if and only if $n=md$, which is decidable.
\end{proof}

\begin{defn}
  A type family $P$ over $\N$ is said to be \define{bounded from above} by $m$ for some natural number $m$, if it comes equipped with a term of type
  \begin{equation*}
    \isbounded_m(P) \defeq \prd{n:\N}P(n)\to (n\leq m).
  \end{equation*}
\end{defn}

\begin{defn}
  Let $P$ be a type family over $\N$, and consider $p:P(n)$. We say that $n$ is the maximal $P$-number if it comes equipped with a term of type
  \begin{equation*}
    \ismaximal_P(n,p) \defeq \prd{m:\N} P(m)\to m\leq n.
  \end{equation*}
\end{defn}

In the following lemma we show that if a decidable family $P$ is bounded from above and inhabited, then it possesses a maximal element.

\begin{lem}\label{lem:maximal}
  Consider a decidable type family $P$ over $\N$ which is bounded from above by $m$. Then there is a function
  \begin{equation*}
    \maximum_P:\Big(\sm{n:\N}P(n)\Big)\to\Big(\sm{n:\N}{p:P(n)}\ismaximal_P(n,p)\Big).
  \end{equation*}
\end{lem}

\begin{proof}
  We define the asserted function by induction on $m$. In the base case, if we have $p:P(n)$, then it follows from $n\leq 0$ that $n=0$. It follows by the boundedness of $P$ that $(n,p)$ is maximal.

  In the inductive step we proceed by case analysis on $P(\succN(m))$. This is allowed because $P$ is decidable. If we have $q:P(\succN(m))$, then it follows by the boundedness of $P$ that $(\succN(m),q)$ is maximal. If $\neg P(\succN(m))$, then it follows that $P$ is bounded by $m$, which allows us to proceed by recursion.
\end{proof}

\begin{defn}
  For any two natural numbers $m,n$ we define the \define{greatest common divisor} $\gcd(m,n)$, which satisfies the following two properties:
  \begin{enumerate}
  \item We have both $\gcd(m,n)\mid m$ and $\gcd(m,n)\mid n$.
  \item For any $d:\N$ we have $d\mid \gcd(m,n)$ if and only if both $d\mid m$ and $d\mid n$ hold.
  \end{enumerate}
\end{defn}

\begin{proof}[Construction]
  Consider the type family $P(d)\defeq (d\mid m)\times (d\mid n)$. Then $P$ is bounded from above by $m$. Moreover, $P(1)$ holds since $1\mid n$ for any natural number $n$. Furthermore, the divisibility relation is decidable, so it follows that $P$ is a family of decidable types. Now the greatest common divisor is defined as the maximal $P$-element, which is obtained by \cref{lem:maximal}
\end{proof}

\subsection{The Euclidean algorithm}

It was immediate from our definition of the greatest common divisor of $a$ and $b$ that it indeed divides both $a$ and $b$, and that it is the greatest such number. However, as a program that is supposed to \emph{compute} the greatest common divisor of $a$ and $b$ it performs rather poorly: it checks for every $n$ from $1$ until either $a$ or $b$ whether it is a divisor of both $a$ and $b$, and only then it gives as output the largest common divisor that it has found. In this section we give a new definition of an operation
\begin{equation*}
  \gcdeuclid:\N \to (\N \to \N)
\end{equation*}
following Euclid's algorithm, with the opposite qualities: it will compute rather quicky a value for $\gcdeuclid(a,b)$, but it will be left as something to show that this value is indeed the greatest common divisor of $a$ and $b$.

\begin{defn}
  We define a binary operation
  \begin{equation*}
    \gcdeuclid:\N \to (\N\to\N).
  \end{equation*}
\end{defn}

\begin{proof}
  We will define the operation $\gcdeuclid$ with the \emph{strong} induction principle for $\N$, which was given as \cref{ex:strong-induction}. Thus it suffices to construct a function $\N\to\N$, which will provide the values for $\gcdeuclid(\zeroN)$, and a function
  \begin{equation*}
    h_a:\Big(\prd{x:\N}(x\leq a) \to \N\to\N\Big)\to (\N\to\N),
  \end{equation*}
  for every $a:\N$, which will provide the values for $\gcdeuclid(a+1)$.

  In the base case, we simply define
  \begin{equation*}
    \gcdeuclid(\zeroN)\defeq\idfunc.
  \end{equation*}
  For the inductive step, consider a family of maps $F_x:\N\to\N$ indexed by $x\leq a$. We think of $F_x(b)$ as the value for $\gcdeuclid(x,b)$, so our assumption of having such a family of maps $F_x$ is really the assumption that $\gcdeuclid(x,b)$ is defined for every $x\leq a$. Our goal is to construct a map
  \begin{equation*}
    \gcdeuclid(a+1):\N\to\N
  \end{equation*}
  We proceed by strong induction on $b:\N$. In the base case, we define
  \begin{equation*}
    \gcdeuclid(a+1,\zeroN)\defeq a+1.
  \end{equation*}
  For the inductive step, assume that we have a number $G_y:\N$ for every $y\leq b$. Observe that $(b\leq a)+(a<b)$ holds for any $b:B$, see \cref{ex:order_N}. Thus we can proceed by case analysis to define
  \begin{equation*}
    h_a(F,b+1)\defeq
    \begin{cases}
      F_{(a+1)-(b+1)}(b+1) & \text{if }b\leq a\\
      G_{(b+1)-(a+1)} & \text{if }a<b.
    \end{cases}
  \end{equation*}
  This completes the inductive step, and hence we obtain a binary operation
  $\gcdeuclid$ that satisfies
  \begin{align*}
    \gcdeuclid(\zeroN,b) & \jdeq b \\
    \gcdeuclid(a+1,\zeroN) & \jdeq a+1 \\
    \gcdeuclid(a+1,b+1) & \jdeq \gcdeuclid((a+1)-(b+1),b+1) & & \text{if }b\leq a.\\
    \gcdeuclid(succN(a),b+1) & \jdeq \gcdeuclid(a+1,(b+1)-(a+1)) & & \text{if }a<b.\qedhere            
  \end{align*}
\end{proof}

\begin{prp}
  For each $a,b:\N$, the number $\gcdeuclid(a,b)$ is the greatest common divisor of $a$ and $b$.
\end{prp}


\subsection{The trial division primality test}

\begin{thm}
  For any $n:\N$, the proposition $\isprime(n)$ is decidable.
\end{thm}

It is important to note that, even when we prove that a type such as $\isprime(n)$ is decidable, it is only after we \emph{evaluate} the proof term that we know whether the type under consideration has a term or not. In other words, for any given $n$ we don't know right away whether it is prime or not. Evaluating whether $n$ is prime can be computationally costly, so it may be desirable in any specific situation to give a separate mathematical \emph{argument} that decides whether or not the number is prime.

\subsection{Prime decomposition}

We will show now that any natural number $n>0$ can be written as a product of primes
\begin{equation*}
  n=p_1^{k_1}\cdots p_{m}^{k_m}
\end{equation*}
This prime decomposition is unique if we require that the primes $p_i<p_{i+1}$ for each $0<i<m$. In order to establish these facts in type theory, we first have to define finite products.

\subsection{The infinitude of primes}

\begin{thm}
  There are infinitely many primes.
\end{thm}

\begin{proof}
  We will show that for every $n:\N$ there is a prime number that is larger than $n$. In other words, we will construct a term of type
  \begin{equation*}
    \prd{n:\N}\sm{p:\N}\isprime(p)\times (n\leq p).
  \end{equation*}
  Note that the number $n!+1$ is relatively prime to any number $m\leq n$. Therefore the primes in its prime factorization must all be larger than $n$. Thus, the function that assigns to $n$ the least prime factor of $n!+1$ shows that for any $n:\N$ there is a prime number $p$ that is larger than $n$.
\end{proof}

\begin{cor}
  There is a function
  \begin{equation*}
    \primetype : \N \to \sm{p:\N}\isprime(p)
  \end{equation*}
  that sends $n$ to the $n$-th prime. This function is strictly monotone, so it is an embedding.
\end{cor}

\begin{exercises}
  \exercise Show that for any $f:\Fin(m)\to\Fin(n)$ and any $i:\Fin(n)$, the type $\fib{f}{i}$ is decidable.
  \exercise Consider a decidable type $P(i)$ indexed by $i:\Fin(n)$.
  \begin{subexenum}
  \item Show that the type
    \begin{equation*}
      \prd{i:\Fin(n)}P(i)
    \end{equation*}
    is decidable.
  \item Show that the type
    \begin{equation*}
      \sm{i:\Fin(n)}P(i)
    \end{equation*}
    is decidable.
  \end{subexenum}
  \exercise
  \begin{subexenum}
  \item Show that $\N$ and $\bool$ have decidable equality. Hint: to show that $\N$ has decidable equality, show first that the successor function is injective.
  \item Show that if $A$ and $B$ have decidable equality, then so do $A+B$ and $A\times B$. Conclude that $\Z$ has decidable equality.
  \item Show that if $A$ is a retract of a type $B$ with decidable equality, then $A$ also has decidable equality.
  \end{subexenum}
  \exercise Define the prime-counting function $\pi:\N\to\N$.
  \exercise (The Cantor-Schr\"oder-Bernstein theorem) Let $X$ and $Y$ be two sets with decidable equality, and consider two maps $f:X\to Y$ and $g:Y\to X$, both of which we assume to be injective. Construct an equivalence $X\simeq Y$.
  \exercise For any $k:\Z$, define a function $i\mapsto i+k \mod n$ of type $\Fin(n)\to\Fin(n)$. Show that this function is an equivalence.
  \exercise For any $k:\Z$, define a function $i\mapsto i\cdot k \mod n$ of type $\Fin(n)\to\Fin(n)$. Show that this function is an equivalence if and only if $\gcd(n,k)=1$.
  \exercise Show that
  \begin{equation*}
    \sum_{i=0}^n \binom{n-i}{i}=F_{n+1}
  \end{equation*}
  \exercise Show that if $2^n-1$ is prime, then $n$ is prime.
  \exercise Prove Fermat's little theorem.
  \exercise Extend the definition of the greatest common divisor to all integers.
  \exercise Show that
  \begin{equation*}
    (\Fin(m)\simeq\Fin(n))\leftrightarrow(m=n).
  \end{equation*}
  \exercise Show that $\N$ satisfies \define{ordinal induction}, i.e., construct for any type family $P$ over $\N$ a function of type
  \begin{equation*}
    \ordindN : \Big(\prd{k:\N} \Big(\prd{m:\N} (m< k) \to P(m)\Big)\to P(k)\Big) \to \prd{n:\N}P(n).
  \end{equation*}
  Moreover, prove that
  \begin{equation*}
    \ordindN(h,n)=h(n,\lam{m}\lam{p}\ordindN(h,m))
  \end{equation*}
  for any $n:\N$ and any $h:\prd{k:\N}\Big(\prd{m:\N}(m<k)\to P(m)\Big)\to P(k)$.
  \exercise
  \begin{subexenum}
  \item Show that if $A$ and $B$ have decidable equality, then so do the types $A+B$ and $A\times B$.
  \item Show that $\Z$ and $\Fin(n)$ have decidable equality, for every $n:\N$.
  \end{subexenum}
  \exercise Let $P:\N\to\classicalprop$ be a decidable subset of $\N$.
  \begin{subexenum}
  \item Show that $\sm{m:\N}{p:P(m)}\isminimal_P(m,p)$ is a proposition.
  \item Show that the map
    \begin{equation*}
      \Big(\sm{n:\N}P(n)\Big)\to\Big(\sm{m:\N}{p:P(m)}\isminimal_P(m,p)\Big)
    \end{equation*}
    is a propositional truncation.
  \end{subexenum}
  \exercise Suppose that $A:I\to \UU$ is a type family over a set $I$ with decidable equality. Show that
  \begin{equation*}
    \Big(\prd{i:I}\iscontr(A_i)\Big)\leftrightarrow \iscontr\Big(\prd{i:I}A_i\Big).
  \end{equation*}
\end{exercises}


\chapter{Function extensionality}

\section{Equivalent forms of function extensionality}
\begin{thm}\label{thm:funext_wkfunext}
The following are equivalent:
\begin{enumerate}
\item The \define{function extensionality principle} holds: For every type family $B:A\to\UU$, and any two dependent functions $f,g:\prd{x:A}B(x)$, the canonical map
\begin{equation*}
\mathsf{htpy\usc{}eq}(f,g) : (\id{f}{g})\to (f\htpy g)
\end{equation*}
by path induction (sending $\refl{f}$ to $\lam{x}\refl{f(x)}$) is an equivalence.
\item The \define{weak function extensionality principle} holds: For every type family $B:A\to\UU$ one has
\begin{equation*}
\Big(\prd{a:A}\iscontr(B(a))\Big)\to\iscontr\Big(\prd{a:A}B(a)\Big).
\end{equation*}
\end{enumerate}
\end{thm}

\begin{proof}
Suppose that each $B(a)$ is contractible with center of contraction $c(a)$ and contraction $C_a:\prd{y:B(a)}c(a)=y$. Then we take $c\defeq \lam{a}c(a)$ to be the center of contraction of $\prd{a:A}B(a)$. To construct the contraction we have to define a term of type
\begin{equation*}
\prd{f:\prd{a:A}B(a)}c=f.
\end{equation*}
Let $f:\prd{a:A}B(a)$. By function extensionality we have a map $(c\htpy f)\to (c=f)$, so it suffices to construct a term of type $c\htpy f$. Here we take $\lam{a}C_a(f(a))$. This completes the proof that function extensionality implies weak function extensionality.

To prove function extensionality from weak function extensionality, observe that it suffices by \autoref{thm:id_fundamental} to show that
\begin{equation*}
\sm{g:\prd{x:A}B(x)}f\htpy g
\end{equation*}
is contractible.

Since the type $\sm{b:B(x)}f(x)=b$ is contractible for each $x:X$, it follows by our assumption of weak function extensionality that the type $\prd{x:A}\sm{b:B(x)}f(x)=b$ is contractible. By \autoref{ex:contr_retr} it therefore suffices to show that
\begin{equation*}
\sm{g:\prd{x:A}B(x)}f\htpy g
\end{equation*}
is a retract of the type $\prd{x:A}\sm{b:B(x)}f(x)=b$. We have the functions
\begin{align*}
\mathsf{pi\usc{}sigma} & \defeq \lam{\pairr{g,H}}\lam{x}\pairr{g(x),H(x)} \\
\mathsf{sigma\usc{}pi} & \defeq \lam{p}\pairr{\lam{x}\proj 1(p(x)),\lam{x}\proj 2(p(x))}.
\end{align*}
It remains to show that $\psi\circ\varphi=\idfunc$. Let $\pairr{g,H}:\sum_g f\htpy g$. 
Then we have
\begin{align*}
\mathsf{sigma\usc{}pi}(\mathsf{pi\usc{}sigma}(g,H)) & \jdeq \mathsf{sigma\usc{}pi}(\lam{x}\pairr{g(x),H(x)}) \\
& \jdeq \pairr{\lam{x}g(x),\lam{x}H(x)} \\
& \jdeq \pairr{g,H}.\qedhere
\end{align*}
\end{proof}

\begin{comment}
\begin{rmk}
Since we assumed the $\eta$-rule for $\Sigma$-types, we also have
\begin{align*}
\mathsf{pi\usc{}sigma}(\mathsf{sigma\usc{}pi}(p)) & \jdeq \mathsf{pi\usc{}sigma}(\pairr{\lam{x}\proj 1(p(x)),\lam{x}\proj 2(p(x))}) \\
& \jdeq \lam{x}\pairr{\proj 1(p(x)),\proj 2(p(x))} \\
& \jdeq \lam{x} p(x) \\
& \jdeq p.
\end{align*}
Therefore, the types $\sum_g f\htpy g$ and $\prod_x\sum_b f(x)=b$ are actually \emph{judgmentally isomorphic}. 
\end{rmk}
\end{comment}

\begin{thm}
Assume function extensionality. Then for any type family $B:A\to\UU$ one has
\begin{equation*}
\Big(\prd{a:A}\istrunc{n}(B(a))\Big)\to \istrunc{n}\Big(\prd{a:A}B(a)\Big).
\end{equation*}
\end{thm}

\begin{proof}
The theorem is proven by induction on $n\geq -2$. The base case is just the weak function extensionality principle, which was shown to follow from function extensionality in \autoref{thm:funext_wkfunext}.

For the inductive hypothesis, assume that the $n$-types are closed under dependent function types. Assume that $B$ is a family of $(n+1)$-types. By function extensionality, the type $f=g$ is equivalent to $f\htpy g$ for any two dependent functions $f,g:\prd{a:A}B(a)$. Now observe that $f\htpy g$ is a dependent product of $n$-types, and therefore it is an $n$-type by our inductive hypotheses. Therefore, it follows by \autoref{thm:ntype_eqv} that $f=g$ is an $n$-type, and hence that $\prd{a:A}B(a)$ is an $(n+1)$-type.
\end{proof}

\section{Universal properties}

\begin{thm}\label{thm:yoneda}
Let $B:A\to\UU$ be a type family, and let $a:A$. Then the map
\begin{equation*}
\Big(\prd{x:A} (a=x)\to B(x)\Big)\to B(a)
\end{equation*}
given by $\lam{f} f(a,\refl{a})$ is an equivalence. 
\end{thm}

\begin{rmk}
\autoref{thm:yoneda} is sometimes called the \emph{type theoretical Yoneda lemma}.
\end{rmk}

\begin{exercises}
\item Show that for any type $A$ and any $n\geq-2$, the type $\istrunc{n}(A)$ is a mere proposition.
\item Show that for any map $f:A\to B$ the type $\isequiv(f)$ is a mere proposition. Conclude that $\eqv{A}{B}$ is a subtype of $A\to B$, and in particular that for any $(f,p),(g,q):\eqv{A}{B}$ the map
\begin{equation*}
\mathsf{ap}_{\proj 1} : ((f,p)= (g,q))\to (f=g)
\end{equation*}
is an equivalence.
\item Let $C$ be a contractible type with center of contraction $c$. Use the function extensionality principle to show that the map $\lam{f}f(c):(C\to A)\to A$ is an equivalence, for each type $A$.
\item Use the function extensionality principle to show that for any $B:A\to\UU$ and $R:\prd{a:A}B(a)\to \UU$, the map
\begin{equation*}
\Big(\sm{f:\prd{a:A}B(a)}\prd{a:A}R(a,f(a))\Big)\to \Big(\prd{a:A}\sm{b:B(a)}R(a,b)\Big)
\end{equation*}
given by $\lam{(f,g)}{a}(f(a),g(a))$ is an equivalence. This is sometimes called the \define{type theoretic principle of choice}.

Conclude that for any $A:\UU$ and any $C:B\to \UU$, the map
\begin{equation*}
\Big(\sm{f:A\to B} \prd{a:A}C(f(a))\Big)\to\Big(A\to\sm{b:B}C(b)\Big)
\end{equation*}
given by $\lam{(f,g)}{a}(f(a),g(a))$ is an equivalence.
\item Let $e:\eqv{A}{B}$ be an equivalence. Show that the map
\begin{equation*}
(B\to X)\to (A\to X)
\end{equation*}
given by $\lam{f}f\circ e$ is an equivalence.
\item Show that the map
\begin{equation*}
(A+B\to X)\to (A\to X)\times (B\to X)
\end{equation*}
given by $f\mapsto (f\circ\inl,f\circ\inr)$ is an equivalence.
\item Show that the map
\begin{equation*}
(A\times B\to X)\to (A\to (B\to X))
\end{equation*}
given by $f\mapsto \lam{a}{b}f(a,b)$ is an equivalence.
\item Show that the map
\begin{equation*}
\Big(\Big(\sm{a:A}B(a)\Big)\to X\Big)\to \Big(\prd{a:A}(B(a)\to X)\Big)
\end{equation*}
is an equivalence.
\item Show that the map
\begin{equation*}
(\unit\to X)\to X
\end{equation*}
given by $\lam{f}f(\ttt)$ is an equivalence. 
\item Show that the map
\begin{equation*}
(\emptyt \to X)\to \unit
\end{equation*}
given by $\lam{f}\ttt$ is an equivalence.
\end{exercises}


\chapter{The univalence axiom}

\section{Type extensionality}
\begin{defn}
We define a family of maps
\begin{equation*}
\mathsf{equiv\usc{}eq}\defeq \mathsf{rec}_{=}(\lam{A}\idfunc[A]) : \prd*{A,B:\UU} (\id{A}{B})\to(\eqv{A}{B}).
\end{equation*}
\end{defn}

\begin{defn}
The \define{univalence axiom} asserts that the family of maps $\mathsf{equiv\usc{}eq}$ is a fiberwise equivalence.
\end{defn}

The univalence axiom asserts that equivalent types are equal. It is considered to be an \emph{extensionality principle} for types.

\begin{lem}
The univalence axiom holds if and only if the type
\begin{equation*}
\sm{B:\UU}\eqv{A}{B}
\end{equation*}
is contractible for each $A:\UU$.
\end{lem}

\begin{proof}
By \autoref{thm:id_fundamental}.
\end{proof}

The following construction enables us to make construction by induction on equivalences, analogous to path induction.

\begin{defn}
Let $A:\UU$, and let $P:\prd{B:\UU} (\eqv{A}{B})\to\type$ be a type family. Using the univalence axiom we construct
\begin{equation*}
\mathsf{equiv\usc{}ind}(P,A) : P(A,\idfunc[A])\to \prd{B:\UU}{e:\eqv{A}{B}}P(B,e).
\end{equation*}
\end{defn}

\begin{constr}
Since $\sm{B:\UU}\eqv{A}{B}$ is contractible we have
\begin{equation*}
P(\idfunc[A])\to\prd{\pairr{B,e}:\sm{B:\UU}\eqv{A}{B}}P(B,e)
\end{equation*}
by \autoref{ex:contr_ind}, so we obtain the desired function by uncurrying.
\end{constr}

From now on we will assume that the univalence axiom holds.

\section{Function extensionality}

The first application of the univalence axiom was Voevodsky's proof of \emph{function extensionality}, which we introduce below.

\begin{defn}
Let $f,g:\prd{x:A}B(x)$ be two dependent functions. We define the function
\begin{equation*}
\mathsf{htpy\usc{}eq}(f,g) : (\id{f}{g})\to (f\htpy g)
\end{equation*}
by path induction, sending $\refl{f}$ to $\lam{x}\refl{f(x)}$. The \define{function extensionality principle} asserts that $\mathsf{htpy\usc{}eq}$ is a fiberwise equivalence, for any $A:\type$ and $B:A\to\type$.
\end{defn}

We first show that function extensionality follows from \emph{weak function extensionality}.

\begin{defn}
The \define{weak function extensionality principle} asserts that for any $B:A\to\type$,
\begin{equation*}
\Big(\prd{x:A}\iscontr(B(x))\Big)\to \iscontr\Big(\prd{x:A}B(x)\Big).
\end{equation*}
\end{defn}

\begin{thm}
Weak function extensionality implies function extensionality.
\end{thm}

\begin{proof}
To prove function extensionality, it suffices by \autoref{thm:id_fundamental} to show that
\begin{equation*}
\sm{g:\prd{x:A}B(x)}f\htpy g
\end{equation*}
is contractible.

Assume that products of contractible types are contractible.
Since the type $\sm{b:B(x)}f(x)=b$ is contractible for each $x:X$, it follows by our assumption of weak function extensionality that the type $\prd{x:A}\sm{b:B(x)}f(x)=b$ is contractible. By \autoref{ex:contr_retr} it therefore suffices to show that
\begin{equation*}
\sm{g:\prd{x:A}B(x)}f\htpy g
\end{equation*}
is a retract of the type $\prd{x:A}\sm{b:B(x)}f(x)=b$. We have the functions
\begin{align*}
\mathsf{pi\usc{}sigma} & \defeq \lam{\pairr{g,H}}\lam{x}\pairr{g(x),H(x)} \\
\mathsf{sigma\usc{}pi} & \defeq \lam{p}\pairr{\lam{x}\proj 1(p(x)),\lam{x}\proj 2(p(x))}.
\end{align*}
It remains to show that $\psi\circ\varphi=\idfunc$. Let $\pairr{g,H}:\sum_g f\htpy g$. 
Then we have
\begin{align*}
\mathsf{sigma\usc{}pi}(\mathsf{pi\usc{}sigma}(g,H)) & \jdeq \mathsf{sigma\usc{}pi}(\lam{x}\pairr{g(x),H(x)}) \\
& \jdeq \pairr{\lam{x}g(x),\lam{x}H(x)} \\
& \jdeq \pairr{g,H}.\qedhere
\end{align*}
\end{proof}

\begin{rmk}
Since we assumed the $\eta$-rule for $\Sigma$-types, we also have
\begin{align*}
\mathsf{pi\usc{}sigma}(\mathsf{sigma\usc{}pi}(p)) & \jdeq \mathsf{pi\usc{}sigma}(\pairr{\lam{x}\proj 1(p(x)),\lam{x}\proj 2(p(x))}) \\
& \jdeq \lam{x}\pairr{\proj 1(p(x)),\proj 2(p(x))} \\
& \jdeq \lam{x} p(x) \\
& \jdeq p.
\end{align*}
Therefore, the types $\sum_g f\htpy g$ and $\prod_x\sum_b f(x)=b$ are actually \emph{judgmentally isomorphic}. 
\end{rmk}

\begin{exercises}
\item \label{ex:tr_ap} Show that for any $P:X\to \UU$ and any $p:x=y$ in $X$, we have
\begin{equation*}
\mathsf{equiv\usc{}eq}(\ap{P}{p})=\mathsf{tr}^P(p).
\end{equation*}
\item Use the univalence axiom to show that the type $\sm{A:\UU}\iscontr(A)$ of all contractible types in $\UU$ is contractible.
\item Use the univalence axiom to show that the type $\sm{P:\prop}P$ is contractible.
\item Show that $\eqv{(\eqv{\bool}{\bool})}{\bool}$, and conclude by the univalence axiom that the universe is not a set.
\item Construct by path induction a family of maps
\begin{equation*}
\prd{A,B:\UU}{a:A}{b:B} (\id{\pairr{A,a}}{\pairr{B,b}})\to \sm{e:\eqv{A}{B}}e(a)=b,
\end{equation*}
and show that this map is an equivalence. In other words, an \emph{identification of pointed types} is a base point preserving equivalence.
\item Let $\pairr{A,a}$ and $\pairr{B,b}$ be two pointed types. Construct by path induction a family of maps
\begin{equation*}
\prd{f,g:A\to B}{p:f(a)=b}{q:g(a)=b} (\id{\pairr{f,p}}{\pairr{g,q}})\to \sm{H:f\htpy g} p = \ct{H(a)}{q},
\end{equation*}
and show that this map is an equivalence. In other words, an \emph{identification of pointed maps} is a base point preserving homotopy.
\item Let $C$ be a contractible type with center of contraction $c$. Use the function extensionality principle to show that the map $\lam{f}f(c):(C\to A)\to A$ is an equivalence, for each type $A$.
\end{exercises}


% !TEX root = hott_intro.tex

\section{The circle}
\index{circle|(}
\index{inductive type!circle|(}

We have seen inductive types, in which we describe a type by its constructors and an induction principle that allows us to construct sections of dependent types. Inductive types are freely generated by their constructors, which describe how we can construct their terms. 

However, many familiar constructions in algebra involve the construction of algebras by generators and relations. 
For example, the free abelian group with two generators is described as the group with generators $x$ and $y$, and the relation $xy=yx$. 

In this chapter we introduce higher inductive types\index{higher inductive type!circle}, where we follow a similar idea: to allow in the specification of inductive types not only \emph{point constructors}, but also \emph{path constructors} that give us relations between the point constructors. 
The ideas behind the definition of higher inductive types are introduced by studying the simplest non-trivial example: the \emph{circle}.

\subsection{The induction principle of the circle}
The \emph{circle} is defined as a higher inductive type $\sphere{1}$\index{S 1@{$\sphere{1}$}|see {circle}} that comes equipped with\index{base@{$\base$}}\index{loop@{$\lloop$}}\index{circle!base@{$\base$}}\index{circle!loop@{$\lloop$}}
\begin{align*}
\base & : \sphere{1} \\
\lloop & : \id{\base}{\base}.
\end{align*}
Just like for ordinary inductive types, the induction principle for higher inductive types provides us with a way of constructing sections of dependent types. However, we need to take the \emph{path constructor}\index{path constructor} $\lloop$ into account in the induction principle. 

By applying a section $f:\prd{x:\sphere{1}}P(x)$ to the base point of the circle, we obtain a term $f(\base):P(\base)$. Moreover, using the dependent action on paths\index{dependent action on paths} of $f$ of \cref{defn:apd} we also obtain for any dependent function $f:\prd{x:\sphere{1}}P(x)$ a path
\begin{align*}
\apd{f}{\lloop} & : \id{\tr_P(\lloop,f(\base))}{f(\base)}
\end{align*}
in the fiber $P(\base)$.

\begin{defn}
Let $P$ be a type family over the circle. The \define{dependent action on generators}\index{dependent action on generators!for the circle} is the map\index{dgen_S1@{$\dgen_{\sphere{1}}$}}
\begin{equation}\label{eq:dgen_circle}
\dgen_{\sphere{1}}:\Big(\prd{x:\sphere{1}}P(x)\Big)\to\Big(\sm{y:P(\base)}\id{\tr_P(\lloop,y)}{y}\Big)
\end{equation}
given by $\dgen_{\sphere{1}}(f)\defeq\pairr{f(\base),\apd{f}{\lloop}}$.
\end{defn}

We now give the full specification of the circle.

\begin{defn}
The \define{circle}\index{circle} is a type $\sphere{1}$\index{S 1@{$\sphere{1}$}} that comes equipped with\index{base@{$\base$}}\index{loop@{$\lloop$}}
\begin{align*}
\base & : \sphere{1} \\
\lloop & : \id{\base}{\base},
\end{align*}
and satisfies the \define{induction principle of the circle}\index{induction principle!of the circle}, which provides for each type family $P$ over $\sphere{1}$ a map
\begin{equation*}
\ind{\sphere{1}}:\Big(\sm{y:P(\base)}\id{\tr_P(\lloop,y)}{y}\Big)\to \Big(\prd{x:\sphere{1}}P(x)\Big),
\end{equation*}
and a homotopy witnessing that $\ind{\sphere{1}}$ is a section of $\dgen_{\sphere{1}}$
\begin{equation*}
\comphtpy{\sphere{1}}:\dgen_{\sphere{1}}\circ \ind{\sphere{1}}\htpy \idfunc
\end{equation*}
for the computation rule\index{computation rules!of the circle}.
\end{defn}

\begin{rmk}\label{rmk:circle-induction}
  The type of identifications $(y,p)=(y',p')$ in the type
  \begin{equation*}
    \sm{y:P(\base)}\tr_P(\lloop,y)=y
  \end{equation*}
  is equivalent to the type of pairs $(\alpha,\beta)$ consisting of an identification $\alpha:y=y'$, and an identification $\beta$ witnessing that the square
  \begin{equation*}
    \begin{tikzcd}[column sep=6em]
      \tr_P(\lloop,y) \arrow[d,equals,swap,"p"] \arrow[r,equals,"\ap{\tr_P(\lloop)}{\alpha}"] & \tr_P(\lloop,y') \arrow[d,equals,"{p'}"] \\
      y \arrow[r,equals,swap,"\alpha"] & y'
    \end{tikzcd}
  \end{equation*}
  commutes. Therefore it follows from the induction principle of the circle that for any $(y,p):\sm{y:P(\base)}\tr_P(\lloop,y)=y$, there is a dependent function $f:\prd{x:\sphere{1}}P(x)$ equipped with an identification
  \begin{equation*}
    \alpha : f(\base)=y,
  \end{equation*}
  and an identification $\beta$ witnessing that the square
  \begin{equation*}
    \begin{tikzcd}[column sep=6em]
      \tr_P(\lloop,f(\base)) \arrow[d,equals,swap,"{\apd{f}{\lloop}}"] \arrow[r,equals,"\ap{\tr_P(\lloop)}{\alpha}"] & \tr_P(\lloop,y) \arrow[d,equals,"{p}"] \\
      f(\base) \arrow[r,equals,swap,"\alpha"] & y
    \end{tikzcd}
  \end{equation*}
  commutes.  
\end{rmk}

\subsection{The (dependent) universal property of the circle}
\subsectionmark{The universal property of the circle}

Our goal is now to use the induction principle of the circle to derive the \define{universal property}\index{universal property!of the circle} of the circle. This universal property states that, for any type $X$ the canonical map
\begin{equation*}
  \Big(\sphere{1}\to X\Big)\to\Big(\sm{x:X}x=x\Big)
\end{equation*}
given by $f\mapsto(f(\base),\ap{f}{\lloop})$ is an equivalence. It turns out that it is easier to prove the \define{dependent universal property}\index{dependent universal property!of the circle} first. The dependent universal property states that for any type family $P$ over the circle, the canonical map
\begin{equation*}
  \Big(\prd{x:\sphere{1}}P(x)\Big)\to\Big(\sm{y:P(\base)}\tr_P(\lloop,y)=y\Big)
\end{equation*}
given by $f\mapsto(f(\base),\apd{f}{\lloop})$ is an equivalence.

\begin{thm}\label{thm:circle-dependent-universal-property}
  For any type family $P$ over the circle, the map
  \begin{equation*}
    \Big(\prd{x:\sphere{1}}P(x)\Big)
    \to
    \Big(\sm{y:P(\base)}\tr_P(\lloop,y)=y\Big)
  \end{equation*}
  given by $f\mapsto(f(\base),\apd{f}{\lloop})$ is an equivalence.
\end{thm}

\begin{proof}
  By the induction principle of the circle we know that the map has a section, i.e., we have
  \begin{align*}
    \ind{\sphere{1}} & : \Big(\sm{y:P(\base)}\tr_P(\lloop,y)=y\Big) \to \Big(\prd{x:\sphere{1}}P(x)\Big) \\
    \comphtpy{\sphere{1}} & : \dgen_{\sphere{1}}\circ\ind{\sphere{1}}\htpy\idfunc
  \end{align*}
  Therefore it remains to construct a homotopy
  \begin{equation*}
    \ind{\sphere{1}}\circ\dgen_{\sphere{1}}\htpy\idfunc.
  \end{equation*}
  Thus, for any $f:\prd{x:\sphere{1}}P(x)$ our task is to construct an identification
  \begin{equation*}
    \ind{\sphere{1}}(\dgen_{\sphere{1}}(f))=f.
  \end{equation*}
  By function extensionality it suffices to construct a homotopy
  \begin{equation*}
    \prd{x:\sphere{1}} \ind{\sphere{1}}(\dgen_{\sphere{1}}(f))(x)= f(x).
  \end{equation*}
  We proceed by the induction principle of the circle using the family of types $E_{g,f}(x)\defeq g(x)=f(x)$ indexed by $x:\sphere{1}$, where $g$ is the function
  \begin{equation*}
    g\defeq\ind{\sphere{1}}(\dgen_{\sphere{1}}(f)).
  \end{equation*}
  Thus, it suffices to construct
  \begin{align*}
    \alpha & : g(\base)=f(\base)\\
    \beta  & : \tr_{E_{g,f}}(\lloop,\alpha)=\alpha. 
  \end{align*}
  An argument by path induction on $p$ yields that
  \begin{equation*}
    \Big(\ct{\apd{g}{p}}{r}=\ct{\ap{\tr_P(p)}{q}}{\apd{f}{p}}\Big)\to\Big(\tr_{E_{g,f}}(p,q)=r\Big),
  \end{equation*}
  for any $f,g:\prd{x:X}P(x)$ and any $p:x=x'$, $q:g(x)=f(x)$ and $r:g(x')=f(x')$.
  Therefore it suffices to construct an identification $\alpha:g(\base)=f(\base)$ equipped with an identification $\beta$ witnessing that the square
  \begin{equation*}
    \begin{tikzcd}[column sep=6em]
      \tr_P(\lloop,g(\base)) \arrow[d,equals,swap,"\apd{g}{\lloop}"] \arrow[r,equals,"\ap{\tr_P(\lloop)}{\alpha}"] & \tr_P(\lloop,f(\base)) \arrow[d,equals,"\apd{f}{\lloop}"] \\
      g(\base) \arrow[r,equals,swap,"\alpha"] & f(\base)"
    \end{tikzcd}
  \end{equation*}
  commutes. Notice that we get exactly such a pair $(\alpha,\beta)$ from the computation rule of the circle, by \cref{rmk:circle-induction}.
\end{proof}

As a corollary we obtain the following uniqueness principle for dependent functions defined by the induction principle of the circle.

\begin{cor}
  Consider a type family $P$ over the circle, and let
  \begin{align*}
    y & : P(\base) \\
    p & : \tr_{P}(\lloop,y)=y.
  \end{align*}
  Then the type of functions $f:\prd{x:\sphere{1}}P(x)$ equipped with an identification
  \begin{equation*}
    \alpha: f(\base)=y
  \end{equation*}
  and an identification $\beta$ witnessing that the square
  \begin{equation*}
    \begin{tikzcd}[column sep=6em]
      \tr_P(\lloop,f(\base)) \arrow[d,equals,swap,"{\apd{f}{\lloop}}"] \arrow[r,equals,"\ap{\tr_P(\lloop)}{\alpha}"] & \tr_P(\lloop,y) \arrow[d,equals,"{p}"] \\
      f(\base) \arrow[r,equals,swap,"\alpha"] & y
    \end{tikzcd}
  \end{equation*}
  commutes, is contractible.
\end{cor}

Now we use the dependent universal property to derive the ordinary universal property of the circle. It would be tempting to say that it is a direct corollary, but we need to address the transport that occurs in the dependent universal property.

\begin{thm}\label{thm:circle_up} 
For each type $X$, the \define{action on generators}\index{action on generators!for the circle}\index{gen_S1@{$\mathsf{gen}_{\sphere{1}}$}}
\begin{equation*}
\mathsf{gen}_{\sphere{1}}:(\sphere{1}\to X)\to \sm{x:X}x=x
\end{equation*}
given by $f\mapsto (f(\base),\ap{f}{\lloop})$ is an equivalence.
\end{thm}

\begin{proof}
  We prove the claim by constructing a commuting triangle
  \begin{equation*}
    \begin{tikzcd}[column sep=-2em]
      \phantom{\Big(\sm{x:X}\tr_{\const_X}(\lloop,x)=x\Big)} & (\sphere{1}\to X) \arrow[dl,swap,"\gen_{\sphere{1}}"] \arrow[dr,"\dgen_{\sphere{1}}"] \\
      \Big(\sm{x:X}x=x\Big) \arrow[rr,swap,"\simeq"] & & \Big(\sm{x:X}\tr_{\const_X}(\lloop,x)=x\Big)
    \end{tikzcd}
  \end{equation*}
  in which the bottom map is an equivalence. Indeed, once we have such a triangle, we use the fact from \cref{thm:circle-dependent-universal-property} that $\dgen_{\sphere{1}}$ is an equivalence to conclude that $\gen_{\sphere{1}}$ is an equivalence.

  To construct the bottom map, we first observe that for any constant type family $\const_B$ over a type $A$, any $p:a=a'$ in $A$, and any $b:B$, there is an identification
  \begin{equation*}
    \mathsf{tr\usc{}const}_B(p,b)=b.
  \end{equation*}
  This identification is easily constructed by path induction on $p$. Now we construct the bottom map as the induced map on total spaces of the family of maps
  \begin{equation*}
    l\mapsto \ct{\mathsf{tr\usc{}const}_X(\lloop,x)}{l},
  \end{equation*}
  indexed by $x:X$. Since concatenating by a path is an equivalence, it follows by \cref{thm:fib_equiv} that the induced map on total spaces is indeed an equivalence.

  To show that the triangle commutes, it suffices to construct for any $f:\sphere{1}\to X$ an identification witnessing that the triangle
  \begin{equation*}
    \begin{tikzcd}[column sep=1em]
      \tr_{\const_X}(\lloop,f(\base)) \arrow[dr,equals,swap,"\apd{f}{\lloop}"] \arrow[rr,equals,"{\mathsf{tr\usc{}const}_X(\lloop,f(\base))}"] & & f(\base) \arrow[dl,equals,"\ap{f}{\lloop}"] \\
      & f(\base) & \phantom{\tr_{\const_X}(\lloop,f(\base))}
    \end{tikzcd}
  \end{equation*}
  commutes. This again follows from general considerations: for any $f:A\to B$ and any $p:a=a'$ in $A$, the triangle
  \begin{equation*}
    \begin{tikzcd}[column sep=1em]
      \tr_{\const_B}(p,f(a)) \arrow[dr,equals,swap,"\apd{f}{p}"] \arrow[rr,equals,"{\mathsf{tr\usc{}const}_B(p,f(a))}"] & & f(a) \arrow[dl,equals,"\ap{f}{p}"] \\
      & f(a') & \phantom{\tr_{\const_B}(p,f(a))}
    \end{tikzcd}
  \end{equation*}
  commutes by path induction on $p$.
\end{proof}

\begin{cor}
  For any loop $l:x=x$ in a type $X$, the type of maps $f:\sphere{1}\to X$ equipped with an identification
  \begin{equation*}
    \alpha : f(\base)=x 
  \end{equation*}
  and an identification $\beta$ witnessing that the square
  \begin{equation*}
    \begin{tikzcd}
      f(\base) \arrow[r,equals,"\alpha"] \arrow[d,equals,swap,"\ap{f}{\lloop}"] & x \arrow[d,equals,"l"] \\
      f(\base) \arrow[r,equals,swap,"\alpha"] & x
    \end{tikzcd}
  \end{equation*}
  commutes, is contractible.
\end{cor}

\subsection{Multiplication on the circle}
\label{sec:mulcircle}

One way the circle arises classically, is as the set of complex numbers at distance $1$ from the origin. It is an elementary fact that $|xy|=|x||y|$ for any two complex numbers $x,y\in\mathbb{C}$, so it follows that when we multiply two complex numbers that both lie on the unit circle, then the result lies again on the unit circle. Thus, using complex multiplication we see that there is a multiplication operation on the circle. And there is a shadow of this operation in type theory, even though our circle arises in a very different way!

\begin{defn}\label{defn:mul-circle}
  We define a binary operation
\begin{equation*}
  \mulcircle : \sphere{1}\to(\sphere{1}\to\sphere{1}).
\end{equation*}
\end{defn}

\begin{proof}[Construction]
  Using the universal property of the circle, we define $\mulcircle$ as the unique map $\sphere{1}\to(\sphere{1}\to\sphere{1})$ equipped with an identification
  \begin{equation*}
    \basemulcircle :\mulcircle(\base)=\idfunc
  \end{equation*}
  and an identification $\loopmulcircle$ witnessing that the square
  \begin{equation*}
    \begin{tikzcd}[column sep=huge]
      \mulcircle(\base) \arrow[r,equals,"\basemulcircle"] \arrow[d,equals,swap,"\ap{\mulcircle}{\lloop}"] & \idfunc \arrow[d,equals,"\eqhtpy(\htpyidcircle)"] \\
      \mulcircle(\base) \arrow[r,equals,swap,"\basemulcircle"] & \idfunc
  \end{tikzcd}
  \end{equation*}
  commutes. Note that in this square we have a homotopy $\htpyidcircle:\idfunc\htpy\idfunc$, which is not yet defined. We  use the dependent universal property of the circle with respect to the family $E_{\idfunc,\idfunc}$ given by
  \begin{equation*}
    E_{\idfunc,\idfunc}(x) \defeq (x=x),
  \end{equation*}
  to define $\htpyidcircle$ as the unique homotopy equipped with an identification
  \begin{equation*}
    \basehtpyidcircle : \htpyidcircle(\base)=\lloop
  \end{equation*}
  and an identification $\loophtpyidcircle$ witnessing that the square
  \begin{equation*}
    \begin{tikzcd}[column sep=8em]
      \tr_{E_{\idfunc,\idfunc}}(\lloop,\htpyidcircle(\base)) \arrow[r,equals,"\ap{\tr_{E_{\idfunc,\idfunc}}(\lloop)}{\basehtpyidcircle}"] \arrow[d,equals,swap,"\apd{\htpyidcircle}{\lloop}"] & \tr_{E_{\idfunc,\idfunc}}(\lloop,\lloop) \arrow[d,equals,"\gamma"] \\
      \htpyidcircle(\base) \arrow[r,equals,swap,"\basehtpyidcircle"] & \lloop
    \end{tikzcd}
  \end{equation*}
  commutes. Now it remains to define the path $\gamma:\tr_{E_{\idfunc,\idfunc}}(\lloop,\lloop)=\lloop$ in the above square. To proceed, we first observe that a simple path induction argument yields a function
  \begin{equation*}
    \Big(\ct{p}{r}=\ct{q}{p}\Big)\to\Big(\tr_{E_{\idfunc,\idfunc}}(p,q)=r\Big),
  \end{equation*}
  for any $p:\base=x$, $q:\base=\base$ and $r:x=x$. In particular, we have a function
  \begin{equation*}
    \Big(\ct{\lloop}{\lloop}=\ct{\lloop}{\lloop}\Big)\to\Big(\tr_{E_{\idfunc,\idfunc}}(\lloop,\lloop)=\lloop\Big).
  \end{equation*}
  Now we apply this function to $\refl{\ct{\lloop}{\lloop}}$ to obtain the desired identification
  \begin{equation*}
    \gamma:\tr_{E_{\idfunc,\idfunc}}(\lloop,\lloop)=\lloop.\qedhere
  \end{equation*}
\end{proof}

\begin{rmk}
  In the definition of $H:\idfunc\htpy\idfunc$ above, it is important that we didn't choose $H$ to be $\reflhtpy$. If we had done so, the resulting operation would be homotopic to $x,y\mapsto y$, which is clearly not what we had in mind with the multiplication operation on the circle. See also \cref{ex:circle-constant}.
\end{rmk}


The left unit law $\mulcircle(\base,x)=x$ holds by the computation rule of the universal property. More precisely, we define
\begin{equation*}
  \leftunit_{\sphere{1}}\defeq \htpyeq(\basemulcircle).
\end{equation*}
For the right unit law, however, we need to give a separate argument that is surprisingly involved, because all the aspects of the definition of $\mulcircle$ will come out and play their part.

\begin{thm}
  The multiplication operation on the circle satisfies the right unit law, i.e., we have
  \begin{equation*}
    \mulcircle(x,\base)=x
  \end{equation*}
  for any $x:\sphere{1}$.
\end{thm}

\begin{proof}
  The proof is by induction on the circle. In the base case we use the left unit law
  \begin{equation*}
    \leftunit_{\sphere{1}}(\base):\mulcircle(\base,\base)=\base.
  \end{equation*}
  Thus, it remains to show that
  \begin{equation*}
    \tr_P(\lloop,\leftunit_{\sphere{1}}(\base))=\leftunit_{\sphere{1}}(\base),
  \end{equation*}
  where $P$ is the family over the circle given by
  \begin{equation*}
    P(x) \defeq \mulcircle(x,\base)=x.
  \end{equation*}
  Now we observe that there is a function
  \begin{equation*}
    \Big(\ct{\htpyeq(\ap{\mulcircle}{p})(\base)}{r}=\ct{q}{p}\Big)\to\Big(\tr_{P}(p,q)=r\Big),
  \end{equation*}
  for any
  \begin{align*}
    p & : \base=x \\
    q & : \mulcircle(\base,\base)=\base \\
    r & : \mulcircle(x,\base)=x.
  \end{align*}
  Thus we see that, in order to construct an identification
  \begin{equation*}
    \tr_{P}(\lloop,\leftunit_{\sphere{1}})=\leftunit_{\sphere{1}},
  \end{equation*}
  it suffices to show that the square
  \begin{equation*}
    \begin{tikzcd}[column sep=8em]
      \mulcircle(\base,\base) \arrow[d,equals,swap,"\htpyeq(\ap{\mulcircle}{\lloop})(\base)"] \arrow[r,equals,"\leftunit_{\sphere{1}}(\base)"] & \base \arrow[d,equals,"\lloop"] \\
      \mulcircle(\base,\base) \arrow[r,equals,swap,"\leftunit_{\sphere{1}}(\base)"] & \base
    \end{tikzcd}
  \end{equation*}
  commutes. Now we note that we have an identification $H(\base)=\lloop$. It is indeed at this point, where it is important that $H$ is not the trivial homotopy, because now we can proceed by observing that the above square commutes if and only if the square
  \begin{equation*}
    \begin{tikzcd}[column sep=12em]
      \mulcircle(\base,\base) \arrow[d,equals,swap,"\htpyeq(\ap{\mulcircle}{\lloop})(\base)"] \arrow[r,equals,"\htpyeq(\basemulcircle)(\base)"] & \base \arrow[d,equals,"H(\base)"] \\
      \mulcircle(\base,\base) \arrow[r,equals,swap,"\htpyeq(\basemulcircle)(\base)"] & \base
    \end{tikzcd}
  \end{equation*}
  commutes. The commutativity of this square easily follows from the identification $\loopmulcircle$ constructed in \cref{defn:mul-circle}.
\end{proof}

\begin{exercises}
  \exercise \label{ex:circle-connected}
  \begin{subexenum}
  \item Let $P:\sphere{1}\to\prop$ be a family of propositions over the circle. Show that
    \begin{equation*}
      P(\base)\to\prd{x:\sphere{1}}P(x).
    \end{equation*}
    In this sense the circle is \emph{connected}.
  \item Show that any embedding $m:\sphere{1}\to\sphere{1}$ is an equivalence.
  \item Show that for any embedding $m:X\to\sphere{1}$, there is a proposition $P$ and an equivalence $e:\eqv{X}{\sphere{1}\times P}$ for which the triangle
    \begin{equation*}
      \begin{tikzcd}[column sep=0]
        X \arrow[dr,swap,"m"] \arrow[rr,"e"] & & \sphere{1}\times P \arrow[dl,"\proj 1"] \\
        \phantom{\sphere{1}\times P} & \sphere{1}
      \end{tikzcd}
    \end{equation*}
    commutes. In other words, all the embeddings into the circle are of the form $\sphere{1}\times P\to \sphere{1}$.
  \end{subexenum}
  \exercise \label{ex:circle-constant}
  Show that for any type $X$ and any $x:X$, the map
  \begin{equation*}
    \ind{\sphere{1}}(x,\refl{x}):\sphere{1}\to X
  \end{equation*}
  is homotopic to the constant map $\mathsf{const}_x$.
  \exercise \label{ex:mulcircle-is-equiv}
  \begin{subexenum}
  \item Show that for any $x:\sphere{1}$, both functions
    \begin{equation*}
      \mulcircle(x,\blank)\qquad\text{and}\qquad\mulcircle(\blank,x)
    \end{equation*}
    are equivalences.
  \item Show that the function
    \begin{equation*}
      \mulcircle : \sphere{1}\to(\sphere{1}\to\sphere{1})
    \end{equation*}
    is an embedding. Compare this fact with \cref{ex:groupop-embedding}.
  \item Show that multiplication on the circle is associative and commutative.
  \end{subexenum}
  \exercise \label{ex:circle_connected}
  \begin{subexenum}
  \item Show that a type $X$ is a set if and only if the map
    \begin{equation*}
      \lam{x}{t} x : X \to (\sphere{1}\to X)
    \end{equation*}
is an equivalence.
\item Show that a type $X$ is a set if and only if the map
  \begin{equation*}
    \lam{f}f(\base) : (\sphere{1}\to X)\to X
  \end{equation*}
  is an equivalence.
  \end{subexenum}
  \exercise Show that the multiplicative operation on the circle is commutative, i.e.~construct an identification
  \begin{equation*}
    \mulcircle(x,y)=\mulcircle(y,x).
  \end{equation*}
  for every $x,y:\sphere{1}$.
  \exercise Show that the circle, equipped with the multiplicative operation $\mulcircle$ is an abelian group, i.e.~construct an inverse operation
  \begin{equation*}
    \invcircle : \sphere{1}\to\sphere{1}
  \end{equation*}
  and construct identifications
  \begin{align*}
    \leftinv_{\sphere{1}} & : \mulcircle(\invcircle(x),x) = \base \\
    \rightinv_{\sphere{1}} & : \mulcircle(x,\invcircle(x)) = \base.
  \end{align*}
  Moreover, show that the square
  \begin{equation*}
    \begin{tikzcd}
      \invcircle(\base) \arrow[d,equals] \arrow[r,equals] & \mulcircle(\base,\invcircle(\base)) \arrow[d,equals] \\
      \mulcircle(\invcircle(\base),\base) \arrow[r,equals] & \base
    \end{tikzcd}
  \end{equation*}
  commutes.
  \exercise Show that for any multiplicative operation
  \begin{equation*}
    \mu:\sphere{1}\to(\sphere{1}\to\sphere{1})
  \end{equation*}
  that satisfies the condition that $\mu(x,\blank)$ and $\mu(\blank,x)$ are equivalences for any $x:\sphere{1}$, there is a term $e:\sphere{1}$ such that
  \begin{equation*}
    \mu(x,y)=\mulcircle(x,\mulcircle(\bar{e},y))
  \end{equation*}
  for every $x,y:\sphere{1}$, where $\bar{e}\defeq\invcircle(e)$ is the complex conjucation of $e$ on $\sphere{1}$.
\end{exercises}


\chapter{Homotopy pullbacks}

Suppose we are given a map $f:A\to B$, and type families $P$ over $A$, and $Q$ over $B$.
Then any fiberwise map
\begin{equation*}
g:\prd{x:A}P(x)\to Q(f(x))
\end{equation*}
gives rise to a commuting square
\begin{equation*}
\begin{tikzcd}[column sep=large]
\sm{x:A}P(x) \arrow[r,"{\total[f]{g}}"] \arrow[d,swap,"\proj 1"] & \sm{y:B}Q(y) \arrow[d,"\proj 1"] \\
A \arrow[r,swap,"f"] & B
\end{tikzcd}
\end{equation*}
where $\total[f]{g}$ is defined as $\lam{(x,p)}(f(x),g(x,y))$. 
We will show in \cref{thm:pb_fibequiv} that $g$ is a fiberwise equivalence if and only if this square is a \emph{pullback square}. This generalization of \cref{thm:fib_equiv} is therefore abstracting away from the notion of fiberwise equivalence, and it serves as our motivating theorem to introduce pullbacks. The connection between pullbacks and fiberwise equivalences has an important role in the descent theorem in \cref{chap:descent}.

\section{Cartesian squares}

Recall that a square
\begin{equation*}
\begin{tikzcd}
C \arrow[r,"q"] \arrow[d,swap,"p"] & B \arrow[d,"g"] \\
A \arrow[r,swap,"g"] & X
\end{tikzcd}
\end{equation*}
is said to \define{commute} if there is a homotopy $H:f\circ p\htpy g\circ q$. 
The pullback property is a \emph{universal property} of the upper left corner of a commuting square (in our case $C$), characterizing the maps \emph{into} it.

To describe the universal property of pullbacks we first need to have a closer look at the \emph{anatomy} of commuting squares.

\begin{defn}
A commuting square
\begin{equation*}
\begin{tikzcd}
C \arrow[r,"q"] \arrow[d,swap,"p"] & B \arrow[d,"g"] \\
A \arrow[r,swap,"f"] & X
\end{tikzcd}
\end{equation*}
with $H:f\circ p\htpy g\circ q$ can be dissected into three parts, consisting of a \emph{cospan}, a type, and a \emph{cone}, where
\begin{enumerate}
\item A \define{cospan}\index{cospan} consists of three types $A$, $X$, and $B$, and maps $f:A\to X$ and $g:B\to X$.
\item Given a type $C$, a \define{cone}\index{cone!on a cospan|textbf} on the cospan $A \stackrel{f}{\rightarrow} X \stackrel{g}{\leftarrow} B$ with \define{vertex} $C$\index{vertex!of a cone|textbf} consists of maps $p:C\to A$, $q:C\to B$ and a homotopy $H:f\circ p\htpy g\circ q$. We write\index{cone(C)@{$\mathsf{cone}(\blank)$}|textbf}
\begin{equation*}
\mathsf{cone}(C)\defeq \sm{p:C\to A}{q:C\to B}f\circ p\htpy g\circ q
\end{equation*}
for the type of cones with vertex $C$.
\end{enumerate}
\end{defn}

Given a cone with vertex $C$ on a span $A\stackrel{f}{\rightarrow} X \stackrel{g}{\leftarrow} B$ and a map $h:C'\to C$, we construct a new cone with vertex $C'$ in the following definition.

\begin{defn}
For any cone $(p,q,H)$ with vertex $C$ and any type $C'$, we define a map\index{cone_map@{$\mathsf{cone\usc{}map}$}|textbf}
\begin{equation*}
\mathsf{cone\usc{}map}(p,q,H):(C'\to C)\to\mathsf{cone}(C')
\end{equation*}
by $h\mapsto (p\circ h,q\circ h,H\circ h)$. 
\end{defn}

\begin{defn}
We say that a commuting square
\begin{equation*}
\begin{tikzcd}
C \arrow[r,"q"] \arrow[d,swap,"p"] & B \arrow[d,"g"] \\
A \arrow[r,swap,"f"] & X
\end{tikzcd}
\end{equation*}
with $H:f\circ p\htpy g\circ q$ is a \define{pullback square}\index{pullback square|textbf}, or that it is \define{cartesian}\index{cartesian square|textbf}, if it satisfies the \define{universal property} of pullbacks\index{universal property!of pullbacks}, which asserts that the map
\begin{equation*}
\mathsf{cone\usc{}map}(p,q,H):(C'\to C)\to\mathsf{cone}(C')
\end{equation*}
is an equivalence for every type $C'$. 
\end{defn}

We often indicate the universal property with a diagram as follows:
\begin{equation*}
\begin{tikzcd}
C' \arrow[drr,bend left=15,"{q'}"] \arrow[dr,densely dotted,"h"] \arrow[ddr,bend right=15,swap,"{p'}"] \\
& C \arrow[r,"q"] \arrow[d,swap,"p"] & B \arrow[d,"g"] \\
& A \arrow[r,swap,"f"] & X
\end{tikzcd}
\end{equation*}
since the universal property states that for every cone $(p',q',H')$ with vertex $C'$, the type of pairs $(h,\alpha)$ consisting of $h:C'\to C$ equipped with $\alpha:\mathsf{cone\usc{}map}((p,q,H),h)=(p',q',H')$ is contractible by \cref{thm:contr_equiv}.

In order to see what goes on in the universal property of pullbacks, we need to first characterize the identity type of $\mathsf{cone}(C)$, for any type $C$.

\begin{lem}\label{lem:id_cone}
Let $(p,q,H)$ and $(p',q',H')$ be cones on a cospan $f:A\rightarrow X \leftarrow B:g$, both with vertex $C$. Then the type $(p,q,H)=(p',q',H')$ is equivalent to the type of triples $(K,L,M)$ consisting of
\begin{align*}
K & : p \htpy p' \\
L & : q \htpy q' \\
M & : \ct{H}{(g\cdot L)} \htpy \ct{(f\cdot K)}{H'}
\end{align*}
\end{lem}

\begin{rmk}
Note for $z:C$, the identification $(\ct{H}{(g\cdot L)})(z)$ is the pointwise concatenation
\begin{equation*}
\begin{tikzcd}
f(p(z)) \arrow[r,equals,"H(z)"] & g(q(z)) \arrow[r,equals,"\ap{g}{L(z)}"] &[1.5em] g(q'(z)),
\end{tikzcd}
\end{equation*}
and the identification $(\ct{(f\cdot K)}{H'})(z)$ is the pointwise concatenation
\begin{equation*}
\begin{tikzcd}
f(p(z)) \arrow[r,equals,"\ap{f}{K(z)}"] &[1.5em] f(p'(z)) \arrow[r,equals,"{H'(z)}"] & g(q'(z)).
\end{tikzcd}
\end{equation*}
Therefore the homotopy $M:\ct{H}{(g\cdot L)} \htpy \ct{(f\cdot K)}{H'}$ shows that for each $z:C$, the \emph{square of identifications}
\begin{equation*}
\begin{tikzcd}[column sep=huge]
f(p(z)) \arrow[r,equals,"\ap{f}{K(z)}"] \arrow[d,equals,swap,"H(z)"] & f(p'(z)) \arrow[d,equals,"{H'(z)}"] \\
g(q(z)) \arrow[r,equals,swap,"\ap{g}{L(z)}"] & g(q'(z))
\end{tikzcd}
\end{equation*}
commutes, as witnessed by $M(z)$. Thus each $M(z)$ is a 2-cell in the sense that it is an identifications of identifications. 
\end{rmk}

\begin{proof}[Proof of \cref{lem:id_cone}]
By the fundamental theorem of identity types (\cref{thm:id_fundamental}) and associativity of $\Sigma$-types (\cref{ex:sigma_assoc}) it suffices to show that the type
\begin{equation*}
\sm{p':C\to A}{q':C\to B}{H':f\circ p'\htpy g\circ q}{K:p\htpy p'}{L:q\htpy q'} \ct{H}{(g\cdot L)} \htpy \ct{(f\cdot K)}{H'}
\end{equation*}
is contractible. Now we apply \cref{ex:sigma_swap} repeatedly to see that this type is equivalent to the type
\begin{equation*}
\sm{p':C\to A}{K: p\htpy p'}{q':C\to B}{L: q\htpy q'}{H':f\circ p'\htpy g\circ q} \ct{H}{(g\cdot L)} \htpy \ct{(f\cdot K)}{H'}.
\end{equation*}
The types $\sm{p':C\to A} p\htpy p'$ and $\sm{q':C\to B} q\htpy q'$ are contractible by function extensionality, and  we have
\begin{samepage}
\begin{align*}
(p,\mathsf{htpy.refl}_p) & : \sm{p':C'\to A} p\htpy p' \\
(q,\mathsf{htpy.refl}_q) & : \sm{q':C'\to B} q\htpy q'.
\end{align*}%
\end{samepage}%
Thus we apply \cref{ex:contr_in_sigma} to see that the type of tuples $(p',K,q',L,H',M)$ is equivalent to the type
\begin{equation*}
\sm{H':f\circ p'\htpy g\circ q} \ct{H}{\mathsf{htpy.refl}_{g\circ q}}\htpy \ct{\mathsf{htpy.refl}_{f\circ p}}{H'}.
\end{equation*}
Of course, the type $\ct{H}{\mathsf{htpy.refl}_{g\circ q}}\htpy \ct{\mathsf{htpy.refl}_{f\circ p}}{H'}$ is equivalent to the type $H\htpy H'$, and $\sm{H':f\circ p'\htpy g\circ q} H\htpy H'$ is contractible.
\end{proof}

As a corollary we obtain the following characterization of the universal property of pullbacks.

\begin{thm}\label{lem:pullback_up}
Consider a commuting square
\begin{equation*}
\begin{tikzcd}
C \arrow[r,"q"] \arrow[d,swap,"p"] & B \arrow[d,"g"] \\
A \arrow[r,swap,"f"] & X
\end{tikzcd}
\end{equation*}
with $H:f\circ p\htpy g\circ q$
Then the following are equivalent:
\begin{enumerate}
\item The square is a pullback square.
\item For every type $C'$ and every cone $(p',q',H')$ with vertex $C'$, the type of quadruples $(h,K,L,M)$ consisting of
\begin{align*}
h & : C'\to C \\
K & : p\circ h \htpy p' \\
L & : q\circ h \htpy q' \\
M & : \ct{(H\cdot h)}{(g\cdot L)} \htpy \ct{(f\cdot K)}{H'}
\end{align*}
is contractible.
\end{enumerate}
\end{thm}

\section{The unique existence of pullbacks}

\begin{defn}
Let $f:A\to X$ and $B\to X$ be maps. Then we define
\begin{align*}
A\times_X B & \defeq \sm{x:A}{y:B}f(x)=g(y) \\
\pi_1 & \defeq \proj 1 & & : A\times_X B\to A \\
\pi_2 & \defeq \proj 1\circ\proj 2 & & : A\times_X B\to B\\
\pi_3 & \defeq \proj 2\circ\proj 2 & & : f\circ \pi_1 \htpy g\circ\pi_2.
\end{align*}
The type $A\times_X B$ is called the \define{canonical pullback}\index{canonical pullback} of $f$ and $g$.
\end{defn}

Note that $A\times_X B$ depends on $f$ and $g$, although this dependency is not visible in the notation.

\begin{thm}
Given maps $f:A\to X$ and $g:B\to X$, the commuting square\index{canonical pullback|textit}
\begin{equation*}
\begin{tikzcd}
A\times_X B \arrow[r,"\pi_2"] \arrow[d,swap,"\pi_1"] & B \arrow[d,"g"] \\
A \arrow[r,swap,"f"] & X,
\end{tikzcd}
\end{equation*}
is a pullback square.
\end{thm}

\begin{proof}
Let $C$ be a type. Our goal is to show that the map
\begin{equation*}
\mathsf{cone\usc{}map}(\pi_1,\pi_2,\pi_3): (C\to A\times_X B)\to \mathsf{cone}(C)
\end{equation*}
is an equivalence. 
By double application of \cref{thm:choice} we obtain equivalences
\begin{align*}
(C\to A\times_X B) & \jdeq C\to \sm{x:A}{y:B}f(x)=g(y) \\
& \eqvsym \sm{p:C\to A}\prd{z:C}\sm{y:B} f(p(z))= y \\
& \eqvsym \sm{p:C\to A}{q:C\to B}\prd{z:C} f(p(z))= g(q(z)) \\
& \jdeq \mathsf{cone}(C)
\end{align*}
The composite of these equivalences is the map
\begin{equation*}
\lam{f}(\lam{z}\proj 1(f(z)),\lam{z} \proj 1(\proj 2(f(z))),\lam{z}\proj 2(\proj 2(f(z)))),
\end{equation*}
which is \emph{exactly} the map $\mathsf{cone\usc{}map}(\pi_1,\pi_2,\pi_3)$, and since it is a composite of equivalences it follows that it is itself an equivalence.
\end{proof}

In the following theorem we establish the uniqueness of pullbacks up to equivalence via a \emph{3-for-2 property} for pullbacks\index{pullback!3-for-2 property}.

\begin{thm}\label{thm:pb_3for2}
Consider the squares
\begin{equation*}
\begin{tikzcd}
C \arrow[r,"q"] \arrow[d,swap,"p"] & B \arrow[d,"g"] & {C'} \arrow[r,"{q'}"] \arrow[d,swap,"{p'}"] & B \arrow[d,"g"] \\
A \arrow[r,swap,"f"] & X & A \arrow[r,swap,"f"] & X
\end{tikzcd}
\end{equation*}
with homotopies $H:f\circ p \htpy g\circ q$ and $H':f\circ p'\htpy g\circ q'$.
Furthermore, suppose we have a map $h:C'\to C$ equipped with
\begin{align*}
K & : p\circ h \htpy p' \\
L & : q\circ h \htpy q' \\
M & : \ct{(H\cdot h)}{(g\cdot L)} \htpy \ct{(f\cdot K)}{H'}.
\end{align*}
If any two of the following three properties hold, so does the third:
\begin{samepage}%
\begin{enumerate}
\item $C$ is a pullback.
\item $C'$ is a pullback.
\item $h$ is an equivalence.
\end{enumerate}%
\end{samepage}%
\end{thm}

\begin{proof}
By the characterization of the identity type of $\mathsf{cone}(C')$ given in \cref{lem:id_cone} we obtain an identification
\begin{equation*}
\mathsf{cone\usc{}map}((p,q,H),h)=(p',q',H')
\end{equation*}
from the triple $(K,L,M)$. 
Let $D$ be a type, and let $k:D\to C'$ be a map. We observe that
\begin{align*}
\mathsf{cone\usc{}map}((p,q,H),(h\circ k)) & \jdeq (p\circ (h\circ k),q\circ (h\circ k),H\circ (h\circ k)) \\
& \jdeq ((p\circ h)\circ k,(q\circ h)\circ k, (H\circ h)\circ k) \\
& \jdeq \mathsf{cone\usc{}map}(\mathsf{cone\usc{}map}((p,q,H),h),k) \\
& = \mathsf{cone\usc{}map}((p',q',H'),k).
\end{align*}
Thus we see that the triangle 
\begin{equation*}
\begin{tikzcd}[column sep=-1em]
(D\to C') \arrow[rr,"{h\circ \blank}"] \arrow[dr,swap,"{\mathsf{cone\usc{}map}(p',q',H')}"] & & (D\to C) \arrow[dl,"{\mathsf{cone\usc{}map}(p,q,H)}"] \\
& \mathsf{cone}(D)
\end{tikzcd}
\end{equation*}
commutes. Therefore it follows from the 3-for-2 property of equivalences established in \cref{ex:3_for_2}, that if any two of the following properties hold, then so does the third:
\begin{enumerate}
\item The map $\mathsf{cone\usc{}map}(p,q,H):(D\to C)\to \mathsf{cone}(D)$ is an equivalence,
\item The map $\mathsf{cone\usc{}map}(p',q',H'):(D\to C')\to \mathsf{cone}(D)$ is an equivalence,
\item The map $h\circ\blank : (D\to C')\to (D\to C)$ is an equivalence.
\end{enumerate}
Thus the 3-for-2 property for pullbacks follows once we show that $h$ is an equivalence if and only if $h\circ\blank : (D\to C')\to (D\to C)$ is an equivalence for any type $D$. We establish this in \cref{lem:postcomp_equiv}.
\end{proof}

\begin{lem}\label{lem:postcomp_equiv}
Let $f:X\to Y$ be a map. Then the following are equivalent:
\begin{enumerate}
\item $f$ is an equivalence.
\item For any type $A$, the map $f\circ\blank : X^A\to Y^A$ is an equivalence.
\end{enumerate}
\end{lem}

\begin{proof}
If $f$ is an equivalence, then it is straightforward to see that the map
\begin{equation*}
f^{-1}\circ\blank : Y^A\to X^A
\end{equation*}
is an inverse of $f\circ \blank : X^A \to Y^A$, for any type $A$.

For the converse, we use that the fibers of $f\circ \blank$ are contractible, for any $A$. In particular, the fiber
\begin{equation*}
\fib{f\circ\blank}{\idfunc[Y]}\jdeq \sm{g:Y\to X} f\circ g=\idfunc[Y]
\end{equation*}
is contractible (choosing $A\jdeq Y$). Thus we obtain a function $g:Y\to X$ and a homotopy $G:f\circ g\htpy \idfunc[Y]$.

It remains to be shown that $g$ is a retract of $f$, i.e.~to construct a homotopy $g\circ f\htpy \idfunc[X]$. To see this, we use that the fiber
\begin{equation*}
\fib{f\circ\blank}{f}\jdeq \sm{h:X\to X} f\circ h=f
\end{equation*}
is contractible (choosing $A\jdeq X$). Of course we have $(\idfunc[X],\refl{f})$ in this fiber. However we claim that there also is an identification $p:f\circ (g\circ f)=f$, showing that $(g\circ f,p)$ is in this fiber. To see this, note that
\begin{align*}
f\circ (g\circ f) & \jdeq (f\circ g)\circ f \\
& = \idfunc[Y]\circ f \\
& \jdeq f
\end{align*}
By the contractibility of the fiber we conclude that $(\idfunc[X],\refl{f})=(g\circ f,p)$, so it follows that $\idfunc[X]=g\circ f$.
\end{proof}

Pullbacks are not only unique in the sense that any two pullbacks of the same cospan are equivalent, they are \emph{uniquely unique} in the sense that 

\begin{cor}
Suppose both commuting squares
\begin{equation*}
\begin{tikzcd}
C \arrow[r,"q"] \arrow[d,swap,"p"] & B \arrow[d,"g"] & {C'} \arrow[r,"{q'}"] \arrow[d,swap,"{p'}"] & B \arrow[d,"g"] \\
A \arrow[r,swap,"f"] & X & A \arrow[r,swap,"f"] & X
\end{tikzcd}
\end{equation*}
with homotopies $H:f\circ p \htpy g\circ q$ and $H':f\circ p'\htpy g\circ q'$ are pullback squares.
Then the type of quadruples $(e,K,L,M)$ consisting of an equivalence $e:\eqv{C'}{C}$ equipped with
\begin{align*}
K & : p\circ e \htpy p' \\
L & : q\circ e \htpy q' \\
M & : \ct{(g\cdot L)}{(H\cdot e)} \htpy \ct{(f\cdot K)}{H'}.
\end{align*}
is contractible.
\end{cor}

\begin{proof}
We have seen that the type of quadruples $(h,K,L,M)$ is equivalent to the fiber of $\mathsf{cone\usc{}map}(p,q,H)$ at $(p',q',H')$. By \cref{thm:pb_3for2} it follows that $h$ is an equivalence. Since $\isequiv(h)$ is a proposition (and hence contractible as soon as it is inhabited) it follows that the type of quadruples $(e,K,L,M)$ is contractible. 
\end{proof}

\section{Fiber products}

An important special case of pullbacks occurs when the cospan is of the form
\begin{equation*}
\begin{tikzcd}
A \arrow[r] & \unit & B. \arrow[l]
\end{tikzcd}
\end{equation*}
In this case, the pullback is just the \emph{cartesian product}.

\begin{lem}\label{lem:prod_pb}
Let $A$ and $B$ be types. Then the square
\begin{equation*}
\begin{tikzcd}
A\times B \arrow[r,"\proj 2"] \arrow[d,swap,"\proj 1"] & B \arrow[d,"\mathsf{const}_{\ttt}"] \\
A \arrow[r,swap,"\mathsf{const}_{\ttt}"] & \unit
\end{tikzcd}
\end{equation*}
which commutes by the homotopy $\mathsf{const}_{\refl{\ttt}}$ is a pullback square.
\end{lem}

\begin{proof}
By \cref{thm:pb_3for2} it suffices to construct an equivalence 
\begin{equation*}
e:\eqv{(A\times B)}{(A\times_\unit B)}
\end{equation*}
equipped with homotopies
\begin{align*}
K & : \pi_1\circ e \htpy \proj 1 \\
L & : \pi_2\circ e \htpy \proj 2 \\
M & : \ct{(\pi_3\cdot e)}{(\mathsf{const}_\ttt\cdot L)}\htpy \ct{(\mathsf{const}_\ttt\cdot K)}{\mathsf{const}_{\refl{\ttt}}}.
\end{align*}

The equivalence $e:\eqv{(A\times B)}{(A\times_\unit B)}$ is given by $\lam{(a,b)}(a,b,\refl{\ttt})$. Its inverse is the map $\lam{(a,b,p)}(a,b)$. The homotopies $K$, $L$, and $M$ are given by
\begin{align*}
K & \defeq \lam{(a,b)}\refl{a} \\
L & \defeq \lam{(a,b)}\refl{b} \\
M & \defeq \lam{(a,b)}\refl{\refl{\ttt}}\qedhere
\end{align*}
\end{proof}

The following generalization of \cref{lem:prod_pb} is the reason why pullbacks are sometimes called \define{fiber products}.

\begin{thm}
Let $P$ and $Q$ be families over a type $X$. Then the square
\begin{equation*}
\begin{tikzcd}[column sep=8em]
\sm{x:X}P(x)\times Q(x) \arrow[r,"{\lam{(x,(p,q))}(x,q)}"] \arrow[d,swap,"{\lam{(x,(p,q))}(x,p)}"] & \sm{x:X}Q(x) \arrow[d,"\proj 1"] \\
\sm{x:X}P(x) \arrow[r,swap,"\proj 1"] & X
\end{tikzcd}
\end{equation*}
is a pullback square.
\end{thm}

\begin{proof}
The square commutes by the homotopy
\begin{equation*}
H\defeq \lam{(x,(p,q))}\refl{x}.
\end{equation*}
To show that the stated square is a pullback square we will construct an equivalence
\begin{equation*}
e:\eqv{\Big(\sm{x:X}P(x)\times Q(x)\Big)}{\Big(\sm{x:X}P(x)\Big)\times_X \Big(\sm{x:X}Q(x)\Big)}
\end{equation*}
equipped with homotopies
\begin{align*}
K & : \pi_1\circ e \htpy \lam{(x,(p,q))}(x,p) \\
L & : \pi_2\circ e \htpy \lam{(x,(p,q))}(x,q) \\
M & : \ct{(\pi_3\cdot e)}{(\proj 1\cdot L)} \htpy \ct{(\proj 1\cdot K)}{H}.
\end{align*}
We define $e(x,(p,q))\defeq ((x,p),(x,q),\refl{x})$. The inverse of $e$ is defined as 
\begin{equation*}
\lam{((x,p),(y,q),\alpha)}(y,(\mathsf{tr}_P(\alpha,p),q)).
\end{equation*}
It is straightforward to see that this is indeed an inverse of $e$.

Then we define
\begin{align*}
K & \defeq \lam{(x,(p,q))}\refl{(x,p)} \\
L & \defeq \lam{(x,(p,q))}\refl{(x,q)}.
\end{align*}
Then we have homotopies
\begin{align*}
\proj 1\cdot K & \htpy \lam{(x,(p,q))}\refl{x} \\
\proj 1\cdot L & \htpy \lam{(x,(p,q))}\refl{x} \\
\pi_3\cdot e & \htpy \lam{(x,(p,q))}\refl{x}.
\end{align*}
Therefore the type of $M$ is equivalent to the type of homotopies
\begin{equation*}
\ct{(\lam{(x,(p,q))}\refl{x})}{(\lam{(x,(p,q))}\refl{x})} \htpy \ct{(\lam{(x,(p,q))}\refl{x})}{(\lam{(x,(p,q))}\refl{x})}
\end{equation*}
Here we have the homotopy $\lam{(x,(p,q))}\refl{\refl{x}}$.
\end{proof}

\begin{cor}
For any $f:A\to X$ and $g:B\to X$, the square
\begin{equation*}
\begin{tikzcd}[column sep=8em]
\sm{x:X}\fib{f}{x}\times\fib{g}{y} \arrow[r,"{\lam{(x,((a,p),(b,q)))}b}"] \arrow[d,swap,"{\lam{(x,((a,p),(b,q)))}a}"] & B \arrow[d,"g"]  \\
A \arrow[r,swap,"f"] & X
\end{tikzcd}
\end{equation*}
is a pullback square.
\end{cor}

\section{Fiber sequences}

\begin{lem}\label{lem:fib_pb}
For any function $f:A\to B$, and any $b:B$, consider the square
\begin{equation*}
\begin{tikzcd}[column sep=large]
\fib{f}{b} \arrow[r,"\mathsf{const}_\ttt"] \arrow[d,swap,"\proj 1"] & \unit \arrow[d,"\mathsf{const}_b"] \\
A \arrow[r,swap,"f"] & B
\end{tikzcd}
\end{equation*}
which commutes by $\proj 2 : \prd{t:\fib{f}{b}} f(\proj 1(t))=b$. This is a pullback square.
\end{lem}

\begin{proof}
To see this, note that by \cref{thm:pb_3for2} it suffices to construct an equivalence $e:\eqv{\fib{f}{b}}{A\times_B \unit}$, and homotopies
\begin{align*}
K & : \pi_1\circ e \htpy \proj 1 \\
L & : \pi_2\circ e \htpy \mathsf{const}_\ttt \\
M & : \ct{(\pi_3 \cdot e)}{(\mathsf{const}_b \cdot L)} \htpy \ct{(f \cdot K)}{\proj 2}.
\end{align*}

The equivalence $e:\eqv{\fib{f}{b}}{A\times_B \unit}$ is defined to be the map $\total{\varphi}$, where 
\begin{equation*}
\varphi : \prd{x:A} (f(x)=b)\to \sm{t:\unit} f(x)=b
\end{equation*}
is given by $\varphi_x(p)\defeq (\ttt,p)$. This is a fiberwise equivalence by \cref{ex:contr_in_sigma}, so $e$ is an equivalence by \cref{thm:fib_equiv}. Then we have
\begin{align*}
K\defeq \lam{(a,p)}\refl{a} & : \pi_1\circ e \htpy \proj 1 \\
L \defeq  \lam{(a,p)}\refl{\ttt} & : \pi_2\circ e \htpy \mathsf{const}_\ttt.
\end{align*}
Now we observe that there are homotopies
\begin{align*}
f\cdot K & \htpy \lam{(a,p)} \refl{f(a)} \\
\mathsf{const}_b \cdot L & \htpy \lam{(a,p)}\refl{b} \\
\pi_3\cdot e & \htpy \proj 2.
\end{align*}
Therefore we obtain the homotopy $M$ by constructing a homotopy
\begin{equation*}
\ct{\proj 2}{(\lam{(a,p)}\refl{b})} \htpy \ct{(\lam{(a,p)} \refl{f(a)})}{\proj 2}.
\end{equation*}
We obtain this homotopy from the left and right unit laws of identity types.
\end{proof}

\cref{lem:fib_pb} motivates the following definition of \emph{fiber sequences}, which play an important role in synthetic homotopy theory (and in algebraic topology). 

\begin{defn}
A \define{fiber sequence} consists of types $F$, $E$, and $B$ with \define{base points} $x:F$, $y:E$, and $b:B$, and maps
\begin{equation*}
\begin{tikzcd}
F \arrow[r,"i"] & E \arrow[r,"p"] & B
\end{tikzcd}
\end{equation*}
preserving the base points in the sense that $i(x)=y$ and $p(y)=b$, such that the square
\begin{equation*}
\begin{tikzcd}
F \arrow[r,"i"] \arrow[d] & E \arrow[d,"p"] \\
\unit \arrow[r,swap,"b"] & B
\end{tikzcd}
\end{equation*}
is a pullback square. We often write $F\hookrightarrow E \twoheadrightarrow B$ to indicate that we have a fiber sequence. 

Given a fiber sequence $F\hookrightarrow E\twoheadrightarrow B$, we call $B$ the \define{base space}, $E$ the \define{total space}, and $F$ the \define{fiber}.
\end{defn}

\begin{eg}
For any type family $B$ over $A$ and any $a:A$ the square
\begin{equation*}
\begin{tikzcd}[column sep=large]
B(a) \arrow[d,swap,"{\lam{y}(a,y)}"] \arrow[r,"\mathsf{const}_\ttt"] & \unit \arrow[d,"\lam{\ttt}a"] \\
\sm{x:A}B(x) \arrow[r,swap,"\proj 1"] & A
\end{tikzcd}
\end{equation*}
is a pullback square. 

To see this we have to construct a equivalence $e:\eqv{B(a)}{\fib{\proj 1}{a}}$ equipped with homotopies
\begin{align*}
K & : \proj 1\circ e \htpy \lam{y}(a,y) \\
L & : \mathsf{const}_\ttt \circ e \htpy \mathsf{const}_\ttt \\
M & : \ct{(\proj 2\circ e)}{((\lam{\ttt}a)\cdot L)} \htpy \ct{(\proj 1\cdot K)}{(\lam{y}\refl{a})}
\end{align*}
We define
\begin{equation*}
e \defeq \lam{y}((a,y),\refl{a})
\end{equation*}
This function is an equivalence by \cref{ex:proj_fiber}. For the homotopies we take
\begin{align*}
K & \defeq \lam{y}\refl{(a,y)} \\
L & \defeq \lam{y}\refl{\ttt} \\
M & \defeq \lam{y}\refl{\refl{a}}.
\end{align*}
This concludes the proof that the asserted square is a pullback square.

Thus we see that if we additionally suppose that there is a term $b:B(a)$, then we obtain a fiber sequence
\begin{equation*}
\begin{tikzcd}
B(a) \arrow[r,hookrightarrow] & \sm{x:A}B(x) \arrow[r,->>] & A.
\end{tikzcd}
\end{equation*}
\end{eg}

\section{Fiberwise equivalences}

\begin{lem}\label{lem:pb_subst}
Let $f:A\to B$, and let $Q$ be a type family over $B$. Then the square
\begin{equation*}
\begin{tikzcd}[column sep=6em]
\sm{x:A}Q(f(x)) \arrow[r,"{\lam{(x,q)}(f(x),q)}"] \arrow[d,swap,"\proj 1"] & \sm{y:B}Q(b) \arrow[d,"\proj 1"] \\
A \arrow[r,swap,"f"] & B
\end{tikzcd}
\end{equation*}
commutes by $H\defeq \lam{(x,q)}\refl{f(x)}$. This is a pullback square.
\end{lem}

\begin{proof}
By \cref{thm:pb_3for2} it suffices to construct an equivalence 
\begin{equation*}
e:\eqv{\Big(\sm{x:A}Q(f(x))\Big)}{A\times_B \Big(\sm{y:B}Q(y)\Big)},
\end{equation*}
and homotopies
\begin{align*}
K & : \pi_1\circ e \htpy \proj 1 \\
L & : \pi_2\circ e \htpy \lam{(x,q)}(f(x),q) \\
M & : \ct{(\pi_3 \cdot e)}{(\proj 1 \cdot L)} \htpy \ct{(f \cdot K)}{H}.
\end{align*}
We define $e$ by
The equivalence $e$ is defined by $\Sigma$-induction, taking
\begin{equation*}
e \defeq \lam{(x,q)}(x,(f(x),q),\refl{f(x)}).
\end{equation*}
The inverse of this map is given by $\lam{(x,((y,q),p))}(x,\mathsf{tr}_Q(p^{-1},q))$, and it is straigthforward to see that these maps are indeed mutual inverses.

Now we define 
\begin{align*}
K \defeq \lam{(x,q)}\refl{x} \\
L \defeq \lam{(x,q)}\refl{(f(x),q)}.
\end{align*}
Then we observe that there are homotopies
\begin{align*}
f \cdot K & \htpy \lam{(x,q)} \refl{f(x)} \\
\proj 1 \cdot L & \htpy \lam{(x,q)} \refl{f(x)} \\
\pi_3 \cdot e & \htpy \lam{(x,q)} \refl{f(x)}.
\end{align*}
Therefore the type of $M$ is equivalent to the type of homotopies
\begin{equation*}
\ct{(\lam{(x,q)} \refl{f(x)})}{(\lam{(x,q)} \refl{f(x)})} \htpy \ct{(\lam{(x,q)} \refl{f(x)})}{(\lam{(x,q)} \refl{f(x)})},
\end{equation*}
and here we have the homotopy $\lam{(x,q)}\refl{\refl{f(x)}}$. 
\end{proof}

\begin{thm}\label{thm:pb_fibequiv}
Let $f:A\to B$, and let $g:\prd{a:A}P(a)\to Q(f(a))$ be a fiberwise transformation\index{fiberwise transformation|textit}. The following are equivalent:
\begin{enumerate}
\item The commuting square
\begin{equation*}
\begin{tikzcd}[column sep=large]
\sm{a:A}P(a) \arrow[r,"{\total[f]{g}}"] \arrow[d,->>] & \sm{b:B}Q(b) \arrow[d,->>] \\
A \arrow[r,swap,"f"] & B
\end{tikzcd}
\end{equation*}
is a pullback square.
\item $g$ is a fiberwise equivalence.
\end{enumerate}
\end{thm}

\begin{proof}
Note that we have the map
\begin{equation*}
\total{g}:\Big(\sm{x:A}P(x)\Big)\to\Big(\sm{x:A}Q(f(x))\Big),
\end{equation*}
and homotopies
\begin{align*}
K & : \proj 1\circ \total{g} \htpy \proj 1 \\
L & : (\lam{(x,q)}(f(x),q))\circ \total{g}\htpy \total[f]{g} \\
M & : \ct{((\lam{(x,q)}\refl{f(x)})\cdot \total{g})}{(\proj 1\cdot L)} \htpy \ct{(f\cdot K)}{(\lam{(x,p)}\refl{f(x)})}
\end{align*}
defined by
\begin{align*}
K & \defeq \lam{(x,p)}\refl{x} \\
L & \defeq \lam{(x,p)}\refl{(f(x),g(x,p))} \\
M & \defeq \lam{(x,p)}\refl{f(x)}.
\end{align*}
Therefore we are in the situation of \cref{thm:pb_3for2}, with the commuting squares
\begin{equation*}
\begin{tikzcd}
\sm{x:A}Q(f(x)) \arrow[r] \arrow[d,swap,"\proj 1"] & \sm{y:B}Q(b) \arrow[d,"\proj 1"] &[-1em] \sm{a:A}P(a) \arrow[r,"{\total[f]{g}}"] \arrow[d,->>] &[1.5em] \sm{b:B}Q(b) \arrow[d,->>] \\
A \arrow[r,swap,"f"] & B & A \arrow[r,swap,"f"] & B.
\end{tikzcd}
\end{equation*}
Since the square on the left is a pullback by \cref{lem:pb_subst}, it follows that the square on the right is a pullback if and only if $\total{g}$ is an equivalence. By \cref{thm:fib_equiv} we know that $\total{g}$ is an equivalence if and only if $g$ is a fiberwise equivalence.
\end{proof}

\begin{cor}\label{cor:pb_fibequiv}
Consider a commuting square
\begin{equation*}
\begin{tikzcd}
C \arrow[r,"q"] \arrow[d,swap,"p"] & B \arrow[d,"g"] \\
A \arrow[r,swap,"f"] & X
\end{tikzcd}
\end{equation*}
with $H:f\circ p\htpy g\circ q$. The following are equivalent:
\begin{enumerate}
\item The square is a pullback square.
\item The induced map on fibers
\begin{equation*}
\lam{x}{(z,\alpha)}(q(z),\ct{H(z)^{-1}}{\ap{f}{\alpha}}):\prd{x:A}\fib{p}{x}\to \fib{g}{f(x)}
\end{equation*}
is a fiberwise equivalence.
\end{enumerate}
\end{cor}

\begin{cor}
Consider a pullback square
\begin{equation*}
\begin{tikzcd}
C \arrow[r,"q"] \arrow[d,swap,"p"] & B \arrow[d,"g"] \\
A \arrow[r,swap,"f"] & X.
\end{tikzcd}
\end{equation*}
If $g$ is a $k$-truncated map, then so is $p$.
\end{cor}

\begin{proof}
Since the square is assumed to be a pullback square, it follows from \cref{cor:pb_fibequiv} that for each $x:A$, the fiber $\fib{p}{x}$ is equivalent to the fiber $\fib{g}{f(x)}$, which is $k$-truncated. Since $k$-truncated types are closed under equivalences by \cref{thm:ktype_eqv}, it follows that $p$ is a $k$-truncated map.
\end{proof}

\begin{cor}\label{cor:pb_equiv}
Consider a commuting square
\begin{equation*}
\begin{tikzcd}
C \arrow[r,"q"] \arrow[d,swap,"p"] & B \arrow[d,"g"] \\
A \arrow[r,swap,"f"] & X.
\end{tikzcd}
\end{equation*}
and suppose that $g$ is an equivalence. Then the following are equivalent:
\begin{enumerate}
\item The square is a pullback square.
\item The map $p:C\to A$ is an equivalence.
\end{enumerate}
\end{cor}

\begin{proof}
If the square is a pullback square, then by \cref{thm:pb_fibequiv} the fibers of $p$ are equivalent to the fibers of $g$, which are contractible by \cref{thm:contr_equiv}. Thus it follows that $p$ is a contractible map, and hence that $p$ is an equivalence.

If $p$ is an equivalence, then by \cref{thm:contr_equiv} both $\fib{p}{x}$ and $\fib{g}{f(x)}$ are contractible for any $x:X$. It follows by \cref{ex:contr_equiv} that the induced map $\fib{p}{x}\to\fib{g}{f(x)}$ is an equivalence. Thus we apply \cref{cor:pb_fibequiv} to conclude that the square is a pullback.
\end{proof}

\section{The pullback pasting property}

\begin{thm}\label{thm:pb_pasting}
Consider a commuting diagram of the form
\begin{equation*}
\begin{tikzcd}
A \arrow[r,"k"] \arrow[d,swap,"f"] & B \arrow[r,"l"] \arrow[d,"g"] & C \arrow[d,"h"] \\
X \arrow[r,swap,"i"] & Y \arrow[r,swap,"j"] & Z
\end{tikzcd}
\end{equation*}
with homotopies $H:i\circ f\htpy g\circ k$ and $K:j\circ g\htpy h\circ l$, and the homotopy
\begin{equation*}
\ct{jH}{Kk}:j\circ i\circ f\htpy h\circ l\circ k
\end{equation*}
witnessing that the outer rectangle commutes. Furthermore, suppose that the square on the right is a pullback square. Then the following are equivalent:
\begin{samepage}%
\begin{enumerate}
\item The square on the left is a pullback square.
\item The outer rectangle is a pullback square.
\end{enumerate}%
\end{samepage}%
\end{thm}

\begin{proof}
The commutativity of the two squares induces fiberwise transformations
\begin{align*}
& \prd{x:X}\fib{f}{x}\to \fib{g}{i(x)} \\
& \prd{y:Y}\fib{g}{y}\to \fib{h}{j(y)}.
\end{align*}
By the assumption that the square on the right is a pullback square, it follows from \cref{cor:pb_fibequiv} that the fiberwise transformation
\begin{equation*}
\prd{y:Y}\fib{g}{y}\to\fib{h}{j(y)}
\end{equation*}
is a fiberwise equivalence. Therefore it follows from 3-for-2 property of equivalences that the fiberwise transformation
\begin{equation*}
\prd{x:X}\fib{f}{x}\to\fib{g}{i(x)}
\end{equation*}
is a fiberwise equivalence if and only if the fiberwise transformation
\begin{equation*}
\prd{x:X}\fib{f}{x}\to\fib{h}{j(i(x))}
\end{equation*}
is a fiberwise equivalence. Now the claim follows from one more application of \cref{cor:pb_fibequiv}.
\end{proof}

\section{The disjointness of coproducts}

As an application of the theory of pullbacks, we show that coproducts are disjoint.

\begin{lem}
Let $X$ be a type. Then we have the pullback squares
\begin{equation*}
\begin{tikzcd}
X \arrow[r,"\mathsf{const}_\ttt"] \arrow[d,swap,"\idfunc"] &[2em] \unit \arrow[d,"\mathsf{const}_{\bfalse}"] & \emptyt \arrow[r] \arrow[d] &[2em] \unit \arrow[d,"\mathsf{const}_{\btrue}"] \\
X \arrow[r,swap,"\mathsf{const}_{\bfalse}"] & \bool & X \arrow[r,swap,"\mathsf{const}_{\bfalse}"] & \bool,
\end{tikzcd}
\end{equation*}
and we have similar pullback squares with the roles of $\bfalse$ and $\btrue$ reversed.
\end{lem}

\begin{proof}
For the first square we observe that both squares and the outer rectangle in the diagram
\begin{equation*}
\begin{tikzcd}[column sep=large]
X \arrow[d] \arrow[r] & \unit \arrow[d] \arrow[r] & \unit \arrow[d,"\mathsf{const}_{\bfalse}"] \\
X \arrow[r,swap,"\mathsf{const}_\ttt"] & \unit \arrow[r,swap,"\mathsf{const}_{\bfalse}"] & \bool.
\end{tikzcd}
\end{equation*}
are pullback squares. To see this, recall that the identity type $\bfalse=\bfalse$ is contractible by \cref{ex:eq_bool}. Therefore it follows that the square on the right is a pullback square by \cref{ex:id_pb}. The square on the left is a pullback square by \cref{cor:pb_equiv}. Therefore the outer rectangle is a pullback square by \cref{thm:pb_pasting}.

For the second square we observe that both squares end the outer rectangle in the diagram
\begin{equation*}
\begin{tikzcd}[column sep=large]
\emptyt \arrow[d] \arrow[r] & \emptyt \arrow[d] \arrow[r] & \unit \arrow[d,"\mathsf{const}_{\btrue}"] \\
X \arrow[r,swap,"\mathsf{const}_\ttt"] & \unit \arrow[r,swap,"\mathsf{const}_{\bfalse}"] & \bool.
\end{tikzcd}
\end{equation*}
are pullback squares.
To see this, recall that the identity type $\bfalse=\btrue$ is equivalent to the empty type by \cref{ex:eq_bool}. Therefore it follows that the square on the right is a pullback. It is also straightforward to verify that the square on the left is a pullback. Therefore it follows from \cref{thm:pb_pasting} that the outer rectangle is a pullback.
\end{proof}

\begin{lem}
For any two types $A$ and $B$, the square
\begin{equation*}
\begin{tikzcd}[column sep=huge]
A \arrow[r,"\mathsf{const}_\ttt"] \arrow[d,swap,"\inl"] & \unit \arrow[d,"\mathsf{const}_{\bfalse}"] \\
A+B \arrow[r,swap,"\mathsf{const}_{\bfalse}+\mathsf{const}_{\btrue}" & \bool
\end{tikzcd}
\end{equation*}
is a pullback square.
\end{lem}

\begin{proof}
The square commutes by the homotopy
\begin{equation*}
H\defeq \mathsf{const}_{\refl{\bfalse}}.
\end{equation*}
To see that the asserted square is a pullback square we will apply \cref{thm:pb_3for2} and construct an equivalence
\begin{equation*}
e:\eqv{A}{(A+B)\times_\bool\unit}
\end{equation*}
equipped with homotopies
\begin{align*}
K & : \pi_1\circ e\htpy \inl \\
L & : \pi_2\circ e\htpy \mathsf{const}_\ttt \\
M & : \ct{(\pi_3\cdot e)}{(\mathsf{const}_{\bfalse})}\htpy \ct{((\mathsf{const}_{\bfalse}+\mathfs{const}_{\btrue})\cdot K)}{(\mathsf{const}_{\refl{\bfalse}})}.
\end{align*}
The map $e:A\to (A+B)\times_\bool\unit$ is defined by
\begin{equation*}
\lam{x}(\inl(x),\ttt,\refl{\bfalse}).
\end{equation*}
\end{proof}

\begin{thm}
For any two types $A$ and $B$, the commuting square
\begin{equation*}
\begin{tikzcd}
\emptyt \arrow[r] \arrow[d] & B \arrow[d,"\inr"] \\
A \arrow[r,swap,"\inl"] & A+B
\end{tikzcd}
\end{equation*}
is a pullback square.
\end{thm}

\begin{proof}
Now consider the commuting diagram
\begin{equation*}
\begin{tikzcd}
\emptyt \arrow[d] \arrow[r] & B \arrow[d,"\inr"] \arrow[r] &[5.5em] \unit \arrow[d,"\lam{\ttt}\inr(\ttt)"] \\
A \arrow[r,swap,"\inl"] & A+B \arrow[r,swap,"{\mathsf{const}_{\inl(\ttt)}+\mathsf{const}_{\inr(\ttt)}}"] & \unit + \unit.
\end{tikzcd}
\end{equation*}
By the first observation the outer rectangle is a pullback square. By the second observation the square on the right is a pullback square. Therefore the square on the left is a pullback square by \cref{thm:pb_pasting}.
\end{proof}

\begin{cor}\label{cor:id_coprod}
Let $A$ and $B$ be types. There are equivalences
\begin{align*}
(\inl(x)=\inl(x')) & \eqvsym (x=_A x') \\
(\inl(x)=\inr(y')) & \eqvsym \emptyt \\
(\inr(y)=\inl(x')) & \eqvsym \emptyt \\
(\inr(y)=\inr(y')) & \eqvsym (y=_B y').
\end{align*}
\end{cor}

\begin{exercises}
\item \label{ex:id_pb}\index{identity type!as pullback}
\begin{subexenum}
\item Show that the square
\begin{equation*}
\begin{tikzcd}
(x=y) \arrow[r] \arrow[d] & \unit \arrow[d,"\mathsf{const}_y"] \\
\unit \arrow[r,swap,"\mathsf{const}_x"] & A
\end{tikzcd}
\end{equation*}
is a pullback square.
\item Show that the square
\begin{equation*}
\begin{tikzcd}[column sep=large]
(x=y) \arrow[r,"\mathsf{const}_{x}"] \arrow[d,swap,"\mathsf{const}_\ttt"] & A \arrow[d,"\delta_A"] \\
\unit \arrow[r,swap,"{\ind{\unit}((x,y))}"] & A\times A
\end{tikzcd}
\end{equation*}
is a pullback square, where $\delta_A:A\to A\times A$ is the diagonal of $A$, defined in \cref{ex:diagonal}.
\end{subexenum}
\item In this exercise we give an alternative characterization of the notion of $k$-truncated map, compared to \cref{thm:trunc_ap} Given a map $f:A\to X$ define $\delta_f:A\to A\times_X A$ by $x\mapsto (x,x,\refl{f(x)})$.
\begin{subexenum}
\item Show that the square
\begin{equation*}
\begin{tikzcd}[column sep=large]
\fib{\apfunc{f}}{p} \arrow[r,"\mathsf{const}_x"] \arrow[d,swap,"\mathsf{const}_\ttt"] & A \arrow[d,"\delta_f"] \\
\unit \arrow[r,swap,"{\ind{\unit}((x,y,p))}"] & A\times_X A
\end{tikzcd}
\end{equation*}
is a pullback square, to obtain an equivalence
\begin{equation*}
\eqv{\fib{\delta_f}{(x,y,p)}}{\fib{\apfunc{f}}{p}}
\end{equation*}
for every $x,y:A$ and $p:f(x)=f(y)$.
\item Show that a map $f:A\to X$ is $(k+1)$-truncated if and only if $\delta_f$ is $k$-truncated.
\end{subexenum}
Conclude that $f$ is an embedding if and only if $\delta_f$ is an equivalence.
\item Consider a commuting square
\begin{equation*}
\begin{tikzcd}
C \arrow[r,"q"] \arrow[d,swap,"p"] & B \arrow[d,"g"] \\
A \arrow[r,swap,"f"] & X
\end{tikzcd}
\end{equation*}
with $H:f\circ p\htpy g\circ q$. Show that the following are equivalent:
\begin{enumerate}
\item The square is a pullback square.
\item For every type $D$, the commuting square
\begin{equation*}
\begin{tikzcd}
C^D \arrow[r,"q\circ\blank"] \arrow[d,swap,"p\circ\blank"] & B^D \arrow[d,"g\circ\blank"] \\
A^D \arrow[r,swap,"f\circ\blank"] & X^D
\end{tikzcd}
\end{equation*}
is a pullback square.
\end{enumerate}
\item Consider a commuting square
\begin{equation*}
\begin{tikzcd}
C \arrow[r,"q"] \arrow[d,swap,"p"] & B \arrow[d,"g"] \\
A \arrow[r,swap,"f"] & X
\end{tikzcd}
\end{equation*}
with $H:f\circ p\htpy g\circ q$. Show that the following are equivalent:
\begin{enumerate}
\item The square is a pullback square.
\item The square
\begin{equation*}
\begin{tikzcd}
C \arrow[r,"g\circ q"] \arrow[d,swap,"{\lam{x}(p(x),q(x))}"] & X \arrow[d,"\delta_X"] \\
A\times B \arrow[r,swap,"f\times g"] & X\times X
\end{tikzcd}
\end{equation*}
which commutes by $\lam{z}\mathsf{eq\usc{}pair}(H(z),\refl{g(q(z))})$ is a pullback square.
\end{enumerate}
\item Show that if
\begin{equation*}
\begin{tikzcd}
C_1 \arrow[r] \arrow[d] & B_1 \arrow[d] & C_2 \arrow[r] \arrow[d] & B_2 \arrow[d] \\
A_1 \arrow[r] & X_1 & A_2 \arrow[r] & X_2
\end{tikzcd}
\end{equation*}
are pullback squares, then so is
\begin{equation*}
\begin{tikzcd}
C_1\times C_2 \arrow[r] \arrow[d] & B_1\times B_2 \arrow[d] \\
A_1 \times A_2 \arrow[r] & X_1\times X_2. 
\end{tikzcd}
\end{equation*}
\item Consider for each $i:I$ a pullback square
\begin{equation*}
\begin{tikzcd}
C_i \arrow[r] \arrow[d] & B_i \arrow[d] \\
A_i \arrow[r] & X_i
\end{tikzcd}
\end{equation*}
with $H_i: f_i\circ p_i\htpy g_i\circ q_i$. 
\begin{subexenum}
\item Show that the commuting square
\begin{equation*}
\begin{tikzcd}
\sm{i:I}C_i \arrow[r] \arrow[d] & \sm{i:I}B_i \arrow[d] \\
\sm{i:I}A_i \arrow[r] & \sm{i:I}X_i
\end{tikzcd}
\end{equation*}
is a pullback square.
\item Show that the commuting square
\begin{equation*}
\begin{tikzcd}
\prd{i:I}C_i \arrow[r] \arrow[d] & \prd{i:I}B_i \arrow[d] \\
\prd{i:I}A_i \arrow[r] & \prd{i:I}X_i
\end{tikzcd}
\end{equation*}
is a pullback square.
\end{subexenum}
\item 
\begin{subexenum}
\item Show that 
\begin{equation*}
\begin{tikzcd}[column sep=8em]
\eqv{A}{B} \arrow[r] \arrow[d] & \unit \arrow[d,"{(\idfunc[A],\idfunc[B])}"] \\
A^B\times B^A \times A^B \arrow[r,swap,"{(h,f,g)\mapsto (h\circ f,f\circ g)}"] & A^A \times B^B
\end{tikzcd}
\end{equation*}
is a pullback square.
\item Show that
\begin{equation*}
\begin{tikzcd}[column sep=6em]
\iscontr(A) \arrow[r,"\mathsf{const}_{\ttt}"] \arrow[d,swap,"\proj 1"] & \unit \arrow[d,"{\lam{\ttt}\idfunc[A]}"] \\
A \arrow[r,swap,"{\lam{x}\mathsf{const}_x}"] & A^A
\end{tikzcd}
\end{equation*}
is a pullback square.
\end{subexenum}
%\item Consider a commuting square
%\begin{equation*}
%\begin{tikzcd}
%C \arrow[r] \arrow[d] & A \arrow[d] \\
%B \arrow[r] & X.
%\end{tikzcd}
%\end{equation*}
%Show that this square is cartesian if and only if the induced map $C\to A\times_X B$ has a retraction.
\item Let $B$ be a type family over $A$. Show that the square
\begin{equation*}
\begin{tikzcd}[column sep=6em]
\prd{x:A}B(x) \arrow[r,"{\lam{f}{x}(x,f(x))}"] \arrow[d] & \Big(\sm{x:A}B(x)\Big)^A \arrow[d,"\proj 1\blank\circ"] \\
\unit \arrow[r,swap,"{\lam{\ttt}\idfunc[A]}"] & A^A
\end{tikzcd}
\end{equation*}
is a pullback square. Conclude that the type $\prd{x:A}B(x)$ is equivalent to the type $\mathsf{sec}(\proj 1)$ of sections of the projection map.
%\end{subexenum}
%\item Suppose that the squares
%\begin{equation*}
%\begin{tikzcd}
%C \arrow[r,"q"] \arrow[d,swap,"p"] & B \arrow[d,"g"] & {C'} \arrow[r,"{q'}"] \arrow[d,swap,"{p'}"] & B \arrow[d,"g"] \\
%A \arrow[r,swap,"f"] & X & A \arrow[r,swap,"f"] & X
%\end{tikzcd}
%\end{equation*}
%with homotopies $H:f\circ p \htpy g\circ q$ and $H':f\circ p'\htpy g\circ q'$ are both pullback squares. Show that the type of equivalences $e:\eqv{C'}{C}$ equipped with an identification
%\begin{equation*}
%\mathsf{cone\usc{}map}((p,q,H),e)=(p',q',H')
%\end{equation*}
%is contractible.
\begin{comment}
\item Consider a \define{natural transformation of cospans}\index{cospan!natural transformation of}, i.e.~a commuting diagram of the form
\begin{equation*}
\begin{tikzcd}
A \arrow[r,"f"] \arrow[d,swap,"i"] & X \arrow[d,swap,"j"] & B \arrow[l,swap,"g"] \arrow[d,"k"] \\
A' \arrow[r,swap,"{f'}"] & X' & B'. \arrow[l,"{g'}"]
\end{tikzcd}
\end{equation*}
Show that the map
\begin{equation*}
(a,b,p)\mapsto (i(a),j(b),\mathsf{ap}_k(p)): A \times_X B \to A'\times_{X'} B'
\end{equation*}
is $k$-truncated if each of the vertical maps is.
\end{comment}
\end{exercises}


\section{Homotopy pushouts}

A common way in topology to construct new spaces is by attaching cells\index{attaching cells} to a given space. A $0$-cell is just a point, an $1$-cell is an interval, a $2$-cell is a disc, a $3$-cell is the solid ball, and so forth. Many spaces can be obtained by attaching cells. For example, the circle\index{circle} is obtained by attaching a $1$-cell to a $0$-cell, so that both end-points of the interval are mapped to the point. More generally, an $n$-sphere is obtained by attaching an $n$-disc to the point, so that its entire boundary gets mapped to the point.

In type theory we can also consider a notion of $n$-cells. Just as in topology, a $0$-cell is just a point (i.e., a term). A $1$-cell, however, is in type theory an identification, i.e., a term of the identity type. A $1$-cell is then an identification of identifications, and so forth. Then we can attach cells to a type by taking a pushout, which is a process dual to taking a pullback. 

The idea of pushouts is to glue two types $A$ and $B$ together using a mediating type $S$ and maps $f:S\to A$ and $g:S\to B$. In other words, we start with a diagram of the form
\begin{equation*}
\begin{tikzcd}
A & S \arrow[l,swap,"f"] \arrow[r,"g"] & B.
\end{tikzcd}
\end{equation*}
We call such a triple $\mathcal{S}\jdeq (S,f,g)$ a \define{span}\index{span} from $A$ to $B$.
A span from $A$ to $B$ can be thought of as a relation\index{relation} from $A$ to $B$, relating $f(s)$ to $g(s)$ for any $s:S$. The pushout of the span $\mathcal{S}$ is then a type $X$ that comes equipped with inclusion maps $i:A\to X$ and $j:B\to X$ and a homotopy $H$ witnessing that the square
\begin{equation*}
  \begin{tikzcd}
    S \arrow[d,swap,"f"] \arrow[r,"g"] & B \arrow[d,"j"] \\
    A \arrow[r,swap,"i"] & X
  \end{tikzcd}
\end{equation*}
Note that this homotopy makes sure that there is a path $H(s):i(f(s))=j(g(s))$ for every $s:S$. In other words, any $x:A$ and $y:B$ that are related by in $\mathcal{S}$ become identified in the pushout. The last requirement of the pushout is that it satisfies a universal property that is dual to the universal property of pullbacks.

There are several equivalent characterizations of pushouts. Two such characterizations are studied in this section, establishing the duality between pullbacks and pushouts. Other characterizations, including the induction principle of pushouts, and the \emph{dependent universal property} of pushouts, are studied in \cref{chap:descent}.

Unlike pullbacks, however, it is not automatically the case that pushouts always exist. We will therefore postulate as an axiom that pushouts always exist. Moreover, we will assume that universes are closed under pushouts.

\subsection{The universal property of pushouts}

\begin{defn}
Consider a span $\mathcal{S}\jdeq (S,f,g)$ from $A$ to $B$, and let $X$ be a type.
A \define{cocone}\index{cocone} with vertex $X$ on $\mathcal{S}$ is a triple $(i,j,H)$ consisting of maps $i:A\to X$ and $j:B\to X$, and a homotopy $H:i\circ f\htpy j\circ g$ witnessing that the square
\begin{equation*}
\begin{tikzcd}
S \arrow[r,"g"] \arrow[d,swap,"f"] & B \arrow[d,"j"] \\
A \arrow[r,swap,"i"] & X
\end{tikzcd}
\end{equation*}
commutes.
We write $\mathsf{cocone}_{\mathcal{S}}(X)$\index{cocone_S(X)@{$\mathsf{cocone}_{\mathcal{S}}(X)$}} for the type of cocones on $\mathcal{S}$ with vertex $X$.
\end{defn}

\begin{rmk}\label{rmk:htpy-cocone}
  Given two cocones $(i,j,H)$ and $(i',j',H')$ with vertex $X$, the type of identifications $(i,j,H)=(i',j',H')$ in $\mathsf{cocone}_{\mathcal{S}}(X)$ is equivalent to the type of triples $(K,L,M)$ consisting of
  \begin{align*}
    K : i\htpy i' \\
    L : j\htpy j',
  \end{align*}
  and a homotopy $M$ witnessing that the square
  \begin{equation*}
    \begin{tikzcd}
      %  ((pr2 (pr2 c)) ∙h (L ·r g)) ~ ((K ·r f) ∙h (pr2 (pr2 c')))
      i\circ f \arrow[d,swap,"H"] \arrow[r,"K\cdot f"] & i' \circ f \arrow[d,"{H'}"] \\
      j\circ g \arrow[r,swap,"L\cdot g"] & j'\circ g
    \end{tikzcd}
  \end{equation*}
  of homotopies commutes.
\end{rmk}

\begin{defn}
Consider a cocone $(i,j,H)$ with vertex $X$ on the span $\mathcal{S}\jdeq (S,f,g)$, as indicated in the following commuting square
\begin{equation*}
\begin{tikzcd}
S \arrow[r,"g"] \arrow[d,swap,"f"] & B \arrow[d,"j"] \\
A \arrow[r,swap,"i"] & X.
\end{tikzcd}
\end{equation*}
For every type $Y$, we define the map\index{cocone_map@{$\mathsf{cocone\usc{}map}$}}
\begin{equation*}
\mathsf{cocone\usc{}map}(i,j,H):(X\to Y)\to \mathsf{cocone}(Y)
\end{equation*}
by $h\mapsto (h\circ i,h\circ j,h\cdot H)$.
\end{defn}

\begin{defn}
  A commuting square
  \begin{equation*}
    \begin{tikzcd}
      S \arrow[r,"g"] \arrow[d,swap,"f"] & B \arrow[d,"j"] \\
      A \arrow[r,swap,"i"] & X.
    \end{tikzcd}
  \end{equation*}
  with $H:i\circ f \htpy j\circ g$ is said to be a \define{(homotopy) pushout square}\index{pushout square} if the cocone $(i,j,H)$ with vertex $X$ on the span $\mathcal{S}\jdeq (S,f,g)$
  satisfies the \define{universal property of pushouts}\index{universal property!of pushouts}, which asserts that the map
  \begin{equation*}
    \mathsf{cocone\usc{}map}(i,j,H):(X\to Y)\to \mathsf{cocone}(Y)
  \end{equation*}
  is an equivalence for any type $Y$. Sometimes pushout squares are also called \define{cocartesian squares}\index{cocartesian square}.
\end{defn}

\begin{lem}\label{lem:unique-mapping-property-pushout}
  Consider a pushout square
  \begin{equation*}
    \begin{tikzcd}
      S \arrow[r,"g"] \arrow[d,swap,"f"] & B \arrow[d,"j"] \\
      A \arrow[r,swap,"i"] & X.
    \end{tikzcd}
  \end{equation*}
  with $H:i\circ f \htpy j\circ g$, and consider a commuting square
  \begin{equation*}
    \begin{tikzcd}
      S \arrow[r,"g"] \arrow[d,swap,"f"] & B \arrow[d,"{j'}"] \\
      A \arrow[r,swap,"{i'}"] & X'.
    \end{tikzcd}
  \end{equation*}
  with $H':i'\circ f \htpy j'\circ g$. Then the type of maps $h:X\to X'$ equipped with homotopies
  \begin{align*}
    K & : h\circ i \htpy i' \\
    L & : h\circ j \htpy j'
  \end{align*}
  and a homotopy $M$ witnessing that the square
  \begin{equation*}
    \begin{tikzcd}
      h\circ i\circ f \arrow[r,"K\cdot f"] \arrow[d,swap,"h\cdot H"] & i' \circ f \arrow[d,"{H'}"] \\
      h\circ j\circ g \arrow[r,swap,"L\cdot g"] & j'\circ g
    \end{tikzcd}
  \end{equation*}
  commutes, is contractible.
\end{lem}

\begin{proof}
  For any map $h:X\to X$', the type of triples $(K,L,M)$ as in the statement of the lemma is equivalent to the type of identifications
  \begin{equation*}
    \mathsf{cocone\usc{}map}((i,j,H),h)=(i',j',H'),
  \end{equation*}
  by \cref{rmk:htpy-cocone}. Therefore it follows that the type of quadruples $(h,K,L,M)$ is equivalent to the fiber of $\mathsf{cocone\usc{}map}(i,j,H)$ at $(i',j',H')$. Since we have assumed that the cocone $(i,j,H)$ satisfies the universal property of the pushout of $\mathcal{S}$, the map $\mathsf{cocone\usc{}map}(i,j,H)$ is an equivalence, and therefore it has contractible fibers by \cref{thm:contr_equiv}.
\end{proof}

\begin{thm}\label{thm:3-for-2-pushout}
  Consider two cocones
  \begin{equation*}
    \begin{tikzcd}
      S \arrow[r,"g"] \arrow[d,swap,"f"] & B \arrow[d,"j"]
      & & S \arrow[r,"g"] \arrow[d,swap,"f"] & B \arrow[d,"{j'}"] \\
      A \arrow[r,swap,"i"] & X
      & & A \arrow[r,swap,"{i'}"] & X'
    \end{tikzcd}
  \end{equation*}
  on a span $\mathcal{S}\jdeq(S,f,g)$, and let $h:X\to X'$ be a map equipped with homotopies
  \begin{align*}
    K & : h\circ i \htpy i' \\
    L & : h\circ j \htpy j'
  \end{align*}
  and a homotopy $M$ witnessing that the square
  \begin{equation*}
    \begin{tikzcd}
      h\circ i\circ f \arrow[r,"K\cdot f"] \arrow[d,swap,"h\cdot H"] & i' \circ f \arrow[d,"{H'}"] \\
      h\circ j\circ g \arrow[r,swap,"L\cdot g"] & j'\circ g
    \end{tikzcd}
  \end{equation*}
  commutes. Then if any two of the following three statements hold, so does the third:
  \begin{enumerate}
  \item The cocone $(i,j,H)$ satisfies the universal property of the pushout of $\mathcal{S}$.
  \item The cocone $(i',j',H')$ satisfies the universal property of the pushotu of $\mathcal{S}$.
  \item The map $h$ is an equivalence.
  \end{enumerate}
\end{thm}

\begin{proof}
  First we observe that we have a commuting triangle
  \begin{equation*}
    \begin{tikzcd}[column sep=0]
      (X'\to Y) \arrow[rr,"\blank\circ h"]
      \arrow[dr,swap,"{\mathsf{cocone\usc{}map}(i',j',H')}"]
      & & (X\to Y) \arrow[dl,"{\mathsf{cocone\usc{}map}(i,j,H)}"] \\
      & \mathsf{cocone}_{\mathcal{S}}(Y) & \phantom{(X'\to Y)}
    \end{tikzcd}
  \end{equation*}
  for any type $Y$. Therefore it follows from the 3-for-2 property of equivalences that if any two of the maps in this triangle is an equivalence, so is the third. Now the claim follows from the observation in \cref{ex:equiv_precomp} that $h$ is an equivalence if and only if the map $\blank\circ h:(X'\to Y)\to (X\to Y)$ is an equivalence for any type $Y$.
\end{proof}

In the following corollary we establish the fact that pushouts are \emph{uniquely unique}.

\begin{cor}
  Consider two pushouts
  \begin{equation*}
    \begin{tikzcd}
      S \arrow[r,"g"] \arrow[d,swap,"f"] & B \arrow[d,"j"]
      & & S \arrow[r,"g"] \arrow[d,swap,"f"] & B \arrow[d,"{j'}"] \\
      A \arrow[r,swap,"i"] & X
      & & A \arrow[r,swap,"{i'}"] & X'
    \end{tikzcd}
  \end{equation*}
  of a given span $\mathcal{S}\jdeq (S,f,g)$. Then the type of equivalences $e:\eqv{X}{X'}$ equipped with homotopies
  \begin{align*}
    K & : h\circ i \htpy i' \\
    L & : h\circ j \htpy j'
  \end{align*}
  and a homotopy $M$ witnessing that the square
  \begin{equation*}
    \begin{tikzcd}
      h\circ i\circ f \arrow[r,"K\cdot f"] \arrow[d,swap,"h\cdot H"] & i' \circ f \arrow[d,"{H'}"] \\
      h\circ j\circ g \arrow[r,swap,"L\cdot g"] & j'\circ g
    \end{tikzcd}
  \end{equation*}
  commutes, is contractible.
\end{cor}

\begin{proof}
  This follows from combining \cref{lem:unique-mapping-property-pushout,thm:3-for-2-pushout}.
\end{proof}

\begin{cor}\label{cor:uniquely-unique-pushout}
  Consider a span
  \begin{equation*}
    \begin{tikzcd}
      A & S \arrow[l,swap,"f"] \arrow[r,"g"] & B
    \end{tikzcd}
  \end{equation*}
  in a universe $\UU$. Then the type
  \begin{equation*}
    \sm{X:\UU}{c:\mathsf{cocone}(X)}\prd{Y:\UU}\mathsf{is\usc{}equiv}(\mathsf{cocone\usc{}map}_Y(c))
  \end{equation*}
  of is a proposition.
\end{cor}

\begin{proof}
  It is routine to verify that the type of quadruples $(e,K,L,M)$ as in \cref{cor:uniquely-unique-pushout} is equivalent to the identity type of the type of pushouts of the span $\mathcal{S}\jdeq (S,f,g)$. The claim then follows, since \cref{cor:uniquely-unique-pushout} asserts that this type of quadruples is contractible. 
\end{proof}

\subsection{Suspensions}
A particularly important class of examples of pushouts are suspensions.

\begin{defn}
  Let $X$ be a type. A \define{suspension}\index{suspension} of $X$ is a type $\susp X$ equipped with a \define{north pole} $\mathsf{N}:\susp X$, a \define{south pole} $\mathsf{S}:\susp X$, and a \define{meridian}
  \begin{equation*}
    \mathsf{merid} : X \to (\mathsf{N}=\mathsf{S}),
  \end{equation*}
  such that the commuting square
  \begin{equation*}
    \begin{tikzcd}
      X \arrow[r,"\mathsf{const}_\ttt"] \arrow[d,swap,"\mathsf{const}_\ttt"] & \unit \arrow[d,"{\mathsf{const}_\south}"] \\
      \unit \arrow[r,swap,"\mathsf{const}_\north"] & \susp X
    \end{tikzcd}
  \end{equation*}
  is a pushout square.
\end{defn}

We can use suspensions to present the spheres in type theory. The $2$-sphere is a space which, like the surface of the earth, has a north pole and a south pole. Moreover, for each point of the equator there is a meridian that connects the north pole to the south pool. Of course, the equator is a circle, so we see that the $2$-sphere is just the suspension of the circle.

Similarly we can see that the $(n+1)$-sphere must be the suspension of the $n$-sphere. The $(n+1)$-sphere is the unit sphere in the vector space $\mathbb{R}^{n+2}$. This vector space has an orthogonal basis $e_1,\ldots,e_{n+2}$. Then the north and the south pole are given by $e_{n+2}$ and $-e_{n+2}$, respectively, and for each unit vector in $\mathbb{R}^{n+1}\subseteq\mathbb{R}^{n+2}$ we have a meridian connecting the north pole with the south pole. The unit sphere in $\mathbb{R}^{n+1}$ is of course the $n$-sphere, so we see that the $(n+1)$-sphere must be a suspension of the $n$-sphere.

These observations suggest that we can define the spheres by recursion on $n$. Note that the spheres in type theory are defined entirely synthetically, i.e., without reference to the ambient topological space $\mathbb{R}^{n+1}$. Indeed, from a homotopical point of view each space $\mathbb{R}^{n}$ is contractible, so in type theory it is just presented as the unit type\footnote{It is an entirely different matter to define the \emph{set} $\mathbb{R}$ rather than the homotopy type of $\mathbb{R}$. See Chapter 11 of \cite{hottbook} for definitions of the Dedekind reals and the Cauchy reals.}.

\begin{defn}
We define the \define{$n$-sphere}\index{n-sphere@{$n$-sphere}} $\sphere{n}$\index{Sn@{$\sphere{n}$}} for any $n:\N$ by induction on $n$, by taking
\begin{align*}
\sphere{0} & \defeq \bool \\
\sphere{n+1} & \defeq \susp{\sphere{n}}.
\end{align*}
\end{defn}

\begin{rmk}
  Note that this recursive definition of the spheres only goes through in type theory if we have (or assume) a universe that is closed under suspensions.
\end{rmk}

In the following lemma we give a slight simplification of the universal property of suspensions, making it just a little easier to work with them.

\begin{lem}
Let $X$ and $Y$ be types, and let $\susp X$ be a suspension of $X$. Then the map\index{universal property!of suspensions}
\begin{equation*}
(\susp{X}\to Y)\to \sm{y,y':Y} X\to (y=y')
\end{equation*}
given by $f\mapsto (f(\north),f(\south),f\cdot\merid)$ is an equivalence.
\end{lem}

\begin{proof}
  Note that we have a commuting triangle
  \begin{equation*}
    \begin{tikzcd}[column sep=-1em]
      \phantom{\sm{y,y':Y}X\to (y=y')} & (\susp X \to Y) \arrow[dl,swap,"\mathsf{cocone\usc{}map}"] \arrow[dr,"{f\mapsto (f(\north),f(\south),f\cdot\merid)}"] &
      \phantom{\mathsf{cocone}_{\mathcal{S}}(Y)} \\
      \mathsf{cocone}_{\mathcal{S}}(Y) \arrow[rr] & & \sm{y,y':Y}X\to (y=y')
    \end{tikzcd}
  \end{equation*}
  where $\mathcal{S}$ is the span $\unit \leftarrow X \rightarrow \unit$. The bottom map is given by $(i,j,H)\mapsto (i(\ttt),j(\ttt),H)$. This map is an equivalence, and the map on the left is an equivalence by the assumption that $\susp X$ is a suspension of $X$. Therefore the claim follows by the 3-for-2 property of equivalences.
\end{proof}

\subsection{The duality of pullbacks and pushouts}
\begin{lem}\label{lem:cocone_pb}
For any span $\mathcal{S}\jdeq (S,f,g)$ from $A$ to $B$, and any type $X$ the square\index{cocone_S(X)@{$\mathsf{cocone}_{\mathcal{S}}(X)$}!as a pullback}
\begin{equation*}
\begin{tikzcd}
\mathsf{cocone}_{\mathcal{S}}(X) \arrow[r,"\pi_2"] \arrow[d,swap,"\pi_1"] & X^B \arrow[d,"\blank\circ g"] \\
X^A \arrow[r,swap,"\blank\circ f"] & X^S,
\end{tikzcd}
\end{equation*}
which commutes by the homotopy $\pi_3' \defeq\lam{(i,j,H)} \mathsf{eq\usc{}htpy}(H)$, is a pullback square.
\end{lem}

\begin{proof}
The gap map $\mathsf{cocone}_{\mathcal{S}}(X)\to X^A\times_{X^S} X^B$ is the function 
\begin{equation*}
\lam{(i,j,H)}(i,j,\mathsf{eq\usc{}htpy}(H)).
\end{equation*}
This is an equivalence by \cref{thm:fib_equiv}, since it is the induced map on total spaces of the family of equivalences $\mathsf{eq\usc{}htpy}$. Therefore, the square is a pullback square by \cref{thm:is_pullback}.
\end{proof}

In the following theorem we establish the duality between pullbacks and pushouts.

\begin{thm}\label{thm:pushout_up}
Consider a commuting square\index{universal property!of pushouts}
\begin{equation*}
\begin{tikzcd}
S \arrow[r,"g"] \arrow[d,swap,"f"] & B \arrow[d,"j"] \\
A \arrow[r,swap,"i"] & X,
\end{tikzcd}
\end{equation*}
with $H:i\circ f\htpy j\circ g$. The following are equivalent:
\begin{enumerate}
\item The square is a pushout square.
\item The square
\begin{equation*}
\begin{tikzcd}
T^X \arrow[r,"\blank\circ j"] \arrow[d,swap,"\blank\circ i"] & T^B \arrow[d,"\blank\circ g"] \\
T^A \arrow[r,swap,"\blank\circ f"] & T^S
\end{tikzcd}
\end{equation*}
which commutes by the homotopy
\begin{equation*}
\lam{h} \mathsf{eq\usc{}htpy}(h\cdot H)
\end{equation*}
is a pullback square, for every type $T$.
%\item The type $X$ satisfies \define{span induction} for the span $A\leftarrow S \rightarrow B$, in the sense that for any type family $P$ over $X$, the map
%\begin{equation*}
%\Big(\prd{x:X}P(x)\Big)\to \Big(\sm{i':\prd{a:A}P(i(a))}{j':\prd{b:B}P(j(b))} i'\htpy_H j'\Big)
%\end{equation*}
%given by $s\mapsto (s\circ i,s\circ j,s\cdot H)$ has a section.
\end{enumerate}
\end{thm}

\begin{proof}
It is straightforward to verify that the triangle
\begin{equation*}
\begin{tikzcd}[column sep=3em]
& T^X \arrow[dl,swap,"{\mathsf{cocone\usc{}map}(i,j,H)}"] \arrow[dr,"{\mathsf{gap}(\blank\circ i,\blank\circ j, \mathsf{eq\usc{}htpy}(\blank\cdot H))}"] \\
\mathsf{cocone}(T) \arrow[rr,swap,"{\mathsf{gap}(i,j,\mathsf{eq\usc{}htpy}(H))}"] & & T^A \times_{T^S} T^B
\end{tikzcd}
\end{equation*}
commutes. Since the bottom map is an equivalence by \cref{lem:cocone_pb}, it follows that if either one of the remaining maps is an equivalence, so is the other. The claim now follows by \cref{thm:is_pullback}.
\end{proof}

\begin{eg}\label{eg:circle_pushout}
  The square
  \begin{equation*}
    \begin{tikzcd}[column sep=huge]
      X^{\sphere{1}} \arrow[r,"\blank\circ\mathsf{const}_{\base}"] \arrow[d,swap,"\blank\circ\mathsf{const}_{\base}"] & X^\unit \arrow[d,"\blank\circ\mathsf{const}_{\ttt}"] \\
      X^\unit \arrow[r,swap,"\blank\circ\mathsf{const}_{\ttt}"] & X^\bool
    \end{tikzcd}
  \end{equation*}
  is a pullback square for each type $X$. Therefore it follows by the second characterization of pushouts in \cref{thm:pushout_up} that the circle is a pushout\index{circle!S1 equiv susp 2@{$\eqv{\sphere{1}}{\susp\bool}$}}
  \begin{equation*}
    \begin{tikzcd}
      \bool \arrow[r] \arrow[d] & \unit \arrow[d] \\
      \unit \arrow[r] & \sphere{1}.
    \end{tikzcd}
  \end{equation*}
  In other words, $\eqv{\sphere{1}}{\susp{\bool}}$. 
\end{eg}

\begin{thm}\label{thm:pushout_pasting}
Consider the following configuration of commuting squares:\index{pushout!pasting property}\index{pasting property!for pushouts}
\begin{equation*}
\begin{tikzcd}
A \arrow[r,"i"] \arrow[d,swap,"f"] & B \arrow[r,"k"] \arrow[d,swap,"g"] & C \arrow[d,"h"] \\
X \arrow[r,swap,"j"] & Y \arrow[r,swap,"l"] & Z
\end{tikzcd}
\end{equation*}
with homotopies $H:j\circ f\htpy g\circ i$ and $K:l\circ g\htpy h\circ k$, and suppose that the square on the left is a pushout square. 
Then the square on the right is a pushout square if and only if the outer rectangle is a pushout square.
\end{thm}

\begin{proof}
Let $T$ be a type. Taking the exponent $T^{(\blank)}$ of the entire diagram of the statement of the theorem, we obtain the following commuting diagram
\begin{equation*}
\begin{tikzcd}
T^Z \arrow[r,"\blank\circ l"] \arrow[d,swap,"\blank\circ h"] & T^Y \arrow[d,swap,"\blank\circ g"] \arrow[r,"\blank\circ j"] & T^X \arrow[d,"\blank\circ f"] \\
T^C \arrow[r,swap,"\blank\circ k"] & T^B \arrow[r,swap,"\blank\circ i"] & T^A.
\end{tikzcd}
\end{equation*}
By the assumption that $Y$ is the pushout of $B\leftarrow A \rightarrow X$, it follows that the square on the right is a pullback square. It follows by \cref{thm:pb_pasting} that the rectangle on the left is a pullback if and only if the outer rectangle is a pullback. Thus the statement follows by the second characterization in \cref{thm:pushout_up}.
\end{proof}

\begin{lem}
Consider a map $f:A\to B$. Then the cofiber of the map $\inr:B\to \mathsf{cofib}_f$ is equivalent to the suspension $\susp{A}$ of $A$. 
\end{lem}

\subsection{Fiber sequences and cofiber sequences}

\begin{defn}
Given a map $f:A\to B$, we define the \define{cofiber}\index{cofiber} $\mathsf{cofib}_f$\index{cofib_f@{$\mathsf{cofib}_f$}} of $f$ as the pushout
\begin{equation*}
\begin{tikzcd}
A \arrow[r,"f"] \arrow[d] & B \arrow[d,"\inr"] \\
\unit \arrow[r,swap,"\inl"] & \mathsf{cofib}_f. 
\end{tikzcd}
\end{equation*}
The cofiber of a map is sometimes also called the \define{mapping cone}\index{mapping cone}.
\end{defn}

\begin{eg}
The suspension $\susp X$ of $X$ is the cofiber of the map $X\to \unit$.\index{suspension!as cofiber} 
\end{eg}

\subsection{Further examples of pushouts}

\begin{defn}
We define the \define{join}\index{join} $\join{X}{Y}$\index{join X Y@{$\join{X}{Y}$}} of $X$ and $Y$ to be the pushout 
\begin{equation*}
\begin{tikzcd}
X\times Y \arrow[r,"\proj 2"] \arrow[d,swap,"\proj 1"] & Y \arrow[d,"\inr"] \\
X \arrow[r,swap,"\inl"] & X \ast Y. 
\end{tikzcd}
\end{equation*}
\end{defn}

\begin{defn}
Suppose $A$ and $B$ are pointed types, with base points $a_0$ and $b_0$, respectively. The \define{(binary) wedge}\index{wedge@(binary) wedge} $A\vee B$ of $A$ and $B$ is defined as the pushout
\begin{equation*}
\begin{tikzcd}
\bool \arrow[r] \arrow[d] & A+B \arrow[d] \\
\unit \arrow[r] & A\vee B.
\end{tikzcd}
\end{equation*}
\end{defn}

\begin{defn}
Given a type $I$, and a family of pointed types $A$ over $i$, with base points $a_0(i)$. We define the \define{(indexed) wedge}\index{wedge@{(indexed) wedge}} $\bigvee_{(i:I)}A_i$ as the pushout
\begin{equation*}
\begin{tikzcd}[column sep=huge]
I \arrow[d] \arrow[r,"{\lam{i}(i,a_0(i))}"] & \sm{i:I}A_i \arrow[d] \\
\unit \arrow[r] & \bigvee_{(i:I)} A_i.
\end{tikzcd}
\end{equation*}
\end{defn}

\begin{defn}
Let $X$ and $Y$ be types with base points $x_0$ and $y_0$, respectively.
We define the \define{wedge} $X\lor Y$ of $X$ and $Y$ to be the pushout
\begin{equation*}
\begin{tikzcd}[column sep=8em]
\bool \arrow[r,"{\ind{\bool}(\inl(x_0),\inr(y_0))}"] \arrow[d,swap,"\mathsf{const}_\ttt"] & X+Y \arrow[d,"\inr"] \\
\unit \arrow[r,swap,"\inl"] & X\lor Y
\end{tikzcd}
\end{equation*}
\end{defn}

\begin{defn}
Let $X$ and $Y$ be types with base points $x_0$ and $y_0$, respectively.
We define a map
\begin{equation*}
\mathsf{wedge\usc{}incl} : X \lor Y \to X\times Y.
\end{equation*}
as the unique map obtained from the commutative square
\begin{equation*}
\begin{tikzcd}[column sep=8em]
\bool \arrow[r,"{\ind{\bool}(\inl(x_0),\inr(y_0))}"] \arrow[d,swap,"\mathsf{const}_\ttt"] & X+Y \arrow[d,"{\ind{X+Y}(\lam{x}\pairr{x,y_0},\lam{y}\pairr{x_0,y})}"] \\
\unit \arrow[r,swap,"\lam{t}\pairr{x_0,y_0}"] & X\times Y.
\end{tikzcd}
\end{equation*}
\end{defn}

\begin{defn}
We define the \define{smash product} $X\wedge Y$ of $X$ and $Y$ to be the pushout
\begin{equation*}
\begin{tikzcd}[column sep=huge]
X\lor Y \arrow[r,"\mathsf{wedge\usc{}incl}"] \arrow[d,swap,"\mathsf{const}_\ttt"] & X\times Y \arrow[d,"\inr"] \\
\unit \arrow[r,swap,"\inl"] & X\wedge Y.
\end{tikzcd}
\end{equation*}
\end{defn}

\begin{exercises}
\exercise \label{ex:pushout_equiv}Use \cref{thm:pushout_up,cor:pb_equiv,ex:equiv_precomp} to show that for any commuting square
\begin{equation*}
\begin{tikzcd}
S \arrow[r,"g"] \arrow[d,swap,"f","{\eqvsym}"'] & B \arrow[d,"j"] \\
A \arrow[r,swap,"i"] & C
\end{tikzcd}
\end{equation*} 
where $f$ is an equivalence, the square is a pushout square if and only if $j:B\to C$ is an equivalence.
Use this observation to conclude the following:
\begin{enumerate}
\item If $X$ is contractible, then $\susp X$ is contractible.
\item The cofiber of any equivalence is contractible.
\item The cofiber of a point in $B$ (i.e., of a map of the type $\unit\to B$) is equivalent to $B$.
\item There is an equivalence $\eqv{X}{\join{\emptyt}{X}}$.
\item If $X$ is contractible, then $\join{X}{Y}$ is contractible. 
\item If $A$ is contractible, then there is an equivalence $\eqv{A\vee B}{B}$ for any pointed type $B$.
\end{enumerate}
\exercise \label{ex:join_propositions}Let $P$ and $Q$ be propositions.
\begin{subexenum}
\item Show that $\join{P}{Q}$ satisfies the \emph{universal property of disjunction}, i.e., that for any proposition $R$, the map
\begin{equation*}
(\join{P}{Q}\to R)\to (P\to R)\times (Q\to R)
\end{equation*}
given by $f\mapsto (f\circ \inl,f\circ \inr)$, is an equivalence.
\item Use the proposition $R\defeq\iscontr(\join{P}{Q})$ to show that $\join{P}{Q}$ is again a proposition.
\end{subexenum}
\exercise Let $Q$ be a proposition, and let $A$ be a type. Show that the following are equivalent:
\begin{enumerate}
\item The map $(Q\to A)\to(\emptyt\to A)$ is an equivalence.
\item The type $A^Q$ is contractible.
\item There is a term of type $Q\to\iscontr(A)$.
\item The map $\inr:A\to \join{Q}{A}$ is an equivalence.
\end{enumerate}
\exercise Let $P$ be a proposition. Show that $\susp P$ is a set, with an equivalence
\begin{equation*}
\eqv{\Big(\inl(\ttt)=\inr(\ttt)\Big)}{P}.
\end{equation*}
\exercise Show that \({A\sqcup^{\mathcal{S}} B} \simeq {B\sqcup^{\mathcal{S}^{\mathsf{op}}} A}\), where $\mathcal{S^{\mathsf{op}}}\defeq (S,g,f)$ is the \define{opposite span} of $\mathcal{S}$. 
\exercise Use \cref{ex:pb_pi} to show that if
\begin{equation*}
\begin{tikzcd}
S \arrow[r] \arrow[d] & Y \arrow[d] \\
X \arrow[r] & Z
\end{tikzcd}
\end{equation*}
is a pushout square, then so is
\begin{equation*}
\begin{tikzcd}
A\times S \arrow[r] \arrow[d] & A\times Y \arrow[d] \\
A\times X \arrow[r] & A\times Z
\end{tikzcd}
\end{equation*}
for any type $A$.
\exercise Use \cref{ex:pb_prod} to show that if
\begin{equation*}
\begin{tikzcd}
S_1 \arrow[r] \arrow[d] & Y_1 \arrow[d] & S_2 \arrow[r] \arrow[d] & Y_2 \arrow[d] \\
X_1 \arrow[r] & Z_1 & X_2 \arrow[r] & Z_2
\end{tikzcd}
\end{equation*}
are pushout squares, then so is
\begin{equation*}
\begin{tikzcd}
S_1+S_2 \arrow[r] \arrow[d] & Y_1+ Y_2 \arrow[d] \\
X_1 +X_2 \arrow[r] & Z_1+Z_2. 
\end{tikzcd}
\end{equation*}
\exercise 
\begin{subexenum}
\item Consider a span $(S,f,g)$ from $A$ to $B$. Use \cref{ex:pb_diagonal} to show that the square
\begin{equation*}
\begin{tikzcd}[column sep=large]
S+S \arrow[d,swap,"{f+g}"] \arrow[r,"{[\idfunc,\idfunc]}"] & S \arrow[d,"{\inr\circ g}"] \\
A+B \arrow[r,swap,"{[\inl,\inr]}"] & A\sqcup^\mathcal{S} B
\end{tikzcd}
\end{equation*}
is again a pushout square.
\item Show that $\eqv{\susp X}{\join{\bool}{X}}$.
\end{subexenum}
\exercise Consider a commuting triangle
\begin{equation*}
\begin{tikzcd}[column sep=tiny]
A \arrow[rr,"h"] \arrow[dr,swap,"f"] & & B \arrow[dl,"g"] \\
& X
\end{tikzcd}
\end{equation*}
with $H:f\htpy g\circ h$. 
\begin{subexenum}
\item Construct a map $\mathsf{cofib}_{(h,H)}: \mathsf{cofib}_{g}\to \mathsf{cofib}_f$.
\item Use \cref{ex:pb_fib} to show that $\eqv{\mathsf{cofib}_{\mathsf{cofib}(h,H)}}{\mathsf{cofib}_h}$.
\end{subexenum}
\exercise \label{ex:sphere_null}Use \cref{ex:circle_connected} to show that for $n\geq 0$, $X$ is an $n$-type if and only if the map
\begin{equation*}
\lam{x}\mathsf{const}_x : X \to (\sphere{n+1}\to X)
\end{equation*}
is an equivalence.
\exercise 
\begin{subexenum}
\item Construct for every $f:X\to Y$ a function
\begin{equation*}
\susp f : \susp X\to \susp Y.
\end{equation*}
\item Show that if $f\htpy g$, then $\susp f \htpy \susp g$. 
\item Show that $\susp \idfunc[X]\htpy\idfunc[\susp X]$
\item Show that
\begin{equation*}
\susp(g\circ f)\htpy (\susp g)\circ (\susp f).
\end{equation*}
for any $f:X\to Y$ and $g:Y\to Z$.
\end{subexenum}
\exercise 
\begin{subexenum}
\item Let $I$ be a type, and let $A$ be a family over $I$. Construct an equivalence
\begin{equation*}
\eqv{\Big(\bigvee\nolimits_{(i:I)}\susp A_i\Big)}{\susp\Big(\bigvee\nolimits_{(i:I)}A_i\Big)}.
\end{equation*}
\item Show that for any type $X$ there is an equivalence
\begin{equation*}
\eqv{\Big(\bigvee\nolimits_{(x:X)}\bool\Big)}{X+\unit}.
\end{equation*}
\item Construct an equivalence
\begin{equation*}
\eqv{\susp(\mathsf{Fin}(n+1))}{\bigvee\nolimits_{(i:\mathsf{Fin}(n))}\sphere{1}}.
\end{equation*}
\end{subexenum}
\exercise Show that $\eqv{\join{\mathsf{Fin}(n+1)}{\mathsf{Fin}(m+1)}}{\bigvee\nolimits_{(i:\mathsf{Fin}(n\cdot m))}\sphere{1}}$, for any $n,m:\N$.
\exercise For any pointed set $X$, show that the squares
  \begin{equation*}
    \begin{tikzcd}
      \sphere{1} \arrow[r] \arrow[d] & \unit \arrow[d] \\
      \bigvee_{(x:X)}\sphere{1} \arrow[r] & \susp{X}
    \end{tikzcd}
    \qquad\text{and}\qquad
    \begin{tikzcd}
      X\times \sphere{1} \arrow[d] \arrow[r] & \unit \arrow[d] \\
      \bigvee_{(x:X)}\sphere{1} \arrow[r] & \susp{X}
    \end{tikzcd}
  \end{equation*}
  are pushout squares.
\exercise Show that the square
  \begin{equation*}
    \begin{tikzcd}
      \sphere{1} \arrow[r] \arrow[d] & \unit \arrow[d] \\
      \sphere{1}\times\sphere{1} \arrow[r] & \sphere{2}\vee\sphere{1}
    \end{tikzcd}
  \end{equation*}
  is a pushout square.
\exercise For any type $X$, show that the mapping cone of the fold map $X+X\to X$ is the suspension of $X+\unit$, i.e.~show that the following square
  \begin{equation*}
    \begin{tikzcd}
      X+X \arrow[d] \arrow[r] & \unit \arrow[d] \\
      X \arrow[r] & \susp{X+\unit}
    \end{tikzcd}
  \end{equation*}
  is a pushout square.
  \exercise Consider a map $f:A\to B$. Show that $f$ is a $k$-truncated map if and only if the square
  \begin{equation*}
    \begin{tikzcd}
      A \arrow[r,"\delta"] \arrow[d,swap,"f"] & A^{\sphere{k+1}} \arrow[d,"f^{\sphere{k+1}}"] \\
      B \arrow[r,swap,"\delta"] & B^{\sphere{k+1}}
    \end{tikzcd}
  \end{equation*}
  is a pullback square.
  \exercise Show that a type $A$ is a proposition if and only if the map $\inl:A\to \join{A}{A}$ is an equivalence.
  \exercise Let $A$ be a type, and let $P$ be a proposition.
  \begin{subexenum}
  \item Show that $\inl:P\to \join{P}{A}$ is an embedding.
  \item Show that $\inl:P\to \join{P}{A}$ is an equivalence if and only if $\brck{A}\to P$ holds.
  \end{subexenum}
\end{exercises}


\section{Cubical diagrams}

In order to proceed with the development of pullbacks and pushouts, it is useful to study commuting diagrams of the form
\begin{equation*}
  \begin{tikzcd}
    & C' \arrow[dl] \arrow[d] \arrow[dr] \\
    A' \arrow[d] & C \arrow[dl] \arrow[dr] & B' \arrow[dl,crossing over] \arrow[d] \\
    A \arrow[dr] & X' \arrow[from=ul,crossing over] \arrow[d] & B \arrow[dl] \\
    & X.
  \end{tikzcd}
\end{equation*}
In these diagrams there are six homotopies witnessing that the faces of the cube commute, as well as a homotopy of homotopies witnessing that the cube as a whole commutes.

Once the basic definitions of cubes are established, we focus on pullbacks and pushouts that appear in different configurations in these cubical diagrams. For example, if all the vertical maps in a commuting cube are equivalences, then the top square is a pullback square if and only if the bottom square is a pullback square. In \cref{chap:descent} we will use cubical diagrams in our formulation of the universality and descent theorems for pushouts.

In the first main theorem of this section we show that given a commuting cube in which the bottom square is a pullback square, the top square is a pullback square if and only if the induced square of fibers of the vertical maps is a pullback square. This theorem should be compared to \cref{cor:pb_fibequiv}, where we showed that a square is a pullback square if and only if it induces equivalences on the fibers of the vertical maps.

In our second main theorem we use the previous result to derive the 3-by-3 properties for pullbacks and pushouts.

\subsection{Commuting cubes}
\begin{defn}\label{defn:cube}
A \define{commuting cube}\index{commuting cube}
\begin{equation*}
\begin{tikzcd}[column sep=large,row sep=large]
& C' \arrow[dl,swap,"{p'}"] \arrow[dr,"{q'}"] \arrow[d,swap,"{h_C}" near end] \\
A' \arrow[d,swap,"{h_A}"] & C \arrow[dl,swap,"{p}" very near start] \arrow[dr,"{q}" very near start] & B' \arrow[dl,crossing over,"{g'}" near end] \arrow[d,"{h_B}"] \\
A \arrow[dr,swap,"f"] & X' \arrow[d,swap,"{h_X}" near start] \arrow[from=ul,crossing over,swap,"{f'}" near end] & B \arrow[dl,"{g}"] \\
& X
\end{tikzcd}
\end{equation*}
consists of types and maps as indicated in the diagram, equipped with
\begin{enumerate}
\item homotopies
  \begin{align*}
    \mathsf{top} & : f' \circ p' \htpy g' \circ q' \\
    \mathsf{back\usc{}left} & : p \circ h_C \htpy h_A \circ p' \\
    \mathsf{back\usc{}right} & : q \circ h_C \htpy h_B \circ q' \\
    \mathsf{front\usc{}left} & : f \circ h_A \htpy h_X \circ f' \\
    \mathsf{front\usc{}right} & : g \circ h_B \htpy h_X \circ g' \\
    \mathsf{bottom} & : f \circ p \htpy g \circ q
  \end{align*}
  witnessing that the 6 faces of the cube commute,
\item and a homotopy 
  \begin{align*}
    % ((((h ·l back-left) ∙h (front-left ·r f')) ∙h (hD ·l top))) ~
    % ((bottom ·r hA) ∙h ((k ·l back-right) ∙h (front-right ·r g')))
\mathsf{coh\usc{}cube} & : \ct{(\ct{(f \cdot \mathsf{back\usc{}left})}{(\mathsf{front\usc{}left}\cdot p')})}{(h_X \cdot \mathsf{top})} \\
& \qquad \htpy \ct{(\mathsf{bottom}\cdot h_C)}{(\ct{(g \cdot \mathsf{back\usc{}right})}{(\mathsf{front\usc{}right}\cdot q')})}
\end{align*}
filling the cube.
\end{enumerate}
\end{defn}

In the following lemma we show that if a cube commutes, then so do its rotations and mirror symmetries (that preserve the directions of the arrows).\footnote{The group acting on commuting cubes of maps is the \emph{dihedral group} $D_3$ which has order $6$.} This fact is obviously true, but there is some `path algebra' involved that we wish to demonstrate at least once.

\begin{lem}
  Consider a commuting cube
  \begin{equation*}
    \begin{tikzcd}
      & C' \arrow[dl] \arrow[d] \arrow[dr] \\
      A' \arrow[d] & C \arrow[dl] \arrow[dr] & B' \arrow[dl,crossing over] \arrow[d] \\
      A \arrow[dr] & X' \arrow[from=ul,crossing over] \arrow[d] & B \arrow[dl] \\
      & X.
    \end{tikzcd}
  \end{equation*}
  Then the cubes

  \begin{center}
  \begin{minipage}{.3\textwidth}
  \begin{equation*}
    \begin{tikzcd}
      & C' \arrow[dl] \arrow[d] \arrow[dr] \\
      C \arrow[d] & B' \arrow[dl] \arrow[dr] & A' \arrow[dl,crossing over] \arrow[d] \\
      B \arrow[dr] & A \arrow[from=ul,crossing over] \arrow[d] & X' \arrow[dl] \\
      & X
    \end{tikzcd}
  \end{equation*}
  \end{minipage}
  \begin{minipage}{.3\textwidth}
  \begin{equation*}
    \begin{tikzcd}
      & C' \arrow[dl] \arrow[d] \arrow[dr] \\
      B' \arrow[d] & A' \arrow[dl] \arrow[dr] & C \arrow[dl,crossing over] \arrow[d] \\
      X' \arrow[dr] & B \arrow[from=ul,crossing over] \arrow[d] & A \arrow[dl] \\
      & X
    \end{tikzcd}
  \end{equation*}
  \end{minipage}

  \begin{minipage}{.3\textwidth}
  \begin{equation*}
    \begin{tikzcd}
      & C' \arrow[dl] \arrow[d] \arrow[dr] \\
      C \arrow[d] & A' \arrow[dl] \arrow[dr] & B' \arrow[dl,crossing over] \arrow[d] \\
      A \arrow[dr] & B \arrow[from=ul,crossing over] \arrow[d] & X' \arrow[dl] \\
      & X.
    \end{tikzcd}
  \end{equation*}
  \end{minipage}
  \begin{minipage}{.3\textwidth}
  \begin{equation*}
    \begin{tikzcd}
      & C' \arrow[dl] \arrow[d] \arrow[dr] \\
      A' \arrow[d] & B' \arrow[dl] \arrow[dr] & C \arrow[dl,crossing over] \arrow[d] \\
      X' \arrow[dr] & A \arrow[from=ul,crossing over] \arrow[d] & B \arrow[dl] \\
      & X.
    \end{tikzcd}
  \end{equation*}
  \end{minipage}
  \begin{minipage}{.3\textwidth}
  \begin{equation*}
    \begin{tikzcd}
      & C' \arrow[dl] \arrow[d] \arrow[dr] \\
      B' \arrow[d] & C \arrow[dl] \arrow[dr] & A' \arrow[dl,crossing over] \arrow[d] \\
      B \arrow[dr] & X' \arrow[from=ul,crossing over] \arrow[d] & A \arrow[dl] \\
      & X.
    \end{tikzcd}
  \end{equation*}
  \end{minipage}
  \end{center}
  also commute.
\end{lem}

\begin{proof}
  We only show that the first cube commutes, which is obtained by a counter-clockwise rotation of the original cube around the axis through $C'$ and $X$. The other cases are similar, and they are formalized in the accompagnying Agda library.

  First we list the homotopies witnessing that the faces of the cube commute:
  \begin{align*}
    \mathsf{top}' & \defeq \mathsf{back\usc{}left} \\
    \mathsf{back\usc{}left}' & \defeq \mathsf{back\usc{}right}^{-1} \\
    \mathsf{back\usc{}right}' & \defeq \mathsf{top}^{-1} \\
    \mathsf{front\usc{}left}' & \defeq \mathsf{bottom}^{-1} \\
    \mathsf{front\usc{}right}' & \defeq \mathsf{front\usc{}left}^{-1} \\
    \mathsf{bottom}' & \defeq \mathsf{front\usc{}right}. 
  \end{align*}
  Thus, to show that the cube commutes, we have to show that there is a homotopy of type
  \begin{align*}
    & \ct{\Big(\ct{(g \cdot \mathsf{back\usc{}right}^{-1})}{(\mathsf{bottom}^{-1}\cdot h_C)}\Big)}{(f \cdot \mathsf{back\usc{}left})} \\
    & \qquad\qquad \htpy \ct{(\mathsf{front\usc{}right}\cdot q')}{\Big(\ct{(h_X \cdot \mathsf{top}^{-1})}{(\mathsf{front\usc{}left}^{-1}\cdot p')}\Big)}.
  \end{align*}
  Recall that $h\cdot H^{-1}\htpy (h\cdot H)^{-1}$ and $H^{-1}\cdot h\htpy (H\cdot h)^{-1}$, so it suffices to construct a homotopy
  \begin{align*}
    & \ct{\Big(\ct{(g \cdot \mathsf{back\usc{}right})^{-1}}{(\mathsf{bottom}\cdot h_C)^{-1}}\Big)}{(f \cdot \mathsf{back\usc{}left})} \\
    & \qquad\qquad \htpy \ct{(\mathsf{front\usc{}right}\cdot q')}{\Big(\ct{(h_X \cdot \mathsf{top})^{-1}}{(\mathsf{front\usc{}left}\cdot p')^{-1}}\Big)}.
  \end{align*}
  Now we note that pointwise, our goal is of the form
  \begin{equation*}
    \ct{(\ct{\varepsilon^{-1}}{\delta^{-1}})}{\alpha}=\ct{\zeta}{(\ct{\gamma^{-1}}{\beta^{-1}})}, %%% check greek alphabet
  \end{equation*}
  whereas the assumption that the original cube commutes yields an identification of the form
  \begin{equation*}
    \ct{(\ct{\alpha}{\beta})}{\gamma}=\ct{\delta}{(\ct{\varepsilon}{\zeta})}
  \end{equation*}
  Indeed, in the case that $\alpha$, $\beta$, $\gamma$, $\delta$, $\varepsilon$, and $\zeta$ are general identifications, we can conclude our goal using path induction on all of them.
\end{proof}

\begin{lem}
Given a commuting cube as in \cref{defn:cube} we obtain a commuting square
\begin{equation*}
\begin{tikzcd}
\fib{f_{1\check{1}1}}{x} \arrow[r] \arrow[d] & \fib{f_{0\check{1}1}}{f_{\check{1}01}(x)} \arrow[d] \\
\fib{f_{1\check{1}0}}{f_{10\check{1}}(x)} \arrow[r] & \fib{f_{0\check{1}0}}{f_{00\check{1}}(x)}
\end{tikzcd}
\end{equation*}
for any $x:A_{101}$. 
\end{lem}

\begin{lem}
Consider a commuting cube
\begin{equation*}
\begin{tikzcd}[column sep=large,row sep=large]
& C' \arrow[dl] \arrow[dr] \arrow[d] \\
A' \arrow[d] & C \arrow[dl] \arrow[dr] & B' \arrow[dl,crossing over] \arrow[d] \\
A \arrow[dr] & X' \arrow[d] \arrow[from=ul,crossing over] & B \arrow[dl] \\
& X,
\end{tikzcd}
\end{equation*}
If the bottom and front right squares are pullback squares, then the back left square is a pullback if and only if the top square is.
\end{lem}

\begin{rmk}\label{rmk:strongly-cartesian}
By rotating the cube we also obtain:
\begin{enumerate}
\item If the bottom and front left squares are pullback squares, then the back right square is a pullback if and only if the top square is.
\item If the front left and front right squares are pullback, then the back left square is a pullback if and only if the back right square is.
\end{enumerate}
By combining these statements it also follows that if the front left, front right, and bottom squares are pullback squares, then if any of the remaining three squares are pullback squares, all of them are. Cubes that consist entirely of pullback squares are sometimes called \define{strongly cartesian}\index{strongly cartesian cube}.
\end{rmk}

\subsection{Families of pullbacks}

\begin{lem}\label{lem:fiberwise-pullback}
Consider a pullback square\index{pullback!Sigma-type of pullbacks@{$\Sigma$-type of pullbacks}}
  \begin{equation*}
    \begin{tikzcd}
      C \arrow[r,"q"] \arrow[d,swap,"p"] & B \arrow[d,"g"] \\
      A \arrow[r,swap,"f"] & X
    \end{tikzcd}
  \end{equation*}
  with $H : f \circ p \htpy g \circ h$. Furthermore, consider type families $P_X$, $P_A$, $P_B$, and $P_C$ over $X$, $A$, $B$, and $C$ respectively, equipped with families of maps
  \begin{align*}
    f' & : \prd{a:A} P_A(a) \to P_X(f(a)) \\
    g' & : \prd{b:B} P_B(b) \to P_X(g(b)) \\
    p' & : \prd{c:C} P_C(c) \to P_A(p(c)) \\
    q' & : \prd{c:C} P_C(c) \to P_B(q(c)),
  \end{align*}
  and for each $c:C$ a homotopy $H'_c$ witnessing that the square
  \begin{equation}\label{eq:family-squares-pullback}
    \begin{tikzcd}
      P_C(c) \arrow[rr,"{q'_c}"] \arrow[d,swap,"{p'_c}"] & &[3em] P_B(q(c)) \arrow[d,"{g'_{q(c)}}"] \\
      P_A(p(c)) \arrow[r,swap,"{f'_{p(c)}}"] & P_X(f(p(c))) \arrow[r,swap,"{\tr_{P_X}(H(c))}"] & P_X(g(q(c)))
    \end{tikzcd}
  \end{equation}
  commutes. Then the following are equivalent:
  \begin{enumerate}
  \item For each $c:C$ the square in \cref{eq:family-squares-pullback} is a pullback square.
  \item The square
    \begin{equation}\label{eq:total-square-pullback}
      \begin{tikzcd}[column sep=huge]
        \sm{c:C}P_C(c)
        \arrow[r,"{\tot[q]{q'}}"] \arrow[d,swap,"{\tot[p]{p'}}"] &
        \sm{b:B}P_B(b) \arrow[d,"{\tot[g]{g'}}"] \\
        \sm{a:A}P_A(a) \arrow[r,swap,"{\tot[f]{f'}}"] & \sm{x:X}P_X(x)
      \end{tikzcd}
    \end{equation}
    is a pullback square.
  \end{enumerate}
\end{lem}


\begin{cor}
Consider a pullback square
\begin{equation*}
\begin{tikzcd}
C \arrow[r,"q"] \arrow[d,swap,"p"] & B \arrow[d,"g"] \\
A \arrow[r,swap,"f"] & X,
\end{tikzcd}
\end{equation*}
with $H:f\circ p\htpy g\circ q$, and let $c_1,c_2:C$. Then the square
\begin{equation*}
\begin{tikzcd}[column sep=8em]
(c_1=c_2) \arrow[r,"\apfunc{q}"] \arrow[d,swap,"\apfunc{p}"] & (q(c_1)=q(c_2)) \arrow[d,"\lam{\beta}\ct{H(c_1)}{\ap{g}{\beta}}"] \\
(p(c_1)=p(c_2)) \arrow[r,swap,"\lam{\alpha}\ct{\ap{f}{\alpha}}{H(c_2)}"] & f(p(c_1))=g(q(c_2)),
\end{tikzcd}
\end{equation*}
commutes and is a pullback square.
\end{cor}


\begin{thm}
  Consider a commuting cube
  \begin{equation*}
    \begin{tikzcd}
      & C' \arrow[dl] \arrow[dr] \arrow[d] \\
      A' \arrow[d] & C \arrow[dl] \arrow[dr] & B' \arrow[crossing over,dl] \arrow[d] \\
      A \arrow[dr] & X' \arrow[d] \arrow[from=ul,crossing over] & B \arrow[dl] \\
      & X
    \end{tikzcd}
  \end{equation*}
  in which the bottom square is a pullback square. Then the following are equivalent:
  \begin{enumerate}
  \item The top square is a pullback square.
  \item The square
    \begin{equation*}
      \begin{tikzcd}
        \fib{\gamma}{c} \arrow[d] \arrow[r] & \fib{\beta}{q(c)} \arrow[d] \\
        \fib{\alpha}{p(c)} \arrow[r] & \fib{\varphi}{f(p(c))}
      \end{tikzcd}
    \end{equation*}
    is a pullback square for each $c:C$.
  \end{enumerate}
\end{thm}


\subsection{The 3-by-3-properties for pullbacks and pushouts}

\begin{thm}
  Consider a commuting diagram of the form
  \begin{equation*}
    \begin{tikzcd}[column sep=large,row sep=large]
      AA \arrow[r,"Af"] \arrow[d,swap,"fA"] \arrow[dr,phantom,"\Rightarrow" description] & AX \arrow[d,swap,"fX"] & AB \arrow[l,swap,"Ag"] \arrow[d,"gB"] \arrow[dl,phantom,"\Leftarrow" description] \\
      XA \arrow[r,"Xf"] & XX & XB \arrow[l,swap,"Xg"] \\
      BA \arrow[u,"gA"] \arrow[r,swap,"Bf"] & BX \arrow[u,"gX"] & BB \arrow[u,swap,"gB"] \arrow[l,"Bg"]
    \end{tikzcd}
  \end{equation*}
  with homotopies
  \begin{align*}
    ff & : Xf \circ fA \htpy Af \circ fX \\
    fg & : Xg \circ gB \htpy Ag \circ fX \\
    gf & : 
  \end{align*}
  filling the (small) squares. Furthermore, consider
  pullback squares
  \begin{equation*}
    \begin{tikzcd}
      AC \arrow[r] \arrow[d] & AB \arrow[d] & XC \arrow[r] \arrow[d] & XB \arrow[d] & BC \arrow[r] \arrow[d] & BB \arrow[d] \\
      AA \arrow[r] & AX & XA \arrow[r] & XX & BA \arrow[r] & BX
    \end{tikzcd}
  \end{equation*}
  \begin{equation*}
    \begin{tikzcd}
      CA \arrow[r] \arrow[d] & BA \arrow[d] & CX \arrow[r] \arrow[d] & BX \arrow[d] & CB \arrow[r] \arrow[d] & BB \arrow[d] \\
      AA \arrow[r] & XA & AX \arrow[r] & XX & AB \arrow[r] & XB.
    \end{tikzcd}
  \end{equation*}
  Finally, consider a commuting square
  \begin{equation*}
    \begin{tikzcd}
      D_3 \arrow[r] \arrow[d] & D_2 \arrow[d] \\
      D_0 \arrow[r] & D_1.
    \end{tikzcd}
  \end{equation*}
  Then the following are equivalent:
  \begin{enumerate}
  \item This square is a pullback square.
  \item The induced square
    \begin{equation*}
      \begin{tikzcd}
        D_3 \arrow[r] \arrow[d] & C_3 \arrow[d] \\
        A_3 \arrow[r] & B_3
      \end{tikzcd}
    \end{equation*}
    is a pullback square.
  \end{enumerate}
\end{thm}

\begin{proof}
  First we construct an equivalence
  \begin{equation*}
    (A_0\times_{B_0}C_0)\times_{(A_1\times_{B_1}C_1)}(A_2\times_{B_2} C_2) \eqvsym (A_0\times_{A_1}A_2)\times_{(B_0\times_{B_1} B_2)} (C_0\times_{C_1}C_2).
  \end{equation*}
  Now it follows that we have an equivalence
  \begin{equation*}
    \mathsf{cone}(f_0,g_0)
  \end{equation*}
\end{proof}


\begin{exercises}
\item Some exercises.
\end{exercises}


\chapter{Descent}\label{chap:descent}

\section{Type families over pushouts}

Given a pushout square
\begin{equation*}
\begin{tikzcd}
S \arrow[r,"g"] \arrow[d,swap,"f"] & B \arrow[d,"j"] \\
A \arrow[r,swap,"i"] & X.
\end{tikzcd}
\end{equation*}
with $H:i\circ f\htpy j\circ g$, and a family $P:X\to\UU$, we obtain
\begin{align*}
P\circ i & : A \to \UU \\
P\circ j & : B \to \UU \\
\lam{x}\mathsf{tr}_P(H(x)) & : \prd{x:S} \eqv{P(i(f(x)))}{P(j(g(x)))}.
\end{align*}
Our goal in the current section is to show that the triple $(P_A,P_B,P_S)$ consisting of $P_A\defeq P\circ i$, $P_B\defeq P\circ j$, and $P_S\defeq \lam{x}\mathsf{tr}_P(H(x))$ characterizes the family $P$ over $X$.

\begin{defn}
Consider a commuting square
\begin{equation*}
\begin{tikzcd}
S \arrow[r,"g"] \arrow[d,swap,"f"] & B \arrow[d,"j"] \\
A \arrow[r,swap,"i"] & X.
\end{tikzcd}
\end{equation*}
with $H:i\circ f\htpy j\circ g$, where all types involved are in $\UU$. The type $\mathsf{Desc}(\mathcal{S})$\index{Desc@{$\mathsf{Desc}(\mathcal{S})$}|textbf} of \define{descent data}\index{descent data|textbf} for $X$, is defined defined to be the type of triples $(P_A,P_B,P_S)$ consisting of
\begin{align*}
P_A & : A \to \UU \\
P_B & : B \to \UU \\
P_S & : \prd{x:S} \eqv{P_A(f(x))}{P_B(g(x))}.
\end{align*}
\end{defn}

\begin{defn}
Given a commuting square
\begin{equation*}
\begin{tikzcd}
S \arrow[r,"g"] \arrow[d,swap,"f"] & B \arrow[d,"j"] \\
A \arrow[r,swap,"i"] & X.
\end{tikzcd}
\end{equation*}
with $H:i\circ f\htpy j\circ g$, we define the map\index{desc_fam@{$\mathsf{desc\usc{}fam}_{\mathcal{S}}$}|textbf}
\begin{equation*}
\mathsf{desc\usc{}fam}_{\mathcal{S}}(i,j,H) : (X\to \UU)\to \mathsf{Desc}(\mathcal{S})
\end{equation*}
by $P\mapsto (P\circ i,P\circ j,\lam{x}\mathsf{tr}_P(H(x)))$.
\end{defn}

\begin{thm}\label{thm:desc_fam}
Consider a pushout square
\begin{equation*}
\begin{tikzcd}
S \arrow[r,"g"] \arrow[d,swap,"f"] & B \arrow[d,"j"] \\
A \arrow[r,swap,"i"] & X.
\end{tikzcd}
\end{equation*}
with $H:i\circ f\htpy j\circ g$, where all types involved are in $\UU$, and suppose we have
\begin{align*}
P_A & : A \to \UU \\
P_B & : B \to \UU \\
P_S & : \prd{x:S} \eqv{P_A(f(x))}{P_B(g(x))}.
\end{align*}
Then the function\index{desc_fam@{$\mathsf{desc\usc{}fam}_{\mathcal{S}}$}!is an equivalence|textit}
\begin{equation*}
\mathsf{desc\usc{}fam}_{\mathcal{S}}(i,j,H) : (X\to \UU)\to \mathsf{Desc}(\mathcal{S})
\end{equation*}
is an equivalence.
\end{thm}

\begin{proof}
By the 3-for-2 property of equivalences it suffices to construct an equivalence $\varphi:\mathsf{cocone}_{\mathcal{S}}(\UU)\to\mathsf{Desc}(\mathcal{S})$ such that the triangle
\begin{equation*}
\begin{tikzcd}[column sep=tiny]
& \UU^X \arrow[dl,swap,"{\mathsf{cocone\usc{}map}_{\mathcal{S}}(i,j,H)}"] \arrow[dr,"{\mathsf{desc\usc{}fam}_{\mathcal{S}}(i,j,H)}"] & \phantom{\mathsf{cocone}_{\mathcal{S}}(\UU)} \\
\mathsf{cocone}_{\mathcal{S}}(\UU) \arrow[rr,densely dotted,"\eqvsym","\varphi"'] & & \mathsf{Desc}(\mathcal{S})
\end{tikzcd}
\end{equation*}
commutes.

Since we have equivalences
\begin{equation*}
\mathsf{equiv\usc{}eq}:\eqv{\Big(P_A(f(x))=P_B(g(x))\Big)}{\Big(\eqv{P_A(f(x))}{P_B(g(x))}\Big)}
\end{equation*}
for all $x:S$, we obtain by \cref{ex:equiv_pi} an equivalence on the dependent products
\begin{equation*}
{\Big(\prd{x:S}P_A(f(x))=P_B(g(x))\Big)}\to{\Big(\prd{x:S}\eqv{P_A(f(x))}{P_B(g(x))}\Big)}.
\end{equation*}
We define $\varphi$ to be the induced map on total spaces. Explicitly, we have
\begin{equation*}
\varphi\defeq \lam{(P_A,P_B,K)}(P_A,P_B,\lam{x}\mathsf{equiv\usc{}eq}(K(x))).
\end{equation*}
Then $\varphi$ is an equivalence by \cref{thm:fib_equiv}, and the triangle commutes by \cref{ex:tr_ap}.
\end{proof}

\begin{cor}\label{cor:desc_fam}
Consider descent data $(P_A,P_B,P_S)$ for a pushout square as in \cref{thm:desc_fam}.
Then the type of quadruples $(P,e_A,e_B,e_S)$ consisting of a family $P:X\to\UU$ equipped with fiberwise equivalences
\begin{samepage}
\begin{align*}
e_A & : \prd{a:A}\eqv{P_A(a)}{P(i(a))} \\
e_B & : \prd{b:B}\eqv{P_B(a)}{P(j(b))}
\end{align*}
\end{samepage}%
and a homotopy $e_S$ witnessing that the square
\begin{equation*}
\begin{tikzcd}[column sep=huge]
P_A(f(x)) \arrow[r,"e_A(f(x))"] \arrow[d,swap,"P_S(x)"] & P(i(f(x))) \arrow[d,"\mathsf{tr}_P(H(x))"] \\
P_B(g(x)) \arrow[r,swap,"e_B(g(x))"] & P(j(g(x)))
\end{tikzcd}
\end{equation*}
commutes, is contractible.
\end{cor}

\begin{proof}
The fiber of this map at $(P_A,P_B,P_S)$ is equivalent to the type of quadruples $(P,e_A,e_B,e_S)$ as described in the theorem, which are contractible by \cref{thm:contr_equiv}.
\end{proof}

\section{The flattening lemma for pushouts}

In this section we consider a pushout square
\begin{equation*}
\begin{tikzcd}
S \arrow[r,"g"] \arrow[d,swap,"f"] & B \arrow[d,"j"] \\
A \arrow[r,swap,"i"] & X.
\end{tikzcd}
\end{equation*}
with $H:i\circ f\htpy j\circ g$, descent data
\begin{align*}
P_A & : A \to \UU \\
P_B & : B \to \UU \\
P_S & : \prd{x:S} \eqv{P_A(f(x))}{P_B(g(x))},
\end{align*}
and a family $P:X\to\UU$ equipped with 
\begin{align*}
e_A & : \prd{a:A}\eqv{P_A(a)}{P(i(a))} \\
e_B & : \prd{b:B}\eqv{P_B(a)}{P(j(b))}
\end{align*}
and a homotopy $e_S$ witnessing that the square
\begin{equation*}
\begin{tikzcd}[column sep=huge]
P_A(f(x)) \arrow[r,"e_A(f(x))"] \arrow[d,swap,"P_S(x)"] & P(i(f(x))) \arrow[d,"\mathsf{tr}_P(H(x))"] \\
P_B(g(x)) \arrow[r,swap,"e_B(g(x))"] & P(j(g(x)))
\end{tikzcd}
\end{equation*}
commutes.

\begin{defn}
We define a commuting square
\begin{equation*}
\begin{tikzcd}
\sm{x:S}P_A(f(x)) \arrow[d,swap,"{f'}"] \arrow[r,"{g'}"] & \sm{b:B}P_B(b) \arrow[d,"{j'}"] \\
\sm{a:A}P_A(a) \arrow[r,swap,"{i'}"] & \sm{x:X}P(x)
\end{tikzcd}
\end{equation*}
with a homotopy $H':i'\circ f'\htpy j'\circ g'$.
\end{defn}

\begin{constr}
We define
\begin{align*}
f' & \defeq \total[f]{\lam{x}\idfunc[P_A(f(x))]} \\
g' & \defeq \total[g]{e_S} \\
i' & \defeq \total[i]{e_A} \\
j' & \defeq \total[j]{e_B}.
\end{align*}
The remaining goal is to construct a homotopy $H':i'\circ f'\htpy j'\circ g'$. Thus, we have to show that
\begin{equation*}
(i(f(x)),e_A(y))=(j(g(x)),e_B(e_S(y)))
\end{equation*}
for any $x:S$ and $y:P_A(f(x))$. We have he identification
\begin{equation*}
\mathsf{eq\usc{}pair}(H(x),e_S(x,y)^{-1})
\end{equation*}
of this type.
\end{constr}

\begin{defn}
We will write $\mathcal{S'}$ for the span
\begin{equation*}
\begin{tikzcd}
\sm{a:A}P_A(a) & \sm{x:S}P_A(f(x)) \arrow[l,swap,"{f'}"] \arrow[r,"{g'}"] & \sm{b:B}P_B(b).
\end{tikzcd}
\end{equation*}
\end{defn}

\begin{lem}[The flattening lemma]\label{lem:flattening}
The commuting square\index{flattening lemma!for pushouts|textit}
\begin{equation*}
\begin{tikzcd}
\sm{x:S}P_A(f(x)) \arrow[d,swap,"{f'}"] \arrow[r,"{g'}"] & \sm{b:B}P_B(b) \arrow[d,"{j'}"] \\
\sm{a:A}P_A(a) \arrow[r,swap,"{i'}"] & \sm{x:X}P(x)
\end{tikzcd}
\end{equation*}
is a pushout square.
\end{lem}

\begin{proof}
We will show that the map
\begin{equation*}
\mathsf{cocone\usc{}map}_{\mathcal{S}'}(i',j',H'): \Big(\Big(\sm{x:X}P(x)\Big)\to Y\Big)\to \mathsf{cocone}_{\mathcal{S}'}(Y)
\end{equation*}
is an equivalence for any type $Y$.
Let $Y$ be a type. Note that the type $\mathsf{cocone}_{\mathcal{S}'}$ is equivalent to the type of triples $(u,v,w)$ consisting of
\begin{align*}
u & : \prd{a:A} P_A(a)\to Y \\
v & : \prd{b:B} P_B(b)\to Y \\
w & : \prd{x:S}{y:P_A(f(x))} u(f(x),y)=v(g(x),e_S(x,y)).
\end{align*}
Now observe that there is an equivalence
\begin{align*}
& \Big(\prd{y:P_A(f(x))} u(f(x),y)=v(g(x),e_S(x,y))\Big) \\
& \qquad \qquad \eqvsym \mathsf{tr}_{(t\mapsto P(t)\to Y)}(H(x),u'(f(x)))=v'(g(x))
\end{align*}
for any $u$ and $v$ as above, and any $x:S$. 
By this equivalence we obtain a commuting square
\begin{equation*}
\begin{tikzcd}
\Big(\Big({\sm{x:X}P(x)}\Big)\to Y\Big) \arrow[r,"\ind{\Sigma}","\eqvsym"'] \arrow[d,swap,"{\mathsf{cocone\usc{}map}_{\mathcal{S}'}(i',j',H')}"] & \prd{x:X}(P(x)\to Y) \arrow[d,"\mathsf{dgen}_{\mathcal{S}}"] \\
\mathsf{cocone}_{\mathcal{S}'}(Y) \arrow[r,"\eqvsym"] & \Psi
\end{tikzcd}
\end{equation*}
where $\Psi$ is the type of triples $(u',v',w')$ consisting of
\begin{align*}
u' & : \prd{a:A} P(i(a))\to Y \\
v' & : \prd{b:B} P(j(b))\to Y \\
w' & : \prd{x:S} \mathsf{tr}_{(t\mapsto P(t)\to Y)}(H(x),u'(f(x)))=v'(g(x)),
\end{align*}
Since the dependent action on generators $\mathsf{dgen}_{\mathcal{S}}$ is an equivalence it follows by the 3-for-2 property of equivalences that $\mathsf{cocone\usc{}map}_{\mathcal{S}'}(i',j',H')$ is an equivalence, as desired.
\end{proof}

\section{The descent property for pushouts}

In the previous section there was a significant role for fiberwise equivalences, and we know by \cref{thm:pb_fibequiv,cor:pb_fibequiv}: fiberwise equivalences indicate the presence of pullbacks. In this section we reformulate the results of the previous section using pullbacks where we used fiberwise equivalences before, to obtain new and useful results. We begin by considering the type of descent data from the perspective of pullback squares.

\begin{defn}
Consider a span $\mathcal{S}$ from $A$ to $B$, and a span $\mathcal{S}'$ from $A'$ to $B'$. A \define{cartesian transformation}of spans\index{cartesian transformation!of spans|textbf} from $\mathcal{S}'$ to $\mathcal{S}$ is a diagram of the form
\begin{equation*}
\begin{tikzcd}
A' \arrow[d,swap,"h_A"]  & S' \arrow[l,swap,"{f'}"] \arrow[r,"{g'}"] \arrow[d,swap,"h_S"] & B' \arrow[d,"h_B"] \\
A & S \arrow[l,"f"] \arrow[r,swap,"g"] & B
\end{tikzcd}
\end{equation*}
with $F:f\circ h_S\htpy h_A\circ f'$ and $G:g\circ h_S\htpy h_B\circ g'$, where both squares are pullback squares. 

The type $\mathsf{cart}(\mathcal{S}',\mathcal{S})$\index{cart(S,S')@{$\mathsf{cart}(\mathcal{S},\mathcal{S}')$}|textbf} of cartesian transformation is the type of tuples
\begin{equation*}
(h_A,h_S,h_B,F,G,p_f,p_g)
\end{equation*}
where $p_f:\mathsf{is\usc{}pullback}(h_S,h_A,F)$ and $p_g:\mathsf{is\usc{}pullback}(h_S,h_B,G)$, and we write
\begin{equation*}
\mathsf{Cart}(\mathcal{S}) \defeq \sm{A',B':\UU}{\mathcal{S}':\mathsf{span}(A',B')}\mathsf{cart}(\mathcal{S}',\mathcal{S}).
\end{equation*}
\end{defn}

\begin{lem}\label{lem:cart_desc}
There is an equivalence\index{cart_desc@{$\mathsf{cart\usc{}desc}_{\mathcal{S}}$}|textit}
\begin{equation*}
\mathsf{cart\usc{}desc}_{\mathcal{S}}:\mathsf{Desc}(\mathcal{S})\to \mathsf{Cart}(\mathcal{S}).
\end{equation*}
\end{lem}

\begin{proof}
Note that by \cref{thm:pb_fibequiv_complete} it follows that the types of triples $(f',F,p_f)$ and $(g',G,p_g)$ are equivalent to the types of fiberwise equivalences
\begin{align*}
& \prd{x:S}\eqv{\fib{h_S}{x}}{\fib{h_A}{f(x)}} \\
& \prd{x:S}\eqv{\fib{h_S}{x}}{\fib{h_B}{g(x)}}
\end{align*} 
respectively. Furthermore, by \cref{thm:fam_proj} the types of pairs $(S',h_S)$, $(A',h_A)$, and $(B',h_B)$ are equivalent to the types $S\to \UU$, $A\to \UU$, and $B\to \UU$, respectively. Therefore it follows that the type $\mathsf{Cart}(\mathcal{S})$ is equivalent to the type of tuples $(Q,P_A,\varphi,P_B,P_S)$ consisting of
\begin{align*}
Q & : S\to \UU \\
P_A & : A \to \UU \\
P_B & : B \to \UU \\
\varphi & : \prd{x:S}\eqv{Q(x)}{P_A(f(x))} \\
P_S & : \prd{x:S}\eqv{Q(x)}{P_B(g(x))}.
\end{align*}
However, the type of $\varphi$ is equivalent to the type $P_A\circ f=Q$. Thus we see that the type of pairs $(Q,\varphi)$ is contractible, so our claim follows.
\end{proof}

\begin{defn}
We define an operation\index{cart map!{$\mathsf{cart\usc{}map}_{\mathcal{S}}$}|textbf}
\begin{equation*}
\mathsf{cart\usc{}map}_{\mathcal{S}}:{\Big(\sm{X':\UU}X'\to X\Big)}\to \mathsf{Cart}(\mathcal{S}).
\end{equation*}
\end{defn}

\begin{constr}
Let $X':\UU$ and $h_X:X'\to X$. Then we define the types
\begin{align*}
A' & \defeq A\times_X X' \\
B' & \defeq B\times_X X'.
\end{align*}
Next, we define a span $\mathcal{S'}\defeq(S',f',g')$ from $A'$ to $B'$. We take
\begin{align*}
S' & \defeq S\times_A A' \\
f' & \defeq \pi_2.
\end{align*}
To define $g'$, let $s:S$, let $(a,x',p):A\times_X X'$, and let $q:f(s)=a$. Our goal is to construct a term of type $B\times_X X'$. We have $g(s):B$ and $x':X'$, so it remains to show that $j(g(s))=h_X(x')$. We construct such an identification as a concatenation
\begin{equation*}
\begin{tikzcd}
j(g(s)) \arrow[r,equals,"H(s)^{-1}"] &[1ex] i(f(s)) \arrow[r,equals,"\ap{i}{q}"] &[1ex] i(a) \arrow[r,equals,"p"] & h_X(x').
\end{tikzcd}
\end{equation*}
To summaze, the map $g'$ is defined as
\begin{equation*}
g' \defeq \lam{(s,(a,x',p),q)}(g(s),x',\ct{H(s)^{-1}}{(\ct{\ap{i}{q}}{p})}).
\end{equation*}
Then we have commuting squares
\begin{equation*}
\begin{tikzcd}
A\times_X X' \arrow[d] & S\times_A A' \arrow[d] \arrow[l] \arrow[r] & B\times_X X' \arrow[d] \\
A & S \arrow[l] \arrow[r] & B.
\end{tikzcd}
\end{equation*}
Moreover, these squares are pullback squares by \cref{thm:pb_pasting}.
\end{constr}

The following theorem is analogous to \cref{thm:desc_fam}.

\begin{thm}[The descent theorem for pushouts]\label{thm:cart_map}\index{descent theorem!for pushouts|textit}
The operation $\mathsf{cart\usc{}map}_{\mathcal{S}}$\index{cart map!{$\mathsf{cart\usc{}map}_{\mathcal{S}}$}!is an equivalence|textit} is an equivalence
\begin{equation*}
\eqv{\Big(\sm{X':\UU}X'\to X\Big)}{\mathsf{Cart}(\mathcal{S})}
\end{equation*}
\end{thm}

\begin{proof}
It suffices to show that the square
\begin{equation*}
\begin{tikzcd}[column sep=huge]
X\to \UU \arrow[r,"{\mathsf{desc\usc{}fam}_{\mathcal{S}}(i,j,H)}"] \arrow[d,swap,"\mathsf{map\usc{}fam}_X"] & \mathsf{Desc}(\mathcal{S}) \arrow[d,"\mathsf{cart\usc{}desc}_{\mathcal{S}}"] \\
\sm{X':\UU}X'\to X \arrow[r,swap,"\mathsf{cart\usc{}map}_{\mathcal{S}}"] & \mathsf{Cart}(\mathcal{S})
\end{tikzcd}
\end{equation*}
commutes. To see that this suffices, note that the operation $\mathsf{map\usc{}fam}_X$ is an equivalence by \cref{thm:fam_proj}, the operation $\mathsf{desc\usc{}fam}_{\mathcal{S}}(i,j,H)$ is an equivalence by \cref{thm:desc_fam}, and the operation $\mathsf{cart\usc{}desc}_{\mathcal{S}}$ is an equivalence by \cref{lem:cart_desc}.

To see that the square commutes, note that the composite
\begin{equation*}
\mathsf{cart\usc{}map}_{\mathcal{S}}\circ \mathsf{map\usc{}fam}_X
\end{equation*}
takes a family $P:X\to \UU$ to the cartesian transformation of spans
\begin{equation*}
\begin{tikzcd}
A\times_X\tilde{P} \arrow[d,swap,"\pi_1"] & S\times_A\Big(A\times_X\tilde{P}\Big) \arrow[l] \arrow[r] \arrow[d,swap,"\pi_1"] & B\times_X\tilde{P} \arrow[d,"\pi_1"] \\
A & S \arrow[l] \arrow[r] & B,
\end{tikzcd}
\end{equation*}
where $\tilde{P}\defeq\sm{x:X}P(x)$.

The composite 
\begin{equation*}
\mathsf{cart\usc{}desc}_{\mathcal{S}}\circ \mathsf{desc\usc{}fam}_X
\end{equation*}
takes a family $P:X\to \UU$ to the cartesian transformation of spans
\begin{equation*}
\begin{tikzcd}
\sm{a:A}P(i(a)) \arrow[d] & \sm{s:S}P(i(f(s))) \arrow[l] \arrow[r] \arrow[d] & \sm{b:B}P(j(b)) \arrow[d] \\
A & S \arrow[l] \arrow[r] & B
\end{tikzcd}
\end{equation*}
These cartesian natural transformations are equal by \cref{lem:pb_subst}
\end{proof}

Since $\mathsf{cart\usc{}map}_{\mathcal{S}}$ is an equivalence it follows that its fibers are contractible. This is essentially the content of the following corollary.

\begin{cor}
Consider a diagram of the form 
\begin{equation*}
\begin{tikzcd}
& S' \arrow[d,swap,"h_S"] \arrow[dl,swap,"{f'}"] \arrow[dr,"{g'}"] \\
A' \arrow[d,swap,"h_A"] & S \arrow[dl,swap,"f"] \arrow[dr,"g"] & B' \arrow[d,"{h_B}"] \\
A \arrow[dr,swap,"i"] & & B \arrow[dl,"j"] \\
& X
\end{tikzcd}
\end{equation*}
with homotopies
\begin{align*}
F & : f\circ h_S \htpy h_A\circ f' \\
G & : g\circ h_S \htpy h_B\circ g' \\
H & : i\circ f \htpy j\circ g,
\end{align*}
and suppose that the bottom square is a pushout square, and the top squares are pullback squares.
Then the type of tuples $((X',h_X),(i',I,p),(j',J,q),(H',C))$ consisting of
\begin{enumerate}
\item A type $X':\UU$ together with a morphism
\begin{equation*}
h_X : X'\to X,
\end{equation*}
\item A map $i':A'\to X'$, a homotopy $I:i\circ h_A\htpy h_X\circ i'$, and a term $p$ witnessing that the square
\begin{equation*}
\begin{tikzcd}
A' \arrow[d,swap,"h_A"] \arrow[r,"{i'}"] & X' \arrow[d,"h_X"] \\
A \arrow[r,swap,"i"] & X
\end{tikzcd}
\end{equation*}
is a pullback square.
\item A map $j':B'\to X'$, a homotopy $J:j\circ h_B\htpy h_X\circ j'$, and a term $q$ witnessing that the square
\begin{equation*}
\begin{tikzcd}
B' \arrow[d,swap,"h_B"] \arrow[r,"{j'}"] & X' \arrow[d,"h_X"] \\
B \arrow[r,swap,"j"] & X
\end{tikzcd}
\end{equation*}
is a pullback square,
\item A homotopy $H':i'\circ f'\htpy j'\circ g'$, and a homotopy
\begin{equation*}
C : \ct{(i\cdot F)}{(\ct{(I\cdot f')}{(h_X\cdot H')})} \htpy \ct{(H\cdot h_S)}{(\ct{(j\cdot G)}{(J\cdot g')})}
\end{equation*}
witnessing that the cube
\begin{equation*}
\begin{tikzcd}
& S' \arrow[dl] \arrow[dr] \arrow[d] \\
A' \arrow[d] & S \arrow[dl] \arrow[dr] & B' \arrow[dl,crossing over] \arrow[d] \\
A \arrow[dr] & X' \arrow[d] \arrow[from=ul,crossing over] & B \arrow[dl] \\
& X,
\end{tikzcd}
\end{equation*}
commutes,
\end{enumerate}
is contractible.
\end{cor}

The following theorem should be compared to the flattening lemma, \cref{lem:flattening}.\index{flattening lemma!for pushouts}

\begin{thm}
Consider a commuting cube
\begin{equation*}
\begin{tikzcd}
& S' \arrow[dl,swap,"{f'}"] \arrow[dr,"{g'}"] \arrow[d,"h_S"] \\
A' \arrow[d,swap,"h_A"] & S \arrow[dl,swap,"f" near start] \arrow[dr,"g" near start] & B' \arrow[dl,crossing over,"{j'}" near end] \arrow[d,"h_B"] \\
A \arrow[dr,swap,"i"] & X' \arrow[d,"h_X" near start] \arrow[from=ul,crossing over,"{i'}"' near end] & B \arrow[dl,"j"] \\
& X.
\end{tikzcd}
\end{equation*}
If each of the vertical squares is a pullback, and the bottom square  is a pushout, then the top square is a pushout.
\end{thm}

\begin{proof}
By \cref{cor:pb_fibequiv} we have fiberwise equivalences
\begin{align*}
F & : \prd{x:S}\eqv{\fib{h_S}{x}}{\fib{h_A}{f(x)}} \\
G & : \prd{x:S}\eqv{\fib{h_S}{x}}{\fib{h_B}{g(x)}} \\
I & : \prd{a:A}\eqv{\fib{h_A}{a}}{\fib{h_X}{i(a)}} \\
J & : \prd{b:B}\eqv{\fib{h_B}{b}}{\fib{h_X}{j(b)}}. 
\end{align*}
Moreover, since the cube commutes we obtain a fiberwise homotopy
\begin{equation*}
K : \prd{x:S} I(f(x))\circ F(x) \htpy J(g(x))\circ G(x).
\end{equation*}
We define the descent data $(P_A,P_B,P_S)$ consisting of $P_A:A\to\UU$, $P_B:B\to\UU$, and $P_S:\prd{x:S}\eqv{P_A(f(x))}{P_B(g(x))}$ by
\begin{align*}
P_A(a) & \defeq \fib{h_A}{a} \\
P_B(b) & \defeq \fib{h_B}{b} \\
P_S(x) & \defeq G(x)\circ F(x)^{-1}.
\end{align*}
We have
\begin{align*}
P & \defeq \fibf{h_X} \\
e_A & \defeq I \\
e_B & \defeq J \\
e_S & \defeq K.
\end{align*}
Now consider the diagram
\begin{equation*}
\begin{tikzcd}
\sm{s:S}\fib{h_S}{s} \arrow[r] \arrow[d] & \sm{s:S}\fib{h_A}{f(s)} \arrow[r] \arrow[d] & \sm{b:B}\fib{h_B}{b} \arrow[d] \\
\sm{a:A}\fib{h_A}{a} \arrow[r] & \sm{a:A}\fib{h_A}{a} \arrow[r] & \sm{x:X}\fib{h_X}{x}
\end{tikzcd}
\end{equation*}
Since the top and bottom map in the left square are equivalences, we obtain from \cref{ex:pushout_equiv} that the left square is a pushout square. Moreover, the right square is a pushout by \cref{lem:flattening}. Therefore it follows by \cref{thm:pushout_pasting} that the outer rectangle is a pushout square.

Now consider the commuting cube
\begin{equation*}
\begin{tikzcd}
& \sm{s:S}\fib{h_S}{s} \arrow[dl] \arrow[dr] \arrow[d] \\
\sm{a:A}\fib{h_A}{a} \arrow[d] & S' \arrow[dl] \arrow[dr] & \sm{b:B}\fib{h_B}{b} \arrow[dl,crossing over] \arrow[d] \\
A' \arrow[dr,swap] & \sm{x:X}\fib{h_X}{x} \arrow[d] \arrow[from=ul,crossing over] & B' \arrow[dl] \\
& X'.
\end{tikzcd}
\end{equation*}
We have seen that the top square is a pushout. The vertical maps are all equivalences, so the vertical squares are all pushout squares. Thus it follows from one more application of \cref{thm:pushout_pasting} that the bottom square is a pushout.
\end{proof}

%\begin{cor}
%For any map $f:A\sqcup^S B\to X$, and any $x:X$, the square
%\begin{equation*}
%\begin{tikzcd}
%\fib{f_S}{x} \arrow[r] \arrow[d] & \fib{f_B}{x} \arrow[d] \\
%\fib{f_A}{x} \arrow[r] & \fib{f}{x}
%\end{tikzcd}
%\end{equation*}
%is a pushout square.
%\end{cor}

\begin{thm}
Consider a commuting cube of types 
\begin{equation*}\label{eq:cube}
\begin{tikzcd}
& S' \arrow[dl] \arrow[dr] \arrow[d] \\
A' \arrow[d] & S \arrow[dl] \arrow[dr] & B' \arrow[dl,crossing over] \arrow[d] \\
A \arrow[dr] & X' \arrow[d] \arrow[from=ul,crossing over] & B \arrow[dl] \\
& X,
\end{tikzcd}
\end{equation*}
and suppose the vertical squares are pullback squares. Then the commuting square
\begin{equation*}
\begin{tikzcd}
A' \sqcup^{S'} B' \arrow[r] \arrow[d] & X' \arrow[d] \\
A\sqcup^{S} B \arrow[r] & X
\end{tikzcd}
\end{equation*}
is a pullback square.
\end{thm}

\begin{proof}
It suffices to show that the pullback 
\begin{equation*}
(A\sqcup^{S} B)\times_{X}X'
\end{equation*}
has the universal property of the pushout. This follows by the descent theorem, since the vertical squares in the cube
\begin{equation*}
\begin{tikzcd}
& S' \arrow[dl] \arrow[dr] \arrow[d] \\
A' \arrow[d] & S \arrow[dl] \arrow[dr] & B' \arrow[dl,crossing over] \arrow[d] \\
A \arrow[dr] & (A\sqcup^{S} B)\times_{X}X' \arrow[d] \arrow[from=ul,crossing over] & B \arrow[dl] \\
& A\sqcup^{S} B
\end{tikzcd}
\end{equation*}
are pullback squares by \cref{thm:pb_pasting}.
\end{proof}

\begin{exercises}
\item Use the characterization of the circle\index{circle} as a pushout given in \cref{eg:circle_pushout} to show that the square
\begin{equation*}
\begin{tikzcd}[column sep=large]
\sphere{1}+\sphere{1} \arrow[r,"{[\idfunc,\idfunc]}"] \arrow[d,swap,"{[\idfunc,\idfunc]}"] & \sphere{1} \arrow[d,"{\lam{t}(t,\base)}"] \\
\sphere{1} \arrow[r,swap,"{\lam{t}(t,\base)}"] & \sphere{1}\times\sphere{1}
\end{tikzcd}
\end{equation*}
is a pushout square.
\item Let $f:A\to B$ be a map. The \define{codiagonal}\index{codiagonal}\index{nabla@{$\nabla_f$}} $\nabla_f$ of $f$ is the map obtained from the universal property of the pushout, as indicated in the diagram
\begin{equation*}
\begin{tikzcd}
A \arrow[d,swap,"f"] \arrow[r,"f"] \arrow[dr, phantom, "\ulcorner", very near end] & B \arrow[d,"\inr"] \arrow[ddr,bend left=15,"{\idfunc[B]}"] \\
A \arrow[r,"\inl"] \arrow[drr,bend right=15,swap,"{\idfunc[B]}"] & B\sqcup^{A} B \arrow[dr,densely dotted,near start,swap,"\nabla_f"] \\
& & B
\end{tikzcd}
\end{equation*}
Show that $\fib{\nabla_f}{b}\eqvsym \susp(\fib{f}{b})$ for any $b:B$.
\item \label{ex:fib_join}Consider two maps $f:A\to X$ and $g:B\to X$. The \define{fiberwise join}\index{fiberwise join} $\join{f}{g}$ is defined by the universal property of the pushout as the unique map rendering the diagram
\begin{equation*}
\begin{tikzcd}
A\times_X B \arrow[d,"\pi_1"] \arrow[r,"\pi_2"] \arrow[dr, phantom, "\ulcorner", very near end] & B \arrow[d,"\inr"] \arrow[ddr,bend left=15,"g"] \\
A \arrow[r,"\inl"] \arrow[drr,bend right=15,swap,"f"] & \join[X]{A}{B} \arrow[dr,densely dotted,near start,swap,"\join{f}{g}"] \\
& & X
\end{tikzcd}
\end{equation*}
commutative, where $\join[X]{A}{B}$ is defined as a pushout, as indicated.
Construct an equivalence
\begin{equation*}
\eqv{\fib{\join{f}{g}}{x}}{\join{\fib{f}{x}}{\fib{g}{x}}}
\end{equation*}
for any $x:X$. 
\item Consider two maps $f:A\to B$ and $g:C\to D$.
The \define{pushout-product}\index{pushout-product}
\begin{equation*}
f\square g : (A\times D)\sqcup^{A\times C} (B\times C)\to B\times D
\end{equation*}
of $f$ and $g$ is defined by the universal property of the pushout as the unique map rendering the diagram
\begin{equation*}
\begin{tikzcd}
A\times C \arrow[r,"{f\times \idfunc[C]}"] \arrow[d,swap,"{\idfunc[A]\times g}"] & B\times C \arrow[d,"\inr"] \arrow[ddr,bend left=15,"{\idfunc[B]\times g}"] \\
A\times D \arrow[r,"\inl"] \arrow[drr,bend right=15,swap,"{f\times\idfunc[D]}"] & (A\times D)\sqcup^{A\times C} (B\times C) \arrow[dr,densely dotted,swap,near start,"f\square g"] \\
& & B\times D
\end{tikzcd}
\end{equation*}
commutative. Construct an equivalence
\begin{equation*}
\eqv{\fib{f\square g}{b,d}}{\join{\fib{f}{b}}{\fib{g}{d}}}
\end{equation*}
for all $b:B$ and $d:D$.
\item Let $A$ and $B$ be pointed types with base points $a_0:A$ and $b_0:B$. The \define{wedge inclusion}\index{wedge inclusion} is defined as follows by the universal property of the wedge:
\begin{equation*}
\begin{tikzcd}[column sep=huge]
\unit \arrow[r] \arrow[d] & B \arrow[d,"\inr"] \arrow[ddr,bend left=15,"{\lam{b}(a_0,b)}"] \\
A \arrow[r,"\inl"] \arrow[drr,bend right=15,swap,"{\lam{a}(a,b_0)}"] & A\vee B \arrow[dr,densely dotted,swap,"{\mathsf{wedge\usc{}in}_{A,B}}"{near start,xshift=1ex}] \\
& & A\times B
\end{tikzcd}
\end{equation*}
Show that the fiber of the wedge inclusion $A\vee B\to A\times B$ is equivalent to $\join{\loopspace{B}}{\loopspace{A}}$.
\item Let $f:X\vee X\to X$ be the map defined by the universal property of the wedge as indicated in the diagram
\begin{equation*}
\begin{tikzcd}
\unit \arrow[d,swap,"x_0"] \arrow[r,"x_0"] \arrow[dr, phantom, "\ulcorner", very near end] & X \arrow[d,"\inr"] \arrow[ddr,bend left=15,"{\idfunc[X]}"] \\
X \arrow[r,"\inl"] \arrow[drr,bend right=15,swap,"{\idfunc[X]}"] & X\vee X \arrow[dr,densely dotted,near start,swap,"f"] \\
& & X.
\end{tikzcd}
\end{equation*}
\begin{subexenum}
\item Show that $\eqv{\fib{f}{x_0}}{\susp\loopspace{X}}$. 
\item Show that $\eqv{\mathsf{cof}_f}{\susp X}$.
\end{subexenum}
\item Consider a pushout square
\begin{equation*}
\begin{tikzcd}
S \arrow[r,"g"] \arrow[d,swap,"f"] & B \arrow[d,"j"] \\
A \arrow[r,swap,"i"] & X,
\end{tikzcd}
\end{equation*}
and suppose that $f$ is an embedding. Show that $j$ is an embedding, and that the square is also a pullback square.
\end{exercises}


\section{Sequential colimits}

\emph{Note: This chapter currently contains only the statements of the definitions and theorems, but no proofs. I hope to make a complete version available soon.}

\subsection{The universal property of sequential colimits}

Type sequences are diagrams of the following form.
\begin{equation*}
\begin{tikzcd}
A_0 \arrow[r,"f_0"] & A_1 \arrow[r,"f_1"] & A_2 \arrow[r,"f_2"] & \cdots.
\end{tikzcd}
\end{equation*}
Their formal specification is as follows.

\begin{defn}
An \define{(increasing) type sequence} $\mathcal{A}$ consists of
\begin{align*}
A & : \N\to\UU \\
f & : \prd{n:\N} A_n\to A_{n+1}. 
\end{align*}
\end{defn}

In this section we will introduce the sequential colimit of a type sequence.
The sequential colimit includes each of the types $A_n$, but we also identify each $x:A_n$ with its value $f_n(x):A_{n+1}$. 
Imagine that the type sequence $A_0\to A_1\to A_2\to\cdots$ defines a big telescope, with $A_0$ sliding into $A_1$, which slides into $A_2$, and so forth.

As usual, the sequential colimit is characterized by its universal property.

\begin{defn}
\begin{enumerate}
\item A \define{(sequential) cocone} on a type sequence $\mathcal{A}$ with vertex $B$ consists of
\begin{align*}
h & : \prd{n:\N} A_n\to B \\
H & : \prd{n:\N} h_n\htpy h_{n+1}\circ f_n.
\end{align*}
We write $\mathsf{cocone}(B)$ for the type of cocones with vertex $B$.
\item Given a cocone $(h,H)$ with vertex $B$ on a type sequence $\mathcal{A}$ we define the map
\begin{equation*}
\mathsf{cocone\usc{}map}(h,H) : (B\to C)\to \mathsf{cocone}(C)
\end{equation*}
given by $f\mapsto (\lam{n}f\circ h_n,\lam{n}{x}\mathsf{ap}_f(H_n(x)))$.
\item We say that a cocone $(h,H)$ with vertex $B$ is \define{colimiting} if $\mathsf{cocone\usc{}map}(h,H)$ is an equivalence for any type $C$.
\end{enumerate}
\end{defn}

\begin{thm}\label{thm:sequential_up}
Consider a cocone $(h,H)$ with vertex $B$ for a type sequence $\mathcal{A}$. The following are equivalent:
\begin{enumerate}
\item The cocone $(h,H)$ is colimiting.
\item The cocone $(h,H)$ is inductive in the sense that for every type family $P:B\to \UU$, the map
\begin{align*}
\Big(\prd{b:B}P(b)\Big)\to {}& \sm{h:\prd{n:\N}{x:A_n}P(h_n(x))}\\ 
& \qquad \prd{n:\N}{x:A_n} \mathsf{tr}_P(H_n(x),h_n(x))={h_{n+1}(f_n(x))}
\end{align*}
given by
\begin{equation*}
s\mapsto (\lam{n}s\circ h_n,\lam{n}{x} \mathsf{apd}_{s}(H_n(x)))
\end{equation*}
has a section.
\item The map in (ii) is an equivalence.
\end{enumerate}
\end{thm}

\subsection{The construction of sequential colimits}

We construct sequential colimits using pushouts.

\begin{defn}
Let $\mathcal{A}\jdeq (A,f)$ be a type sequence. We define the type $A_\infty$ as a pushout
\begin{equation*}
\begin{tikzcd}[column sep=large]
\tilde{A}+\tilde{A} \arrow[r,"{[\idfunc,\sigma_{\mathcal{A}}]}"] \arrow[d,swap,"{[\idfunc,\idfunc]}"] & \tilde{A} \arrow[d,"\inr"] \\
\tilde{A} \arrow[r,swap,"\inl"] & A_\infty.
\end{tikzcd}
\end{equation*}
\end{defn}

\begin{defn}
The type $A_\infty$ comes equipped with a cocone structure consisting of
\begin{align*}
\mathsf{seq\usc{}in} & : \prd{n:\N} A_n\to A_\infty \\
\mathsf{seq\usc{}glue} & : \prd{n:\N}{x:A_n} \mathsf{in}_n(x)=\mathsf{in}_{n+1}(f_n(x)).
\end{align*}
\end{defn}

\begin{constr}
We define
\begin{align*}
\mathsf{seq\usc{}in}(n,x)\defeq \inr(n,x) \\
\mathsf{seq\usc{}glue}(n,x)\defeq \ct{\glue(\inl(n,x))^{-1}}{\glue(\inr(n,x))}.
\end{align*}
\end{constr}

\begin{thm}
Consider a type sequence $\mathcal{A}$, and write $\tilde{A}\defeq\sm{n:\N}A_n$. Moreover, consider the map
\begin{equation*}
\sigma_{\mathcal{A}}:\tilde{A}\to\tilde{A}
\end{equation*}
defined by $\sigma_{\mathcal{A}}(n,a)\defeq (n+1,f_n(a))$. Furthermore, consider a cocone $(h,H)$ with vertex $B$.
The following are equivalent:
\begin{enumerate}
\item The cocone $(h,H)$ with vertex $B$ is colimiting.
\item The defining square
\begin{equation*}
\begin{tikzcd}[column sep=large]
\tilde{A}+\tilde{A} \arrow[r,"{[\idfunc,\sigma_{\mathcal{A}}]}"] \arrow[d,swap,"{[\idfunc,\idfunc]}"] & \tilde{A} \arrow[d,"{\lam{(n,x)}h_n(x)}"] \\
\tilde{A} \arrow[r,swap,"{\lam{(n,x)}h_n(x)}"] & A_\infty,
\end{tikzcd}
\end{equation*}
of $A_\infty$ is a pushout square.
\end{enumerate}
\end{thm}

\subsection{Descent for sequential colimits}

\begin{defn}
The type of \define{descent data} on a type sequence $\mathcal{A}\jdeq (A,f)$ is defined to be
\begin{equation*}
\mathsf{Desc}(\mathcal{A}) \defeq \sm{B:\prd{n:\N}A_n\to\UU}\prd{n:\N}{x:A_n}\eqv{B_n(x)}{B_{n+1}(f_n(x))}.
\end{equation*}
\end{defn}

\begin{defn}
We define a map
\begin{equation*}
\mathsf{desc\usc{}fam} : (A_\infty\to\UU)\to\mathsf{Desc}(\mathcal{A})
\end{equation*}
by $B\mapsto (\lam{n}{x}B(\mathsf{seq\usc{}in}(n,x)),\lam{n}{x}\mathsf{tr}_B(\mathsf{seq\usc{}glue}(n,x)))$.
\end{defn}

\begin{thm}
The map 
\begin{equation*}
\mathsf{desc\usc{}fam} : (A_\infty\to\UU)\to\mathsf{Desc}(\mathcal{A})
\end{equation*}
is an equivalence.
\end{thm}

\begin{defn}
A \define{cartesian transformation} of type sequences from $\mathcal{A}$ to $\mathcal{B}$ is a pair $(h,H)$ consisting of
\begin{align*}
h & : \prd{n:\N} A_n\to B_n \\
H & : \prd{n:\N} g_n\circ h_n \htpy h_{n+1}\circ f_n,
\end{align*}
such that each of the squares in the diagram
\begin{equation*}
\begin{tikzcd}
A_0 \arrow[d,swap,"h_0"] \arrow[r,"f_0"] & A_1 \arrow[d,swap,"h_1"] \arrow[r,"f_1"] & A_2 \arrow[d,swap,"h_2"] \arrow[r,"f_2"] & \cdots \\
B_0 \arrow[r,swap,"g_0"] & B_1 \arrow[r,swap,"g_1"] & B_2 \arrow[r,swap,"g_2"] & \cdots
\end{tikzcd}
\end{equation*}
is a pullback square. We define
\begin{align*}
\mathsf{cart}(\mathcal{A},\mathcal{B}) & \defeq\sm{h:\prd{n:\N}A_n\to B_n} \\
& \qquad\qquad \sm{H:\prd{n:\N}g_n\circ h_n\htpy h_{n+1}\circ f_n}\prd{n:\N}\mathsf{is\usc{}pullback}(h_n,f_n,H_n),
\end{align*}
and we write
\begin{equation*}
\mathsf{Cart}(\mathcal{B}) \defeq \sm{\mathcal{A}:\mathsf{Seq}}\mathsf{cart}(\mathcal{A},\mathcal{B}).
\end{equation*}
\end{defn}

\begin{defn}
We define a map
\begin{equation*}
\mathsf{cart\usc{}map}(\mathcal{B}) : \Big(\sm{X':\UU}X'\to X\Big)\to\mathsf{Cart}(\mathcal{B}).
\end{equation*}
which associates to any morphism $h:X'\to X$ a cartesian transformation of type sequences into $\mathcal{B}$.
\end{defn}

\begin{thm}
The operation $\mathsf{cart\usc{}map}(\mathcal{B})$ is an equivalence.
\end{thm}

\subsection{The flattening lemma for sequential colimits}

The flattening lemma for sequential colimits essentially states that sequential colimits commute with $\Sigma$. 

\begin{lem}
Consider
\begin{align*}
B & : \prd{n:\N}A_n\to\UU \\
g & : \prd{n:\N}{x:A_n}\eqv{B_n(x)}{B_{n+1}(f_n(x))}.
\end{align*}
and suppose $P:A_\infty\to\UU$ is the unique family equipped with
\begin{align*}
e & : \prd{n:\N}\eqv{B_n(x)}{P(\mathsf{seq\usc{}in}(n,x))}
\end{align*}
and homotopies $H_n(x)$ witnessing that the square
\begin{equation*}
\begin{tikzcd}[column sep=7em]
B_n(x) \arrow[r,"g_n(x)"] \arrow[d,swap,"e_n(x)"] & B_{n+1}(f_n(x)) \arrow[d,"e_{n+1}(f_n(x))"] \\
P(\mathsf{seq\usc{}in}(n,x)) \arrow[r,swap,"{\mathsf{tr}_P(\mathsf{seq\usc{}glue}(n,x))}"] & P(\mathsf{seq\usc{}in}(n+1,f_n(x)))
\end{tikzcd}
\end{equation*}
commutes. Then $\sm{t:A_\infty}P(t)$ satisfies the universal property of the sequential colimit of the type sequence
\begin{equation*}
\begin{tikzcd}
\sm{x:A_0}B_0(x) \arrow[r,"{\tot[f_0]{g_0}}"] & \sm{x:A_1}B_1(x) \arrow[r,"{\tot[f_1]{g_1}}"] & \sm{x:A_2}B_2(x) \arrow[r,"{\tot[f_2]{g_2}}"] & \cdots.
\end{tikzcd}
\end{equation*}
\end{lem}

In the following theorem we rephrase the flattening lemma in using cartesian transformations of type sequences.

\begin{thm}
Consider a commuting diagram of the form
\begin{equation*}
\begin{tikzcd}[column sep=small,row sep=small]
A_0 \arrow[rr] \arrow[dd] & & A_1 \arrow[rr] \arrow[dr] \arrow[dd] &[-.9em] &[-.9em] A_2 \arrow[dl] \arrow[dd] & & \cdots \\
& & & X \arrow[from=ulll,crossing over] \arrow[from=urrr,crossing over] \arrow[from=ur,to=urrr] \\
B_0 \arrow[rr] \arrow[drrr] & & B_1 \arrow[rr] \arrow[dr] & & B_2 \arrow[rr] \arrow[dl] & & \cdots \arrow[dlll] \\
& & & Y \arrow[from=uu,crossing over] 
\end{tikzcd}
\end{equation*}
If each of the vertical squares is a pullback square, and $Y$ is the sequential colimit of the type sequence $B_n$, then $X$ is the sequential colimit of the type sequence $A_n$. 
\end{thm}

\begin{cor}
Consider a commuting diagram of the form
\begin{equation*}
\begin{tikzcd}[column sep=small,row sep=small]
A_0 \arrow[rr] \arrow[dd] & & A_1 \arrow[rr] \arrow[dr] \arrow[dd] &[-.9em] &[-.9em] A_2 \arrow[dl] \arrow[dd] & & \cdots \\
& & & X \arrow[from=ulll,crossing over] \arrow[from=urrr,crossing over] \arrow[from=ur,to=urrr] \\
B_0 \arrow[rr] \arrow[drrr] & & B_1 \arrow[rr] \arrow[dr] & & B_2 \arrow[rr] \arrow[dl] & & \cdots \arrow[dlll] \\
& & & Y \arrow[from=uu,crossing over] 
\end{tikzcd}
\end{equation*}
If each of the vertical squares is a pullback square, then the square
\begin{equation*}
\begin{tikzcd}
A_\infty \arrow[r] \arrow[d] & X \arrow[d] \\
B_\infty \arrow[r] & Y
\end{tikzcd}
\end{equation*} 
is a pullback square.
\end{cor}

\subsection{Constructing the propositional truncation}\label{sec:propositional-truncation-constr}
The propositional truncation can be used to construct the image of a map, so we construct that first. We construct the propositional truncation of $A$ via a construction called the \define{join construction}, as the colimit of the sequence of join-powers of $A$
\begin{equation*}
  \begin{tikzcd}
    A \arrow[r] & \join{A}{A} \arrow[r] & \join{A}{(\join{A}{A})} \arrow[r] & \cdots
  \end{tikzcd}
\end{equation*}
The join-powers of $A$ are defined recursively on $n$, by taking\footnote{In this definition, the case $A^{\ast1}\defeq A$ is slightly redundant because we have an equivalence
\begin{equation*}
  \join{A}{\emptyt}\simeq A.
\end{equation*}
Nevertheless, it is nice to have that $A^{\ast 1}\jdeq A$.}
\begin{align*}
  A^{\ast0} & \defeq \emptyt \\
  A^{\ast 1} & \defeq A \\
  A^{\ast(n+2)} & \defeq \join{A}{A^{\ast (n+1)}}.
\end{align*}
We will write $A^{\ast\infty}$ for the colimit of the sequence
\begin{equation*}
  \begin{tikzcd}
    A \arrow[r,"\inr"] & \join{A}{A} \arrow[r,"\inr"] & \join{A}{(\join{A}{A})} \arrow[r,"\inr"] & \cdots.
  \end{tikzcd}
\end{equation*}
The sequential colimit $A^{\ast\infty}$ comes equipped with maps $\inseq_n:A^{\ast (n+1)}\to A^{\ast\infty}$, and we will write
\begin{equation*}
  \eta\defeq\inseq_0:A\to A^{\ast\infty}.
\end{equation*}
Our goal is to show $A^{\ast\infty}$ is a proposition, and that $\eta:A\to A^{\ast\infty}$ satisfies the universal property of the propositional truncation of $A$. Before showing that $A^{\ast\infty}$ is indeed a proposition, let us show in two steps that for any proposition $P$, the map
\begin{equation*}
  (A^{\ast\infty}\to P)\to (A\to P)
\end{equation*}
is indeed an equivalence. 

\begin{lem}\label{lem:extend_join_prop}
Suppose $f:A\to P$, where $A$ is any type and $P$ is a proposition.
Then the precomposition function
\begin{equation*}
\blank\circ\inr:(\join{A}{B}\to P)\to (B\to P)
\end{equation*}
is an equivalence, for any type $B$.
\end{lem}

\begin{proof}
  Since the precomposition function
  \begin{equation*}
    \blank\circ\inr:(\join{A}{B}\to P)\to (B\to P)
  \end{equation*}
  is a map between propositions, it suffices to construct a map
  \begin{equation*}
    (B\to P)\to (\join{A}{B}\to P).
  \end{equation*}
  Let $g:B\to P$. Then the square
  \begin{equation*}
    \begin{tikzcd}
      A\times B \arrow[r,"\proj 2"] \arrow[d,swap,"\proj 1"] & B \arrow[d,"g"] \\
      A \arrow[r,swap,"f"] & P
    \end{tikzcd}
  \end{equation*}
  commutes since $P$ is a proposition. Therefore we obtain a map $\join{A}{B}\to P$ by the universal property of the join.
\end{proof}

\begin{prp}\label{prp:universal-property-brck}
Let $A$ be a type, and let $P$ be a proposition. Then the function
\begin{equation*}
\blank\circ \eta : (A^{\ast\infty}\to P)\to (A\to P)
\end{equation*}
is an equivalence. 
\end{prp}

\begin{proof}
  Since the map
  \begin{equation*}
    \blank\circ \eta : (A^{\ast\infty}\to P)\to (A\to P)
  \end{equation*}
  is a map between propositions, it suffices to construct a map in the converse direction.

  Let $f:A\to P$. First, we show by recursion that there are maps
  \begin{equation*}
    f_n:A^{\ast(n+1)}\to P.
  \end{equation*}
  The map $f_0$ is of course just defined to be $f$. Given $f_n:A^{\ast(n+1)}$ we obtain $f_{n+1}:\join{A}{A^{\ast(n+1)}}\to P$ by \cref{lem:extend_join_prop}. Because $P$ is assumed to be a proposition it is immediate that the maps $f_n$ form a cocone with vertex $P$ on the sequence
  \begin{equation*}
    \begin{tikzcd}
      A \arrow[r,"\inr"] & \join{A}{A} \arrow[r,"\inr"] & \join{A}{(\join{A}{A})} \arrow[r,"\inr"] & \cdots.
    \end{tikzcd}
  \end{equation*}
  From this cocone we obtain the desired map $(A^{\ast\infty}\to P)$.
\end{proof}

\begin{prp}\label{prp:isprop-infjp}
The type $A^{\ast\infty}$ is a proposition for any type $A$.
\end{prp}

\begin{proof}
  By \cref{lem:isprop_eq} it suffices to show that
  \begin{equation*}
    A^{\ast\infty}\to \iscontr(A^{\ast\infty}).
  \end{equation*}
  Since the type $\iscontr(A^{\ast\infty})$ is already known to be a proposition by \cref{ex:isprop_istrunc}, it follows from \cref{prp:universal-property-brck} that it suffices to show that
\begin{equation*}
A\to \iscontr(A^{\ast\infty}).
\end{equation*}

Let $x:A$. To see that $A^{\ast\infty}$ is contractible it suffices by \cref{ex:seqcolim_contr} to show that $\inr:A^{\ast n}\to A^{\ast(n+1)}$ is homotopic to the constant function $\const_{\inl(x)}$. However, we get a homotopy $\const_{\inl(x)}\htpy \inr$ immediately from the path constructor $\glue$.  
\end{proof}

All the definitions are now in place to define the propositional truncation of a type.

\begin{defn}
  For any type $A$ we define the type
  \begin{equation*}
    \trunc{-1}{A}\defeq A^{\ast\infty},
  \end{equation*}
  and we define $\eta:A\to\trunc{-1}{A}$ to be the constructor $\seqin_0$ of the sequential colimit $A^{\ast\infty}$. Often we simply write $\brck{A}$ for $\trunc{-1}{A}$.
\end{defn}

The type $\trunc{-1}{A}$ is a proposition by \cref{prp:isprop-infjp}, and
\begin{equation*}
  \eta:A\to\trunc{-1}{A}
\end{equation*}
satisfies the universal property of propositional truncation by \cref{prp:universal-property-brck}.

\begin{prp}
  The propositional truncation operation is functorial in the sense that for any map $f:A\to B$ there is a unique map $\brck{f}:\brck{A}\to\brck{B}$ such that the square
  \begin{equation*}
    \begin{tikzcd}
      A \arrow[r,"f"] \arrow[d,swap,"\eta"] & B \arrow[d,"\eta"] \\
      \brck{A} \arrow[r,swap,"\brck{f}"] & \brck{B}
    \end{tikzcd}
  \end{equation*}
  commutes. Moreover, there are homotopies
  \begin{align*}
    \brck{\idfunc[A]} & \htpy \idfunc[\brck{A}] \\
    \brck{g\circ f} & \htpy \brck{g}\circ\brck{f}.
  \end{align*}
\end{prp}

\begin{proof}
  The functorial action of propositional truncation is immediate by the universal property of propositional truncation. To see that the functorial action preserves the identity, note that the type of maps $\brck{A}\to\brck{A}$ for which the square
  \begin{equation*}
    \begin{tikzcd}
      A \arrow[r,"\idfunc"] \arrow[d,swap,"\eta"] & A \arrow[d,"\eta"] \\
      \brck{A} \arrow[r,densely dotted] & \brck{A}
    \end{tikzcd}
  \end{equation*}
  commutes is contractible. Since this square commutes for both $\brck{\idfunc}$ and for $\idfunc$, it must be that they are homotopic. The proof that the functorial action of propositional truncation preserves composition is similar.
\end{proof}

\subsection{Proving type theoretical replacement}

Our goal is now to show that the image of a map $f:A\to B$ from an essentially small type $A$ into a locally small type $B$ is again essentially small. This property is called the type theoretic replacement property. In order to prove this property, we have to find another construction of the image of a map. In order to make this construction, we define a join operation on maps.

\begin{defn}
  Consider two maps $f:A\to X$ and $g:B\to X$ with a common codomain $X$.
  \begin{enumerate}
  \item We define the type $\join[X]{A}{B}$ as the pushout
    \begin{equation*}
      \begin{tikzcd}
        A\times_X B \arrow[r,"\pi_2"] \arrow[d,swap,"\pi_1"] & B \arrow[d,"\inr"] \\
        A \arrow[r,swap,"\inl"] & \join[X]{A}{B}.
      \end{tikzcd}
    \end{equation*}
  \item We define the \define{join} $\join{f}{g}:\join[X]{A}{B}\to X$ to be the unique map for which the diagram
        \begin{equation*}
      \begin{tikzcd}
        A\times_X B \arrow[r,"\pi_2"] \arrow[d,swap,"\pi_1"] & B \arrow[d,"\inr"] \arrow[ddr,bend left=15,"g"] \\
        A \arrow[r,swap,"\inl"] \arrow[drr,bend right=15,swap,"f"]  & \join[X]{A}{B} \arrow[dr,densely dotted,swap,"\join{f}{g}"] \\
        & & X
      \end{tikzcd}
    \end{equation*}
  \end{enumerate}
\end{defn}

The reason to call the map $\join{f}{g}$ the join of $f$ and $g$ is that the fiber of $\join{f}{g}$ at any $x:X$ is equivalent to the join of the fibers of $f$ and $g$ at $x$.

\begin{lem}
  Consider two maps $f:A\to X$ and $g:B\to X$. Then there is an equivalence
  \begin{equation*}
    \fib{\join{f}{g}}{x}\simeq\join{\fib{f}{x}}{\fib{g}{x}}
  \end{equation*}
  for any $x:X$.
\end{lem}

\begin{proof}
  Consider the commuting cube
  \begin{equation*}
    \begin{tikzcd}
      & \fib{f}{x}\times\fib{g}{x} \arrow[dl] \arrow[dr] \arrow[d] \\
      \fib{f}{x} \arrow[d] & A\times_X B \arrow[dl] \arrow[dr] & \fib{g}{x} \arrow[d] \arrow[dl,crossing over] \\
      A \arrow[dr] & \unit \arrow[from=ul,crossing over] \arrow[d] & B \arrow[dl] \\
      & X
    \end{tikzcd}
  \end{equation*}
  In this cube, the bottom square is a canonical pullback square. The two squares in the front are pullbacks by \cref{lem:fib_pb}, and the top square is a pullback square by \cref{lem:prod_pb}. Therefore it follows by \cref{rmk:strongly-cartesian} that all the faces of this cube are pullback squares, and hence by \cref{thm:effectiveness-pullback} we obtain that the square
  \begin{equation*}
    \begin{tikzcd}
      \join{\fib{f}{x}}{\fib{g}{x}} \arrow[d,densely dotted] \arrow[r] & \unit \arrow[d] \\
      \join[X]{A}{B} \arrow[r,swap,"\join{f}{g}"] & X
    \end{tikzcd}
  \end{equation*}
  is a pullback square. Now the claim follows by the uniqueness of pullbacks, which was shown in \cref{cor:uniquely-unique-pullback}.
\end{proof}

\begin{lem}
Consider a map $f:A\to X$, an embedding $m:U\to X$, and $h:\mathrm{hom}_X(f,m)$. Then the map
\begin{equation*}
\mathrm{hom}_X(\join{f}{g},m)\to \mathrm{hom}_X(g,m)
\end{equation*}
is an equivalence for any $g:B\to X$.
\end{lem}

\begin{proof}
Note that both types are propositions, so any equivalence can be used to prove the claim. Thus, we simply calculate
\begin{align*}
\mathrm{hom}_X(\join{f}{g},m) & \eqvsym \prd{x:X}\fib{\join{f}{g}}{x}\to \fib{m}{x} \\
& \eqvsym \prd{x:X}\join{\fib{f}{x}}{\fib{g}{x}}\to\fib{m}{x} \\
& \eqvsym \prd{x:X}\fib{g}{x}\to\fib{m}{x} \\
& \eqvsym \mathrm{hom}_X(g,m).
\end{align*}
The first equivalence holds by \cref{ex:triangle_fib}; the second equivalence holds by \cref{ex:fib_join}, also using \cref{ex:equiv_precomp,lem:postcomp_equiv} where we established that that pre- and postcomposing by an equivalence is an equivalence; the third equivalence holds by \cref{lem:extend_join_prop,lem:postcomp_equiv}; the last equivalence again holds by \cref{ex:triangle_fib}.
\end{proof}

For the construction of the image of $f:A\to X$ we observe that if we are given an embedding $m:U\to X$ and a map $(i,I):\mathrm{hom}_X(f,m)$, then $(i,I)$ extends uniquely along $\inr:A\to \join[X]{A}{A}$ to a map $\mathrm{hom}_X(\join{f}{f},m)$. This extension again extends uniquely along $\inr:\join[X]{A}{A}\to \join[X]{A}{(\join[X]{A}{A})}$ to a map $\mathrm{hom}_X(\join{f}{(\join{f}{f})},m)$ and so on, resulting in a diagram of the form
\begin{equation*}
\begin{tikzcd}
A \arrow[dr] \arrow[r,"\inr"] & \join[X]{A}{A} \arrow[d,densely dotted] \arrow[r,"\inr"] & \join[X]{A}{(\join[X]{A}{A})} \arrow[dl,densely dotted] \arrow[r,"\inr"] & \cdots \arrow[dll,densely dotted,bend left=10] \\
& U
\end{tikzcd}
\end{equation*}

\begin{defn}
Suppose $f:A\to X$ is a map. Then we define the \define{fiberwise join powers} 
\begin{equation*}
f^{\ast n}:A_X^{\ast n} X.
\end{equation*}
\end{defn}

\begin{constr}
Note that the operation $(B,g)\mapsto (\join[X]{A}{B},\join{f}{g})$ defines an endomorphism on the type
\begin{equation*}
\sm{B:\UU}B\to X.
\end{equation*}
We also have $(\emptyt,\ind{\emptyt})$ and $(A,f)$ of this type. For $n\geq 1$ we define
\begin{align*}
A_X^{\ast (n+1)} & \defeq \join[X]{A}{A_X^{\ast n}} \\
f^{\ast (n+1)} & \defeq \join{f}{f^{\ast n}}.\qedhere
\end{align*}
\end{constr}

\begin{defn}
We define $A_X^{\ast\infty}$ to be the sequential colimit of the type sequence
\begin{equation*}
\begin{tikzcd}
A_X^{\ast 0} \arrow[r] & A_X^{\ast 1} \arrow[r,"\inr"] & A_X^{\ast 2} \arrow[r,"\inr"] & \cdots.
\end{tikzcd}
\end{equation*}
Since we have a cocone
\begin{equation*}
\begin{tikzcd}
A_X^{\ast 0} \arrow[r] \arrow[dr,swap,"f^{\ast 0}" near start] & A_X^{\ast 1} \arrow[r,"\inr"] \arrow[d,swap,"f^{\ast 1}" near start] & A_X^{\ast 2} \arrow[r,"\inr"] \arrow[dl,swap,"f^{\ast 2}" xshift=1ex] & \cdots \arrow[dll,bend left=10] \\
& X
\end{tikzcd}
\end{equation*}
we also obtain a map $f^{\ast\infty}:A_X^{\ast\infty}\to X$ by the universal property of $A_X^{\ast\infty}$. 
\end{defn}

\begin{lem}\label{lem:finfjp_up}
Let $f:A\to X$ be a map, and let $m:U\to X$ be an embedding. Then the function
\begin{equation*}
\blank\circ \seqin_0: \mathrm{hom}_X(f^{\ast\infty},m)\to \mathrm{hom}_X(f,m)
\end{equation*}
is an equivalence. 
\end{lem}

\begin{thm}\label{lem:isprop_infjp}
For any map $f:A\to X$, the map $f^{\ast\infty}:A_X^{\ast\infty}\to X$ is an embedding that satisfies the universal property of the image inclusion of $f$.
\end{thm}

\begin{lem}
Consider a commuting square
\begin{equation*}
\begin{tikzcd}
A \arrow[r] \arrow[d] & B \arrow[d] \\
C \arrow[r] & D.
\end{tikzcd}
\end{equation*}
\begin{enumerate}
\item If the square is cartesian, $B$ and $C$ are essentially small, and $D$ is locally small, then $A$ is essentially small.
\item If the square is cocartesian, and $A$, $B$, and $C$ are essentially small, then $D$ is essentially small. 
\end{enumerate}
\end{lem}

\begin{cor}
Suppose $f:A\to X$ and $g:B\to X$ are maps from essentially small types $A$ and $B$, respectively, to a locally small type $X$. Then $A\times_X B$ is again essentially small. 
\end{cor}

\begin{lem}
Consider a type sequence
\begin{equation*}
\begin{tikzcd}
A_0 \arrow[r,"f_0"] & A_1 \arrow[r,"f_1"] & A_2 \arrow[r,"f_2"] & \cdots
\end{tikzcd}
\end{equation*}
where each $A_n$ is essentially small. Then its sequential colimit is again essentially small. 
\end{lem}

\begin{thm}\label{thm:replacement}
  For any map $f:A\to B$ from an essentially small type $A$ into a locally small type $B$, the image of $f$ is again essentially small.
\end{thm}

\begin{cor}
  Consider a $\UU$-small type $A$, and an equivalence relation $R$ over $A$ valued in the $\UU$-small propositions. Then the set quotient $A/R$ is essentially small.
\end{cor}

\begin{exercises}
\exercise \label{ex:seqcolim_shift}
Show that the sequential colimit of a type sequence
\begin{equation*}
\begin{tikzcd}
A_0 \arrow[r,"f_0"] & A_1 \arrow[r,"f_1"] & A_2 \arrow[r,"f_2"] & \cdots
\end{tikzcd}
\end{equation*}
is equivalent to the sequential colimit of its shifted type sequence
\begin{equation*}
\begin{tikzcd}
A_1 \arrow[r,"f_1"] & A_2 \arrow[r,"f_2"] & A_3 \arrow[r,"f_3"] & \cdots.
\end{tikzcd}
\end{equation*}
  \exercise Let
  \begin{tikzcd}
    P_0 \arrow[r] & P_1 \arrow[r] & P_2 \arrow[r] & \cdots
  \end{tikzcd}
  be a sequence of propositions. Show that
  \begin{equation*}
    \eqv{\colim_n(P_n)}{\exists_{(n:\N)} P_n}.
  \end{equation*}
\exercise \label{ex:seqcolim_contr}Consider a type sequence
\begin{equation*}
\begin{tikzcd}
A_0 \arrow[r,"f_0"] & A_1 \arrow[r,"f_1"] & A_2 \arrow[r,"f_2"] & \cdots
\end{tikzcd}
\end{equation*}
and suppose that $f_n\htpy \mathsf{const}_{a_{n+1}}$ for some $a_n:\prd{n:\N}A_n$. Show that the sequential colimit is contractible.
\exercise Define the $\infty$-sphere $\sphere{\infty}$ as the sequential colimit of
\begin{equation*}
\begin{tikzcd}
\sphere{0} \arrow[r,"f_0"] & \sphere{1} \arrow[r,"f_1"] & \sphere{2} \arrow[r,"f_2"] & \cdots
\end{tikzcd}
\end{equation*}
where $f_0:\sphere{0}\to\sphere{1}$ is defined by $f_0(\bfalse)\jdeq \inl(\ttt)$ and $f_0(\btrue)\jdeq \inr(\ttt)$, and $f_{n+1}:\sphere{n+1}\to\sphere{n+2}$ is defined as $\susp(f_n)$. Use \cref{ex:seqcolim_contr} to show that $\sphere{\infty}$ is contractible.
\exercise Consider a type sequence
\begin{equation*}
\begin{tikzcd}
A_0 \arrow[r,"f_0"] & A_1 \arrow[r,"f_1"] & A_2 \arrow[r,"f_2"] & \cdots
\end{tikzcd}
\end{equation*}
in which $f_n:A_n\to A_{n+1}$ is weakly constant in the sense that
\begin{equation*}
\prd{x,y:A_n} f_n(x)=f_n(y)
\end{equation*}
Show that $A_\infty$ is a mere proposition.
\exercise Show that $\N$ is the sequential colimit of
\begin{equation*}
  \begin{tikzcd}
    \Fin(0) \arrow[r,"\inl"] & \Fin(1) \arrow[r,"\inl"] & \Fin(2) \arrow[r,"\inl"] & \cdots.
  \end{tikzcd}
\end{equation*}
\end{exercises}


\section{The image of a map and the replacement axiom}\label{chap:image}

The idea of the image of a map $f:A\to X$ is that it is, in a way, the least subtype of $X$ that contains all the values of $f$. More precisely, the image of $f$ is an embedding $i:\im(f)\hookrightarrow X$ that fits in a commuting triangle
\begin{equation*}
  \begin{tikzcd}[column sep=tiny]
    A \arrow[rr,"q"] \arrow[dr,swap,"f"] & & \im(f) \arrow[dl,hook,"i"] \\
    \phantom{\im(f)} & X
  \end{tikzcd}
\end{equation*}
and satisfies the \emph{universal property} of the image of $f$. The universal property of the image of $f$ asserts that if a subtype $B\hookrightarrow X$ contains all the values of $f$, then it contains the image of $f$.
%In other words, for  asserts that there is a unique map $h:\im(f)\to B$ for which the tetrahedron
%\begin{equation*}
%  \begin{tikzcd}[column sep=large]
%    A \arrow[rr] \arrow[dr,"q"] \arrow[dddr,swap,"f"] & & B \arrow[dddl,"m"] \\
%    & \im(f) \arrow[ur,densely dotted,"h"] \arrow[dd,"i"] \\ \\
%    & X
%  \end{tikzcd}
%\end{equation*}
%commutes.
The image of a map can be constructed using the propositional truncation operation. In fact, we can also go the other way around: The propositional truncation of a type $A$ is the image of the map $A\to\unit$.

The final topic of this section is the type theoretic replacement axiom. A specific instance of the replacement axiom asserts that the image of any map $f:A\to\UU$ is equivalent to a type in $\UU$, provided that $A$ is equivant to a type in $\UU$. This property will be used to construct quotients in type theory, much in the same way as quotients are constructed in set theory.

We should note that the existence of the propositional truncation operation and the replacement axiom will be assumed for now. However, once we assume that universes are closed under pushouts, we will be able to construct the propositional truncations and we will be able to prove the replacement axiom. These constructions will be given in \cref{sec:join-construction}.

\subsection{The image of a map}\label{sec:image-construction}
 %Note that there is quite a lot of information in this diagram: not only are there the three small commuting triangles; there is also the large commuting triange in the back, and there is a three-dimensional solid filling the space between the four triangles. We make the following definition, in order to express the universal property of the image efficiently.

\begin{defn}
  Let $f:A\to X$ and $g:B\to X$ be maps. A \define{morphism} from $f$ to $g$ over $X$ consists of a map $h:A\to B$ equipped with a homotopy $H:f\htpy g\circ h$ witnessing that the triangle
\begin{equation*}
\begin{tikzcd}[column sep=tiny]
A \arrow[rr,"h"] \arrow[dr,swap,"f"] & & B \arrow[dl,"g"] \\
& X
\end{tikzcd}
\end{equation*}
commutes. Thus, we define the type
\begin{equation*}
\mathrm{hom}_X(f,g)\defeq\sm{h:A\to B}f\htpy g\circ h.
\end{equation*}
Composition of morphisms over $X$ is defined by
\begin{equation*}
  (k,K)\circ (h,H) \defeq (k\circ h,\ct{H}{(K\cdot h)}).
\end{equation*}
\end{defn}

\begin{defn}
Consider a commuting triangle
\begin{equation*}
\begin{tikzcd}[column sep=tiny]
A \arrow[rr,"q"] \arrow[dr,swap,"f"] & & I \arrow[dl,"i"] \\
& X
\end{tikzcd}
\end{equation*}
with $H:f\htpy i\circ q$, where $i$ is an embedding\index{embedding}.
We say that $i$ has the \define{universal property of the image of $f$}\index{universal property!of the image} if the map
\begin{equation*}
\blank\circ(q,H) : \mathrm{hom}_X(i,m)\to\mathrm{hom}_X(f,m)
\end{equation*}
is an equivalence for every embedding $m:B\to X$. 
\end{defn}

\begin{rmk}
  Consider a commuting triangle
\begin{equation*}
\begin{tikzcd}[column sep=tiny]
A \arrow[rr,"q"] \arrow[dr,swap,"f"] & & I \arrow[dl,"i"] \\
& X
\end{tikzcd}
\end{equation*}
with $H:f\htpy i\circ q$, where $i$ is an embedding. Then it is not hard to see that the embedding $i$ satisfies the universal property of the image inclusion if and only if for every commuting triangle
\begin{equation*}
  \begin{tikzcd}[column sep=tiny]
    A \arrow[dr,swap,"f"] \arrow[rr,"g"] & & B \arrow[dl,"m"] \\
    & X
  \end{tikzcd}
\end{equation*}
with $G:f\htpy m\circ g$, where $m$ is an embedding, the type of quadruples $(h,K,L,M)$ consisting of
\begin{enumerate}
\item a map $h:I\to B$,
\item a homotopy $K:i\htpy m\circ h$ witnessing that the triangle
  \begin{equation*}
    \begin{tikzcd}[column sep=tiny]
      I \arrow[rr,"h"] \arrow[dr,swap,"i"] & & B \arrow[dl,"m"] \\
      & X
    \end{tikzcd}
  \end{equation*}
  commutes,
\item a homotopy $L:g\htpy h\circ q$ witnessing that the triangle
  \begin{equation*}
    \begin{tikzcd}[column sep=tiny]
      A \arrow[rr,"q"] \arrow[dr,swap,"g"] & & I \arrow[dl,"h"] \\
      & B
    \end{tikzcd}
  \end{equation*}
  commutes,
\item a homotopy $M:\ct{H}{(K\cdot q)}\htpy\ct{G}{(m\cdot L)}$ witnessing that the square
  \begin{equation*}
    \begin{tikzcd}
      f \arrow[d,swap,"H"] \arrow[r,"G"] & m\circ g \arrow[d,"m\cdot L"] \\
      i\circ q \arrow[r,swap,"K\cdot q"] & m\circ h\circ g
    \end{tikzcd}
  \end{equation*}
  commutes,
\end{enumerate}
is contractible. However, the situation is in fact much simpler, because the type $\mathrm{hom}_X(f,m)$ is a proposition whenever $m$ is an embedding.
\end{rmk}

\begin{rmk}
  Suppose that the map $f:A\to X$ has a section. Then the identity function
  \begin{equation*}
    \idfunc:X\to X
  \end{equation*}
  satisfies the universal property of the image of $f$. 
\end{rmk}

\begin{rmk}
  Suppose that $f:A\to X$ is already an embedding. Then $f$ itself satisfies the universal property of the image of $f$.
\end{rmk}

\begin{lem}
For any $f:A\to X$ and any embedding\index{embedding} $m:B\to X$, the type $\mathrm{hom}_X(f,m)$ is a proposition.
\end{lem}

\begin{proof}
  Recall from \cref{ex:triangle_fib} that the type $\mathrm{hom}_X(f,m)$ is equivalent to the type
  \begin{equation*}
    \prd{x:X}\fib{f}{x}\to\fib{m}{x}.
  \end{equation*}
  Therefore it suffices to show that this type is a proposition. Recall from \cref{cor:prop_emb} that a map is an embedding if and only if its fibers are propositions.
  Thus we see that the type $\prd{x:X}\fib{f}{x}\to\fib{m}{x}$ is a product of propositions, hence it is a proposition by \cref{thm:trunc_pi}.
\end{proof}

\begin{prp}\label{prp:simplifly-universal-property-image}
  Consider a commuting triangle
  \begin{equation*}
    \begin{tikzcd}[column sep=tiny]
      A \arrow[rr,"q"] \arrow[dr,swap,"f"] & & I \arrow[dl,"i"] \\
      & X
\end{tikzcd}
  \end{equation*}
  with $H:f\htpy i\circ q$, where $i$ is an embedding. Then the following are equivalent:
  \begin{enumerate}
  \item The embedding $i$ satisfies the universal property of the image inclusion of $f$.
  \item For every embedding $m:B\to X$ there is a map
    \begin{equation*}
      \mathrm{hom}_X(f,m)\to\mathrm{hom}_X(i,m).
    \end{equation*}
  \end{enumerate}
\end{prp}

\begin{proof}
Since $\mathrm{hom}_X(f,m)$ is a proposition for every every embedding $m:B\to X$, the claim follows immediately by \cref{ex:prop_equiv}.
\end{proof}

Just as in the cases for pullbacks and pushouts, the universal property of the image implies that the image is determined uniquely. We will show here that the type of image factorizations of any map is a proposition. In \cref{sec:image-construction} we will construct the image, after constructing the propositional truncation.

\begin{prp}
  Let $f$ be a map, and consider two commuting triangles
  \begin{equation*}
    \begin{tikzcd}[column sep=tiny]
      A \arrow[dr,swap,"f"] \arrow[rr,"q"] & & B \arrow[dl,"i"] &[2em] A \arrow[dr,swap,"f"] \arrow[rr,"{q'}"] & & B' \arrow[dl,"{i'}"] \\
      & X & & \phantom{B'} & X
    \end{tikzcd}
  \end{equation*}
  with $I:f\htpy i\circ q$ and $I':f\htpy i'\circ q'$, in which $i$ and $i'$ are assumed to be embeddings. Moreover, consider
  \begin{equation*}
    (h,H):\mathrm{hom}_X(i,i')
  \end{equation*}
  equipped with an identification $(h,H)\circ(q,I)=(q',I')$ in $\mathrm{hom}_X(f,i')$. Then, if any two of the following properties hold, so does the third:
  \begin{enumerate}
  \item The embedding $i$ satisfies the universal property of the image inclusion of $f$.
  \item The embedding $i'$ satisfies the universal property of the image inclusion of $f$.
  \item The map $h$ is an equivalence.
  \end{enumerate}
\end{prp}

\begin{proof}
  Consider an embedding $m:C\to X$. Then we have a commuting triangle
  \begin{equation*}
    \begin{tikzcd}[column sep=-1em]
      \mathrm{hom}_X(i',m) \arrow[rr,"{\blank\circ(h,H)}"] \arrow[dr,swap,"{\blank\circ(q',I')}"] & & \mathrm{hom}_X(i,m) \arrow[dl,"{\blank\circ(q,I)}"] \\
      & \mathrm{hom}_X(f,m), & \phantom{\mathrm{hom}_X(i',m)}
    \end{tikzcd}
  \end{equation*}
  so it follows that if any two of these maps are equivalences, then so is the third. The claim now follows by the observation that $\blank\circ(h,H)$ is an equivalence for every embedding $m:C\to X$ if and only if $h$ is an equivalence.
\end{proof}

\begin{cor}\label{cor:uniqueness-image}
  Consider two image factorizations
  \begin{equation*}
    \begin{tikzcd}[column sep=tiny]
      A \arrow[dr,swap,"f"] \arrow[rr,"q"] & & B \arrow[dl,"i"] &[2em] A \arrow[dr,swap,"f"] \arrow[rr,"{q'}"] & & B' \arrow[dl,"{i'}"] \\
      & X & & \phantom{B'} & X
    \end{tikzcd}
  \end{equation*}
  of a map $f$, with $I:f\htpy i\circ q$ and $I':f\htpy i'\circ q'$. Then the type of $(e,H):\mathrm{hom}_X(i,i')$ in which $e$ is an equivalence, equipped with an identification
  \begin{equation*}
    (e,H)\circ(q,I)=(q',I')
  \end{equation*}
  in $\mathrm{hom}_X(f,i')$, is contractible.
\end{cor}

The image of a map $f:A\to X$ can now be defined using the propositional truncation:

\begin{defn}
For any map $f:A\to X$ we define the \define{image}\index{image} of $f$ to be the type
\begin{equation*}
\im(f) \defeq \sm{x:X}\brck{\fib{f}{x}}.
\end{equation*}
Furthermore, we define:
\begin{enumerate}
\item The \define{image inclusion}
  \begin{equation*}
    i_f:\im(f)\to X
  \end{equation*}
  to be the projection $\proj 1$.
\item The map
  \begin{equation*}
    q_f:A\to\im(f)
  \end{equation*}
  to be the map given by $q_f(x)\defeq(f(x),\eta(x,\refl{f(x)}))$.
\item The homotopy $I_f:f\htpy i_f\circ q_f$ witnessing that the triangle
  \begin{equation*}
    \begin{tikzcd}[column sep=tiny]
      A \arrow[rr,"q_f"] \arrow[dr,swap,"f"] & & \im(f) \arrow[dl,"i_f"] \\
      \phantom{\im(f)} & X
    \end{tikzcd}
  \end{equation*}
  commutes, to be given by $I_f(x)\defeq\refl{f(x)}$.
\end{enumerate}
\end{defn}

\begin{prp}
  The image inclusion $i_f:\im(f)\to X$ of any map $f:A\to X$ is an embedding.
\end{prp}

\begin{proof}
  The fiber of $i_f$ at $x:X$ is equivalent to the type $\brck{\fib{f}{x}}$. In particular we see that the fibers are propositions, so $i_f$ is an embedding.
\end{proof}

\begin{thm}
  The image inclusion $i_f:\im(f)\to X$ of any map $f:A\to X$ satisfies the universal property of the image inclusion of $f$.
\end{thm}

\begin{proof}
  Consider an embedding $m:B\to X$. Note that we have a commuting square
  \begin{equation*}
    \begin{tikzcd}[column sep=6em]
      \mathrm{hom}_X(i_f,m) \arrow[d] \arrow[r] & \mathrm{hom}_X(f,m) \arrow[d] \\
      \Big(\prd{x:X}\fib{i_f}{x}\to\fib{m}{x}\Big) \arrow[r,swap,"h\mapsto{\lam{x}h_x\circ\varphi_x}"] & \Big(\prd{x:X}\fib{f}{x}\to\fib{m}{x}\Big)
    \end{tikzcd}
  \end{equation*}
  The vertical maps are of the form
  \begin{equation*}
    (h,H) \mapsto \lam{x}{(y,p)}(h(y),\ct{H(y)^{-1}}{p}),
  \end{equation*}
  and they are both equivalences. The map
  \begin{equation*}
    \varphi_x:\fib{f}{x}\to\fib{i_f}{x}
  \end{equation*}
  given by $\varphi_x(a,p)\defeq((h(a),\eta(a,p)),p)$ is a propositional truncation for every $x:X$. Therefore it follows that the map
  \begin{equation*}
    (\fib{i_f}{x}\to\fib{m}{x})\to(\fib{f}{x}\to\fib{m}{x})
  \end{equation*}
  is an equivalence, for every $x:X$. Thus we conclude that the bottom map in the above square is an equivalence, which implies that the top map is an equivalence. 
\end{proof}

\begin{eg}
  An important special case of the homotopy image of a map is the image of the terminal projection
\begin{equation*}
  \const_\ttt : A \to \unit,
\end{equation*}
which results in an embedding $I\hookrightarrow \unit$. Embeddings into the unit type are in fact just propositions. To see this, note that
\begin{align*}
\sm{A:\UU}{f:A\to\unit}\isemb(f)
& \eqvsym \sm{A:\UU}\isemb(\const_\ttt) \\
& \eqvsym \sm{A:\UU}\prd{x:\unit}\isprop(\fib{\const_\ttt}{x}) \\
& \eqvsym \sm{A:\UU}\isprop(\fib{\const_\ttt}{\ttt}) \\
& \eqvsym \sm{A:\UU}\isprop(A).
\end{align*}
Therefore, the universal property of the image of the map $A\to\unit$ is equivalently described as a proposition $P$ satisfying the universal property of the propositional truncation.
\end{eg}

\subsection{Surjective maps}

Another application of the propositional truncation is the notion of surjective map.

\begin{defn}
A map $f:A\to B$ is said to be \define{surjective} if there is a term of type
\begin{equation*}
\issurj(f)\defeq \prd{y:B}\brck{\fib{f}{b}}.
\end{equation*}
\end{defn}

\begin{eg}
Any equivalence is a surjective map, and so is any map that has a section (those are sometimes called \define{split epimorphisms}). Other examples include the base point inclusion $\unit\to\sphere{n}$ for any $n\geq 1$. 
\end{eg}

\begin{prp}\label{prp:surjective}
  Consider a map $f:A\to B$. Then the following are equivalent:
  \begin{enumerate}
  \item The map $f:A\to B$ is surjective.
  \item For any family $P$ of propositions over $B$, the precomposition map
    \begin{equation*}
      \blank\circ f : \Big(\prd{y:B}P(y)\Big)\to\Big(\prd{x:A}P(f(x))\Big)
    \end{equation*}
    is an equivalence.
  \end{enumerate}
\end{prp}

\begin{proof}
  Suppose first that $f$ is surjective, and consider the commuting square
  \begin{equation*}
    \begin{tikzcd}[column sep=6em]
      \Big(\prd{y:B}P(y)\Big) \arrow[r,"\blank\circ f"] \arrow[d,swap,"h\mapsto\lam{y}\const_{h(y)}"] & \Big(\prd{x:A}P(f(x))\Big)  \\
      \Big(\prd{y:B}\brck{\fib{f}{y}}\to P(y)\Big) \arrow[r,swap,"h\mapsto\lam{y}h(y)\circ\eta"] & \Big(\prd{y:B}\fib{f}{y}\to P(y)\Big) \arrow[u,swap,"{h\mapsto\lam{x}h(f(x),(x,\refl{f(x)}))}"]
    \end{tikzcd}
  \end{equation*}
  In this square, the bottom map is an equivalence by the universal property of the propositional truncation of $\fib{f}{y}$. The map on the right is also easily seen to be an equivalence. Furthermore, the map on the left is an equivalence by the assumption that $f$ is surjective, from which it follows that the types $\brck{\fib{f}{y}}$ are contractible. Therefore it follows that the top map is an equivalence, which completes the proof that (i) implies (ii).

  For the converse, it follows immediately from the assumption (ii) that
  \begin{equation*}
    \blank\circ f : \Big(\prd{y:B}\brck{\fib{f}{y}}\Big)\to\Big(\prd{x:A}\brck{\fib{f}{f(x)}}\Big)
  \end{equation*}
  is an equivalence. Hence it suffices to construct a term of type $\brck{\fib{f}{f(x)}}$ for each $x:A$. This is easy, because we have
  \begin{equation*}
    \eta(x,\refl{f(x)}):\brck{\fib{f}{f(x)}}.\qedhere.
  \end{equation*}
\end{proof}

\begin{thm}\label{thm:surjective}
Consider a commuting triangle
\begin{equation*}
\begin{tikzcd}[column sep=tiny]
A \arrow[rr,"q"] \arrow[dr,swap,"f"] & & B \arrow[dl,"m"] \\
& X
\end{tikzcd}
\end{equation*}
in which $m$ is an embedding. Then the following are equivalent:
\begin{enumerate}
\item The embedding $m$ satisfies the universal property of the image inclusion of $f$.
\item The map $q$ is surjective.
\end{enumerate}
\end{thm}

\begin{proof}
  First assume that $m$ satisfies the universal property of the image inclusion of $f$, and consider the composite function
  \begin{equation*}
    \begin{tikzcd}
      \Big(\sm{y:B}\brck{\fib{q}{y}}\Big) \arrow[r,"\proj 1"] & B \arrow[r,"m"] & X.
    \end{tikzcd}
  \end{equation*}
  Note that $m\circ\proj 1$ is a composition of embeddings, so it is an embedding. By the universal property of $m$ there is a unique map $h$ for which the triangle
  \begin{equation*}
    \begin{tikzcd}[column sep=0]
      B \arrow[dr,swap,"m"] \arrow[rr,densely dotted,"h"] & & \sm{y:B}\brck{\fib{q}{y}} \arrow[dl,"m\circ\proj 1"] \\
      \phantom{\sm{y:B}\brck{\fib{q}{y}}} & X
    \end{tikzcd}
  \end{equation*}
  commutes. Now note that $\proj 1\circ h$ is a map such that $m\circ (\proj 1\circ h)\htpy m$. The identity function is another map for which we have $m\circ\idfunc\htpy m$, so it follows by uniqueness that $\proj 1\circ h\htpy \idfunc$. In other words, the map $h$ is a section of the projection map. Therefore we obtain by \cref{ex:pi_sec} a dependent function
  \begin{equation*}
    \prd{b:B}\brck{\fib{q}{b}},
  \end{equation*}
  showing that $q$ is surjective.

  For the converse, suppose that $q$ is surjective. To prove that $m$ satisfies the universal property of the image factorization of $f$, it suffices to construct an equivalence
  \begin{equation*}
    \mathrm{hom}_X(f,m')\to\mathrm{hom}_X(m,m'),
  \end{equation*}
  for any embedding $m':B'\to X$. To see that there is such an equivalence, we make the following calculation
  \begin{align*}
    \mathrm{hom}_X(m,m') & \simeq \prd{x:X}\fib{m}{x}\to\fib{m'}{x} \\
                         & \simeq \prd{b:B}\fib{m'}{m(b)} \\
                         & \simeq \prd{a:A}\fib{m'}{m(q(a))} \\
                         & \simeq \prd{a:A}\fib{m'}{f(a)} \\
                         & \simeq \prd{x:X}\fib{f}{x}\to\fib{m'}{x} \\
                         & \simeq \mathrm{hom}_X(f,m').
  \end{align*}
  In this calculation, the first and last equivalence hold by \cref{ex:triangle_fib}. The second and second to last equivalences hold by \cref{ex:pi-fib}. The third equivalence holds by \cref{prp:surjective}, since $q$ is assumed to be surjective, and the fourth equivalence holds since we have a homotopy $f\htpy m\circ f$.
\end{proof}

\begin{cor}
  Every map factors uniquely as a surjective map followed by an embedding.
\end{cor}

\begin{proof}
  Consider a map $f:A\to X$, and two factorizations
  \begin{equation*}
    \begin{tikzcd}[column sep=tiny]
      A \arrow[rr,"q"] \arrow[dr,swap,"f"] & & B \arrow[dl,"i"] &[3em] A \arrow[rr,"{q'}"] \arrow[dr,swap,"f"] & & B' \arrow[dl,"{i'}"] \\
      & X & & & X
    \end{tikzcd}
  \end{equation*}
  of $f$ where $m$ and $m'$ are embeddings, and $q$ and $q'$ are surjective. Then both $m$ and $m'$ satisfy the universal property of the image factorization of $f$ by \cref{thm:surjective}. Now it follows by \cref{cor:uniqueness-image} that the type of $(e,H):\mathrm{hom}_X(i,i')$ in which $e$ is an equivalence, equipped with an identification
  \begin{equation*}
    (e,H)\circ(q,I)=(q',I')
  \end{equation*}
  in $\mathrm{hom}_X(f,i')$, is contractible.
\end{proof}

\subsection{Type theoretic replacement}

\begin{comment}
We have constructed the set quotient $A/R$ as the image of the equivalence relation
\begin{equation*}
  R:A\to \UU^A.
\end{equation*}
However, the type $\UU^A$ is itself in the next universe $\UU^+$. Hence the quotient is also in the universe $\UU^+$. We prove in this section that $A/R$ is nevertheless equivalent to a type in $\UU$. In other words, we show that $A/R$ is \emph{essentially} small.
\end{comment}

\begin{defn}\label{defn:ess_small}
\begin{enumerate}
\item A type $A$ is said to be \define{essentially small}\index{essentially small!type} if there is a type $X:\UU$ and an equivalence $\eqv{A}{X}$. We write\index{ess_small(A)@{$\mathsf{ess\usc{}small}(A)$}}
\begin{equation*}
\mathsf{ess\usc{}small}(A)\defeq\sm{X:\UU}\eqv{A}{X}.
\end{equation*}
\item A map $f:A\to B$ is said to be \define{essentially small}\index{essentially small!map} if for each $b:B$ the fiber $\fib{f}{b}$ is essentially small.
We write\index{ess_small(f)@{$\mathsf{ess\usc{}small}(f)$}}
\begin{equation*}
\mathsf{ess\usc{}small}(f)\defeq\prd{b:B}\mathsf{ess\usc{}small}(\fib{f}{b}).
\end{equation*}
\item A type $A$ is said to be \define{locally small}\index{locally small!type} if for every $x,y:A$ the identity type $x=y$ is essentially small.
We write\index{loc_small(A)@{$\mathsf{loc\usc{}small}(A)$}}
\begin{equation*}
\mathsf{loc\usc{}small}(A)\defeq \prd{x,y:A}\mathsf{ess\usc{}small}(x=y).
\end{equation*}
\end{enumerate}
\end{defn}

\begin{eg}
  \begin{enumerate}
  \item Any essentially $\UU$-small type is also locally $\UU$-small.
  \item Any univalent universe $\UU$ is locally $\UU$-small, because by the univalence axiom we have equivalences
    \begin{equation*}
      (A=B)\simeq (A\simeq B)
    \end{equation*}
    for each $A,B:\UU$, and the type $A\simeq B$ is in $\UU$.
  \item Any proposition is locally small with respect to any universe $\UU$.
  \item For any family $P$ of locally $\UU$-small types over a essentially $\UU$-small type $A$, the dependent product $\prd{x:A}P(x)$ is locally $\UU$-small. In particular, any type $A\to B$ of functions from an essentially small type into a locally small type is again locally small.
  \end{enumerate}
\end{eg}

\begin{lem}\label{lem:isprop_ess_small}
The type $\mathsf{ess\usc{}small}(A)$ is a proposition for any type $A$.\index{essentially small!is a proposition}
\end{lem}

\begin{proof}
Let $A$ be a type, not necessarily in $\UU$. In order to show that $\mathsf{ess\usc{}small}(A)$ is a proposition, we will use \cref{lem:isprop_eq} and show that for any $X:\UU$ and any equivalence $e:A\simeq X$, the type
\begin{equation*}
\sm{Y:\UU}\eqv{A}{Y}
\end{equation*}
is contractible. Note that we have an equivalence
\begin{equation*}
\eqv{\Big(\sm{Y:\UU}\eqv{X}{Y}\Big)}{\Big(\sm{Y:\UU}\eqv{A}{Y}\Big)}
\end{equation*}
because precomposing with the equivalence $e:A \simeq X$ is an equivalence. However, the type $\sm{Y:\UU}\eqv{X}{Y}$ is contractible by \cref{thm:univalence}. This shows that $\mathsf{ess\usc{}small}(A)$ is equivalent to a contractible type, assuming that $A$ is essentially small.
\end{proof}

\begin{cor}
For each function $f:A\to B$, the type $\mathsf{ess\usc{}small}(f)$ is a proposition, and for each type $X$ the type $\mathsf{loc\usc{}small}(X)$ is a proposition.
\end{cor}

\begin{proof}
This follows from the fact that propositions are closed under dependent products, established in \cref{thm:trunc_pi}.
\end{proof}

Recall that in set theory, the replacement axiom asserts that for any family of sets $\{X_i\}_{i\in I}$ indexed by a set $I$, there is a set $X[I]$ consisting of precisely those sets $x$ for which there exists an $i\in I$ such that $x\in X_i$. In other words: the image of a set-indexed family of sets is again a set. Without the replacement axiom, $X[I]$ would be a class. In the following corollary we establish a type-theoretic analogue of the replacement axiom: the image of a family of small types indexed by a small type is again (essentially) small.

\begin{axiom}\label{axiom:replacement}
  For any map $f:A\to B$ from an essentially small type $A$ into a locally small type $B$, the image of $f$ is again essentially small.
\end{axiom}

\begin{eg}
  For any type $A:\UU$, the image of the constant map $\const_A:\unit\to \UU$ is essentially small. This image is called the \define{connected component} of the universe at $A$. To see why, let us calculate
  \begin{align*} 
    \im(\const_A) & \jdeq \sm{X:\UU}\Brck{\fib{\const_A}{X}} \\
                  & \jdeq \sm{X:\UU}\Brck{\sm{t:\unit}A=X} \\
                  & \simeq \sm{X:\UU}\brck{A=X}.
  \end{align*}
  We see that the image of $\const_A:\unit\to\UU$ is the type of all types that are \emph{merely} equal to $A$. In other words, they are equal to $A$ in an unspecified way.
\end{eg}

\begin{eg}
  The type $\F$ of all finite types is defined to be the image of the map
  \begin{equation*}
    \Fin : \N\to\UU_0
  \end{equation*}
  By the replacement axiom, this type is essentially small. 
\end{eg}

\begin{exercises}
  \exercise Consider a map $f:A\to P$ into a proposition $P$. Show that the following are equivalent:
  \begin{enumerate}
  \item The map $f$ is a propositional truncation of $A$.
  \item The map $f$ is surjective.
  \end{enumerate}
  \exercise Consider a map $f:A\to B$. Show that the following are equivalent:
  \begin{enumerate}
  \item $f$ is an equivalence.
  \item $f$ is both surjective and an embedding.
  \end{enumerate}
  \exercise Consider a commuting triangle
  \begin{equation*}
    \begin{tikzcd}[column sep=tiny]
      A \arrow[rr,"h"] \arrow[dr,swap,"f"] & & B \arrow[dl,"g"] \\
      & X
    \end{tikzcd}
  \end{equation*}
  with $H:f\htpy g\circ h$, and assume that $h$ is surjective. Show that the following are equivalent:
  \begin{enumerate}
  \item The map $f$ is surjective.
  \item The map $g$ is surjective.
  \end{enumerate}
  \exercise \label{ex:surjective-precomp}Consider a map $f:A\to B$. Show that the following are equivalent:
  \begin{enumerate}
  \item The map $f$ is surjective.
  \item For every set $C$, the precomposition function
    \begin{equation*}
      \blank\circ f:(B\to C)\to (A\to C)
    \end{equation*}
    is an embedding.
  \end{enumerate}
  Hint: To show that (ii) implies (i), use the assumption with the set $C\jdeq\prop_\UU$, where $\UU$ is a univalent universe containing both $A$ and $B$.
  \exercise Let us say that a type family $B$ over $A$ is \define{univalent} if the map
  \begin{equation*}
    (x=y)\to (B(x)\simeq B(y))
  \end{equation*}
  is an equivalence, for every $x,y:A$.
  \begin{subexenum}
  \item Show that a family $B:A\to\UU$ is univalent if and only if the map $B:A\to\UU$ is an embedding.
  \item For any family $B:A\to\UU$, show that the type family $\hat{B}:\hat{A}\to\UU$ defined by
    \begin{align*}
      \hat{A} & \defeq \im(B) \\
      \hat{B}(X,p) & \defeq X
    \end{align*}
    is univalent.
  \item For any two families $B:A\to\UU$ and $D:C\to\mathcal{V}$, define the type of \define{cartesian morphisms}
    \begin{equation*}
      \carthomFam((A,B),(C,D)) \defeq \sm{f:A\to C}\prd{x:A}B(x)\simeq D(f(x)).
    \end{equation*}
    Construct a cartesian morphism
    \begin{equation*}
      (\eta,\alpha) : \carthomFam((A,B),(\hat{A},\hat{B})).
    \end{equation*}
  \item Show that for any family $B:A\to\UU$ and any \emph{univalent} family $D:C\to\mathcal{V}$, the map
    \begin{equation*}
      \carthomFam((\hat{A},\hat{B}),(C,D))\to\carthomFam((A,B),(C,D))
    \end{equation*}
    given by
    \begin{equation*}
      (f,e)\mapsto (f\circ\eta,\lam{x}e_x\circ \alpha_x)
    \end{equation*}
    is an equivalence. This is the \define{universal property} of the univalent completion of $A$.
  \end{subexenum}
  %\exercise \label{also}(Mart\'in Escard\'o) For any two propositions $P$ and $Q$, define
  %\begin{equation*}
  %P\boxplus Q \defeq ((P\to Q)\to Q)\times ((Q\to P)\to P).
  %\end{equation*}
  %\begin{subexenum}
  %\item Show that $P\lor Q\to P\boxplus Q$ and $P\boxplus Q\to\neg(\neg P\land \neg Q)$.
  %\end{subexenum}
  %\item \label{ex:brck_comp} Formulate the computation rule corresponding to the path constructor $\mu$. That is, compute the type of $\apd{\rec{\brck{\blank}}(f,g)}{\mu(x,y)}$, and find a canonical element in it.
  %\exercise Let $f:A\to X$ be a map. Construct an equivalence
  %\begin{equation*}
  %\eqv{\Big(\sm{y:\mathsf{join\usc{}power}_X(n,A)}f(x)=f^{\ast n}(y)\Big)}{\Big(\sm{y:A}f(x)=f(y)\Big)^{\ast n}}
  %\end{equation*}
  %for any $x:A$.
\end{exercises}

\endinput

\begin{thm}
Consider a commuting triangle
\begin{equation*}
\begin{tikzcd}[column sep=small]
A \arrow[rr,"i"] \arrow[dr,swap,"f"] & & B \arrow[dl,"m"] \\
& X
\end{tikzcd}
\end{equation*}
with $I:f\htpy m\circ i$, where $m$ is an embedding. The following are equivalent:
\begin{enumerate}
\item $m$ satisfies the universal property of the image of $f$.
\item for each $x:X$, the proposition $\fib{m}{x}$ satisfies the universal property of the propositional truncation of $\fib{f}{x}$.
\end{enumerate}
\end{thm}


\section{Type theoretic replacement}

\subsection{Essentially small types and maps}
It is a trivial observation, but nevertheless of fundamental importance, that by the univalence axiom the identity types of $\UU$ are equivalent to types in $\UU$, because it provides an equivalence $\eqv{(A=B)}{(\eqv{A}{B})}$, and the type $\eqv{A}{B}$ is in $\UU$ for any $A,B:\UU$. Since the identity types of $\UU$ are equivalent to types in $\UU$, we also say that the universe is \emph{locally small}.

\begin{defn}\label{defn:ess_small}
\begin{enumerate}
\item A type $A$ is said to be \define{essentially small}\index{essentially small!type} if there is a type $X:\UU$ and an equivalence $\eqv{A}{X}$. We write\index{ess_small(A)@{$\mathsf{ess\usc{}small}(A)$}}
\begin{equation*}
\mathsf{ess\usc{}small}(A)\defeq\sm{X:\UU}\eqv{A}{X}.
\end{equation*}
\item A map $f:A\to B$ is said to be \define{essentially small}\index{essentially small!map} if for each $b:B$ the fiber $\fib{f}{b}$ is essentially small.
We write\index{ess_small(f)@{$\mathsf{ess\usc{}small}(f)$}}
\begin{equation*}
\mathsf{ess\usc{}small}(f)\defeq\prd{b:B}\mathsf{ess\usc{}small}(\fib{f}{b}).
\end{equation*}
\item A type $A$ is said to be \define{locally small}\index{locally small!type} if for every $x,y:A$ the identity type $x=y$ is essentially small.
We write\index{loc_small(A)@{$\mathsf{loc\usc{}small}(A)$}}
\begin{equation*}
\mathsf{loc\usc{}small}(A)\defeq \prd{x,y:A}\mathsf{ess\usc{}small}(x=y).
\end{equation*}
\end{enumerate}
\end{defn}

\begin{lem}\label{lem:isprop_ess_small}
The type $\mathsf{ess\usc{}small}(A)$ is a proposition for any type $A$.\index{essentially small!is a proposition}
\end{lem}

\begin{proof}
Let $X$ be a type. Our goal is to show that the type
\begin{equation*}
\sm{Y:\UU}\eqv{X}{Y}
\end{equation*}
is a proposition. Suppose there is a type $X':\UU$ and an equivalence $e:\eqv{X}{X'}$, then the map
\begin{equation*}
(\eqv{X}{Y})\to (\eqv{X'}{Y})
\end{equation*}
given by precomposing with $e^{-1}$ is an equivalence. This induces an equivalence on total spaces
\begin{equation*}
\eqv{\Big(\sm{Y:\UU}\eqv{X}{Y}\Big)}{\Big(\sm{Y:\UU}\eqv{X'}{Y}\Big)}
\end{equation*}
However, the codomain of this equivalence is contractible by \cref{thm:univalence}. Thus it follows by \cref{cor:contr_prop} that the asserted type is a proposition.
\end{proof}

\begin{cor}
For each function $f:A\to B$, the type $\mathsf{ess\usc{}small}(f)$ is a proposition, and for each type $X$ the type $\mathsf{loc\usc{}small}(X)$ is a proposition.
\end{cor}

\begin{proof}
This follows from the fact that propositions are closed under dependent products, established in \cref{thm:trunc_pi}.
\end{proof}

\begin{thm}\label{thm:fam_proj}
For any small type $A:\UU$ there is an equivalence
\begin{equation*}
\mathsf{map\usc{}fam}_A:\eqv{(A\to \UU)}{\Big(\sm{X:\UU} X\to A\Big)}.
\end{equation*}
\end{thm}

\begin{proof}
Note that we have the function
\begin{equation*}
\varphi :\lam{B} \Big(\sm{x:A}B(x),\proj 1\Big) : (A\to \UU)\to \Big(\sm{X:\UU}X\to A\Big).
\end{equation*}
The fiber of this map at $(X,f)$ is by univalence and function extensionality equivalent to the type
\begin{equation*}
\sm{B:A\to \UU}{e:\eqv{(\sm{x:A}B(x))}{X}} \proj 1\htpy f\circ e.
\end{equation*}
By \cref{ex:triangle_fib} this type is equivalent to the type
\begin{equation*}
\sm{B:A\to \UU}\prd{a:A} \eqv{B(a)}{\fib{f}{a}},
\end{equation*}
and by `type theoretic choice', which was established in \cref{thm:choice}, this type is equivalent to
\begin{equation*}
\prd{a:A}\sm{X:\UU}\eqv{X}{\fib{f}{a}}.
\end{equation*}
We conclude that the fiber of $\varphi$ at $(X,f)$ is equivalent to the type $\mathsf{ess\usc{}small}(f)$. However, since $f:X\to A$ is a map between small types it is essentially small. Moreover, since being essentially small is a proposition by \cref{lem:isprop_ess_small}, it follows that $\fib{\varphi}{(X,f)}$ is contractible for every $f:X\to A$. In other words, $\varphi$ is a contractible map, and therefore it is an equivalence.
\end{proof}

\begin{rmk}
The inverse of the map
\begin{equation*}
\varphi : (A\to \UU)\to \Big(\sm{X:\UU}X\to A\Big).
\end{equation*}
constructed in \cref{thm:fam_proj} is the map $(X,f)\mapsto \fibf{f}$.
\end{rmk}

\begin{thm}\label{thm:classifier}
Let $f:A\to B$ be a map. Then there is an equivalence
\begin{equation*}
\eqv{\mathsf{ess\usc{}small}(f)}{\mathsf{is\usc{}classified}(f)},
\end{equation*}
where $\mathsf{is\usc{}classified}(f)$\index{is_classified(f)@{$\mathsf{is\usc{}classified}(f)$}} is the type of quadruples $(F,\tilde{F},H,p)$ consisting of maps
$F:B\to \UU$ and $\tilde{F}:A\to \sm{X:\UU}X$, a homotopy $H:F\circ f\htpy \proj 1\circ \tilde{F}$,  such that the commuting square
\begin{equation*}
\begin{tikzcd}
A \arrow[r,"\tilde{F}"] \arrow[d,swap,"f"] & \sm{X:\UU}X \arrow[d,"\proj 1"] \\
B \arrow[r,swap,"F"] & \UU
\end{tikzcd}
\end{equation*}
is a pullback square, as witnessed by $p$\footnote{The universal property of the pullback is not expressible by a type. However, we may take the type of $p:\isequiv(h)$, where $h:A\to B\times_\UU\big(\sm{X:\UU}X\big)$ is the map obtained by the universal property of the canonical pullback.}. If $f$ comes equipped with a term of type $\mathsf{is\usc{}classified}(f)$, we also say that $f$ is \define{classified}\index{classified by the universal family} by the universal family. 
\end{thm}

\begin{proof}
From \cref{ex:sq_fib} we obtain that the type of pairs $(\tilde{F},H)$ is equivalent to the type of fiberwise transformations
\begin{equation*}
\prd{b:B}\fib{f}{b}\to F(b).
\end{equation*}
By \cref{cor:pb_fibequiv} the square is a pullback square if and only if the induced map
\begin{equation*}
\prd{b:B}\fib{f}{b}\to F(b)
\end{equation*}
is a fiberwise equivalence. Thus the data $(F,\tilde{F},H,pb)$ is equivalent to the type of pairs $(F,e)$ where $e$ is a fiberwise equivalence from $\fibf{f}$ to $F$. By \cref{thm:choice} the type of pairs $(F,e)$ is equivalent to the type $\mathsf{ess\usc{}small}(f)$. 
\end{proof}

\begin{rmk}
For any type $A$ (not necessarily small), and any $B:A\to \UU$, the square\index{Sigma-type@{$\Sigma$-type}!as pullback of universal family}
\begin{equation*}
\begin{tikzcd}[column sep=6em]
\sm{x:A}B(x) \arrow[d,swap,"\proj 1"] \arrow[r,"{\lam{(x,y)}(B(x),y)}"] & \sm{X:\UU}X \arrow[d,"\proj 1"] \\
A \arrow[r,swap,"B"] & \UU
\end{tikzcd}
\end{equation*}
is a pullback square. Therefore it follows that for any family $B:A\to\UU$ of small types, the projection map $\proj 1:\sm{x:A}B(x)\to A$ is an essentially small map.
To see that the claim is a direct consequence of \cref{lem:pb_subst} we write the asserted square in its rudimentary form:
\begin{equation*}
%\begin{gathered}[b]
\begin{tikzcd}[column sep=6em]
\sm{x:A}\mathrm{El}(B(x)) \arrow[d,swap,"\proj 1"] \arrow[r,"{\lam{(x,y)}(B(x),y)}"] & \sm{X:\UU}\mathrm{El}(X) \arrow[d,"\proj 1"] \\
A \arrow[r,swap,"B"] & \UU.
\end{tikzcd}%\\[-\dp\strutbox]\end{gathered}\qedhere
\end{equation*}
\end{rmk}

In the following theorem we show that a type is small if and only if its diagonal is classified by $\UU$.

\begin{thm}
Let $A$ be a type. The following are equivalent:
\begin{enumerate}
\item $A$ is locally small.\index{locally small}
\item There are maps $I:A\times A\to\UU$ and $\tilde{I}:A\to\sm{X:\UU}X$, and a homotopy $H:I\circ \delta_A\htpy \proj 1\circ\tilde{I}$
such that the commuting square
\begin{equation*}
\begin{tikzcd}
A \arrow[r,"\tilde{I}"] \arrow[d,swap,"\delta_A"] & \sm{X:\UU}X \arrow[d,"\proj 1"] \\
A\times A \arrow[r,swap,"{I}"] & \UU
\end{tikzcd}
\end{equation*}
is a pullback square.\index{diagonal!of a type}
\end{enumerate}
\end{thm}

\begin{proof}
In \cref{ex:diagonal} we have established that the identity type $x=y$ is the fiber of $\delta_A$ at $(x,y):A\times A$. Therefore it follows that $A$ is locally small if and only if the diagonal $\delta_A$ is essentially small.
Now the result follows from \cref{thm:classifier}.
\end{proof}

\subsection{Smallness of images}
However, the construction of the fiberwise join in \cref{ex:fib_join} suggests that we can also define the image of $f$ as the infinite join power $f^{\ast\infty}$, where we repeatedly take the fiberwise join of $f$ with itself. The reasons for defining the image in this way are twofold: we will be able to use this construction to show that the set-quotients of a small type are small, and second, we many interesting types appear in this construction.

\begin{lem}
Consider a map $f:A\to X$, an embedding $m:U\to X$, and $h:\mathrm{hom}_X(f,m)$. Then the map
\begin{equation*}
\mathrm{hom}_X(\join{f}{g},m)\to \mathrm{hom}_X(g,m)
\end{equation*}
is an equivalence for any $g:B\to X$.
\end{lem}

\begin{proof}
Note that both types are propositions, so any equivalence can be used to prove the claim. Thus, we simply calculate
\begin{align*}
\mathrm{hom}_X(\join{f}{g},m) & \eqvsym \prd{x:X}\fib{\join{f}{g}}{x}\to \fib{m}{x} \\
& \eqvsym \prd{x:X}\join{\fib{f}{x}}{\fib{g}{x}}\to\fib{m}{x} \\
& \eqvsym \prd{x:X}\fib{g}{x}\to\fib{m}{x} \\
& \eqvsym \mathrm{hom}_X(g,m).
\end{align*}
The first equivalence holds by \cref{ex:triangle_fib}; the second equivalence holds by \cref{ex:fib_join}, also using \cref{ex:equiv_precomp,lem:postcomp_equiv} where we established that that pre- and postcomposing by an equivalence is an equivalence; the third equivalence holds by \cref{lem:extend_join_prop,lem:postcomp_equiv}; the last equivalence again holds by \cref{ex:triangle_fib}.
\end{proof}

For the construction of the image of $f:A\to X$ we observe that if we are given an embedding $m:U\to X$ and a map $(i,I):\mathrm{hom}_X(f,m)$, then $(i,I)$ extends uniquely along $\inr:A\to \join[X]{A}{A}$ to a map $\mathrm{hom}_X(\join{f}{f},m)$. This extension again extends uniquely along $\inr:\join[X]{A}{A}\to \join[X]{A}{(\join[X]{A}{A})}$ to a map $\mathrm{hom}_X(\join{f}{(\join{f}{f})},m)$ and so on, resulting in a diagram of the form
\begin{equation*}
\begin{tikzcd}
A \arrow[dr] \arrow[r,"\inr"] & \join[X]{A}{A} \arrow[d,densely dotted] \arrow[r,"\inr"] & \join[X]{A}{(\join[X]{A}{A})} \arrow[dl,densely dotted] \arrow[r,"\inr"] & \cdots \arrow[dll,densely dotted,bend left=10] \\
& U
\end{tikzcd}
\end{equation*}

\begin{defn}
Suppose $f:A\to X$ is a map. Then we define the \define{fiberwise join powers} 
\begin{equation*}
f^{\ast n}:A_X^{\ast n} X.
\end{equation*}
\end{defn}

\begin{constr}
Note that the operation $(B,g)\mapsto (\join[X]{A}{B},\join{f}{g})$ defines an endomorphism on the type
\begin{equation*}
\sm{B:\UU}B\to X.
\end{equation*}
We also have $(\emptyt,\ind{\emptyt})$ and $(A,f)$ of this type. For $n\geq 1$ we define
\begin{align*}
A_X^{\ast (n+1)} & \defeq \join[X]{A}{A_X^{\ast n}} \\
f^{\ast (n+1)} & \defeq \join{f}{f^{\ast n}}.\qedhere
\end{align*}
\end{constr}

\begin{defn}
We define $A_X^{\ast\infty}$ to be the sequential colimit of the type sequence
\begin{equation*}
\begin{tikzcd}
A_X^{\ast 0} \arrow[r] & A_X^{\ast 1} \arrow[r,"\inr"] & A_X^{\ast 2} \arrow[r,"\inr"] & \cdots.
\end{tikzcd}
\end{equation*}
Since we have a cocone
\begin{equation*}
\begin{tikzcd}
A_X^{\ast 0} \arrow[r] \arrow[dr,swap,"f^{\ast 0}" near start] & A_X^{\ast 1} \arrow[r,"\inr"] \arrow[d,swap,"f^{\ast 1}" near start] & A_X^{\ast 2} \arrow[r,"\inr"] \arrow[dl,swap,"f^{\ast 2}" xshift=1ex] & \cdots \arrow[dll,bend left=10] \\
& X
\end{tikzcd}
\end{equation*}
we also obtain a map $f^{\ast\infty}:A_X^{\ast\infty}\to X$ by the universal property of $A_X^{\ast\infty}$. 
\end{defn}

\begin{lem}\label{lem:finfjp_up}
Let $f:A\to X$ be a map, and let $m:U\to X$ be an embedding. Then the function
\begin{equation*}
\blank\circ \seqin_0: \mathrm{hom}_X(f^{\ast\infty},m)\to \mathrm{hom}_X(f,m)
\end{equation*}
is an equivalence. 
\end{lem}

\begin{thm}\label{lem:isprop_infjp}
For any map $f:A\to X$, the map $f^{\ast\infty}:A_X^{\ast\infty}\to X$ is an embedding that satisfies the universal property of the image inclusion of $f$.
\end{thm}

\begin{exercises}
\exercise
\begin{subexenum}
\item Show that any proposition is locally small.\index{proposition!is locally small}
\item Show that any essentially small type is locally small.\index{essentially small!type!is locally small}
\item Show that the function type $A\to X$ is locally small whenever $A$ is essentially small and $X$ is locally small.
\end{subexenum}
\exercise Let $f:A\to B$ be a map. Show that the following are equivalent:
\begin{enumerate}
\item The map $f$ is \define{locally small}\index{locally small!map} in the sense that for every $x,y:A$, the action on paths of $f$
\begin{equation*}
\apfunc{f}:(x=y)\to (f(x)=f(y))
\end{equation*}
is an essentially small map.
\item The diagonal $\delta_f$ of $f$ as defined in \cref{ex:trunc_diagonal_map} is classified by the universal fibration.
\end{enumerate}
\exercise \label{ex:span_rel}Use \cref{thm:choice,thm:fam_proj,cor:times_up_out} to show that the type 
\begin{equation*}
\mathsf{span}(A,B)\defeq \sm{S:\UU} (S\to A)\times (S\to B)
\end{equation*}
of small spans from $A$ to $B$ is equivalent to the type $A\to (B\to\UU)$ of small relations from $A$ to $B$.
\end{exercises}


\chapter{The classifying type of a group}

\chapter{Truncations}

\section{The truncations}

\begin{exercises}
\item Show that a map $f:A\to B$ is $n$-truncated if and only if
\begin{equation*}
\begin{tikzcd}
A \arrow[r,"f"] \arrow[d,swap,"\tproj{n}{\blank}"] & B \arrow[d,"\tproj{n}{\blank}"] \\
\trunc{n}{A} \arrow[r,swap,"\trunc{n}{f}"] & \trunc{n}{B}
\end{tikzcd}
\end{equation*}
is a pullback square.
\end{exercises}


\chapter{Homotopy groups of types}

\section{Pointed types}
\begin{defn}
We introduce the `category' of pointed types:
\begin{enumerate}
\item A pointed type consists of a type $X$ equipped with a base point $x:X$. We will write $\UU_\ast$ for the type $\sm{X:\UU}X$ of all pointed types.
\item A pointed map $(f,p):(X,x)\to_\ast (Y,y)$ consists of a map $f:X\to Y$ and an identification $p:f(x)=y$. 
\item A pointed homotopy $(H,q):(f,p)\htpy_\ast (f',p')$ consists of a homotopy $H:f\htpy f'$ and an identification $q:p=\ct{H(x)}{p'}$ witnessing that the square
\begin{equation*}
\begin{tikzcd}
f(x) \arrow[rr,equals,"H(x)"] \arrow[dr,equals,swap,"p"] & & f'(x) \arrow[dl,equals,"{p'}"] \\
& y
\end{tikzcd}
\end{equation*}
commutes.
\end{enumerate}
\end{defn}

\begin{eg}
The circle $\sphere{1}$ is a pointed type with base point $\base:\sphere{1}$.
\end{eg}

\begin{eg}
If $X$ is a pointed type, then in the suspension of $X$ we have the canonical identification $\merid(\ast_X):\north=\south$. Therefore we do not have to worry about whether to choose $\north$ or $\south$ as the base point of $\susp{X}$. 
\end{eg} 

To consider higher pointed homotopies it is useful to first consider pointed families, and pointed $\Pi$-types.

\begin{defn}
\begin{enumerate}
\item Let $(X,\ast_X)$ be a pointed type. A \define{pointed family} over $(X,\ast_X)$ consists of a type family $P:X\to \UU_\ast$ equipped with a base point $\ast_P:P(\ast_X)$. 
\item Let $(P,\ast_P)$ be a pointed family over $(X,\ast_X)$. A \define{pointed section} of $(P,\ast_P)$ consists of a dependent function $f:\prd{x:X}P(x)$ and an identification $p:f(\ast_X)=\ast_P$. We define the \define{pointed $\Pi$-type} to be the type of pointed sections:
\begin{equation*}
\Pi^\ast_{(x:X)}P(x) \defeq \sm{f:\prd{x:X}P(x)}f(\ast_X)=\ast_P
\end{equation*}
\item Given any two pointed sections $f$ and $g$ of a pointed family $P$ over $X$, we define the type of pointed homotopies
\begin{equation*}
f\htpy_\ast g \defeq \Pi^\ast_{(x:X)} f(x)=g(x),
\end{equation*}
where the family $x\mapsto f(x)=g(x)$ is equipped with the base point $\ct{p}{q^{-1}}$. 
\end{enumerate}
\end{defn}

Since pointed homotopies are now defined as certain pointed sections, we can use the same definition of pointed homotopies again to consider pointed homotopies between pointed homotopies, and so on.

\section{Loop spaces}
\begin{defn}
Let $X$ be a pointed type with base point $x$. We define the \define{loop space} $\loopspace{X,x}$ of $X$ at $x$ to be the pointed type $x=x$ with base point $\refl{x}$. 
\end{defn}

\begin{defn}
The loop space operation $\loopspacesym$ is \emph{functorial} in the sense that
\begin{enumerate}
\item For every pointed map 
\end{enumerate}
\end{defn}

\section{Homotopy groups}
\begin{defn}
For $n\geq 1$, the \define{$n$-th homotopy group} of a type $X$ at a base point $x:X$ consists of the type
\begin{equation*}
|\pi_n(X,x)| \defeq \trunc{0}{\loopspace[n]{X,x}}
\end{equation*}
equipped with the group operations inherited from the path operations on $\loopspace[n]{X,x}$. 
Often we will simply write $\pi_n(X)$ when it is clear from the context what the base point of $X$ is.

For $n\jdeq 0$ we define $\pi_0(X,x)\defeq \trunc{0}{X}$. 
\end{defn}

\begin{eg}
In \autoref{circle_loopspace} we established that $\loopspace{\sphere{1}}=\Z$. It follows that
\begin{equation*}
\pi_1(\sphere{1})=\Z \qquad\text{ and }\qquad\pi_n(\sphere{1})=0\qquad\text{for $n\geq 2$.}
\end{equation*}
Furthermore, we have seen in \autoref{circle_conn} that $\trunc{0}{\sphere{1}}$ is contractible. 
Therefore we also have $\pi_0(\sphere{1})=0$.
\end{eg}

\begin{thm}[The Eckmann-Hilton argument]
For $n\geq 2$, the $n$-th homotopy group is abelian.
\end{thm}

\begin{exercises}
\item Show that the type of pointed families over a pointed type $(X,x)$ is equivalent to the type
\begin{equation*}
\sm{Y:\UU_\ast} Y\to_\ast X.
\end{equation*}
\item Given two pointed types $A$ and $X$, we say that $A$ is a (pointed) retract of $X$ if we have $i:A\to_\ast X$, a retraction $r:X\to_\ast A$, and a pointed homotopy $H:r\circ_\ast i\htpy_\ast \idfunc^\ast$. 
\begin{subexenum}
\item Show that if $A$ is a pointed retract of $X$, then $\loopspace{A}$ is a pointed retract of $\loopspace{X}$. 
\item Show that if $A$ is a pointed retract of $X$ and $\pi_n(X)$ is a trivial group, then $\pi_n(A)$ is a trivial group.
\end{subexenum}
\item Show that if $A\leftarrow S\rightarrow B$ is a span of pointed types, then for any pointed type $X$ the square
\begin{equation*}
\begin{tikzcd}
(A\sqcup^S B \to_\ast X) \arrow[r] \arrow[d] & (B \to_\ast X) \arrow[d] \\
(A\to_\ast X) \arrow[r] & (S\to_\ast X)
\end{tikzcd}
\end{equation*}
is a pullback square.
\item In this exercise we prove the suspension-loopspace adjunction.
\begin{subexenum}
\item Construct a pointed equivalence
\begin{equation*}
\tau_{X,Y}:(\susp(X)\to_\ast Y) \eqvsym_\ast (X\to \loopspace{Y})
\end{equation*}
for any two pointed spaces $X$ and $Y$.
\item Show that for any $f:X\to_\ast X'$ and $g:Y'\to_\ast Y$, there is a pointed homotopy witnessing that the square
\begin{equation*}
\begin{tikzcd}[column sep=large]
(\susp(X')\to_\ast Y') \arrow[r,"\tau_{X',Y'}"] \arrow[d,swap,"h\mapsto g\circ h\circ \susp(f)"] & (X'\to_\ast \loopspace{Y'}) \arrow[d,"h\mapsto\loopspace{g}\circ h\circ f"] \\
(\susp(X)\to_\ast Y) \arrow[r,swap,"\tau_{X,Y}"] & (X\to_\ast \loopspace{Y})
\end{tikzcd}
\end{equation*}
\end{subexenum}
\item Show that if
\begin{equation*}
\begin{tikzcd}
C \arrow[r] \arrow[d] & B \arrow[d] \\
A \arrow[r] & X
\end{tikzcd}
\end{equation*}
is a pullback square of pointed types, then so is
\begin{equation*}
\begin{tikzcd}
\loopspace{C} \arrow[r] \arrow[d] & \loopspace{B} \arrow[d] \\
\loopspace{A} \arrow[r] & \loopspace{X}.
\end{tikzcd}
\end{equation*}
\end{exercises}


\chapter{The long exact sequence of homotopy groups}

\section{The Hopf fibration}
Our goal in this chapter is to construct the Hopf fibration, i.e.~a fiber sequence
\begin{equation*}
\sphere{1}\hookrightarrow\sphere{3}\twoheadrightarrow\sphere{2}.
\end{equation*}


% !TEX root = hott_intro.tex

\section{Connected types and maps}

In this section we introduce the concept of $k$-connected types and maps. We define $k$-connected types to be types with contractible $k$-truncation, and a $k$-connected map is just a map of which the fibers are $k$-connected. The idea is that a type is $k$-connected if and only if its homotopy groups $\pi_i(X)$ are trivial for all $i\leq k$.

One of the main theorems in this section is a characterization of $k$-connected maps in terms of their action on homotopy groups: A map $f:X\to Y$ is $k$-connected if and only if it induces isomorphisms
\begin{equation*}
  \pi_i(f,x):\pi_i(X,x)\to\pi_i(Y,f(x))
\end{equation*}
of homotopy groups, for each $i\leq k$ and each $x:X$, and a \emph{surjective} group homomorphism
\begin{equation*}
  \pi_{k+1}(f,x):\pi_{k+1}(X,x)\to\pi_{k+1}(Y,f(x))
\end{equation*}
on the $(k+1)$-st homotopy group, for each $x:X$. If one drops the condition that $f$ induces a surjective group homomorphism on the $(k+1)$-st homotopy group, then the map is only a $k$-equivalence, i.e., a map of which $\trunc{k}{f}$ is an equivalence. We see from the above characterization that any $k$-connected map is a $k$-equivalence, and also that any $(k+1)$-equivalence is a $k$-connected map. Nevertheless, the difference between the classes of $k$-equivalences and $k$-connected maps is somewhat subtle.

We will study $k$-equivalences and $k$-connected maps synchronously, because understanding the subtle differences between the results about either of them will increase the understanding of both classes of maps. For instance, we will show that the $k$-connected maps enjoy a dependent elimination property, while the $k$-equivalences only satisfy a non-dependent elimination property. We will see that the $k$-equivalences satisfy the 3-for-2 property, while one of the cases of the 3-for-2 property fails for $k$-connected maps.

The $k$-connected maps can be characterized as the class of maps that is left orthogonal to the class of $k$-truncated maps, where a map $f:A\to B$ is said to be left orthogonal to a map $g:X\to Y$ if the type of diagonal fillers of any commuting square of the form
\begin{equation*}
  \begin{tikzcd}
    A \arrow[d,swap,"f"] \arrow[r] & X \arrow[d,"g"] \\
    B \arrow[r] \arrow[ur,densely dotted] & Y
  \end{tikzcd}
\end{equation*}
is contractible. Similarly, the class of $k$-equivalences is the class of maps that is left orthogonal to any map between $k$-truncated types. However, this result is not entirely sharp, because there are more maps that the $k$-equivalences are left orthogonal to. It turns out that a map is a $k$-equivalence if and only if it is left orthogonal to any map $g:X\to Y$ for which the naturality square
\begin{equation*}
  \begin{tikzcd}[column sep=large]
    X \arrow[r,"g"] \arrow[d,swap,"\eta"] & Y \arrow[d] \\
    \trunc{k}{X} \arrow[r,swap,"\trunc{k}{g}"] & \trunc{k}{Y}
  \end{tikzcd}
\end{equation*}
is a pullback square. Such maps are called $k$-\'etale, and they induce isomorphisms
\begin{equation*}
  \pi_i(g,x):\pi_i(X,x)\to\pi_i(Y,g(x))
\end{equation*}
on homotopy groups for $i>k$.

In the final part of this section we will use the results about $k$-equivalences to show that the $n$-sphere is $(n-1)$-connected, for each $n:\N$, and that the join $\join{A}{B}$ is $(k+l+2)$-connected if $A$ is $k$-connected and $B$ is $l$-connected.

\subsection{Connected types}

\begin{defn}
  A type $X$ is said to be \define{$k$-connected} if its $k$-truncation $\trunc{k}{X}$ is contractible. We define
  \begin{equation*}
    \isconn_k(X)\defeq\iscontr\trunc{k}{X}.
  \end{equation*}
\end{defn}

\begin{rmk}
  Since the $(-2)$-truncation of any type is just $\unit$, it follows that every type is $(-2)$-connected. Furthermore, since any proposition is contractible as soon as it comes equipped with a term, it follows that any type is $(-1)$-connected as soon as it is inhabited.

    In \cref{thm:conn-succ} below, we will see that a type $X$ is $0$-connected if and only if it is inhabited and every two points are connected by an unspecified path. In this sense $0$-connected types are also called \define{path connected}, or just \define{connected}. Thus, it is immediate that the circle is an example of a connected type.

  Similarly, in the case where $k\jdeq 0$ the theorem states that a type $X$ is $1$-conneced if and only if it is inhabited and for every $x,y:X$ the identity type $x=y$ is path connected. In other words, a type is \define{simply connected} if it is $1$-connected! The $2$-sphere is an example of a simply connected type. This fact is shown in \cref{cor:conn-sphere} below, where we will show more generally that the $n$-sphere is $(n-1)$-connected, for each $n:\N$.
\end{rmk}

\begin{lem}
  If a type is $(k+1)$-connected, then it is also $k$-connected.
\end{lem}

\begin{proof}
  This follows from the fact that $\trunc{k}{\trunc{k+1}{X}}\simeq\trunc{k}{X}$. Indeed, if $\trunc{k+1}{X}$ is contractible, then its $k$-truncation is also contractible, so it follows that $\trunc{k}{X}$ is contractible.
\end{proof}

For the following theorem, recall that a type $X$ is said to be inhabited if it comes equipped with a term $\trunc{-1}{X}$.

\begin{thm}\label{thm:conn-succ}
  Consider a type $X$. Then the following are equivalent:
  \begin{enumerate}
  \item The type $X$ is $(k+1)$-connected.
  \item The type $X$ is inhabited, and the type $x=y$ is $k$-connected for each $x,y:X$.
  \end{enumerate}
\end{thm}

\begin{proof}
  Suppose first that $X$ is $(k+1)$-connected. It is immediate that $X$ is inhabited in this case. Moreover, since we have equivalences
  \begin{equation*}
    (\eta(x)=\eta(y))\simeq \trunc{k}{x=y}
  \end{equation*}
  for each $x,y:X$, it follows from the assumption that $\trunc{k+1}{X}$ is contractible that the type $\trunc{k}{x=y}$ is equivalent to a contractible type. This proves that (i) implies (ii).

  To see that (ii) implies (i), suppose that $X$ is inhabited and that its identity types are $k$-connected. Our goal is to construct a term of type
  \begin{equation*}
    \iscontr\trunc{k+1}{X},
  \end{equation*}
  which is a proposition, so we may eliminate the assumption that $X$ is inhabited and assume to have $x:X$. Now we simply take $\eta(x)$ for the center of contraction of $\trunc{k+1}{X}$. To construct the contraction, note that by the dependent universal property of $(k+1)$-truncation we have an equivalence
  \begin{equation*}
    \Big(\prd{y:\trunc{k+1}{X}}\eta(x)=y\Big)\simeq\Big(\prd{y:X}\eta(x)=\eta(y)\Big).
  \end{equation*}
  Therefore it suffices to construct an identification $\eta(x)=\eta(y)$ for every $y:X$. However, this type is contractible, since it is equivalent to the contractible type $\trunc{k}{x=y}$. This completes the proof of (ii) implies (i).
\end{proof}

In the case where $k\geq -1$ we can improve \cref{thm:conn-succ} and characterize a high degree of connectedness entirely in terms of the triviality of homotopy groups. This is what connectedness is all about.

\begin{thm}\label{thm:conn-htpy-groups}
  Consider a type $X$, and suppose that $k\geq 0$. Then the following are equivalent:
  \begin{enumerate}
  \item The type $X$ is $k$-connected.
  \item The type $X$ is connected, and for every $x:X$ the loop space
    \begin{equation*}
      \loopspace{X,x}
    \end{equation*}
    is $(k-1)$-connected.
  \item For each $i\leq k$ and each $x:X$, the $i$-th homotopy group $\pi_i(X,x)$ is trivial.
  \end{enumerate}
\end{thm}

\begin{proof}
  If $X$ is $k$-connected for $k\geq 0$, then it is certainly connected, and $\loopspace{X,x}$ is $(k-1)$-connected by \cref{thm:conn-succ}. Thus, the fact that (i) implies (ii) is immediate.

  To see that (ii) implies (i), note that if $X$ is connected and its loop spaces are $(k-1)$-connected, then all its identity types are $(k-1)$-connected, since we have
  \begin{align*}
    \prd{x,y:X}\iscontr(\trunc{k-1}{x=y}) & \simeq \prd{x,y:X}\trunc{-1}{x=y}\to\iscontr(\trunc{k-1}{x=y}) \\
    & \simeq \prd{x,y:X}(x=y)\to\iscontr(\trunc{k-1}{x=y}) \\
    & \simeq \prd{x:X}\iscontr(\trunc{k-1}{x=x}).
  \end{align*}
  In the first step of this calculation we use that $X$ is connected, so $\trunc{-1}{x=y}$ is contractible; then we use that $\iscontr(\trunc{k-1}{x=y})$ is a proposition; and finally we use the universal property of identity types to arrive at our assumption that the loop spaces of $X$ are $(k-1)$-connected. Since we have shown that the identity types are $(k-1)$-connected, it follows by \cref{thm:conn-succ} that $X$ is $k$-connected, which concludes the proof that (ii) implies (i).

  It is easy to see by induction on $k\geq 0$ that (ii) holds if and only if (iii) holds, since we have
  \begin{equation*}
    \pi_{i+1}(X,x)=\pi_i(\loopspace{X,x}).\qedhere
  \end{equation*}
\end{proof}

\begin{rmk}
  If $X$ is assumed to be a pointed type in \cref{thm:conn-htpy-groups}, then conditions (ii) and (iii) only have to be checked at the base point.
\end{rmk}

\subsection{\texorpdfstring{$k$}{k}-Equivalences and \texorpdfstring{$k$}{k}-connected maps}

We now study two classes of maps that differ only slightly: the $k$-equivalences and the $k$-connected maps. 

\begin{defn}
  ~
  \begin{enumerate}
  \item A map $f:X\to Y$ is said to be \define{$k$-connected} if its fibers are $k$-connected. We will write
  \begin{equation*}
    \isconn_k(f)\defeq\prd{y:Y}\isconn_k(\fib{f}{y}).
  \end{equation*}
  \item A map $f:X\to Y$ is said to be a \define{$k$-equivalence} if
    \begin{equation*}
      \trunc{k}{f}:\trunc{k}{X}\to\trunc{k}{Y}
    \end{equation*}
    is an equivalence. We will write
    \begin{equation*}
      \isequiv_k(f)\defeq\isequiv(\trunc{k}{f}).
    \end{equation*}
  \end{enumerate}
\end{defn}

\begin{eg}
  Any equivalence is a $k$-connected map, as well as a $k$-equivalence. Moreover, for any $k$-connected type $X$ the map $\const_\ttt:X\to\unit$ is $k$-connected. It is also immediate that \emph{any} map between $k$-connected types is a $k$-equivalence.
\end{eg}

\begin{eg}
  A $(-1)$-connected map is a map $f:X\to Y$ for which the propositionally truncated fibers
  \begin{equation*}
    \trunc{-1}{\fib{f}{y}}
  \end{equation*}
  are contractible. Since propositions are contractible as soon as they are inhabited, we see that a map is $(-1)$-connected if and only if it is surjective.

  A $(-1)$-equivalence, on the other hand, is just a map $f:X\to Y$ that induces an equivalence $\trunc{-1}{X}\simeq\trunc{-1}{Y}$. The map $\const_\btrue : \unit\to\bool$ is an example of such a map, showing that $(-1)$-equivalences don't need to be surjective.

  However, it is the case that every surjective map $f:X\to Y$ is in fact $(-1)$-equivalence. To see this, we need to show that
  \begin{equation*}
    \trunc{-1}{Y}\to\trunc{-1}{X}.
  \end{equation*}
  Such a map is constructed by the universal property of $(-1)$-truncation. Thus, it suffices to construct a function $Y\to\trunc{-1}{X}$. Since we have assumed that $f$ is surjective, we have for every $y:Y$ a term
  \begin{equation*}
    s(y):\trunc{-1}{\fib{f}{y}}.
  \end{equation*}
  Thus, we define a function $Y\to\trunc{-1}{X}$ by
  \begin{equation*}
    y\mapsto\trunc{-1}{\proj 1}(s(y)).
  \end{equation*}
  This concludes the proof that $f$ is a $(-1)$-equivalence, since we have shown that $\trunc{-1}{X}\leftrightarrow\trunc{-1}{Y}$. 
\end{eg}

\begin{rmk}
  An immediate difference between the classes of $k$-equivalences and $k$-connected maps is that the $k$-connected maps are stable under base change, while the $k$-equivalences are not. By this, we mean that for any pullback square
  \begin{equation*}
    \begin{tikzcd}
      E' \arrow[d,swap,"{p'}"] \arrow[r,"g"] & E \arrow[d,"p"] \\
      B' \arrow[r,swap,"f"] & B,
    \end{tikzcd}
  \end{equation*}
  if the map $p$ is $k$-connected, then the map $p'$ is also $k$-connected. In such a pullback diagram, the map $p'$ is sometimes called the \define{base change} of $p$ along $f$. By \cref{cor:pb_fibequiv} we have an equivalence
  \begin{equation*}
    \fib{p'}{b'}\simeq\fib{p}{f(b')}
  \end{equation*}
  for any $b':B'$, so it is indeed the case that if the fibers of $p$ are $k$-connected, then so are the fibers of $p'$.

  An example showing that the $k$-equivalences are not stable under base change is given by the pullback square
  \begin{equation*}
    \begin{tikzcd}
      \loopspace{\sphere{k+1}} \arrow[r] \arrow[d] & \unit \arrow[d] \\
      \unit \arrow[r] & \sphere{k+1}
    \end{tikzcd}
  \end{equation*}
  We will show in \cref{cor:conn-sphere} that the $(k+1)$-sphere is $k$-connected, so the map $\unit\to\sphere{k+1}$ is a $k$-equivalence. However, its loop space is only $(k-1)$-connected, and indeed we will show in \cref{far-future} that
  $\pi_{k+1}(\sphere{k+1})=\Z$ for $k\geq 0$, showing that $\loopspace{\sphere{k+1}}$ is \emph{not} $k$-connected. Thus, the map $\loopspace{\sphere{k+1}}\to\unit$ is not a $k$-equivalence.
\end{rmk}

\subsubsection{Elimination properties}
We will show that a map $f:X\to Y$ is a $k$-equivalence if and only if the precomposition function
\begin{equation*}
  \blank\circ f : (Y\to Z)\to (X\to Z)
\end{equation*}
is an equivalence for every $k$-type $Z$. On the other hand, we will show that $f$ is $k$-connected if and only if the precomposition function
\begin{equation*}
  \blank\circ f : \Big(\prd{y:Y}P(y)\Big)\to\Big(\prd{x:X}P(f(x))\Big)
\end{equation*}
is an equivalence for every family $P$ of $k$-types over $Y$. In other words, the $k$-connected maps satisfy a \emph{dependent} unique elimination property, while the $k$-equivalences only satisfy a \emph{non-dependent} unique elimination property.

\begin{thm}\label{thm:k-equiv-precomp}
  Consider a function $f:X\to Y$. Then the following are equivalent
  \begin{enumerate}
  \item The map $f$ is a $k$-equivalence.
  \item For every $k$-type $Z$, the precomposition function
    \begin{equation*}
      \blank\circ f:(Y\to Z)\to(X\to Z)
    \end{equation*}
    is an equivalence.
  \end{enumerate}
\end{thm}

\begin{thm}\label{thm:conn-dup}
  Let $f:X\to Y$ be a map. The following are equivalent:
  \begin{enumerate}
  \item The map $f$ is $k$-connected.
  \item For every family $P$ of $k$-truncated types over $Y$, the precomposition map
    \begin{equation*}
      \blank\circ f : \Big(\prd{y:Y}P(y)\Big)\to\Big(\prd{x:X}P(f(x))\Big)
    \end{equation*}
    is an equivalence.
  \end{enumerate}
\end{thm}

\begin{proof}
  Suppose $f$ is $k$-connected and let $P$ be a family of $k$-types over $Y$. Now we may consider the following commuting diagram
  \begin{equation*}
    \begin{tikzcd}[column sep=-10em]
      \phantom{\prd{x:X}{y:Y}{p:f(x)=y}P(y)} & \prd{y:Y}P(y) \arrow[r,"\blank\circ f"] \arrow[dl] &[10em] \prd{x:X}P(f(x)) \\
      \prd{y:Y}\trunc{k}{\fib{f}{y}}\to P(y) \arrow[dr] & \phantom{\prd{y:Y}{x:X}{p:f(x)=y}P(y)} & & \prd{x:X}{y:Y}{p:f(x)=y}P(y) \arrow[ul] \\
      & \prd{y:Y}\fib{f}{y}\to P(y) \arrow[r] & \prd{y:Y}{x:X}{p:f(x)=y}P(y) \arrow[ur]
    \end{tikzcd}
  \end{equation*}
  which commutes by $\reflhtpy$. In this diagram, the five maps going around counter clockwise are all equivalences for obvious reasons, so it follows that the top map is an equivalence.

  Now suppose that $f$ satisfies the dependent elimination property stated in (ii). In order to construct a center of contraction of $\trunc{k}{\fib{f}{y}}$ for every $y:Y$, we use the dependent elimination property with respect to the family $P$ given by $P(y)\defeq\trunc{k}{\fib{f}{y}}$. 
\end{proof}

\begin{cor}
  For any type $X$, the unit $\eta:X\to\trunc{k}{X}$ of the $k$-truncation is a $k$-connected map. 
\end{cor}

\subsubsection{The inclusions}

We will prove the following implications
\begin{equation*}
  \begin{tikzcd}[column sep=8em]
    \isequiv_{k+1}(f) \arrow[r,"\text{\cref{prp:is-k-equiv-is-k-conn}}"] & \isconn_k(f) \arrow[r,"\text{\cref{prp:is-k-conn-is-succk-equiv}}"] & \isequiv_k(f)
  \end{tikzcd}
\end{equation*}
showing that the class of $k$-connected maps is contained in the class of $k$-equivalences, and that the class of $(k+1)$-equivalences is contained in the class of $k$-connected maps. Neither of these implications reverses.

\begin{prp}\label{prp:is-k-equiv-is-k-conn}
  Any $k$-connected map is a $k$-equivalence.
\end{prp}

\begin{prp}\label{prp:is-k-conn-is-succk-equiv}
  Any $(k+1)$-equivalence is $k$-connected.
\end{prp}

\begin{proof}
  Consider a $(k+1)$-equivalence $f:X\to Y$. Recall that the map $\trunc{k+1}{f}$ comes equipped with a homotopy $H:\trunc{k+1}{f}\circ\eta\htpy\eta\circ f$ witnessing that the square
  \begin{equation*}
    \begin{tikzcd}[column sep=large]
      X \arrow[r,"f"] \arrow[d,swap,"\eta"] & Y \arrow[d,"\eta"] \\
      \trunc{k+1}{X} \arrow[r,swap,"\trunc{k+1}{f}"] & \trunc{k+1}{Y}
    \end{tikzcd}
  \end{equation*}
  commutes. We be using this homotopy, and we will use \cref{thm:conn-dup} to show that $f$ is $k$-connected. Thus, our goal is to show that
  \begin{equation*}
    \blank\circ f:\Big(\prd{y:Y}P(y)\Big)\to\Big(\prd{x:X}P(f(x))\Big)
  \end{equation*}
  is an equivalence for any family $P$ of $k$-types over $Y$.

  Note that any family $P$ of $k$-types over $Y$ extends to a family $\tilde{P}$ of $k$-types over $\trunc{k+1}{Y}$, since any univalent universe of $k$-types that contains $P$ is itself a $(k+1)$-type by \cref{ex:istrunc_UUtrunc}. The extended family $\tilde{P}$ of $k$-types over $\trunc{k+1}{Y}$ comes equipped with a family of equivalences
  \begin{equation*}
    e:\prd{y:Y}\tilde{P}(\eta(y))\simeq P(y).
  \end{equation*}
  Now consider the commuting diagram
  \begin{equation*}
    \begin{tikzcd}[column sep=-9em]
      \phantom{\prd{x:X}\tilde{P}(\trunc{k+1}{f}(\eta(x)))} & \phantom{\prd{x:X}\tilde{P}(\trunc{k+1}{f}(\eta(x)))} & \prd{y:\trunc{k+1}{Y}}\tilde{P}(y) \arrow[r,"\blank\circ\trunc{k+1}{f}"] \arrow[ddll,swap,"\blank\circ\eta"] &[11em] \prd{x:\trunc{k+1}{X}}\tilde{P}(\trunc{k+1}{f}(x)) \arrow[dr,"\blank\circ\eta" near end] & & \phantom{\prd{x:X}\tilde{P}(\trunc{k+1}{f}(\eta(x)))} \\
      & & \phantom{\prd{x:\trunc{k+1}{X}}\tilde{P}(\trunc{k+1}{f}(x))} & & \prd{x:X}\tilde{P}(\trunc{k+1}{f}(\eta(x))) \arrow[dr,"{h\mapsto\lam{x}\tr_{\tilde{P}}(H(x),h(x))}" near end] \\
      \prd{y:Y}\tilde{P}(\eta(y)) \arrow[drr,swap,"{h\mapsto\lam{y}e_y(h(y))}" near start] & & & & & \prd{x:X}\tilde{P}(\eta(f(x))) \arrow[dll,"{h\mapsto\lam{x}e_{f(x)}(h(x))}" near start] \\
      & & \prd{y:Y}P(y) \arrow[r,swap,"\blank\circ f"] & \prd{x:X}P(f(x)).
    \end{tikzcd}
  \end{equation*}
  This diagram commutes by the homotopy
  \begin{equation*}
    \lam{h}\eqhtpy(\lam{x}\ap{e(f(x))}{\apd{h}{H(x)}}^{-1}).
  \end{equation*}
  In this diagrams all the maps pointing downwards are equivalences for obvious reasons: the two maps $\blank\circ\eta$ are equivalences since $\tilde{P}$ is a family of $k$-types, and the remaining three maps pointing downwards are all postcomposing with an equivalence. The top map is an equivalence since $\trunc{k+1}{f}$ is assumed to be an equivalence. Thus we conclude that the bottom map $\blank\circ f$ is an equivalence.
\end{proof}

\subsubsection{The 3-for-2 property}
An important distinction between the class of $k$-equivalences and the class of $k$-connected maps is that the $k$-equivalences satisfy the 3-for-2 property, while the $k$-connected maps do not.

\begin{rmk}\label{rmk:conn-3-for-2}
  It is not hard to see that the $k$-connected maps don't satisfy the 3-for-2 property. For example, consider the following commuting triangle
  \begin{equation*}
    \begin{tikzcd}[column sep=tiny]
      \sphere{1} \arrow[rr,"d_2"] \arrow[dr] & & \sphere{1} \arrow[dl] \\
      & \unit,
    \end{tikzcd}
  \end{equation*}
  where $d_2:\sphere{1}\to\sphere{1}$ is the degree $2$ map. Since the circle is a $0$-connected type, it follows that the maps $\sphere{1}\to\unit$ are $0$-connected. However, the fiber of $d_2$ at the base point is equivalent to the booleans, which is a non-contractible set so it is certainly not $0$-connected.
  \end{rmk}

\begin{lem}
  The $k$-equivalences satisfy the 3-for-2 property, i.e., for any commuting triangle
  \begin{equation*}
    \begin{tikzcd}[column sep=tiny]
      A \arrow[rr,"h"] \arrow[dr,swap,"f"] & & B \arrow[dl,"g"] \\
      & X,
    \end{tikzcd}   
  \end{equation*}
  if any two of the three maps are $k$-equivalences, then so is the third.
\end{lem}

\begin{proof}
  This follows immediately from the fact that equivalences satisfy the 3-for-2 property.
\end{proof}

\begin{prp}
  Consider a commuting triangle
  \begin{equation*}
    \begin{tikzcd}[column sep=tiny]
      A \arrow[rr,"h"] \arrow[dr,swap,"f"] & & B \arrow[dl,"g"] \\
      & X
    \end{tikzcd}
  \end{equation*}
  with $H:f\htpy g\circ h$. The following three statements hold:
  \begin{enumerate}
  \item If $f$ and $h$ are $k$-connected, then $g$ is $k$-connected.
  \item If $g$ and $h$ are $k$-connected, then $f$ is $k$-connected.
  \item If $f$ and $g$ are $k$-connected, then $h$ is a $k$-equivalence.
  \end{enumerate}
\end{prp}

\begin{proof}
  The first two statements combined assert that if $h$ is $k$-connected, then $f$ is $k$-connected if and only if $g$ is $k$-connected. To see that this equivalence holds, consider for any family $P$ of $k$-truncated types over $X$ the commuting square
  \begin{equation*}
    \begin{tikzcd}[column sep=10em]
      \prd{x:X}P(x) \arrow[r,"\blank\circ g"] \arrow[d,swap,"\blank\circ f"] & \prd{b:B}P(g(b)) \arrow[d,"\blank\circ h"] \\
      \prd{a:A}P(f(a)) \arrow[r,swap,"{\lam{s}{a}\tr_P(H(a),s(a))}"] & \prd{a:A}P(g(h(a)))
    \end{tikzcd}
  \end{equation*}
  In this square, the bottom map is given by postcomposing with the family of equivalences $\tr_P(H(a))$ indexed by $a:A$, so it is an equivalence. The map on the right is an equivalence by \cref{thm:conn-dup}, using the assumption that $h$ is a $k$-connected map. The square commutes by the homotopy
  \begin{equation*}
    \lam{s}\eqhtpy\big(\lam{a}\apd{s}{H(a)}\big).
  \end{equation*}
  Therefore it follows that the precomposition map $\blank\circ f$ is an equivalence if and only if the precomposition map $\blank\circ g$ is. By \cref{thm:conn-dup} we conclude that $f$ is connected if and only if $g$ is. This proves statements (i) and (ii).

  Statement (iii) follows from the facts that any $k$-connected map is a $k$-equivalence by \cref{cor:k-equiv-k-conn} and that the $k$-equivalences satisfy the 3-for-2 property \cref{lem:3-for-2-k-equiv}.
\end{proof}

\subsubsection{The action on homotopy groups}

\begin{thm}
  Consider a map $f:X\to Y$, and suppose that $k\geq -1$. The following are equivalent:
  \begin{enumerate}
  \item The map $f$ is a $k$-equivalence.
  \item The map $f$ is a $(-1)$-equivalence, and for every $0\leq i\leq k$ and every $x:X$, the induced group homomorphism
    \begin{equation*}
      \pi_i(f,x):\pi_i(X,x)\to\pi_i(Y,f(x))
    \end{equation*}
    is an isomorphism.
  \end{enumerate}
\end{thm}

\begin{defn}
  A map $f:X\to Y$ is said to be a \define{weak equivalence} if it is a $0$-equivalence, and it induces an isomorphism
  \begin{equation*}
    \pi_i(f,x):\pi_i(X,x)\cong\pi_i(Y,f(x))
  \end{equation*}
  on homotopy groups, for every $x:X$ and every $i\geq 1$. 
\end{defn}

The following corollary is an instance of Whitehead's principle, which asserts that a map between any two spaces is a homotopy equivalence if and only if it is a weak equivalence. Thus, by the following corollary, Whitehead's principle holds for $k$-types.

\begin{cor}
  Consider two $k$-types $X$ and $Y$, and consider a map $f:X\to Y$ between them. Then the following are equivalent:
  \begin{enumerate}
  \item The map $f$ is an equivalence.
  \item The map $f$ is a weak equivalence.
  \end{enumerate}
\end{cor}

\begin{thm}
  Consider a map $f:X\to Y$. The following are equivalent:
  \begin{enumerate}
  \item The map $f$ is $(k+1)$-connected.
  \item The map $f$ is surjective, and for each $x,x':X$ the action on paths
    \begin{equation*}
      \apfunc{f} : (x=x')\to (f(x)=f(x'))
    \end{equation*}
    is $k$-connected.
  \end{enumerate}
\end{thm}

\begin{thm}
  Consider a surjective map $f:X\to Y$. The following are equivalent:
  \begin{enumerate}
  \item The map $f$ is $k$-connected.
  \item The induced maps on loop spaces
    \begin{equation*}
      \loopspace{f,x}:\loopspace{X,x}\to\loopspace{Y,f(x)}
    \end{equation*}
    is $(k-1)$-connected for every $x:X$.
  \item The induced maps on homotopy groups
    \begin{equation*}
      \pi_i(f,x):\pi_i(X,x)\to\pi_i(Y,f(x))
    \end{equation*}
    are isomorphisms for $0\leq i\leq k$, and it is surjective for $i=k+1$. 
  \end{enumerate}
\end{thm}

\begin{rmk}
  If $f:X\to Y$ is a pointed map between connected types, then conditions (ii) and (iii) in \cref{thm:htpy-groups-conn-map} only have to be checked at the base point.
\end{rmk}

\subsection{Orthogonality}

The idea of orthogonality is that a map $f:A\to B$ is left orthogonal to a map $g:X\to Y$ if for every commuting square of the form
\begin{equation*}
  \begin{tikzcd}
    A \arrow[d,swap,"f"] \arrow[r,"h"] & X \arrow[d,"g"] \\
    B \arrow[r,swap,"i"] & Y,
  \end{tikzcd}
\end{equation*}
with $H:(i\circ f)\htpy (g\circ h)$, the type of diagonal fillers is contractible. The type of diagonal fillers is the type of maps $j:B\to X$ equipped with homotopies
\begin{align*}
  K & : j\circ f\htpy h \\
  L & : g\circ j\htpy i
\end{align*}
and a homotopy $M$ witnessing that the triangle
\begin{equation*}
  \begin{tikzcd}[column sep=small]
    g\circ j\circ f \arrow[rr,"g\cdot K"] \arrow[dr,swap,"L\cdot f"] & & h\circ g \\
    & i\circ f \arrow[ur,swap,"H"]
  \end{tikzcd}
\end{equation*}
commutes. A slicker way to express this condition is to assert that the map
\begin{equation*}
   (B\to X)\to \sm{h:A\to X}{i:B\to Y} i\circ f\htpy g\circ h
\end{equation*}
given by $j\mapsto(j\circ f,g\circ j,\reflhtpy)$ is an equivalence. Indeed, the type of triples $(h,i,H)$ in the codomain is the type of commuting squares with respect to which we stated the orthogonality condition. Now we may even recognize the above map as a gap map of a commuting square, and we arrive at our actual definition of orthogonality.

\begin{defn}
  A map $f:A\to B$ is said to be \define{left orthogonal} to a map $g:X\to Y$, or equivalently the map $g$ is said to be \define{right orthogonal} to $f$, if the commuting square
  \begin{equation*}
    \begin{tikzcd}[column sep=large]
      X^B \arrow[r,"\blank\circ f"] \arrow[d,swap,"g\circ\blank"] & X^A \arrow[d,"g\circ\blank"] \\
      Y^B \arrow[r,swap,"\blank\circ f"] & Y^A
    \end{tikzcd}
  \end{equation*}
  is a pullback square.
\end{defn}

\begin{thm}
Let $f:A\to B$ be a map. The following are equivalent:
\begin{enumerate}
\item The map $f$ is $k$-connected.
\item The map $f$ is left orthogonal to every $k$-truncated map.
is a pullback square.
\end{enumerate}
\end{thm}

\begin{thm}
  Let $f:A\to B$ be a map. The following are equivalent:
  \begin{enumerate}
  \item The map $f$ is a $k$-equivalence.
  \item The map $f$ is left orthogonal to every map between $k$-truncated types.
  \item The map $f$ is left orthogonal to every map $g:X\to Y$ for which the naturality square
    \begin{equation*}
      \begin{tikzcd}
        X \arrow[r,"g"] \arrow[d,swap,"\eta"] & Y \arrow[d,"\eta"] \\
        \trunc{k}{X} \arrow[r,swap,"\trunc{k}{g}"] & \trunc{k}{Y}
      \end{tikzcd}
    \end{equation*}
    is a pullback square. Such maps are called \define{$k$-\'etale}.
  \end{enumerate}
\end{thm}

\subsection{The connectedness of suspensions}

We will use connected maps to prove the connectedness of suspensions.

\begin{prp}\label{prp:conn-pushout}
  Consider a pushout square
  \begin{equation*}
    \begin{tikzcd}
      S \arrow[d,swap,"f"] \arrow[r,"g"] & B \arrow[d,"j"] \\
      A \arrow[r,swap,"i"] & X.
    \end{tikzcd}
  \end{equation*}
  If the map $f:S\to A$ is $k$-connected, then so is the map $j:B\to X$.
\end{prp}

\begin{proof}
  We claim that the map $j:B\to X$ is left orthogonal to any $k$-truncated map $p:Y\to Z$, which is equivalent to the property that $j$ is $k$-connected. To see that $j$ is left orthogonal to $p$, consider the commuting cube
  \begin{equation*}
    \begin{tikzcd}
      & Y^X \arrow[dl] \arrow[d] \arrow[dr] & \\
      Y^A \arrow[d] & Y^B \arrow[dl] \arrow[dr] & Z^X \arrow[dl,crossing over] \arrow[d] \\
      Y^S \arrow[dr] & Z^A \arrow[d] \arrow[from=ul,crossing over] & Z^B \arrow[dl] \\
      & Z^S.
    \end{tikzcd}
  \end{equation*}
  In this cube, the front left square is a pullback square because the map $f:S\to A$ is assumed to be $k$-connected, and therefore it is left orthogonal to the $k$-truncated map $p$. The back left and front right squares are pullback squares by the pullback property of pushouts. Therefore it follows that the back right square is a pullback square. This shows that $j$ is left orthogonal to $p$.
\end{proof}

\begin{lem}\label{lem:conn-mismatch}
  A pointed type $X$ is $(k+1)$-connected if and only if the point inclusion
  \begin{equation*}
    \unit\to X
  \end{equation*}
  is a $k$-connected map.
\end{lem}

\begin{proof}
  Since $X$ is assumed to have a base point $x_0:X$, it follows that $X$ is $(k+1)$-connected if and only if its identity types $(x=y)$ are $k$-connected. Now the claim follows from the fact that there is an equivalence
  \begin{equation*}
    \fib{\const_{x_0}}{y}\simeq (x_0=y).\qedhere
  \end{equation*}
\end{proof}

\begin{thm}\label{thm:conn-suspension}
  If $X$ is an $k$-connected type, then its suspension $\susp X$ is $(k+1)$-connected.
\end{thm}

\begin{proof}
  The type $X$ is $k$-connected if and only if the map $\const_\ttt:X\to\unit$ is a $k$-connected map. Recall that the suspension of $X$ is a pushout
  \begin{equation*}
    \begin{tikzcd}
      X \arrow[d,swap,"\const_\ttt"] \arrow[r,"\const_\ttt"] & \unit \arrow[d,"\south"] \\
      \unit \arrow[r,swap,"\north"] & \susp X.
    \end{tikzcd}
  \end{equation*}
  Therefore we see by \cref{prp:conn-pushout} that the point inclusions $\north,\south:\unit\to\susp X$ are both $k$-connected maps. By \cref{lem:conn-mismatch} it follows that $\susp X$ is a $(k+1)$-connected type.
\end{proof}

\begin{cor}\label{cor:conn-sphere}
  The $n$-sphere is $(n-1)$-connected.
\end{cor}

\begin{proof}
  The $0$-sphere is $(-1)$-connected, since it contains a point. Thus the claim follows by induction on $n:\N$, using \cref{thm:conn-suspension}.
\end{proof}

\subsection{The join connectivity theorem}

\begin{thm}
  If $X$ is $k$-connected and $Y$ is $l$-connected, then their join $\join{X}{Y}$ is $(k+l+2)$-connected.
\end{thm}

\begin{thm}
Consider a pullback square
\begin{equation*}
\begin{tikzcd}
C \arrow[r] \arrow[d] & B \arrow[d] \\
A \arrow[r] & X.
\end{tikzcd}
\end{equation*}
If the maps $A\to X$ and $B\to X$ are $k$- and $l$-connected, respectively, then the map $A\sqcup^C B\to X$ is $(k+l+2)$-connected.
\end{thm}

\begin{thm}
  The connected maps contain the equivalences, are closed under coproducts, pushouts, retracts, and transfinite compositions.
\end{thm}

\begin{exercises}
  \exercise Show that every type is equivalent to a disjoint union of connected components, i.e., show that for every type $X$ there is a family of connected types $B_i$ by a set $I$, with an equivalence
  \begin{equation*}
    X \eqvsym \sm{i:I}B_i.
  \end{equation*}
\exercise Let $f:A\to_\ast B$ be a pointed map between pointed $n$-connected types, for $n\geq -1$. Show that the following are equivalent:
\begin{enumerate}
\item $f$ is an equivalence.
\item $\loopspace[n+1]{f}$ is an equivalence. 
\end{enumerate}
\exercise Show that if
\begin{equation*}
\begin{tikzcd}
A \arrow[r] \arrow[d,swap,"f"] & B \arrow[d,"g"] \\
X \arrow[r] & Y
\end{tikzcd}
\end{equation*}
is \define{$k$-cocartesian} in the sense that the cogap map is $k$-connected, then the map $\mathsf{cofib}(f)\to \mathsf{cofib}(g)$ is $k$-connected.
\exercise Show that if $f:X\to Y$ is a $k$-connected map, then so is
\begin{equation*}
  \begin{tikzcd}
    \trunc{l}{f}:\trunc{l}{X}\to\trunc{l}{Y}
  \end{tikzcd}
\end{equation*}
for any $l\geq-2$.
\exercise Consider a commuting square
\begin{equation*}
\begin{tikzcd}
A \arrow[d,swap,"f"] \arrow[r] & B \arrow[d,"g"] \\
X \arrow[r] & Y
\end{tikzcd}
\end{equation*}
\begin{subexenum}
\item Show that if the square is $k$-cartesian and $g$ is $k$-connected, then so is $f$.
\item Show that if $f$ is $k$-connected and $g$ is $(k+1)$-connected, then the square is $k$-cartesian. 
\end{subexenum}
\exercise
\begin{subexenum}
\item Show that any sequential colimit of $k$-connected types is again $k$-connected.
\item Show that if every map in a type sequence
  \begin{equation*}
    \begin{tikzcd}
      A_0\arrow[r] & A_1 \arrow[r] & A_2 \arrow[r] & \cdots
    \end{tikzcd}
  \end{equation*}
  is $k$-connected, then so is the transfinite composition $A_0\to A_\infty$.
\end{subexenum}
\exercise Recall that a commuting square is called $k$-cartesian, if its gap map is $k$-connected. Show that $(k+1)$-truncation preserves $l$-cartesian squares for any $l\leq k$, i.e., show that for any $l\leq k$, if a square
\begin{equation*}
  \begin{tikzcd}
    C \arrow[r,"q"] \arrow[d,swap,"p"] & B \arrow[d,"g"] \\
    A \arrow[r,swap,"f"] & X.
  \end{tikzcd}
\end{equation*}
is $l$-cartesian, then the square
\begin{equation*}
  \begin{tikzcd}[column sep=large]
    \trunc{k+1}{C} \arrow[r,"\trunc{k+1}{q}"] \arrow[d,swap,"\trunc{k+1}{p}"] & \trunc{k+1}{B} \arrow[d,"\trunc{k+1}{g}"] \\
    \trunc{k+1}{A} \arrow[r,swap,"\trunc{k+1}{f}"] & \trunc{k+1}{X}
  \end{tikzcd}
\end{equation*}
is $l$-cartesian.
\exercise Generalize \cref{rmk:conn-3-for-2} to show that for every $k\geq-1$, the $k$-connected maps do not satisfy the 3-for-2 property.
\exercise Consider a commuting square
\begin{equation*}
\begin{tikzcd}
A \arrow[d,swap,"f"] \arrow[r] & B \arrow[d,"g"] \\
X \arrow[r] & Y
\end{tikzcd}
\end{equation*}
Show that the following are equivalent:
\begin{enumerate}
\item The map $A\to X\times_Y B$ is $n$-connected. In this case the square is called \define{$n$-cartesian}.
\item For each $x:X$ the map
\begin{equation*}
\fib{f}{x}\to \fib{g}{f(x)}
\end{equation*}
is $n$-connected.
\end{enumerate}
\exercise Consider a map $f:A\to B$. Show that the following are equivalent:
  \begin{enumerate}
  \item The map $f$ is a weak equivalence.
  \item The map $f$ is $\infty$-connected, in the sense that $f$ is $k$-connected for each $k$.
  \item The map $f$ is left orthogonal to any map between truncated types of any truncation level.
  \item The map $f$ is left orthogonal to any truncated map, for any truncation level.
  \end{enumerate}
  Thus we see that, while the classes of $k$-connected maps and $k$-equivalences differ for finite $k\geq-1$, they come to agree at $\infty$.
  \exercise Consider a pointed $(k+1)$-connected type $X$. Show that every $k$-truncated map $f:A\to X$ trivializes, in the sense that there is a $k$-type $B$ and an equivalence $e:\eqv{A}{X\times B}$ for which the triangle
  \begin{equation*}
    \begin{tikzcd}[column sep=0]
      A \arrow[rr,"e"] \arrow[dr,swap,"f"] & & X\times B \arrow[dl,"\proj 1"] \\
      \phantom{X\times B} & X
    \end{tikzcd}
  \end{equation*}
  commutes.
  \exercise Consider a $k$-equivalence $f:B'\to B$. Show that the base-change functor induces an equivalence
  \begin{equation*}
    \Big(\sm{E:\UU}{p:E\to B}\isetale_k(p)\Big)\simeq\Big(\sm{E':\UU}{p':E'\to B'}\isetale_k(p')\Big).
  \end{equation*}
  In other words, for every $k$-\'etale map $p':E'\to B'$ there is a unique $k$-\'etale map $p:E\to B$ equipped with a map $q:E'\to E$ such that the square
  \begin{equation*}
    \begin{tikzcd}
      E' \arrow[d,swap,"{p'}"] \arrow[r,densely dotted,"q"] & E \arrow[d,densely dotted,"p"] \\
      B' \arrow[r,swap,"f"] & B
    \end{tikzcd}
  \end{equation*}
  commutes and is a pullback square. In this sense $k$-\'etale maps descend along $k$-equivalences.
\end{exercises}


\section{The wedge and smash product of pointed types}

\begin{defn}
Let $A$ and $B$ be pointed types.
\begin{enumerate}
\item We define the \define{wedge} $A\vee B$ of $A$ and $B$ to be the pushout
\begin{equation*}
\begin{tikzcd}
\unit \arrow[r] \arrow[d] & B \arrow[d] \\
A \arrow[r] & A\vee B
\end{tikzcd}
\end{equation*}
\item We define the \define{smash product} $A\wedge B$ of $A$ and $B$ to be the cofiber of the cogap map of the square
\begin{equation*}
\begin{tikzcd}
\unit \arrow[r] \arrow[d] & B \arrow[d] \\
A \arrow[r] & A\times B
\end{tikzcd}
\end{equation*}
That is, the smash product is defined as the cofiber of the canonical map $A\vee B\to A\times B$. 
\end{enumerate}
\end{defn}

For any two pointed types $A$ and $B$, there is a pointed map
\begin{equation*}
\mathsf{pair}_\ast : A \to_\ast (B\to_\ast A\wedge B).
\end{equation*}

\begin{thm}\label{thm:smash_adj}
Let $A$, $B$, and $X$ be pointed types. Then the pointed map
\begin{equation*}
(A\wedge B \to_\ast X)\to_\ast (A \to_\ast (B\to_\ast X))
\end{equation*}
given by $f\mapsto f\mathbin{\circ_\ast}\mathsf{pair}_\ast$ is an equivalence.
Moreover, these equivalences are natural in $A$, $B$, and $X$ in the sense that...
\end{thm}

\begin{cor}
For any $m,n:\N$ we have an equivalence
\begin{equation*}
\eqv{{\sphere{m}}\wedge{\sphere{n}}}{\sphere{m+n}}.
\end{equation*}
\end{cor}

\begin{proof}
We have
\begin{align*}
({\sphere{m}}\wedge{\sphere{n}}\to_\ast X) & \eqvsym (\sphere{m}\to_\ast (\sphere{n}\to_\ast X)) \\
& \eqvsym \loopspace[m]{\loopspace[n]{X}} \\
& \eqvsym \loopspace[m+n]{X} \\
& \eqvsym (\sphere{m+n}\to_\ast X)
\end{align*}
By the naturality of the equivalences in \cref{thm:smash_adj} it follows that the composite equivalence is given by precomposition by the pointed map 
\begin{equation*}
\sphere{m}\wedge\sphere{n}\to_\ast \sphere{m+n}
\end{equation*}
that corresponds to the identity map $\sphere{m+n}\to_\ast \sphere{m+n}$. Thus it follows by \cref{ex:yoneda_ptd_types} that this pointed map is an equivalence.
\end{proof}

\begin{thm}
Given two pointed spaces, there is an equivalence
\begin{equation*}
\eqv{\join{X}{Y}}{\susp(X\wedge Y)}.
\end{equation*}
\end{thm}

\begin{exercises}
\exercise 
\begin{subexenum}
\item Show that $Y$ is equivalent to the mapping cone of $X\to X\vee Y$.
\item Show that the pushout of $X \leftarrow X\vee Y \rightarrow Y$ is contractible.
\end{subexenum}
\exercise Let $A$ and $B$ be pointed types. Show that the square
\begin{equation*}
\begin{tikzcd}
A+B \arrow[r] \arrow[d] & A\times B \arrow[d] \\
1+1 \arrow[r] & A\wedge B
\end{tikzcd}
\end{equation*}
is cocartesian.
\exercise Show that if
\begin{equation*}
\begin{tikzcd}
S_1 \arrow[r] \arrow[d] & Y_1 \arrow[d] & S_2 \arrow[r] \arrow[d] & Y_2 \arrow[d] \\
X_1 \arrow[r] & Z_1 & X_2 \arrow[r] & Z_2
\end{tikzcd}
\end{equation*}
are pushout squares, where all types, maps and homotopies are pointed, then so is
\begin{equation*}
\begin{tikzcd}
S_1\vee S_2 \arrow[r] \arrow[d] & Y_1\vee Y_2 \arrow[d] \\
X_1 \vee X_2 \arrow[r] & Z_1\vee Z_2. 
\end{tikzcd}
\end{equation*}
\exercise Show that if
\begin{equation*}
\begin{tikzcd}
S \arrow[r] \arrow[d] & Y \arrow[d] \\
X \arrow[r] & Z
\end{tikzcd}
\end{equation*}
is a cocartesian square of pointed spaces, then the cofiber of $X\vee Y\to Z$ is equivalent to $\susp(S)$.
\exercise Show that there is an equivalence
\begin{equation*}
\eqv{\susp(X\times Y)}{\susp(X\vee Y)\vee \susp(X\wedge Y)}
\end{equation*}
\exercise Show that $\susp(X\vee Y)$ is a retract of $\susp(X\times Y)$. 
\exercise Show that if $f:A\to X$ is a constant of pointed spaces, then $\eqv{M_f}{X\vee \susp(A)}$. 
\exercise Show that the cofiber of the diagonal $\delta:\sphere{1}\to \sphere{1}\times\sphere{1}$ is equivalent to $\sphere{2}\vee\sphere{2}$.
\exercise Show that $\eqv{\mathsf{Fin}(n+1)\wedge \mathsf{Fin}(m+1)}{\mathsf{Fin}(n\cdot m)+\unit}$.
\end{exercises}


% !TEX root = hott_intro.tex

\chapter{The Blakers-Massey theorem}
The Blakers-Massey theorem is a connectivity theorem which can be used to prove the Freudenthal suspension theorem, giving rise to the field of \emph{stable homotopy theory}. It was proven in the setting of homotopy type theory by Lumsdaine et al, and their proof was the first that was given entirely in an elementary way, using only constructions that are invariant under homotopy equivalence. 

Consider a span $A \leftarrow S \rightarrow B$, consisting of an $m$-connected map $f:S\to A$ and an $n$-connected map $g:S\to B$. We take the pushout of this span, and subsequently the pullback of the resulting cospan, as indicated in the diagram
\begin{equation}\label{eq:BM}
\begin{tikzcd}
S \arrow[drr,bend left=15,"g"] \arrow[ddr,bend right=15,swap,"f"] \arrow[dr,densely dotted,"u" near end] \\
& A \times_{(A \sqcup^S B)} B \arrow[r,"\pi_2"] \arrow[d,"\pi_1"] & B \arrow[d,"\inr"] \\
& A \arrow[r,swap,"\inl"] & A \sqcup^S B.
\end{tikzcd}
\end{equation}
The universal property of the pullback determines a unique map $u:S\to A \times_{(A\sqcup^S B)} B$ as indicated.

\begin{thm}[Blakers-Massey]
The map $u:S\to A \times_{(A\sqcup^S B)} B$ of \autoref{eq:BM} is $(n+m)$-connected.
\end{thm}

\begin{exercises}
\item Show that if $X$ is $m$-connected and $f:X\to Y$ is $n$-connected, then the map
\begin{equation*}
X \to \fib{m_f}{\ast}
\end{equation*}
where $m_f:Y\to M_f$ is the inclusion of $Y$ into the cofiber of $f$, is $(m+n)$-connected.
\item Suppose that $X$ is a connected type, and let $f:X\to Y$ be a map.
Show that the following are equivalent:
\begin{enumerate}
\item $f$ is $n$-connected.
\item The mapping cone of $f$ is $(n+1)$-connected.
\end{enumerate}
\item Apply the Blakers-Massey theorem to the defining pushout square of the smash product to show that if $A$ and $B$ are $m$- and $n$-connected respectively, then there is a $(m+n+\min(m,n)+2)$-connected map
\begin{equation*}
\join{\loopspace{A}}{\loopspace{B}}\to \loopspace{A \wedge B}.
\end{equation*}
\item Show that the square
\begin{equation*}
\begin{tikzcd}
\unit \arrow[r] \arrow[d] & \bool \arrow[d] \\
X \arrow[r] & X+\unit
\end{tikzcd}
\end{equation*}
is both a pullback and a pushout. Conclude that the result of the Blakers-Massey theorem is not always sharp.
\end{exercises}


%\chapter{Open problems in homotopy type theory}
We list only the open problems of synthetic homotopy theory, omitting the open problems related to the semantics of homotopy type theory.
\begin{itemize}
\item Compute the homotopy groups of the spheres.
\item Define the Grassmannians.
\item Prove Freyd's generating hypothesis
\item Hurewicz theorem
\item Barratt-Priddy theorem
\item (Semi-)simplicial types.
\item Ring structure on the sphere spectrum
\item Higher Van Kampen theorems
\item Higher Blakers-Massey theorems
\item Find a delooping of $\sphere{3}$, and define the octonionic Hopf fibration.
\end{itemize}


\backmatter

\printbibliography

\printindex

\end{document}
