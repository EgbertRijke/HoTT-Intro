\chapter{Function extensionality}

\section{Equivalent forms of function extensionality}
\begin{thm}\label{thm:funext_wkfunext}
The following are equivalent:
\begin{enumerate}
\item The principle of \define{homotopy induction}: for every $f:\prd{x:A}B(x)$ and every type family
\begin{equation*}
P : \prd{g:\prd{x:A}B(x)}(f\htpy g)\to\mathsf{Type},
\end{equation*}
the map
\begin{equation*}
\Big(\prd{g:\prd{x:A}B(x)}{H:f\htpy g}P(g,H)\Big)\to P(f,\mathsf{htpy.refl}_f)
\end{equation*}
given by $s\mapsto s(f,\mathsf{htpy.refl}_f)$ has a section.
\item The \define{function extensionality principle}: For every type family $B:A\to\UU$, and any two dependent functions $f,g:\prd{x:A}B(x)$, the canonical map
\begin{equation*}
\mathsf{htpy\usc{}eq}(f,g) : (\id{f}{g})\to (f\htpy g)
\end{equation*}
by path induction (sending $\refl{f}$ to $\lam{x}\refl{f(x)}$) is an equivalence.
\item The \define{weak function extensionality principle} holds: For every type family $B:A\to\UU$ one has
\begin{equation*}
\Big(\prd{a:A}\iscontr(B(a))\Big)\to\iscontr\Big(\prd{a:A}B(a)\Big).
\end{equation*}
\end{enumerate}
\end{thm}

\begin{proof}
Suppose that each $B(a)$ is contractible with center of contraction $c(a)$ and contraction $C_a:\prd{y:B(a)}c(a)=y$. Then we take $c\defeq \lam{a}c(a)$ to be the center of contraction of $\prd{a:A}B(a)$. To construct the contraction we have to define a term of type
\begin{equation*}
\prd{f:\prd{a:A}B(a)}c=f.
\end{equation*}
Let $f:\prd{a:A}B(a)$. By function extensionality we have a map $(c\htpy f)\to (c=f)$, so it suffices to construct a term of type $c\htpy f$. Here we take $\lam{a}C_a(f(a))$. This completes the proof that function extensionality implies weak function extensionality.

To prove function extensionality from weak function extensionality, observe that it suffices by \autoref{thm:id_fundamental} to show that
\begin{equation*}
\sm{g:\prd{x:A}B(x)}f\htpy g
\end{equation*}
is contractible.

Since the type $\sm{b:B(x)}f(x)=b$ is contractible for each $x:X$, it follows by our assumption of weak function extensionality that the type $\prd{x:A}\sm{b:B(x)}f(x)=b$ is contractible. By \autoref{ex:contr_retr} it therefore suffices to show that
\begin{equation*}
\sm{g:\prd{x:A}B(x)}f\htpy g
\end{equation*}
is a retract of the type $\prd{x:A}\sm{b:B(x)}f(x)=b$. We have the functions
\begin{align*}
\mathsf{pi\usc{}sigma} & \defeq \lam{\pairr{g,H}}\lam{x}\pairr{g(x),H(x)} \\
\mathsf{sigma\usc{}pi} & \defeq \lam{p}\pairr{\lam{x}\proj 1(p(x)),\lam{x}\proj 2(p(x))}.
\end{align*}
It remains to show that $\psi\circ\varphi=\idfunc$. Let $\pairr{g,H}:\sum_g f\htpy g$. 
Then we have
\begin{align*}
\mathsf{sigma\usc{}pi}(\mathsf{pi\usc{}sigma}(g,H)) & \jdeq \mathsf{sigma\usc{}pi}(\lam{x}\pairr{g(x),H(x)}) \\
& \jdeq \pairr{\lam{x}g(x),\lam{x}H(x)} \\
& \jdeq \pairr{g,H}.\qedhere
\end{align*}
\end{proof}

\begin{comment}
\begin{rmk}
Since we assumed the $\eta$-rule for $\Sigma$-types, we also have
\begin{align*}
\mathsf{pi\usc{}sigma}(\mathsf{sigma\usc{}pi}(p)) & \jdeq \mathsf{pi\usc{}sigma}(\pairr{\lam{x}\proj 1(p(x)),\lam{x}\proj 2(p(x))}) \\
& \jdeq \lam{x}\pairr{\proj 1(p(x)),\proj 2(p(x))} \\
& \jdeq \lam{x} p(x) \\
& \jdeq p.
\end{align*}
Therefore, the types $\sum_g f\htpy g$ and $\prod_x\sum_b f(x)=b$ are actually \emph{judgmentally isomorphic}. 
\end{rmk}
\end{comment}

\begin{thm}
Assume function extensionality. Then for any type family $B:A\to\UU$ one has
\begin{equation*}
\Big(\prd{a:A}\istrunc{n}(B(a))\Big)\to \istrunc{n}\Big(\prd{a:A}B(a)\Big).
\end{equation*}
\end{thm}

\begin{proof}
The theorem is proven by induction on $n\geq -2$. The base case is just the weak function extensionality principle, which was shown to follow from function extensionality in \autoref{thm:funext_wkfunext}.

For the inductive hypothesis, assume that the $n$-types are closed under dependent function types. Assume that $B$ is a family of $(n+1)$-types. By function extensionality, the type $f=g$ is equivalent to $f\htpy g$ for any two dependent functions $f,g:\prd{a:A}B(a)$. Now observe that $f\htpy g$ is a dependent product of $n$-types, and therefore it is an $n$-type by our inductive hypotheses. Therefore, it follows by \autoref{thm:ntype_eqv} that $f=g$ is an $n$-type, and hence that $\prd{a:A}B(a)$ is an $(n+1)$-type.
\end{proof}

\section{Universal properties}
The function extensionality principle allows us to prove \emph{universal properties}: characterizations of all maps out of (or into) a given type. In the following theorem we prove the universal property of dependent pair types.

\begin{thm}
Let $B$ be a type family over $A$, and let $X$ be a type. Then the map
\begin{equation*}
\mathsf{ev\usc{}pair}:\Big(\Big(\sm{a:A}B(a)\Big)\to X\Big)\to \Big(\prd{a:A}(B(a)\to X)\Big)
\end{equation*}
given by $f\mapsto\lam{a}{b}f(\mathsf{pair}(a,b))$ is an equivalence.
\end{thm}

This theorem justifies that we usually write $f(a,b)$ rather than $f(\mathsf{pair}(a,b))$, since $f:(\sm{a:A}B(a))\to X$ is uniquely determined by its action on terms of the form $\mathsf{pair}(a,b)$.

\begin{proof}
We define $\varphi$ in the converse direction by dependent pair induction.
Given $f:\prd{a:A}(B(a)\to X)$ we obtain $\tilde{f}:\big(\sm{a:A}B(a)\big)\to X$ satisfying $\tilde{f}(\mathsf{pair}(a,b))\jdeq f(a,b)$. Thus it is immediate by an application of function extensionality that $\mathsf{ev\usc{}pair}\circ\varphi\htpy\idfunc$. 

To show that $\varphi\circ\mathsf{ev\usc{}pair}\htpy\idfunc$ we will also apply function extensionality. Thus, it suffices to show that $\varphi(\lam{a}{b}f(\mathsf{pair}(a,b)))=f$. We apply function extensionality again, so it suffices to show that
\begin{equation*}
\prd{x:\sm{a:A}B(a)}\varphi(\lam{a}{b}f(\mathsf{pair}(a,b)))(x)=f(x).
\end{equation*}
This follows immediately by dependent pair induction. 
\end{proof}

\begin{cor}
Let $A$, $B$, and $X$ be types. Then the map
\begin{equation*}
\mathsf{ev\usc{}pair}: (A\times B \to X)\to (A\to (B\to X))
\end{equation*}
given by $f\mapsto\lam{a}{b}f(\mathsf{pair}(a,b))$ is an equivalence.
\end{cor}

The universal property of identity types is sometimes called the \emph{type theoretical Yoneda lemma}: families of maps out of the identity type are uniquely determined by their action on the reflexivity identification.

\begin{thm}\label{thm:yoneda}
Let $B$ be a type family over $A$, and let $a:A$. Then the map
\begin{equation*}
\mathsf{ev\usc{}refl}:\Big(\prd{x:A} (a=x)\to B(x)\Big)\to B(a)
\end{equation*}
given by $\lam{f} f(a,\refl{a})$ is an equivalence. 
\end{thm}

\begin{proof}
The inverse $\varphi$ is defined by path induction, taking $b:B(a)$ to the function $f$ satisfying $f(a,\refl{a})\jdeq b$. It is immediate that $\mathsf{ev\usc{}refl}\circ\varphi\htpy \idfunc$.

To see that $\varphi\circ \mathsf{ev\usc{}refl}\htpy\idfunc$, let $f:\prd{x:A}(a=x)\to B(x)$. To show that $\varphi(f(a,\refl{a}))=f$ we use function extensionality (twice), so it suffices to show that
\begin{equation*}
\prd{x:A}{p:a=x} \varphi(f(a,\refl{a}),x,p)=f(x,p).
\end{equation*}
This follows by path induction on $p$, since $\varphi(f(a,\refl{a}),a,\refl{a})\jdeq f(a,\refl{a})$.
\end{proof}



\begin{exercises}
\item Show that for any type $A$ and any $n\geq-2$, the type $\istrunc{n}(A)$ is a mere proposition.
\item Let $f:A\to B$ be a function.
\begin{subexenum}
\item Show that if $f$ is an equivalence, then the type $\sm{g:B\to A}f\circ g\htpy \idfunc$ of sections of $f$ is contractible.
\item Show that if $f$ is an equivalence, then the type $\sm{h:B\to A}h\circ f\htpy \idfunc$ of retractions of $f$ is contractible.
\item Show that $\isequiv(f)$ is a mere proposition.
\item Conclude that $\eqv{A}{B}$ is a subtype of $A\to B$, and in particular that for any $(f,p),(g,q):\eqv{A}{B}$ the map
\begin{equation*}
\mathsf{ap}_{\proj 1} : ((f,p)= (g,q))\to (f=g)
\end{equation*}
is an equivalence.
\end{subexenum}
\item Let $C$ be a contractible type with center of contraction $c$. Use the function extensionality principle to show that the map $\lam{f}f(c):(C\to A)\to A$ is an equivalence, for each type $A$.
\item Use the function extensionality principle to show that for any $B:A\to\UU$ and $R:\prd{a:A}B(a)\to \UU$, the map
\begin{equation*}
\Big(\sm{f:\prd{a:A}B(a)}\prd{a:A}R(a,f(a))\Big)\to \Big(\prd{a:A}\sm{b:B(a)}R(a,b)\Big)
\end{equation*}
given by $\lam{(f,g)}{a}(f(a),g(a))$ is an equivalence. This is sometimes called the \define{type theoretic principle of choice}.

Conclude that for any $A:\UU$ and any $C:B\to \UU$, the map
\begin{equation*}
\Big(\sm{f:A\to B} \prd{a:A}C(f(a))\Big)\to\Big(A\to\sm{b:B}C(b)\Big)
\end{equation*}
given by $\lam{(f,g)}{a}(f(a),g(a))$ is an equivalence.
\item Let $f:A\to B$ be a map. Show that the following are equivalent:
\begin{enumerate}
\item $f$ is an equivalence.
\item For any type family $P:B\to\UU$ the map
\begin{equation*}
\Big(\prd{y:B}P(y)\Big)\to\Big(\prd{x:A}P(f(x))\Big)
\end{equation*}
given by $s\mapsto s\circ f$ is an equivalence.
\item For any type $X$ the map
\begin{equation*}
(B\to X)\to (A\to X)
\end{equation*}
given by $g\mapsto g\circ f$ is an equivalence. 
\end{enumerate}
\item Let $f:A\to B$ be a map. Use \autoref{ex:coh_intro} and \autoref{defn:inv_equiv} to show that
\begin{equation*}
\eqv{\isequiv(f)}{\mathsf{half\usc{}adj}(f)},
\end{equation*}
where $\mathsf{half\usc{}adj}(f)$ is the type consisting of functions $g:B\to A$ equipped with
\begin{align*}
H_r & : f\circ g\htpy \idfunc \\
H_l & : g\circ f\htpy \idfunc \\
H_{coh} & : g\cdot H_r\htpy H_l\cdot g,
\end{align*}
witnessing that $f$ is a \define{half-adjoint equivalence}.
\item Show that the map
\begin{equation*}
(\emptyt \to X)\to \unit
\end{equation*}
given by $\lam{f}\ttt$ is an equivalence.
\item Show that the map
\begin{equation*}
(A+B\to X)\to (A\to X)\times (B\to X)
\end{equation*}
given by $f\mapsto (f\circ\inl,f\circ\inr)$ is an equivalence.
\item Show that the map
\begin{equation*}
(\unit\to X)\to X
\end{equation*}
given by $\lam{f}f(\ttt)$ is an equivalence. 
\end{exercises}
