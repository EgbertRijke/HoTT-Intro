% arara: makechapters: {items: [dtt, inductive, identity, equivalences, fundamental, hierarchy, univalence]}

\documentclass[11pt]{memoir} %[ebook,10pt,oneside]

\title{Introduction to homotopy type theory}
\author{Egbert Rijke}
\date{Carnegie Mellon University\\Pittsburgh PA\\Spring 2018}
%\address{Carnegie Mellon University}
%\email{erijke@andrew.cmu.edu}

\pretitle{\begin{center}\textsc\bgroup\LARGE}
\posttitle{\egroup\end{center}\vspace{2cm}}
\preauthor{\begin{center}\textsc\bgroup\Large}\postauthor{\egroup\end{center}\vfill}
\predate{\begin{center}\textsc\bgroup}{\postdate{\egroup\end{center}}

\usepackage{hott}

\makeatletter
\renewcommand{\@chapapp}{Lecture}
\makeatother

\newlist{exenum}{enumerate}{1}
\setlist[exenum]{noitemsep,label=\thechapter.\arabic*}%,ref=\thechapter.\arabic*}
%\crefalias{exenumi}{Exercise} 
\crefname{exenumi}{Exercise}{Exercises}

\newlist{subexenum}{enumerate}{1}
\setlist[subexenum]{noitemsep,label=(\alph*),ref=\theexenumi.\alph*}
\crefname{subexenumi}{Exercise}{Exercises}

\newenvironment{exercises}%
{%
\section*{Exercises}\addcontentsline{toc}{section}{Exercises}\sectionmark{Exercises}
\begin{exenum}}
{%
\end{exenum}}

\begin{document}

\begin{titlingpage}
\maketitle 
\end{titlingpage}

\frontmatter
\tableofcontents

\chapter{Introduction}
%\addcontentsline{toc}{chapter}{Introduction}
These are course notes for an introduction to homotopy type theory taught at Carnegie Mellon University in the spring semester of 2018.

\chapter{Syllabus}

\mainmatter 

\chapter{Dependent type theory}
\label{ch:dtt}

In this lecture we describe the deductive system of dependent type theory, without introducing yet any ways of actually forming types. In other words, we just give the rules of type dependency.

\section{The primitive judgments of type theory}

The theory of type dependency is formulated as a deductive system in which derivations establish that a given construction is well-formed. In any dependent type theory there are four \define{primitive judgments}\index{primitive judgment}:
\begin{enumerate}
%\item `\emph{$\Gamma$ is a well-formed context.}'
\item `\emph{$A$ is a (well-formed) \define{type}\index{well-formed type}\index{type} in context $\Gamma$.}'\index{primitive judgment!type in context}
\item `\emph{$A$ and $B$ are \define{judgmentally equal types} in context $\Gamma$.}'\index{primitive judgment!equal types in context}\index{judgmental equality!of types}
\item `\emph{$a$ is a (well-formed) \define{term}\index{well-formed term}\index{term} of type $A$ in context $\Gamma$.}'\index{primitive judgment!term of a type in context}
\item `\emph{$a$ and $b$ are \define{judgmentally equal terms} of type $A$ in context $\Gamma$.}'\index{primitive judgment!equal terms of a type in context}\index{judgmental equality!of terms}
\end{enumerate}
\begin{samepage}
The symbolic expressions for these four primitive judgments are as follows:
\begin{align*}
%& \vdash \Gamma~\mathrm{ctx} & & \vdash \Gamma\jdeq \Gamma'~\mathrm{ctx} \\
\Gamma & \vdash A~\textrm{type} & \Gamma & \vdash A\jdeq B~\textrm{type}\\
\Gamma & \vdash a:A & \Gamma & \vdash a\jdeq b:A.
\end{align*}
\end{samepage}
In these judgments, well-formedness of a type $A$ in context $\Gamma$ just means that the type $A$ has been formed in accordance with the rules of type theory, which we are about to describe.

A well-formed \define{context}\index{context|textbf} is an expression of the form
\begin{equation*}
x_1:A_1,~x_2:A_2(x_1),~\ldots,~x_n:A_n(x_1,\ldots,x_{n-1}),
\end{equation*}
which we often simply write as $x_1:A_1,~x_2:A_2,~\ldots,~x_n:A_n$,
satisfying the condition that for each $1\leq k\leq n$ we have that $A_k$ is a well-formed type in context $x_1:A_1,x_2:A_2,\ldots,x_{k-1}:A_{k-1}$, i.e.
\begin{equation*}
x_1:A_1,x_2:A_2,\ldots,x_{k-1}:A_{k-1} \vdash A_k~\textrm{type}.
\end{equation*}
We say that a context $x_1:A_1,~\ldots,~x_n:A_n$ \define{declares the variables}\index{variable declaration} $x_1,\ldots,x_n$. In other words, a type $A(x_1,\ldots,x_n)$ in context $\Gamma$ is well-formed if all its variables $x_1,\ldots,x_n$ are assigned well-formed types in the context $\Gamma$.

We may use variable names other than $x_1,\ldots,x_n$, as long as \emph{no variable is declared more than once.} For example, we will often use the variable names $x$, $y$, and $z$ when they are assigned a general type, variables $f$, $g$, and $h$ for function types, and so on.
%For example we used the variable names $A,\mu,u_l,p,u_r,q$ when we displayed the context of \autoref{lem:unit}.

In the special case where $n=0$, the list $x_1:A_1,x_2:A_2,\ldots,x_n:A_n$ is empty, which satisfies the well-formedness condition vacuously. In other words, the \define{empty context}\index{context!empty context|textbf}\index{empty context|textbf} is well-formed. A well-formed type in the empty context is also called a \define{closed type}\index{closed type|textbf}, and a well-formed term of a closed type is called a \define{closed term}\index{closed term|textbf}.

When $B$ is a type in context $\Gamma,x:A$, we also say that $B$ is a \define{family of types}\index{family!of types|textbf} over $A$ (in context $\Gamma$).

\section{Renaming variables}
In some situations one might want to change the name of a variable in a context. This is allowed, provided that the new variable does not occur anywhere else in the context, so that also after renaming no variable is declared more than once. 
The inference rules that rename a variable are sometimes called \define{$\alpha$-conversion}\index{alpha-conversion@{$\alpha$-conversion}}\index{conversion rules!alpha-conversion@{$\alpha$-conversion}}\index{rule!alpha conversion@{$\alpha$-conversion}} rules. 

If we are given a type $A$ in context $\Gamma$, then for any type $B$ in context $\Gamma,x:A,\Delta$ we can form the type $B[x'/x]$ in context $\Gamma,x':A,\Delta[x'/x]$, where $B[x'/x]$ is an abbreviation for
\begin{equation*}
B(x_1,\ldots,x_{n-1},x',x_{n+1},\ldots,x_{n+m-1})
\end{equation*}
This definition of \define{renaming}\index{variable renaming|textbf} the variable $x$ by $x'$ is understood to be recursive over the length of $\Delta$. The first variable renaming rule\index{rule!variable renaming} postulates that the renaming of a variable preserves well-formedness of types:
\begin{prooftree}
\AxiomC{$\Gamma,x:A,\Delta\vdash B~\mathrm{type}$}
\RightLabel{$x'/x$}
\UnaryInfC{$\Gamma,x':A,\Delta[x'/x]\vdash B[x'/x]~\mathrm{type}$}
\end{prooftree}

Similarly we obtain for any term $b:B$ in context $\Gamma,x:A,\Delta$ a term $b[x'/x]:B[x'/x]$, and there is a variable renaming rule postulating that the renaming of a variable preserves the well-formedness of terms.
In fact, there is variable renaming rule for each of the primitive judgments. To avoid having to state essentially the same rule four times in a row, we postulate the four variable renaming rules all at once using a \emph{generic judgment}\index{generic judgment} $\mathcal{J}$. 
\begin{prooftree}
\AxiomC{$\Gamma,x:A,\Delta\vdash \mathcal{J}$}
\RightLabel{$x'/x$}
\UnaryInfC{$\Gamma,x':A,\Delta[x'/x]\vdash \mathcal{J}[x'/x]$}
\end{prooftree}
where $\mathcal{J}$ may be a typing judgment, a judgment of equality of types, a term judgment, or a judgment of equality of terms.
We will use generic judgments extensively to postulate the rest of the rules of dependent type theory.

\section{Inference rules governing judgmental equality}

\begin{samepage}
Both on types and on terms, we postulate that judgmental equality is an equivalence relation. That is, we provide inference rules for the reflexivity, symmetry and transitivity of both kinds of judgmental equality\index{judgmental equality!equivalence relation}:
\begin{center}
\begin{small}
\begin{minipage}{.2\textwidth}
\begin{prooftree}
\AxiomC{$\Gamma\vdash A~\textrm{type}$}
\UnaryInfC{$\Gamma\vdash A\jdeq A~\textrm{type}$}
\end{prooftree}
\end{minipage}
\begin{minipage}{.25\textwidth}
\begin{prooftree}
\AxiomC{$\Gamma\vdash A\jdeq A'~\textrm{type}$}
\UnaryInfC{$\Gamma\vdash A'\jdeq A~\textrm{type}$}
\end{prooftree}
\end{minipage}
\begin{minipage}{.5\textwidth}
\begin{prooftree}
\AxiomC{$\Gamma\vdash A\jdeq A'~\textrm{type}$}
\AxiomC{$\Gamma\vdash A'\jdeq A''~\textrm{type}$}
\BinaryInfC{$\Gamma\vdash A\jdeq A''~\textrm{type}$}
\end{prooftree}
\end{minipage}
\\
\bigskip
\begin{minipage}{.2\textwidth}
\begin{prooftree}
\AxiomC{$\Gamma\vdash a:A$}
\UnaryInfC{$\Gamma\vdash a\jdeq a : A$}
\end{prooftree}
\end{minipage}
\begin{minipage}{.25\textwidth}
\begin{prooftree}
\AxiomC{$\Gamma\vdash a\jdeq a':A$}
\UnaryInfC{$\Gamma\vdash a'\jdeq a: A$}
\end{prooftree}
\end{minipage}
\begin{minipage}{.5\textwidth}
\begin{prooftree}
\AxiomC{$\Gamma\vdash a\jdeq a' : A$}
\AxiomC{$\Gamma\vdash a'\jdeq a'': A$}
\BinaryInfC{$\Gamma\vdash a\jdeq a'': A$}
\end{prooftree}
\end{minipage}
\end{small}
\end{center}
\end{samepage}

Apart from the rules postulating that judgmental equality is an equivalence relation, there are also \define{variable conversion rules}\index{judgmental equality!conversion rules}\index{variable conversion rules}\index{conversion rule!variable}\index{rule!variable conversion}.
Informally, these are rules stating that if $A$ and $A'$ are judgmentally equal types in context $\Gamma$, then any valid judgment in context $\Gamma,x:A$ is also a valid judgment in context $\Gamma,x:A'$. In other words: we can convert the type of a variable to a judgmentally equal type. We state this with a generic judgment $\mathcal{J}$
\begin{prooftree}
\AxiomC{$\Gamma\vdash A\jdeq A'~\textrm{type}$}
\AxiomC{$\Gamma,x:A,\Delta\vdash \mathcal{J}$}
\RightLabel{$A'/A$}
\BinaryInfC{$\Gamma,x:A',\Delta\vdash \mathcal{J}$}
\end{prooftree}
An analogous \emph{term conversion rule}\index{term conversion rule}\index{conversion rule!term}\index{rule!term conversion}, stated in \cref{ex:term_conversion}, converting the type of a term to a judgmentally equal type, is derivable using the `structural rules' of type theory described in the next section.


\section{Structural rules of type theory}

We complete the specification of dependent type theory by postulating rules for \emph{weakening} and \emph{substitution}, and the \emph{variable rule}:
\begin{enumerate}
\item If we are given a type $A$ in context $\Gamma$, then any judgment made in a longer context $\Gamma,\Delta$ can also be made in the context $\Gamma,x:A,\Delta$, for a fresh variable $x$. The \define{weakening rule}\index{weakening}\index{rule!weakening} asserts that weakening by a type $A$ in context preserves well-formedness and judgmental equality of types and terms.
\begin{prooftree}
\AxiomC{$\Gamma\vdash A~\textrm{type}$}
\AxiomC{$\Gamma,\Delta\vdash \mathcal{J}$}
\RightLabel{$W_A$}
\BinaryInfC{$\Gamma,x:A,\Delta \vdash \mathcal{J}$}
\end{prooftree}
This process of expanding the context by a fresh variable of type $A$ is called \define{weakening (by $A$)}.

For example, when we have two types $A$ and $B$ in context $\Gamma$, we can weaken $B$ by $A$ as follows
\begin{prooftree}
  \AxiomC{$\Gamma\vdash A~\textrm{type}$}
  \AxiomC{$\Gamma\vdash B~\textrm{type}$}
  \RightLabel{$W_A$}
  \BinaryInfC{$\Gamma,x:A\vdash B~\mathrm{type}$}
\end{prooftree}
in order to form the type $B$ in context $\Gamma,x:A$. The type $B$ in context $\Gamma,x:A$ is also called the \define{constant family}\index{family!constant family}\index{constant family} $B$, or the \define{trivial family}\index{family!trivial family}\index{trivial family} $B$.
\item If we are given a type $A$ in context $\Gamma$, then $x$ is a well-formed term of type $A$ in context $\Gamma,x:A$.
\begin{prooftree}
\AxiomC{$\Gamma\vdash A~\textrm{type}$}
\RightLabel{$\delta_A$}
\UnaryInfC{$\Gamma,x:A\vdash x:A$}
\end{prooftree}
This is called the \define{variable rule}\index{variable rule}\index{rule!variable rule|textbf}. It provides an \emph{identity function}\index{identity function} on the type $A$ in context $\Gamma$.
\item If we are given a term $a:A$ in context $\Gamma$, then for any type $B$ in context $\Gamma,x:A,\Delta$ we can form the type $B[a/x]$ in context $\Gamma,\Delta[a/x]$, where $B[a/x]$ is an abbreviation for
\begin{equation*}
B(x_1,\ldots,x_{n-1},a(x_1,\ldots,x_{n-1}),x_{n+1},\ldots,x_{n+m-1})
\end{equation*}
This definition of substituting $a$ for $x$ is understood to be recursive over the length of $\Delta$. Similarly we obtain for any term $b:B$ in context $\Gamma,x:A,\Delta$ a term $b[a/x]:B[a/x]$. The \define{substitution rule}\index{substitution}\index{rule!substitution} asserts that substitution preserves well-formedness and judgmental equality of types and terms:
\begin{prooftree}
\AxiomC{$\Gamma\vdash a:A$}
\AxiomC{$\Gamma,x:A,\Delta\vdash \mathcal{J}$}
\RightLabel{$S_a$}
\BinaryInfC{$\Gamma,\Delta[a/x]\vdash \mathcal{J}[a/x]$}
\end{prooftree}
Furthermore, we postulate that substitution by judgmentally equal terms results in judgmentally equal types
\begin{prooftree}
\AxiomC{$\Gamma\vdash a\jdeq a':A$}
\AxiomC{$\Gamma,x:A,\Delta\vdash B~\mathrm{type}$}
\BinaryInfC{$\Gamma,\Delta[a/x]\vdash B[a/x]\jdeq B[a'/x]~\mathrm{type}$}
\end{prooftree}
and it also results in judgmentally equal terms
\begin{prooftree}
\AxiomC{$\Gamma\vdash a\jdeq a':A$}
\AxiomC{$\Gamma,x:A,\Delta\vdash b:B$}
\BinaryInfC{$\Gamma,\Delta[a/x]\vdash b[a/x]\jdeq b[a'/x]:B[a/x]$}
\end{prooftree}
When $B$ is a family of types over $A$ and $a:A$, we also say that $B[a/x]$ is the \define{fiber}\index{family!fiber of}\index{fiber!of a family} of $B$ at $a$. Often we write $B(a)$ for $B[a/x]$.
\end{enumerate}




\begin{comment}
\bigskip
\begin{minipage}{.45\textwidth}
\begin{prooftree}
\AxiomC{$\Gamma\vdash A~\textrm{type}$}
\AxiomC{$\Gamma,\Delta\vdash B~\textrm{type}$}
\RightLabel{$W_A$}
\BinaryInfC{$\Gamma,x:A,\Delta \vdash B~\textrm{type}$}
\end{prooftree}
\end{minipage}\hfill
\begin{minipage}{.45\textwidth}
\begin{prooftree}
\AxiomC{$\Gamma\vdash A~\textrm{type}$}
\AxiomC{$\Gamma,\Delta\vdash b:B$}
\RightLabel{$W_A$}
\BinaryInfC{$\Gamma,x:A,\Delta \vdash b:B$}
\end{prooftree}
\end{minipage}

\noindent
\begin{prooftree}
\AxiomC{$\Gamma\vdash A~\textrm{type}$}
\RightLabel{$\delta_A$}
\UnaryInfC{$\Gamma,x:A\vdash x:A$}
\end{prooftree}

\noindent
\begin{minipage}{.5\textwidth}
\begin{prooftree}
\AxiomC{$\Gamma\vdash a:A$}
\AxiomC{$\Gamma,x:A,\Delta\vdash B~\textrm{type}$}
\RightLabel{$S_a$}
\BinaryInfC{$\Gamma,\Delta[a/x]\vdash B[a/x]~\textrm{type}$}
\end{prooftree}
\end{minipage}\hfill
\begin{minipage}{.5\textwidth}
\begin{prooftree}
\AxiomC{$\Gamma\vdash a:A$}
\AxiomC{$\Gamma,x:A,\Delta\vdash b:B$}
\RightLabel{$S_a$}
\BinaryInfC{$\Gamma,\Delta[a/x]\vdash b[a/x] : B[a/x]$}
\end{prooftree}
\end{minipage}

\bigskip
\end{comment}


\begin{eg}
To give an example of how the deductive system works, we give a deduction for the \define{interchange rule}\index{rule!interchange}\index{interchange rule}
\begin{prooftree}
\AxiomC{$\Gamma\vdash B~\textrm{type}$}
\AxiomC{$\Gamma,x:A,y:B,\Delta\vdash \mathcal{J}$}
\BinaryInfC{$\Gamma,y:B,x:A,\Delta\vdash \mathcal{J}$}
\end{prooftree}
In other words, if we have two types $A$ and $B$ in context $\Gamma$, and we make a judgment in context $\Gamma,x:A,y:B$, then we can make that same judgment in context $\Gamma,y:B,x:A$.
The derivation is as follows:
\begin{small}
\begin{prooftree}
\AxiomC{$\Gamma\vdash B~\textrm{type}$}
\RightLabel{$\delta_B$}
\UnaryInfC{$\Gamma,y:B\vdash y:B$}
\RightLabel{$W_{W_B(A)}$}
\UnaryInfC{$\Gamma,y:B,x:A\vdash y:B$}
\AxiomC{$\Gamma,x:A,y:B,\Delta\vdash \mathcal{J}$}
\RightLabel{$y'/y$}
\UnaryInfC{$\Gamma,x:A,y':B,\Delta[y'/y]\vdash \mathcal{J}[y'/y]$}
\RightLabel{$W_B$}
\UnaryInfC{$\Gamma,y:B,x:A,y':B,\Delta[y'/y]\vdash \mathcal{J}[y'/y]$}
\RightLabel{$S_{W_A(y)}$}
\BinaryInfC{$\Gamma,y:B,x:A,\Delta\vdash \mathcal{J}$}
\end{prooftree}
\end{small}
\end{eg}


\begin{comment}
For $A\in T_n$ we define $T_{n+k+1}(A):= \{B\in T_{n+k}\mid \mathrm{ft}^{k+1}(B)=A\}$. Similarly we define $\tilde{T}_{n+k+1}(A):=\{b\in\tilde{T}_{n+k+1}\mid\mathrm{ft}^{k+1}(\partial(b))=A\}$. For any closed type $A$ we maintain the convention that $T_{k}(\mathrm{ft}(A)):= T_k$.
\begin{enumerate}
\item For all $A\in T_n$, and all $k\in\N$, \define{weakening} operations
\begin{align*}
W_A & : T_{n+k}(\mathrm{ft}(A)) \to T_{n+k+1}(A) \\
\tilde{W}_A & : \tilde{T}_{n+k}(\mathrm{ft}(A))\to \tilde{T}_{n+k+1}(A)
\end{align*}
subject to the conditions $\mathrm{ft}(W_A(B))=W_A(\mathrm{ft}(B))$ if $B\in T_{n+k}$ with $k\geq 1$, and $\partial(\tilde{W}_A(b))=W_A(\partial(b))$.
\item For all $A\in T_n$ a term $\delta_A\in \tilde{T}_{n+1}$ satisfying $\partial(\delta_A)=W_A(A)$. 
\item For all $a\in \tilde{T}_n$ satisfying $\partial(a)=A$, and all $k\in\N$, \define{substitution} operations
\begin{align*}
S_a & : \{B\in T_{n+k+1}\mid \mathrm{ft}^{k+1}(B)= A\}\to T_k \\
\tilde{S}_a & : \{b\in \tilde{T}_{n+k+1}\mid \mathrm{ft}^{k+1}(\partial(b))=A\}\to \tilde{T}_{n+k}
\end{align*}
subject to the conditions $\mathrm{ft}(S_a(B))=\mathrm{ft}(A)$ if $B\in T_{n+1}$, $\mathrm{ft}(S_a(B))=S_a(\mathrm{ft}(B))$ if $B\in T_{n+k+1}$ with $k\geq 1$, and $\partial(\tilde{S}_a(b))=S_a(\partial(b))$.
\end{enumerate}
\end{comment}

%\section{Axioms for weakening, substitution, and the diagonal}
\begin{comment}
\begin{prooftree}
\AxiomC{$\Gamma\vdash A~\textrm{type}$}
  \AxiomC{$\Gamma,x:A,\Delta\vdash B~\textrm{type}$}
    \AxiomC{$\Gamma,x:A,\Delta,y:B,E\vdash C~\textrm{type}$}
\TrinaryInfC{$\Gamma,\Delta[a/x],E[b/y][a/x]\vdash C[b/y][a/x]\jdeq C[a/x][b[a/x]/y']~\textrm{type}$}
\end{prooftree}
\begin{prooftree}
\AxiomC{$\Gamma\vdash A~\textrm{type}$}
  \AxiomC{$\Gamma,x:A,\Delta\vdash B~\textrm{type}$}
    \AxiomC{$\Gamma,x:A,\Delta,y:B,E\vdash c:C$}
\TrinaryInfC{$\Gamma,\Delta[a/x],E[b/y][a/x]\vdash c[b/y][a/x]\jdeq c[a/x][b[a/x]/y']:C[b/y][a/x]$}
\end{prooftree}
\begin{prooftree}
\AxiomC{$\Gamma\vdash a:A$}
  \AxiomC{$\Gamma,\Delta\vdash B~\textrm{type}$}
\RightLabel{($x$ not free in $B$)}
\BinaryInfC{$\Gamma,\Delta\vdash B[a/x]\jdeq B~\textrm{type}$}
\end{prooftree}
\end{comment}


\begin{comment}
\section{An algebraic presentation of dependent type theory}

%Let us write $T_n$ for the set of well-formed contexts of length $n$, for $n>1$. Then any well-formed context of length $n+1$ is of the form $\Gamma,x:A$, where $\Gamma$ is a well-formed context of length $n$. Thus we see that there are maps $\eft:T_{n+1}\to T_n$ for $n>1$. Similarly, if we write $\tilde{T}_n$ for the set of all terms of a type in a context of length $n-1$, we see that there is a map $\tilde{T}_n\to T_n$. The following picture emerges:
%\begin{equation*}
%\begin{tikzcd}
%\tilde{T}_3 \arrow[dr,"\partial"] & \vdots \arrow[d,"\mathrm{ft}"] \\
%\tilde{T}_2 \arrow[dr,"\partial"] & T_3 \arrow[d,"\mathrm{ft}"] \\
%\tilde{T}_1 \arrow[dr,"\partial"] & T_2 \arrow[d,"\mathrm{ft}"] \\
%& T_1
%\end{tikzcd}
%\end{equation*}

Observe that given a type $A$ in context $\Gamma$ and a type $B$ in context $\Gamma,\Delta$ we can weaken twice by first weakening by $B$ and then by $A$, as indicated in the following derivation:
\begin{prooftree}
\AxiomC{$\Gamma\vdash A~\textrm{type}$}
\AxiomC{$\Gamma,\Delta\vdash B~\textrm{type}$}
  \AxiomC{$\Gamma,\Delta,\mathrm{E}\vdash \mathcal{J}$}
\BinaryInfC{$\Gamma,\Delta,y:B,\mathrm{E}\vdash \mathcal{J}$}
\BinaryInfC{$\Gamma,x:A,\Delta,y:B,\mathrm{E}\vdash \mathcal{J}$}
\end{prooftree}
However, we can also first weaken by $A$, and then by `$B$ weakened by $A$', as indicated in the following derivation:
\begin{prooftree}
\AxiomC{$\Gamma\vdash A~\textrm{type}$}
  \AxiomC{$\Gamma,\Delta\vdash B~\textrm{type}$}
\BinaryInfC{$\Gamma,x:A,\Delta\vdash B~\textrm{type}$}
  \AxiomC{$\Gamma\vdash A~\textrm{type}$}
    \AxiomC{$\Gamma,\Delta,\mathrm{E}\vdash \mathcal{J}$}
  \BinaryInfC{$\Gamma,x:A,\Delta,\mathrm{E}\vdash \mathcal{J}$}
\BinaryInfC{$\Gamma,x:A,\Delta,y:B,\mathrm{E}\vdash \mathcal{J}$}
\end{prooftree}
For the end result it does not matter in what order the two weakenings are performed. We can express this conveniently as an equation:
\begin{equation*}
W_A(W_B(\mathcal{J}))\jdeq W_{W_A(B)}(W_A(\mathcal{J})).
\end{equation*}
The complete list of rules (in alphabetic order) is
\begin{align*}
S_a(\delta_B) & \jdeq \delta_{S_a(B)} \\
S_a(\delta_A) & \jdeq a \\
S_a(S_b(\mathcal{J})) & \jdeq S_{S_a(b)}(S_a(\mathcal{J})) \\
S_a(W_A(\mathcal{J})) & \jdeq \mathcal{J} \\
S_a(W_B(\mathcal{J})) & \jdeq W_{S_a(B)}(S_a(\mathcal{J})) \\
S_{\delta_A}(W_{W_A}(\mathcal{J})) & \jdeq \mathcal{J} \\
W_A(\delta_B) & \jdeq \delta_{W_A(B)} \\
W_A(S_b(\mathcal{J})) & \jdeq S_{W_A(b)}(W_A(\mathcal{J})) \\
W_A(W_B(\mathcal{J})) & \jdeq W_{W_A(B)}(W_A(\mathcal{J}))
\end{align*}
Here $A$ is a type in context $\Gamma$ and $a$ is a term of type $A$, $B$ is a type in context $\Gamma,x:A,\Delta$ and $b$ is a term of type $B$.
\end{comment}

%\begin{rmk}
%To avoid overly long proof trees maintain as a convention that every derivation with hypotheses $\mathcal{H}_1,\ldots,\mathcal{H}_n$ and conclusion $\mathcal{C}$ can be used as a rule
%\begin{prooftree}
%\AxiomC{$\mathcal{H}_1$}
%\AxiomC{$\cdots$}
%\AxiomC{$\mathcal{H}_n$}
%\TrinaryInfC{$\mathcal{C}$}
%\end{prooftree}
%in later derivations.
%\end{rmk}

\begin{exercises}
\item \label{ex:term_conversion}Give a derivation for the following conversion rule\index{term conversion rule}\index{term conversion rule}\index{rule!term conversion}\index{term conversion rule}\index{conversion rule!term}:
\begin{prooftree}
\AxiomC{$\Gamma\vdash A\jdeq A'~\textrm{type}$}
\AxiomC{$\Gamma\vdash a:A$}
\BinaryInfC{$\Gamma\vdash a:A'$}
\end{prooftree}
\begin{comment}
\item Consider a type $A$ in context $\Gamma$. In this exercise we establish a connection between types in context $\Gamma,x:A$, and uniform choices of types $B_a$, where $a$ ranges over terms of $A$ in a uniform way. A similar connection is made for terms.
\begin{subexenum}
\item We define a \define{uniform family} over $A$ to consist of a type
\begin{equation*}
\Delta,\Gamma\vdash B_a~\mathrm{type}
\end{equation*}
for every context $\Delta$, and every term $\Delta,\Gamma\vdash a:A$, subject to the condition that one can derive
\begin{prooftree}
\AxiomC{$\Delta\vdash d:D$}
\AxiomC{$\Delta,y:D,\Gamma\vdash a:A$}
\BinaryInfC{$\Delta,\Gamma\vdash B_a[d/y]\jdeq B_{a[d/y]}~\mathrm{type}$}
\end{prooftree}
Define a bijection between types in context $\Gamma,x:A$ and uniform families over $A$. 
\item Consider a type $\Gamma,x:A\vdash B$. We define a \define{uniform term} of $B$ over $A$ to consist of a type
\begin{equation*}
\Delta,\Gamma\vdash b_a:B[a/x]~\mathrm{type}
\end{equation*}
for every context $\Delta$, and every term $\Delta,\Gamma\vdash a:A$, subject to the condition that one can derive
\begin{prooftree}
\AxiomC{$\Delta\vdash d:D$}
\AxiomC{$\Delta,y:D,\Gamma\vdash a:A$}
\BinaryInfC{$\Delta,\Gamma\vdash b_a[d/y]\jdeq b_{a[d/y]}:B[a/x][d/y]$}
\end{prooftree}
Define a bijection between terms of $B$ in context $\Gamma,x:A$ and uniform terms of $B$ over $A$. 
\end{subexenum}
\end{comment}
\end{exercises}


\chapter{Inductive types}

\section{The idea of inductive types}

Many other types can also be specified as inductive types, similar to the natural numbers. The unit type, the empty type, and the booleans are the simplest examples of this way of defining types. Just like the type of natural numbers, other inductive types are also specified by their \emph{constructors}, an \emph{induction principle}, and their \emph{computation rules}: 
\begin{enumerate}
\item The constructors tell what structure the inductive type comes equipped with. There may be multiple constructors, or no constructors at all in the specification of an inductive type. 
\item The induction principle specifies the data that should be provided in order to construct a section of an arbitrary dependent type over the inductive type. 
\item The computation rules assert that the inductively defined section agrees on the constructors with the data that was used to define the section. Thus, there is a computation rule for every constructor.
\end{enumerate}
For a more general treatment of inductive types, we refer to Chapter 5 of \cite{hottbook}.
\begin{defn}
We define the \define{unit type} to be a type $\unit$ equipped with
\begin{equation*}
\ttt:\unit,
\end{equation*}
satisfying the induction principle that for any type family $P:\unit\to\type$, there is a term
\begin{equation*}
\ind{\unit} : P(\ttt)\to\prd{x:\unit}P(x)
\end{equation*}
for which the computation rule
\begin{equation*}
\ind{\unit}(p_{\ttt},\ttt) \jdeq p_\ttt
\end{equation*}
holds.
\end{defn}

The empty type is a degenerate example of an inductive type. It does \emph{not} come equipped with any constructors, and therefore there are also no computation rules. The induction principle merely asserts that any type family has a section.
\begin{defn}
We define the \define{empty type} to be a type $\emptyt$ satisfying the induction principle that for any type family $P:\emptyt\to\type$, there is a term
\begin{equation*}
\ind{\emptyt} : \prd{x:\emptyt}P(x).
\end{equation*}
\end{defn}

\begin{defn}
We define the \define{booleans} to be a type $\bool$ that comes equipped with
\begin{align*}
\bfalse & : \bool \\
\btrue & : \bool
\end{align*}
satisfying the induction principle that for any type family $P:\bool\to\type$, there is a term
\begin{equation*}
\ind{\bool} : P(\bfalse)\to P(\btrue)\to \prd{x:\bool}P(x)
\end{equation*}
for which the computation rules
\begin{align*}
\ind{\bool}(p_0,p_1,\bfalse) & \jdeq p_0 \\
\ind{\bool}(p_0,p_1,\btrue) & \jdeq p_1
\end{align*}
hold.
\end{defn}

\begin{defn}
Let $A$ and $B$ be types. We define the \define{disjoint sum} $A+B$ to be a type that comes equipped with
\begin{align*}
\inl & : A \to A+B \\
\inr & : B \to A+B
\end{align*}
satisfying the induction principle that for any type family $P:(A+B)\to\type$, there is a term
\begin{equation*}
\ind{A+B} : \Big(\prd{x:A}P(\inl(x))\Big)\to\Big(\prd{y:B}P(\inr(y))\Big)\to\prd{z:A+B}P(z)
\end{equation*}
for which the computation rules
\begin{align*}
\ind{A+B}(f,g,\inl(x)) & \jdeq f(x) \\
\inr{A+B}(f,g,\inr(y)) & \jdeq g(y)
\end{align*}
hold.
\end{defn}

\section{The type of integers}
\begin{defn}
We define the \define{integers} to be the type $\Z\defeq\nat_{\geq 1}+\unit+\nat_{\geq 1}$.
\end{defn}

\begin{defn}
We construct a successor function $S:\Z\to\Z$.
\end{defn}

\begin{constr}

\end{constr}

\section{Dependent pair types}

\begin{defn}
Let $A$ be a type, and let $P:A\to\type$ be a type family over $A$.
The \define{dependent pair type} $\sm{x:A}P(x)$ is defined to be a type equipped with a \define{pairing function}
\begin{equation*}
\pairr{\blank,\blank}:\prd{x:A} P(x)\to \Big(\sm{y:A}P(y)\Big)
\end{equation*}
and \define{projection maps}
\begin{align*}
\proj 1 & : \Big(\sm{x:A}P(x)\Big)\to A \\
\proj 2 & : \prd{w:\sm{x:A}P(x)}P(\proj 1(w))
\end{align*}
satisfying the equations
\begin{align*}
\proj 1(\pairr{a,p}) & \jdeq a \tag{$\beta_1$}\\
\proj 2(\pairr{a,p}) & \jdeq p \tag{$\beta_2$}\\
\pairr{\proj 1(w),\proj 2(w)} & \jdeq w. \tag{$\eta$}
\end{align*}
\end{defn}

\begin{exercises}
\item For any type $A$, show that $(A+\neg A)\to(\neg\neg A\to A)$. 
\item \label{ex:int_group_ops}Define operations $k,l\mapsto k+l:\Z\to\Z\to\Z$ and $k\mapsto -k:\Z\to \Z$.
\item \label{ex:int_order}Define the relations $\leq$ and $<$ on and $\Z$.
\end{exercises}


\chapter{Identity types}
From the perspective of types as proof-relevant propositions, how should we think of \emph{equality} in type theory? Given a type $A$, and two terms $x,y:A$, the equality $\id{x}{y}$ should again be a type. Indeed, we want to \emph{use} type theory to prove equalities. \emph{Dependent} type theory provides us with a convenient setting for this: the equality type $\id{x}{y}$ is dependent on $x,y:A$. 

Then, if $\id{x}{y}$ is to be a type, how should we think of the terms of $\id{x}{y}$. A term $p:\id{x}{y}$ witnesses that $x$ and $y$ are equal terms of type $A$. In other words $p:\id{x}{y}$ is an \emph{identification} of $x$ and $y$. In a proof-relevant world, there might be many terms of type $\id{x}{y}$. I.e.~there might be many identifications of $x$ and $y$. And, since $\id{x}{y}$ is itself a type, we can form the type $\id{p}{q}$ for any two identifications $p,q:\id{x}{y}$. That is, since $\id{x}{y}$ is a type, we may also use the type theory to prove things \emph{about} identifications (for instance, that two given such identifications can themselves be identified), and we may use the type theory to perform constructions with them. As we will see shortly, we can give every type a groupoid-like structure.

Clearly, the equality type should not just be any type dependent on $x,y:A$. Then how do we form the equality type, and what ways are there to use identifications in constructions in type theory? The answer to both these questions is that we will form the identity type as an \emph{inductive} type, generated by just a reflexivity term providing an identification of $x$ to itself. The induction principle then provides us with a way of performing constructions with identifications, such as concatenating them, inverting them, and so on. Thus, the identity type is equipped with a reflexivity term, and further possesses the structure that are generated by its induction principle and by the type theory. This inductive construction of the identity type is elegant, beautifully simple, but far from trivial!

The situation where two terms can be identified in possibly more than one way is analogous to the situation in \emph{homotopy theory}, where two points of a space can be connected by possibly more than one \emph{path}. Indeed, for any two points $x,y$ in a space, there is a \emph{space of paths} from $x$ to $y$. Moreover, between any two paths from $x$ to $y$ there is a space of \emph{homotopies} between them, and so on. This analogy has been made precise by the construction of \emph{homotopical models} of type theory, and it has led to the fruitful research area of \emph{synthetic homotopy theory}, the subfield of \emph{homotopy type theory} that is the topic of this course.

\section{The inductive definition of identity types}

Let $A$ be a type in context $\Gamma$. The \define{identity type} of $A$ at $a:A$ is the inductive type family 
\begin{equation*}
\Gamma,x:A,y:A\vdash x =_A y~\mathrm{type}
\end{equation*}
with constructor
\begin{equation*}
\Gamma,x:A\vdash \refl{x} : x=_A x.
\end{equation*}
The induction principle that for any type family
\begin{equation*}
\Gamma,x:A,y:A,\alpha: x=_A y\vdash P(x,y,\alpha)~\mathrm{type}
\end{equation*}
%we can derive
%\begin{prooftree}
%\AxiomC{$\Gamma,\Delta[a/x,\refl{a}/p]\vdash \alpha : P(a,\refl{a})$}
%\UnaryInfC{$\Gamma,x:A,p:\idtypevar{A}(a,x),\Delta\vdash \ind{a=}(\alpha,x,p):P(x,p)$}
%\end{prooftree}
there is a term
\begin{equation*}
\ind{x=} : P(x,x,\refl{x})\to \prd{y:A}{\alpha:x=_A y}P(x,y,\alpha)
\end{equation*}
in context $\Gamma,x:A$, satisfying the computation rule
\begin{equation*}
\ind{x=}(p,x,\refl{x})\jdeq p.
\end{equation*}
A term of type $x=_A y$ is also called an \define{identification} of $x$ with $y$, and sometimes it is called a \define{path} from $x$ to $y$.
The induction principle for identity types is sometimes called \define{identification elimination} or \define{path induction}. We also write $\idtypevar{A}$ for the identity type on $A$. 

We also assume that the universe $\UU$ is closed under identity types, i.e. that there is a map
\begin{equation*}
\check{\mathsf{Id}}:\prd{A:\UU}\mathrm{El}(A)\to\mathrm{El}(A)\to\UU
\end{equation*}
satisfying
\begin{equation*}
\mathrm{El}(\check{\mathsf{Id}}(A,x,y))\jdeq x=_{\mathrm{El}(A)} y.
\end{equation*}

In the following lemma we show that the identity type on $A$ is contained in any reflexive relation on $A$.

\begin{lem}
Let $\Gamma,x:A,y:A\vdash R(x,y)~\mathrm{type}$, and suppose that $R$ is reflexive in the sense that there is a term
\begin{equation*}
\rho:\prd{x:A}R(x,x)
\end{equation*}
Then there is a term of type
\begin{equation*}
\prd{y:A} (x=_A y)\to R(x,y)
\end{equation*}
in context $\Gamma,x:A$.
\end{lem}

\begin{constr}
By weakening the reflexive relation $R$ we obtain
\begin{equation*}
\Gamma,x:A,y:A,\alpha:x=_A y\vdash R(x,y)~\mathrm{type},
\end{equation*}
on which the induction principle is applicable.
Thus we see that by the induction principle for identity types we have a term
\begin{equation*}
\ind{x=} : R(x,x)\to \prd{y:A}(x=_A y)\to R(x,y)
\end{equation*}
so it suffices to construct a term of type $R(x,x)$, which we have by reflexivity of $R$.
\end{constr}

\section{The groupoid structure of types}\label{sec:groupoid}
We show that identifications can be \emph{concatenated} and \emph{inverted}, which corresponds to the transitivity and symmetry of the identity type. 

Furthermore, we observe that we can iteratively take identity types, i.e.~we can take identity types of identity types, 
\begin{equation*}
p =_{(x=_Ay)} q,
\end{equation*}
and so on. In other words, for any two identifications $p,q:x=_A y$, there is a type of identifications of $p$ with $y$. One way to think about this is that the identifications $p,q:x=_A y$ are paths in the type (space) $A$, and an identification of $p$ with $q$ is a \emph{higher path} from $p$ to $q$, i.e.~a \emph{homotopy}.

Using the observation that identity types can be iterated we show that concatenation is \emph{associative}, satisfies the left and right \emph{unit laws}, and satisfies the left and right \emph{inverse laws}. These are the \define{groupoid operations} on the identity type.

\begin{defn}\label{defn:id_concat}
Let $A$ be a type. We define the \define{concatenation} operation
\begin{equation*}
\mathsf{concat} : \prd{x,y,z:A} (\id{x}{y})\to(\id{y}{z})\to (\id{x}{z}).
\end{equation*}
We will write $\ct{p}{q}$ for $\mathsf{concat}(p,q)$. Also, we will \emph{associate to the right}, i.e.~by $\ct{p}{q}{r}$ we mean $\ct{p}{(\ct{q}{r})}$.
\end{defn}

\begin{constr}
We construct the concatenation operation by path induction. It suffices to construct
\begin{equation*}
\mathsf{concat}(\refl{x}):\prd{z:A} (x=z)\to(x=z).
\end{equation*}
Here we take $\mathsf{concat}(\refl{x})_z \jdeq \idfunc[(x=z)]$. 
Explicitly, the term we have constructed is
\begin{equation*}
\lam{x}\rec{x=}(\lam{z}\idfunc[(\id{x}{z})]):\prd{x,y:A} (x=y)\to \prd{z:A} (y=z)\to (x=z).
\end{equation*}
To obtain a term of the asserted type we need to swap the order of the arguments $p:x=y$ and $z:A$, using \autoref{ex:swap}.
\end{constr}

\begin{defn}\label{defn:id_inv}
Let $A$ be a type. We define the \define{inverse operation} 
\begin{equation*}
\mathsf{inv}:\prd{x,y:A} (x=y)\to (y=x).
\end{equation*}
Most of the time we will write $p^{-1}$ for $\mathsf{inv}(p)$.
\end{defn}

\begin{constr}
We construct the inverse operation by path induction. It suffices to construct
\begin{equation*}
\mathsf{inv}(\refl{x}): x=x,
\end{equation*}
for any $x:A$. Here we take $\mathsf{inv}(\refl{x})\defeq \refl{x}$.
\end{constr}

\begin{defn}\label{defn:id_assoc}
Let $A$ be a type. We define the \define{associativity operation}, which assigns to each $p:x=y$, $q:y=z$, and $r:z=w$ the \define{associator}
\begin{equation*}
\mathsf{assoc}(p,q,r) : \ct{(\ct{p}{q})}{r}=\ct{p}{(\ct{q}{r})}.
\end{equation*}
\end{defn}

\begin{constr}
By identification elimination it suffices to show that
\begin{equation*}
\prd{z:A}{q:x=z}{z':A}{r:z=w} \ct{(\ct{\refl{x}}{q})}{r}= \ct{\refl{x}}{(\ct{q}{r})}.
\end{equation*}
Let $q:x=z$ and $r:z=w$. Note that by the computation rule $\ct{\refl{x}}{q}\jdeq q$, so $\ct{(\ct{\refl{x}}{q})}{r}\jdeq \ct{q}{r}$. Similarly we have $\ct{\refl{x}}{(\ct{q}{r})}\jdeq \ct{q}{r}$. Therefore we can simply take $\refl{\ct{q}{r}}$.
\end{constr}

\begin{defn}\label{defn:id_unit}
Let $A$ be a type. We define the left and right \define{unit operations}, which assigns to each $p:x=y$ the terms
\begin{align*}
\mathsf{left\usc{}unit}(p) & : \ct{\refl{x}}{p}=p \\
\mathsf{right\usc{}unit}(p) & : \ct{p}{\refl{y}}=p,
\end{align*}
respectively.
\end{defn}

\begin{constr}
By identification elimination it suffices to construct
\begin{align*}
\mathsf{left\usc{}unit}(\refl{x}) & : \ct{\refl{x}}{\refl{x}} = \refl{x} \\
\mathsf{right\usc{}unit}(\refl{x}) & : \ct{\refl{x}}{\refl{x}} = \refl{x}.
\end{align*}
In both cases we take $\refl{\refl{x}}$.
\end{constr}

\begin{defn}\label{defn:id_invlaw}
Let $A$ be a type. We define left and right \define{inverse operations}
\begin{align*}
\mathsf{left\usc{}inv}(p) & : \ct{p^{-1}}{p} = \refl{y} \\
\mathsf{right\usc{}inv}(p) & : \ct{p}{p^{-1}} = \refl{x}.
\end{align*}
\end{defn}

\begin{constr}
By identification elimination it suffices to construct
\begin{align*}
\mathsf{left\usc{}inv}(\refl{x}) & : \ct{\refl{x}^{-1}}{\refl{x}} = \refl{x} \\
\mathsf{right\usc{}inv}(\refl{x}) & : \ct{\refl{x}}{\refl{x}^{-1}} = \refl{x}.
\end{align*}
Using the computation rules we see that
\begin{equation*}
\ct{\refl{x}^{-1}}{\refl{x}}\jdeq \ct{\refl{x}}{\refl{x}}\jdeq\refl{x},
\end{equation*}
so we define $\mathsf{left\usc{}inv}(\refl{x})\defeq \refl{\refl{x}}$. Similarly it follows from the computation rules that
\begin{equation*}
\ct{\refl{x}}{\refl{x}^{-1}} \jdeq \refl{x}^{-1}\jdeq \refl{x}
\end{equation*}
so we again define $\mathsf{right\usc{}inv}(\refl{x})\defeq\refl{\refl{x}}$. 
\end{constr}

\section{The action on paths of functions}

Using the induction principle of the identity type we can show that every function preserves identifications.
In other words, every function sends identified terms to identified terms.
Note that this is a form of continuity for functions in type theory: if there is a path that identifies two points $x$ and $y$ of a type $A$, then there also is a path that identifies the values $f(x)$ and $f(y)$ in the codomain of $f$. 

\begin{defn}\label{defn:ap}
Let $f:A\to B$ be a map. We define the \define{action on paths} of $f$ as an operation
\begin{equation*}
\apfunc{f} : \prd*{x,y:A} (\id{x}{y})\to(\id{f(x)}{f(y)}).
\end{equation*}
Moreover, there are operations
\begin{align*}
\mathsf{ap.idfun}_A & : \prd*{x,y:A}{p:\id{x}{y}} \id{p}{\ap{\idfunc[A]}{p}} \\
\mathsf{ap.comp}(f,g) & : \prd*{x,y:A}{p:\id{x}{y}} \id{\ap{g}{\ap{f}{p}}}{\ap{g\circ f}{p}}.
\end{align*}
\end{defn}

\begin{constr}
First we define $\apfunc{f}$ by identity elimination, taking
\begin{equation*}
\apfunc{f}(\refl{x})\defeq \refl{f(x)}.
\end{equation*}
Next, we construct $\mathsf{ap.idfun}_A$ by identity elimination, taking
\begin{equation*}
\mathsf{ap.idfun}_A(\refl{x}) \defeq \refl{\refl{x}}.
\end{equation*}
Finally, we construct $\mathsf{ap.comp}(f,g)$ by identity elimination, taking
\begin{equation*}
\mathsf{ap.comp}(f,g,\refl{x}) \defeq \refl{g(f(x))}.\qedhere
\end{equation*}
\end{constr}

\begin{defn}
Let $f:A\to B$ be a map. Then there are identifications
\begin{align*}
\mathsf{ap.refl}(f,x) & : \id{\ap{f}{\refl{x}}}{\refl{f}(x)} \\
\mathsf{ap.inv}(f,p) & : \id{\ap{f}{p^{-1}}}{\ap{f}{p}^{-1}} \\
\mathsf{ap.concat}(f,p,q) & : \id{\ap{f}{\ct{p}{q}}}{\ct{\ap{f}{p}}{\ap{f}{q}}}
\end{align*}
for every $p:\id{x}{y}$ and $q:\id{x}{y}$.
\end{defn}

\begin{constr}
To construct $\mathsf{ap.refl}(f,x)$ we simply observe that ${\ap{f}{\refl{x}}}\jdeq {\refl{f}(x)}$, so we take
\begin{equation*}
\mathsf{ap.refl}(f,x)\defeq\refl{\refl{f(x)}}.
\end{equation*}
We construct $\mathsf{ap.inv}(f,p)$ by identification elimination on $p$, taking
\begin{equation*}
\mathsf{ap.inv}(f,\refl{x}) \defeq \refl{\ap{f}{\refl{x}}}.
\end{equation*}
Finally we construct $\mathsf{ap.concat}(f,p,q)$ by identification elimination on $p$, taking
\begin{equation*}
\mathsf{ap.concat}(f,\refl{x},q)  \defeq \refl{\ap{f}{q}}.\qedhere
\end{equation*}
\end{constr}

\section{Transport}

\begin{table}
\begin{center}
\caption{The homotopy interpretation}
\begin{tabular}{ll}
\toprule
\emph{Type theory} &  \emph{Homotopy theory} \\
\midrule
Types  & Spaces \\
Dependent types & Fibrations \\
Terms & Points \\
Dependent pair type & Total space \\
Identity type & Path fibration\\
\bottomrule
\end{tabular}
\end{center}
\end{table}

Dependent types also come with an action on paths: the \emph{transport} functions.
Given an identification $p:\id{x}{y}$ in the base type $A$, we can transport any term $b:B(x)$ to the fiber $B(y)$.
The transport functions have many applications, which we will encounter throughout this course.

\begin{defn}
Let $A$ be a type, and let $B$ be a type family over $A$.
We will construct a \define{transport} operation
\begin{equation*}
\mathsf{tr}_B:\prd*{x,y:A} (\id{x}{y})\to (B(x)\to B(y)).
\end{equation*}
We will write $\trans{p}{b}$ for $\mathsf{tr}_B(p,b)$.
\end{defn}

\begin{constr}
We construct $\mathsf{tr}_B(p)$ by induction on $p:x=_A y$, taking
\begin{equation*}
\mathsf{tr}_B(\refl{x}) \defeq \idfunc[B(x)].\qedhere
\end{equation*}
\end{constr}

Thus we see that type theory cannot distinguish between identified terms $x$ and $y$, because for any type family $B$ over $A$ one gets a term of $B(y)$ as soon as $B(x)$ has a term.

As an application of the transport function we construct the \emph{dependent} action on paths of a dependent function $f:\prd{x:A}B(x)$. Note that for such a dependent function $f$, and an identification $p:\id[A]{x}{y}$, it does not make sense to directly compare $f(x)$ and $f(y)$, since the type of $f(x)$ is $B(x)$ whereas the type of $f(y)$ is $B(y)$, which might not be exactly the same type. However, we can first \emph{transport} $f(x)$ along $p$, so that we obtain the term $\mathsf{tr}_B(p,f(x))$ which is of type $B(y)$. Now we can ask whether it is the case that $\mathsf{tr}_B(p,f(x))=f(y)$. The dependent action on paths of $f$ establishes this identification.

\begin{defn}\label{defn:apd}
Given a dependent function $f:\prd{a:A}B(a)$ and a path $p:\id{x}{y}$ in $A$, we construct a path
\begin{equation*}
\apd{f}{p} : \id{\mathsf{tr}_B(p,f(x))}{f(y)}.
\end{equation*}
\end{defn}

\begin{constr}
The path $\apd{f}{p}$ is constructed by path induction on $p$. Thus, it suffices to construct a path
\begin{equation*}
\apd{f}{\refl{x}}:\id{\mathsf{tr}_B(\refl{x},f(x))}{f(x)}.
\end{equation*}
Since transporting along $\refl{x}$ is the identity function on $B(x)$, we simply take $\apd{f}{\refl{x}}\defeq\refl{f(x)}$. 
\end{constr}

%\begin{defn}\label{defn:path_lifting}
%Let $A$ be a type, and let $B:A\to\type$ be a type family over $A$.
%We will construct a \define{path lifting} operation
%\begin{equation*}
%\mathsf{lift}^B : \prd*{x,y:A}{p:\id{x}{y}}{b:B(x)} \id{\pairr{x,b}}{\pairr{y,\trans{p}{b}}}.
%\end{equation*}
%\end{defn}
%
%\autoref{defn:path_lifting} gives a way to lift a path $p:x=y$ in the base type of a type family, to a path in the $\Sigma$-type. This, along with the basic groupoid operations developed in \autoref{sec:groupoid}, inspired the \emph{homotopy interpretation} of type theory.

\begin{exercises}
\item \label{ex:trans_concat}Let $B$ be a family over a type $A$. Construct for any two identifications $p:x=_A y$ and $q:y=_A z$, and any $b:B(x)$ an identification
\begin{equation*}
\mathsf{tr}_B(q,\mathsf{tr}_B(p,x))=\mathsf{tr}_B(\ct{p}{q},x).
\end{equation*}
\item \label{ex:inv_assoc}Let $p:\id{x}{y}$ and $q:\id{y}{z}$. Construct an identification
\begin{align*}
\mathsf{inv\usc{}assoc}(p,q):\id{(\ct{p}{q})^{-1}}{\ct{q^{-1}}{p^{-1}}}.
\end{align*}
\item \label{ex:trans_triv}Consider two types $A$ and $B$, and let $p:x=y$ in $A$, and $b:B$. Construct an identification
\begin{align*}
\mathsf{tr\usc{}triv}(p,b):\mathsf{tr}_{W_A(B)}(p,b)=b
\end{align*}
where $W_A(B)$ is the family $B$ weakened by $A$.
%\item In this exercise we show that the action on paths of a function preserves the groupoid-structure of a type.
%\begin{subexenum}
%\item Construct an identification
%\begin{equation*}
%\mathsf{ap.assoc}(f,p,q,r)
%\end{equation*}
%witnessing that the diagram
%\begin{equation*}
%\begin{tikzcd}[column sep=large]
%\ap{f}{\ct{(\ct{p}{q})}{r}} \arrow[r,equals,"\ap{\apfunc{f}}{\mathsf{assoc}(p,q,r)}"] \arrow[d,swap,equals,"{\mathsf{ap.ct}(f,%\ct{p}{q},r)}"] & \ap{f}{\ct{p}{(\ct{q}{r})}} \arrow[d,equals,"{\mathsf{ap.ct}(f,p,\ct{q}{r})}"] \\ 
%\ct{\ap{f}{\ct{p}{q}}}{\ap{f}{r}} \arrow[dd,equals,near start,"{\mathsf{whisk\usc{}r}(\mathsf{ap.ct}(f,p,q),\ap{f}{r})}"]   & %\ct{\ap{f}{p}}{\ap{f}{\ct{q}{r}}} \arrow[dd,equals,swap,near end,"{\mathsf{whisk\usc{}l}(\ap{f}{p},\mathsf{ap.ct}(f,q,r))}"]  %\\
%\\
%\ct{(\ct{\ap{f}{p}}{\ap{f}{q}})}{\ap{f}{r}} \arrow[r,equals,swap,"{\mathsf{assoc}(\ap{f}{p},\ap{f}{q},\ap{f}{r})}"yshift=-1em] & \ct{\ap{f}{p}}{(\ct{\ap{f}{q}}{\ap{f}{r}})}
%\end{tikzcd}
%\end{equation*}
%commutes.
%\end{subexenum}
\item \label{ex:trans_ap}Let $f:A\to B$ be a map, and consider $p:x=y$ in $A$. 
\begin{subexenum}
\item Construct for any $q:f(x)=b$ in $B$ an identification
\begin{equation*}
\mathsf{tr\usc{}ap}(p,q):\id{\trans{p}{q}}{\ct{\ap{f}{p}^{-1}}{q}}.
\end{equation*}
\item Similarly, construct for any $q':b=f(x)$ in $B$ an identification
\begin{equation*}
\mathsf{tr\usc{}ap'}(p,q'):\id{\trans{p}{q}}{\ct{q}{\ap{f}{p}}}.
\end{equation*}
\end{subexenum}
\item \label{ex:inv_con}For any $p:x=y$, $q:y=z$, and $r:x=z$, construct maps
\begin{align*}
\mathsf{inv\usc{}con}(p,q,r) & : (\ct{p}{q}=r)\to (q=\ct{p^{-1}}{r}) \\
\mathsf{con\usc{}inv}(p,q,r) & : (\ct{p}{q}=r)\to (p=\ct{r}{q^{-1}}).
\end{align*}
\item Let $A$ be a type, and let $B:A\to\type$ be a type family over $A$.
Construct the \define{path lifting} operation
\begin{equation*}
\mathsf{lift}_B : \prd*{x,y:A}{p:\id{x}{y}}{b:B(x)} \id{\pairr{x,b}}{\pairr{y,\trans{p}{b}}}.
\end{equation*}
In other words, a path in the \emph{base type} $A$ lifts to a path in the total space $\sm{x:A}B(x)$ for every term over the domain.
\item Show that the operations of addition and multiplication on the natural numbers satisfy the following laws:
\begin{align*}
m+(n+k) & =(m+n)+k & m\cdot (n\cdot k) & = (m\cdot n)\cdot k \\
m+0 & = m & m\cdot 1 & = m \\
0+m & = m & 1\cdot m & = m \\
m+n & = n+m & m\cdot n & = n\cdot m\\
& & m\cdot (n+k) & = m\cdot n + m\cdot k.
\end{align*}
\end{exercises}


% !TEX root = hott_intro.tex

\section{Equivalences}

\subsection{Homotopies}
In homotopy type theory, a homotopy is just a pointwise equality between two functions $f$ and $g$.

\begin{defn}
Let $f,g:\prd{x:A}P(x)$ be two dependent functions. The type of \define{homotopies}\index{homotopy|textbf} from $f$ to $g$ is defined as
\begin{equation*}
f\htpy g \defeq \prd{x:A} f(x)=g(x).
\end{equation*}
\end{defn}

Since we formulated homotopies using dependent functions, we may also consider homotopies \emph{between}\index{homotopy!iterated} homotopies, and further homotopies between those higher homotopies. 
Explicitly, if $H,K:f\htpy g$, then the type $H\htpy K$ of homotopies is just the type
\begin{equation*}
\prd{x:A} H(x)=K(x).
\end{equation*}

In the following definition we define the groupoid-like structure of homotopies. Note that we implement the groupoid-laws as \emph{homotopies} rather than as identifications.

\begin{defn}\label{defn:htpy_groupoid}\index{groupoid laws!of homotopies|textbf}
For any dependent type $B:A\to\type$ there are operations
\begin{align*}
& \mathsf{htpy\usc{}refl} & & : \prd{f:\prd{x:A}B(x)}f\htpy f \\
& \mathsf{htpy\usc{}inv} & & : \prd*{f,g:\prd{x:A}B(x)} (f\htpy g)\to(g\htpy f) \\
& \mathsf{htpy\usc{}concat} & & : \prd*{f,g,h:\prd{x:A}B(x)} (f\htpy g)\to ((g\htpy h)\to (f\htpy h)).
\end{align*}
We will write $H^{-1}$ for $\mathsf{htpy\usc{}inv}(H)$, and $\ct{H}{K}$ for $\mathsf{htpy\usc{}concat}(H,K)$. 

Furthermore, we define
\begin{align*}
& \mathsf{htpy\usc{}assoc}(H,K,L) & & : \ct{(\ct{H}{K})}{L}\htpy\ct{H}{(\ct{K}{L})} \\
& \mathsf{htpy\usc{}left\usc{}unit}(H) & & : \ct{\mathsf{htpy\usc{}refl}_f}{H}\htpy H \\
& \mathsf{htpy\usc{}right\usc{}unit}(H) & & : \ct{H}{\mathsf{htpy\usc{}refl}_g}\htpy H \\
& \mathsf{htpy\usc{}left\usc{}inv}(H) & & : \ct{H^{-1}}{H} \htpy \mathsf{htpy\usc{}refl}_g \\
& \mathsf{htpy\usc{}right\usc{}inv}(H) & & : \ct{H}{H^{-1}} \htpy \mathsf{htpy\usc{}refl}_f
\end{align*}
for any $H:f\htpy g$, $K:g\htpy h$ and $L:h\htpy i$, where $f,g,h,i:\prd{x:A}B(x)$.
\end{defn}

\begin{constr}
We define
\begin{align*}
\mathsf{htpy\usc{}refl}(f) & \defeq \lam{x} \refl{f(x)} \\
\mathsf{htpy\usc{}inv}(H) & \defeq \lam{x} H(x)^{-1} \\
\mathsf{htpy\usc{}concat}(H,K) & \defeq \lam{x}\ct{H(x)}{K(x)},
\end{align*}
where $H:f\htpy g$ and $K:g\htpy h$ are homotopies. Furthermore, we define
\begin{align*}
\mathsf{htpy\usc{}assoc}(H,K,L) & \defeq \lam{x}\mathsf{assoc}(H(x),K(x),L(x)) \\
\mathsf{htpy\usc{}left\usc{}unit}(H) & \defeq \lam{x}\mathsf{left\usc{}unit}(H(x)) \\
\mathsf{htpy\usc{}right\usc{}unit}(H) & \defeq \lam{x}\mathsf{right\usc{}unit}(H(x)) \\
\mathsf{htpy\usc{}left\usc{}inv}(H) & \defeq \lam{x}\mathsf{left\usc{}inv}(H(x)) \\
\mathsf{htpy\usc{}right\usc{}inv}(H) & \defeq \lam{x}\mathsf{right\usc{}inv}(H(x)).\qedhere
\end{align*}
\end{constr}


Apart from the groupoid operations and their laws, we will occasionally need \emph{whiskering} operations.

\begin{defn}
We define the following \define{whiskering}\index{homotopy!whiskering operations|textbf}\index{whiskering operations!of homotopies|textbf} operations on homotopies:
\begin{enumerate}
\item Suppose $H:f\htpy g$ for two functions $f,g:A\to B$, and let $h:B\to C$. We define
\begin{equation*}
h\cdot H\defeq \lam{x}\ap{h}{H(x)}:h\circ f\htpy h\circ g.
\end{equation*}
\item Suppose $f:A\to B$ and $H:g\htpy h$ for two functions $g,h:B\to C$. We define
\begin{equation*}
H\cdot f\defeq\lam{x}H(f(x)):h\circ f\htpy g\circ f.
\end{equation*}
\end{enumerate}
\end{defn}

We also use homotopies to express the commutativity of diagrams. For example, we say that a triangle
\begin{equation*}
  \begin{tikzcd}[column sep=tiny]
    A \arrow[rr,"h"] \arrow[dr,swap,"f"] & & B \arrow[dl,"g"] \\
    & X
  \end{tikzcd}
\end{equation*}
commutes if it comes equipped with a homotopy $H:f\htpy g\circ h$, and we say that a square
\begin{equation*}
  \begin{tikzcd}
    A \arrow[r,"g"] \arrow[d,swap,"f"] & A' \arrow[d,"{f'}"] \\
    B \arrow[r,swap,"h"] & B'
  \end{tikzcd}
\end{equation*}
if it comes equipped with a homotopy $h \circ f~g\circ f'$.

\subsection{Bi-invertible maps}
\begin{defn}
Let $f:A\to B$ be a function. We say that $f$ has a \define{section}\index{section!of a map|textbf} if there is a term of type\index{sec(f)@{$\mathsf{sec}(f)$}|textbf}
\begin{equation*}
\mathsf{sec}(f) \defeq \sm{g:B\to A} f\circ g\htpy \idfunc[B].
\end{equation*}
Dually, we say that $f$ has a \define{retraction}\index{retraction} if there is a term of type\index{retr(f)@{$\mathsf{retr}(f)$}|textbf}
\begin{equation*}
\mathsf{retr}(f) \defeq \sm{h:B\to A} h\circ f\htpy \idfunc[A].
\end{equation*}
If a map $f:A \to B$ has a retraction, we also say that $A$ is a \define{retract}\index{retract!of a type} of $B$.
We say that a function $f:A\to B$ is an \define{equivalence}\index{equivalence|textbf}\index{bi-invertible map|see {equivalence}} if it has both a section and a retraction, i.e., if it comes equipped with a term of type\index{is_equiv@{$\isequiv$}|textbf}
\begin{equation*}
\isequiv(f)\defeq\mathsf{sec}(f)\times\mathsf{retr}(f).
\end{equation*}
We will write $\eqv{A}{B}$\index{equiv@{$\eqv{A}{B}$}|textbf} for the type $\sm{f:A\to B}\isequiv(f)$.
\end{defn}

\begin{rmk}
An equivalence, as we defined it here, can be thought of as a \emph{bi-invertible} map, since it comes equipped with a separate left and right inverse. Explicitly, if $f$ is an equivalence, then there are
\begin{align*}
g & : B\to A & h & : B\to A \\
G & : f\circ g \htpy \idfunc[B] & H & : h\circ f \htpy \idfunc[A].
\end{align*}
Clearly, if $f$ is \define{invertible}\index{invertible map} in the sense that it comes equipped with a function $g:B\to A$ such that $f\circ g\htpy\idfunc[B]$ and $g\circ f\htpy\idfunc[A]$, then $f$ is an equivalence. We write\index{is_invertible@{$\mathsf{is\usc{}invertible}$}|textbf}
\begin{equation*}
\mathsf{has\usc{}inverse}(f)\defeq\sm{g:B\to A} (f\circ g\htpy \idfunc[B])\times (g\circ f\htpy\idfunc[A]).
\end{equation*}
\end{rmk}

\begin{defn}\label{defn:inv_equiv}
Any equivalence can be given the structure of an invertible map.\index{equivalence!invertibility of}
\end{defn}

\begin{constr}
First we construct for any equivalence $f$ with right inverse $g$ and left inverse $h$ a homotopy $K:g\htpy h$. For any $y:B$, we have 
\begin{equation*}
\begin{tikzcd}[column sep=huge]
g(y) \arrow[r,equals,"H(g(y))^{-1}"] & hfg(y) \arrow[r,equals,"\ap{h}{G(y)}"] & h(y).
\end{tikzcd}
\end{equation*} 
Therefore we define a homotopy $K:g\htpy h$ by $K\defeq \ct{(H\cdot g)^{-1}}{h\cdot G}$.
Using the homotopy $K$ we are able to show that $g$ is also a left inverse of $f$. For $x:A$ we have the identification
\begin{equation*}
\begin{tikzcd}[column sep=large]
gf(x) \arrow[r,equals,"K(f(x))"] & hf(x) \arrow[r,equals,"H(x)"] & x.
\end{tikzcd}\qedhere
\end{equation*}
\end{constr}

\begin{cor}
The inverse of an equivalence is again an equivalence.
\end{cor}

\begin{proof}
Let $f:A\to B$ be an equivalence. By \cref{defn:inv_equiv} it follows that the section of $f$ is also a retraction. Therefore it follows that the section is itself an invertible map, with inverse $f$. Hence it is an equivalence.
\end{proof}

\begin{thm}\label{thm:id_equiv}
For any type $A$, the identity function $\idfunc[A]$ is an equivalence.\index{identity function!is an equivalence|textit}
\end{thm}

\begin{proof}
The identity function is trivially its own section and its own retraction.
\end{proof}

\begin{eg}
  For any type $C(x,y)$ indexed by $x:A$ and $y:B$, the function
\begin{equation*}
\sigma:\Big(\prd{x:A}{y:B}C(x,y)\Big)\to\Big(\prd{y:B}{x:A}C(x,y)\Big)
\end{equation*}
that swaps the order of the arguments $x$ and $y$ is an equivalence by \cref{ex:swap}.\index{swap function!is an equivalence|textit}
\end{eg}

\subsection{The identity type of a \texorpdfstring{$\Sigma$-}{dependent pair }type}

In this section we characterize the identity type of a $\Sigma$-type as a $\Sigma$-type of identity types. In this course we will be characterizing the identity types of many types, so we will follow the general outline of how such a characterization goes:
\begin{enumerate}
\item First we define a binary relation $R:A\to A\to \UU$ on the type $A$ that we are interested in. This binary relation is intended to be equivalent to its identity type.
\item Then we will show that this binary relation is reflexive, by constructing a term of type
  \begin{equation*}
    \prd{x:A}R(x,x)
  \end{equation*}
\item Using the reflexivity we will show that there is a canonical map
  \begin{equation*}
    (x=y)\to R(x,y)
  \end{equation*}
  for every $x,y:A$. This map is just constructed by path induction, using the reflexivity of $R$.
\item Finally, it has to be shown that the map
  \begin{equation*}
    (x=y)\to R(x,y)
  \end{equation*}
  is an equivalence for each $x,y:A$. 
\end{enumerate}
The last step is usually the most difficult, and we will refine our methods for this step in \cref{chap:fundamental}, where we establish the Fundamental Theorem of Identity Types.

In this section we consider a type family $B$ over $A$. Given two pairs
\begin{equation*}
  (x,y),(x',y'):\sm{x:A}B(x),
\end{equation*}
if we have a path $\alpha:x=x'$ then we can compare $y:B(x)$ to $y':B(x')$ by first transporting $y$ along $\alpha$, i.e., we consider the identity type
\begin{equation*}
  \mathsf{tr}_B(\alpha,y)=y'.
\end{equation*}
Thus it makes sense to think of $(x,y)$ to be identical to $(x',y')$ if there is an identification $\alpha:x=x'$ and an identification $\beta:\mathsf{tr}_B(\alpha,y)=y'$. In the following definition we turn this idea into a binary relation on the $\Sigma$-type.

\begin{defn}
  We will define a relation
  \begin{equation*}
    \mathsf{Eq}_{\Sigma} : \Big(\sm{x:A}B(x)\Big)\to\Big(\sm{x:A}B(x)\Big)\to\UU
  \end{equation*}
  by defining
  \begin{equation*}
    \mathsf{Eq}_{\Sigma}(s,t)\defeq\sm{\alpha:\proj 1(s)=\proj 1(t)}\mathsf{tr}_B(\alpha,\proj 2(s))=\proj 2 (t).
  \end{equation*}
\end{defn}

\begin{lem}
  The relation $\mathsf{Eq}_{\Sigma}$ is reflexive, i.e., there is a term
  \begin{equation*}
    \mathsf{reflexive\usc{}Eq}_{\Sigma}:\prd{s:\sm{x:A}B(x)}\mathsf{Eq}_{\Sigma}(s,s).
  \end{equation*}
\end{lem}

\begin{constr}
  This term is constructed by $\Sigma$-induction on $s:\sm{x:A}B(x)$. Thus, it suffices to construct a term of type
  \begin{equation*}
    \prd{x:A}{y:B(x)}\sm{\alpha:x=x}\mathsf{tr}_B(\alpha,y)=y.
  \end{equation*}
  Here we take $\lam{x}{y}(\refl{x},\refl{y})$.
\end{constr}

\begin{defn}
  Consider a type family $B$ over $A$. Then for any $s,t:\sm{x:A}B(x)$ we define a map\index{pair_eq@{$\mathsf{pair\usc{}eq}$}|textbf}
  \begin{equation*}
    \mathsf{pair\usc{}eq}: (s=t)\to \mathsf{Eq}_\Sigma(s,t)
  \end{equation*}
  by path induction, taking $\mathsf{pair\usc{}eq}(\refl{s})\defeq\mathsf{reflexive\usc{}Eq}_\Sigma(s)$.
\end{defn}

\begin{thm}\label{thm:eq_sigma}
  Let $B$ be a type family over $A$. Then the map
  \begin{equation*}
    \mathsf{pair\usc{}eq}: (s=t)\to \mathsf{Eq}_\Sigma(s,t)
  \end{equation*}
  is an equivalence for every $s,t:\sm{x:A}B(x)$.\index{Sigma type@{$\Sigma$-type}!identity types of|textit}\index{identity type!of a Sigma-type@{of a $\Sigma$-type}|textit}
\end{thm}

\begin{proof}
The maps in the converse direction\index{eq_pair@{$\mathsf{eq\usc{}pair}$}}
\begin{equation*}
\mathsf{eq\usc{}pair} : \mathsf{Eq}_\Sigma(s,t)\to(\id{s}{t})
\end{equation*}
are defined by repeated $\Sigma$-induction. By $\Sigma$-induction on $s$ and $t$  we see that it suffices to define a map
\begin{equation*}
\mathsf{eq\usc{}pair} : \Big(\sm{p:x=x'}\id{\mathsf{tr}_B(p,y)}{y'}\Big)\to(\id{(x,y)}{(x',y')}).
\end{equation*}
A map of this type is again defined by $\Sigma$-induction. Thus it suffices to define a dependent function of type
\begin{equation*}
\prd{p:x=x'} (\id{\mathsf{tr}_B(p,y)}{y'}) \to (\id{(x,y)}{(x',y')}).
\end{equation*}
Such a dependent function is defined by double path induction by sending $\pairr{\refl{x},\refl{y}}$ to $\refl{\pairr{x,y}}$. This completes the definition of the function $\mathsf{eq\usc{}pair}$.

Next, we must show that $\mathsf{eq\usc{}pair}$ is a section of $\mathsf{pair\usc{}eq}$. In other words, we must construct an identification
\begin{equation*}
\mathsf{pair\usc{}eq}(\mathsf{eq\usc{}pair}(\alpha,\beta))=\pairr{\alpha,\beta}
\end{equation*}
for each $\pairr{\alpha,\beta}:\sm{\alpha:x=x'}\id{\mathsf{tr}_B(\alpha,y)}{y'}$. We proceed by path induction on $\alpha$, followed by path induction on $\beta$. Then our goal becomes to construct a term of type
\begin{equation*}
\mathsf{pair\usc{}eq}(\mathsf{eq\usc{}pair}\pairr{\refl{x},\refl{y}})=\pairr{\refl{x},\refl{y}}
\end{equation*}
By the definition of $\mathsf{eq\usc{}pair}$ we have $\mathsf{eq\usc{}pair}\pairr{\refl{x},\refl{y}}\jdeq \refl{\pairr{x,y}}$, and by the definition of $\mathsf{pair\usc{}eq}$ we have $\mathsf{pair\usc{}eq}(\refl{\pairr{x,y}})\jdeq\pairr{\refl{x},\refl{y}}$. Thus we may take $\refl{\pairr{\refl{x},\refl{y}}}$ to complete the construction of the homotopy $\mathsf{pair\usc{}eq}\circ\mathsf{eq\usc{}pair}\htpy\idfunc$.

To complete the proof, we must show that $\mathsf{eq\usc{}pair}$ is a retraction of $\mathsf{pair\usc{}eq}$. In other words, we must construct an identification
\begin{equation*}
\mathsf{eq\usc{}pair}(\mathsf{pair\usc{}eq}(p))=p
\end{equation*}
for each $p:s=t$. We proceed by path induction on $p:s=t$, so it suffices to construct an identification 
\begin{equation*}
\mathsf{eq\usc{}pair}\pairr{\refl{\proj 1(s)},\refl{\proj 2(s)}}=\refl{s}.
\end{equation*}
Now we proceed by $\Sigma$-induction on $s:\sm{x:A}B(x)$, so it suffices to construct an identification
\begin{equation*}
\mathsf{eq\usc{}pair}\pairr{\refl{x},\refl{y}}=\refl{(x,y)}.
\end{equation*}
Since $\mathsf{eq\usc{}pair}\pairr{\refl{x},\refl{y}}$ computes to $\refl{(x,y)}$, we may simply take $\refl{\refl{(x,y)}}$.
\end{proof}

\begin{exercises}
%  \item Show that for any term $a:A$ the functions
%    \begin{align*}
%      \ind{\unit}(a) & : \unit \to A \\
%      \mathsf{const}_a & : \unit \to A
%    \end{align*}
%    are homotopic.
%  \item Let $A$ and $B$ be types, and consider the constant map $\mathsf{const}_b:A\to B$ for some $b:B$. Construct a homotopy
%    \begin{equation*}
%      \mathsf{ap}_{\mathsf{const}_b}(x,y)\htpy \mathsf{const}_{\refl{b}}
%    \end{equation*}
%    for any $x,y:A$.
\item \label{ex:equiv_grpd_ops}Show that the functions
  \begin{align*}
    \mathsf{inv} & :(\id{x}{y})\to(\id{y}{x}) \\
    \mathsf{concat}(p) & : (\id{y}{z})\to(\id{x}{z}) \\
    \mathsf{concat'}(q) & : (\id{x}{y}) \to (\id{x}{z}) \\
    \mathsf{tr}_B(p) & :B(x)\to B(y)
  \end{align*}
  are equivalences, where $\mathsf{concat'}(q,p)\defeq \ct{p}{q}$\index{concat'@{$\mathsf{concat'}$}}. Give their inverses explicitly.
\item Show that the maps
  \begin{align*}
    \inl & : X \to X+\emptyt &     \proj 1 & : \emptyt \times X \to \emptyt \\
    \inr & : X \to \emptyt+X &    \proj 2 & : X \times \emptyt \to \emptyt
  \end{align*}
  are equivalences.
\item
  \begin{subexenum}
  \item \label{ex:htpy_equiv}\index{equivalence!homotopic maps} Consider two functions $f,g:A\to B$ and a homotopy $H:f\htpy g$. Then
    \begin{equation*}
      \isequiv(f)\leftrightarrow\isequiv(g).
    \end{equation*}
  \item Show that for any two homotopic equivalences $e,e':\eqv{A}{B}$, their inverses are also homotopic.
  \end{subexenum}
\item \label{ex:3_for_2}\index{equivalence!three@{3-for-2 property}}\index{3-for-2 property!of equivalences}
  Consider a commuting triangle
  \begin{equation*}
    \begin{tikzcd}[column sep=tiny]
      A \arrow[rr,"h"] \arrow[dr,swap,"f"] & & B \arrow[dl,"g"] \\
      & X.
    \end{tikzcd}
  \end{equation*}
  with $H:f\htpy g\circ h$.
  \begin{subexenum}
  \item Suppose that the map $h$ has a section $s:B \to A$. Show that the triangle
    \begin{equation*}
      \begin{tikzcd}[column sep=tiny]
        B \arrow[rr,"s"] \arrow[dr,swap,"g"] & & A \arrow[dl,"f"] \\
        & X.
      \end{tikzcd}
    \end{equation*}
    commutes, and that $f$ has a section if and only if $g$ has a section.
  \item Suppose that the map $g$ has a retraction $r:X\to B$. Show that the triangle
    \begin{equation*}
      \begin{tikzcd}[column sep=tiny]
        A \arrow[rr,"f"] \arrow[dr,swap,"h"] & & X \arrow[dl,"r"] \\
        & B.
      \end{tikzcd}
    \end{equation*}
    commutes, and that $f$ has a retraction if and only if $h$ has a retraction.
  \item (The \define{3-for-2 property} for equivalences.) Show that if any two of the functions
    \begin{equation*}
      f,\qquad g,\qquad h
    \end{equation*}
    are equivalences, then so is the third.
  \end{subexenum}
\item \label{ex:neg_equiv} 
  \begin{subexenum}
  \item Define the negation function on the booleans, and show that it is an equivalence.\index{negation function!is an equivalence}
  \item Use the observational equality on the booleans, defined in \cref{ex:obs_bool}, to show that $\btrue\neq\bfalse$.
  \item Show that for any $b:\bool$, the constant function $\mathsf{const}_b$ is not an equivalence.
  \end{subexenum}
\item \label{ex:succ_equiv} Show that the successor function on the integers is an equivalence.\index{successor function!on Z@{on $\Z$}!is an equivalence}
\item \label{ex:comm_prod}Construct a equivalences $\eqv{A+B}{B+A}$ and $\eqv{A\times B}{B\times A}$.\index{coproduct!is symmetric}
\item \label{ex:retr_id} Consider a section-retraction pair
  \begin{equation*}
    \begin{tikzcd}
      A \arrow[r,"i"] & B \arrow[r,"r"] & A,
    \end{tikzcd}
  \end{equation*}
  with $H:r\circ i\htpy \idfunc$. Show that $\id{x}{y}$ is a retract of $\id{i(x)}{i(y)}$.\index{retract!identity types of}
\item \label{ex:sigma_assoc}\index{Sigma type@{$\Sigma$-type}!associativity of}Let $B$ be a family of types over $A$, and let $C$ be a family of types indexed by $x:A,y:B(x)$. Construct an equivalence
  \begin{equation*}
    \Sigma\mathsf{\usc{}assoc}:\eqv{\Big(\sm{p:\sm{x:A}B(x)}C(\proj 1(p),\proj 2(p))\Big)}{\Big(\sm{x:A}\sm{y:B(x)}C(x,y)\Big)}.
  \end{equation*}
\item \label{ex:sigma_swap}Let $A$ and $B$ be types, and let $C$ be a family over $x:A,y:B$. Construct an equivalence
  \begin{equation*}
    \Sigma\mathsf{\usc{}swap}:\eqv{\Big(\sm{x:A}{y:B}C(x,y)\Big)}{\Big(\sm{y:B}{x:A}C(x,y)\Big)}.
  \end{equation*}
  %\item \label{ex:sigma_base_equiv}Consider an equivalence $e:A'\eqv A$ and a type family $B$ over $A$. Show that the map
  %\begin{equation*}
  %\lam{(x',y)}(e(x'),y) : \Big(\sm{x':A'}B(e(x'))\Big)\to\Big(\sm{x:A}B(x)\Big)
  %\end{equation*}
  %is an equivalence.
\item \label{ex:int_group_laws}\index{Z@{$\Z$}!group laws} In this exercise we will show that the laws for abelian groups hold for addition on the integers. Note: these are obvious facts, but the proof terms that show \emph{how} the group laws hold are nevertheless fairly involved. This exercise is perfect for a formalization project. 
  \begin{subexenum}
  \item Show that addition satisfies the left and right unit laws, i.e., construct terms
    \begin{align*}
      \mathsf{left\usc{}unit\usc{}law\usc{}add\usc{}}\Z  & : \prd{x:\Z} 0 + x = x \\
      \mathsf{right\usc{}unit\usc{}law\usc{}add\usc{}}\Z  & : \prd{x : \Z} x + 0 = x.
    \end{align*}
  \item Show that addition respects predecessors and successor on both sides, i.e., construct terms
    \begin{align*}
      \mathsf{left\usc{}predecessor\usc{}law\usc{}add\usc{}}\Z & : \prd{x,y:\Z} \mathsf{pred}_{\mathbb{Z}}(x)+y = \mathsf{pred}_{\mathbb{Z}}(x+y) \\
      \mathsf{right\usc{}predecessor\usc{}law\usc{}add\usc{}}\Z & : \prd{x,y:\Z} x+\mathsf{pred}_{\mathbb{Z}}(y) = \mathsf{pred}_{\mathbb{Z}}(x+y) \\
      \mathsf{left\usc{}successor\usc{}law\usc{}add\usc{}}\Z & : \prd{x,y:\Z} \mathsf{succ}_{\mathbb{Z}}(x)+y = \mathsf{succ}_{\mathbb{Z}}(x+y) \\
      \mathsf{right\usc{}successor\usc{}law\usc{}add\usc{}}\Z & : \prd{x,y:\Z} x+\mathsf{succ}_{\mathbb{Z}}(y) = \mathsf{succ}_{\mathbb{Z}}(x+y).
    \end{align*}
    Hint: to avoid an excessive number of cases, use induction on $x$ but not on $y$. You may need to use the homotopies $\mathsf{succ}_{\mathbb{Z}}\circ \mathsf{pred}_{\mathbb{Z}}\htpy \idfunc$ and $\mathsf{pred}_{\mathbb{Z}}\circ\mathsf{succ}_{\mathbb{Z}}$ constructed in exercise \cref{ex:succ_equiv}.
  \item Use part (b) to show that addition on the integers is associative and commutative, i.e., construct terms
    \begin{align*}
      \mathsf{assoc\usc{}add\usc{}}\Z & : \prd{x,y,z:\Z} (x+y)+z = x + (y+z) \\
      \mathsf{comm\usc{}add\usc{}}\Z & : \prd{x,y:\Z} x+y = y+x.
    \end{align*}
    Hint: Especially in the construction of the associator there is a risk of running into an unwieldy amount of cases if you use $\Z$-induction on all arguments. Avoid induction on $y$ and $z$.
  \item Show that addition satisfies the left and right inverse laws:
    \begin{align*}
      \mathsf{left\usc{}inverse\usc{}law\usc{}add\usc{}}\Z & : \prd{x : \Z} (-x)+x=0 \\
      \mathsf{right\usc{}inverse\usc{}law\usc{}add\usc{}}\Z & : \prd{x : \Z} x + (-x)=0.
    \end{align*}
    Conclude that the functions $y \mapsto x + y$ and $x\mapsto x + y$ are equivalences for any $x:\Z$ and $y:\Z$, respectively.
  \end{subexenum}
\item \label{ex:coproduct_functor}In this exercise we will construct the \define{functorial action} of coproducts.
  \begin{subexenum}
  \item Construct for any two maps $f:A \to A'$ and $g:B \to B'$, a map
    \begin{equation*}
      f+g:A+B \to A'+B'.
    \end{equation*}
  \item Show that if $H:f\htpy f'$ and $K:g\htpy g'$, then there is a homotopy
    \begin{equation*}
      H+K:(f+g)\htpy (f'+g').
    \end{equation*}
  \item Show that $\idfunc[A]+\idfunc[B]\htpy \idfunc[A+B]$.
  \item Show that for any $f:A\to A'$, $f':A'\to A''$, $g:B\to B'$, and $g':B'\to B''$ there is a homotopy
    \begin{equation*}
      (f'\circ f)+(g'\circ g) \htpy (f'+g')\circ (f\circ g).
    \end{equation*}
  \item \label{ex:coproduct_functor_equivalence}Show that if $f$ and $g$ are equivalences, then so is $f+g$. (The converse of this statement also holds, see \cref{ex:is-equiv-is-equiv-functor-coprod}.)
  \end{subexenum}
\item Construct equivalences
  \begin{align*}
    \mathsf{Fin}(m+n) & \simeq \mathsf{Fin}(m)+\mathsf{Fin}(n) \\
    \mathsf{Fin}(mn) & \simeq \mathsf{Fin}(m)\times\mathsf{Fin}(n).
  \end{align*}
\end{exercises}


% !TEX root = hott_intro.tex

\section{The fundamental theorem of identity types}\label{chap:fundamental}
\sectionmark{The fundamental theorem}

\index{fundamental theorem of identity types|(}
\index{characterization of identity type!fundamental theorem of identity types|(}
For many types it is useful to have a characterization of their identity types. For example, we have used a characterization of the identity types of the fibers of a map in order to conclude that any equivalence is a contractible map. The fundamental theorem of identity types is our main tool to carry out such characterizations, and with the fundamental theorem it becomes a routine task to characterize an identity type whenever that is of interest.

In our first application of the fundamental theorem of identity types we show that any equivalence is an embedding. Embeddings are maps that induce equivalences on identity types, i.e., they are the homotopical analogue of injective maps. In our second application we characterize the identity types of coproducts.

Throughout the rest of this book we will encounter many more occasions to characterize identity types. For example, we will show in \cref{thm:eq_nat} that the identity type of the natural numbers is equivalent to its observational equality, and we will show in \cref{thm:eq-circle} that the loop space of the circle is equivalent to $\Z$.

In order to prove the fundamental theorem of identity types, we first prove the basic fact that a family of maps is a family of equivalences if and only if it induces an equivalence on total spaces. 

\subsection{Families of equivalences}

\index{family of equivalences|(}
\begin{defn}
Consider a family of maps
\begin{equation*}
f : \prd{x:A}B(x)\to C(x).
\end{equation*}
We define the map\index{total(f)@{$\tot{f}$}}
\begin{equation*}
\tot{f}:\sm{x:A}B(x)\to\sm{x:A}C(x)
\end{equation*}
by $\lam{(x,y)}(x,f(x,y))$.
\end{defn}

\begin{lem}\label{lem:fib_total}
  For any family of maps $f:\prd{x:A}B(x)\to C(x)$ and any $t:\sm{x:A}C(x)$,
  there is an equivalence\index{fiber!of total(f)@{of $\tot{f}$}}\index{total(f)@{$\tot{f}$}!fiber}
  \begin{equation*}
    \eqv{\fib{\tot{f}}{t}}{\fib{f(\proj 1(t))}{\proj 2(t)}}.
  \end{equation*}
\end{lem}

\begin{proof}
  For any $p:\fib{\tot{f}}{t}$ we define $\varphi(t,p):\fib{\proj 1(t)}{\proj 2(t)}$ by $\Sigma$-induction on $p$. Therefore it suffices to define $\varphi(t,(s,\alpha)):\fib{\proj 1(t)}{\proj 2 (t)}$ for any $s:\sm{x:A}B(x)$ and $\alpha:\tot{f}(s)=t$. Now we proceed by path induction on $\alpha$, so it suffices to define $\varphi(\tot{f}(s),(s,\refl{})):\fib{f(\proj 1(\tot{f}(s)))}{\proj 2(\tot{f}(s))}$. Finally, we use $\Sigma$-induction on $s$ once more, so it suffices to define
  \begin{equation*}
    \varphi((x,f(x,y)),((x,y),\refl{})):\fib{f(x)}{f(x,y)}.
  \end{equation*}
  Now we take as our definition
  \begin{equation*}
    \varphi((x,f(x,y)),((x,y),\refl{}))\defeq(y,\refl{}).
  \end{equation*}

  For the proof that this map is an equivalence we construct a map
  \begin{equation*}
    \psi(t) : \fib{f(\proj 1(t))}{\proj 2(t)}\to\fib{\tot{f}}{t}
  \end{equation*}
  equipped with homotopies $G(t):\varphi(t)\circ\psi(t)\htpy\idfunc$ and $H(t):\psi(t)\circ\varphi(t)\htpy\idfunc$. In each of these definitions we use $\Sigma$-induction and path induction all the way through, until an obvious choice of definition becomes apparent. We define $\psi(t)$, $G(t)$, and $H(t)$ as follows:
  \begin{align*}
    \psi((x,f(x,y)),(y,\refl{})) & \defeq ((x,y),\refl{}) \\
    G((x,f(x,y)),(y,\refl{})) & \defeq \refl{} \\
    H((x,f(x,y)),((x,y),\refl{})) & \defeq \refl{}.\qedhere
  \end{align*}
\end{proof}

\begin{thm}\label{thm:fib_equiv}
  Let $f:\prd{x:A}B(x)\to C(x)$ be a family of maps. The following are equivalent:
  \index{is an equivalence!total(f) of family of equivalences@{$\tot{f}$ of family of equivalences}}
  \index{total(f)@{$\tot{f}$}!of family of equivalences is an equivalence}\index{is family of equivalences!if total(f) is an equivalence@{iff $\tot{f}$ is an equivalence}}
\begin{enumerate}
\item For each $x:A$, the map $f(x)$ is an equivalence. In this case we say that $f$ is a \define{family of equivalences}.
\item The map $\tot{f}:\sm{x:A}B(x)\to\sm{x:A}C(x)$ is an equivalence.
\end{enumerate}
\end{thm}

\begin{proof}
By \cref{thm:equiv_contr,thm:contr_equiv} it suffices to show that $f(x)$ is a contractible map for each $x:A$, if and only if $\tot{f}$ is a contractible map. Thus, we will show that $\fib{f(x)}{c}$ is contractible if and only if $\fib{\tot{f}}{x,c}$ is contractible, for each $x:A$ and $c:C(x)$. However, by \cref{lem:fib_total} these types are equivalent, so the result follows by \cref{ex:contr_equiv}.
\end{proof}

Now consider the situation where we have a map $f:A\to B$, and a family $C$ over $B$. Then we have the map
\begin{equation*}
  \lam{(x,z)}(f(x),z):\sm{x:A}C(f(x))\to\sm{y:B}C(y).
\end{equation*}
We claim that this map is an equivalence when $f$ is an equivalence. The technique to prove this claim is the same as the technique we used in \cref{thm:fib_equiv}: first we note that the fibers are equivalent to the fibers of $f$, and then we use the fact that a map is an equivalence if and only if its fibers are contractible to finish the proof.

The converse of the following lemma does not hold. Why not?

\begin{lem}\label{lem:total-equiv-base-equiv}
  Consider an equivalence $e:A\simeq B$, and let $C$ be a type family over $B$. Then the map
  \begin{equation*}
    \sigma_f(C) \defeq\lam{(x,z)}(f(x),z):\sm{x:A}C(f(x))\to\sm{y:B}C(y)
  \end{equation*}
  is an equivalence.
\end{lem}

\begin{proof}
  We claim that for each $t:\sm{y:B}C(y)$ there is an equivalence
  \begin{equation*}
    \fib{\sigma_f(C)}{t}\simeq \fib{f}{\proj 1(t)}.
  \end{equation*}
  We obtain such an equivalence by constructing the following functions and homotopies:
  \begin{align*}
    \varphi(t) & : \fib{\sigma_f(C)}{t}\to\fib{f}{\proj 1 (t)} & \varphi((f(x),z),((x,z),\refl{})) & \defeq (x,\refl{}) \\
    \psi(t) & : \fib{f}{\proj 1(t)} \to\fib{\sigma_f(C)}{t} & \psi((f(x),z),(x,\refl{})) & \defeq ((x,z),\refl{}) \\
    G(t) & : \varphi(t)\circ\psi(t)\htpy\idfunc & G((f(x),z),(x,\refl{})) & \defeq \refl{} \\
    H(t) & : \psi(t)\circ\varphi(t)\htpy\idfunc & H((f(x),z),((x,z),\refl{})) & \defeq \refl{}.
  \end{align*}
  Now the claim follows, since we see that $\varphi$ is a contractible map if and only if $f$ is a contractible map.
\end{proof}

We now combine \cref{thm:fib_equiv,lem:total-equiv-base-equiv}.

\begin{defn}
  Consider a map $f:A\to B$ and a family of maps
  \begin{equation*}
    g:\prd{x:A}C(x)\to D(f(x)),
  \end{equation*}
  where $C$ is a type family over $A$, and $D$ is a type family over $B$. In this situation we also say that $g$ is a \define{family of maps over $f$}. Then we define\index{total f(g)@{$\tot[f]{g}$}}
  \begin{equation*}
    \tot[f]{g}:\sm{x:A}C(x)\to\sm{y:B}D(y)
  \end{equation*}
  by $\tot[f]{g}(x,z)\defeq (f(x),g(x,z))$.
\end{defn}

\begin{thm}\label{thm:equiv-toto}
  Suppose that $g$ is a family of maps over $f$, and suppose that $f$ is an equivalence. Then the following are equivalent:
  \begin{enumerate}
  \item The family of maps $g$ over $f$ is a family of equivalences.
  \item The map $\tot[f]{g}$ is an equivalence.
  \end{enumerate}
\end{thm}

\begin{proof}
  Note that we have a commuting triangle
  \begin{equation*}
    \begin{tikzcd}[column sep=0]
      \sm{x:A}C(x) \arrow[rr,"{\tot[f]{g}}"] \arrow[dr,swap,"\tot{g}"]& & \sm{y:B}D(y) \\
      & \sm{x:A}D(f(x)) \arrow[ur,swap,"{\lam{(x,z)}(f(x),z)}"]
    \end{tikzcd}
  \end{equation*}
  By the assumption that $f$ is an equivalence, it follows that the map $\sm{x:A}D(f(x))\to \sm{y:B}D(y)$ is an equivalence. Therefore it follows that $\tot[f]{g}$ is an equivalence if and only if $\tot{g}$ is an equivalence. Now the claim follows, since $\tot{g}$ is an equivalence if and only if $g$ if a family of equivalences.
\end{proof}
\index{family of equivalences|)}

\subsection{The fundamental theorem}

\index{identity system|(}
Many types come equipped with a reflexive relation that possesses a similar
structure as the identity type. The observational equality on the natural
numbers is such an example. We have see that it is a reflexive, symmetric, and
transitive relation, and moreover it is contained in any other reflexive
relation. Thus, it is natural to ask whether observational equality on the natural numbers is equivalent to the identity type.

The fundamental theorem of identity types (\cref{thm:id_fundamental}) is a general theorem that can be used to answer such questions. It describes a necessary and sufficient condition on a type family $B$ over a type $A$ equipped with a point $a:A$, for there to be a family of equivalences $\prd{x:A}(a=x)\simeq B(x)$. In other words, it tells us when a family $B$ is a characterization of the identity type of $A$.

Before we state the fundamental theorem of identity types we introduce the notion of \emph{identity systems}. Those are families $B$ over a $A$ that satisfy an induction principle that is similar to the path induction principle, where the `computation rule' is stated with an identification.

\begin{defn}
  Let $A$ be a type equipped with a term $a:A$. A \define{(unary) identity system} on $A$ at $a$ consists of a type family $B$ over $A$ equipped with $b:B(a)$, such that for any family of types $P(x,y)$ indexed by $x:A$ and $y:B(x)$,
  the function
  \begin{equation*}
    h\mapsto h(a,b):\Big(\prd{x:A}\prd{y:B(x)}P(x,y)\Big)\to P(a,b)
  \end{equation*}
  has a section.
\end{defn}

The most important implication in the fundamental theorem is that (ii) implies (i). Occasionally we will also use the third equivalent statement. We note that the fundamental theorem also appears as Theorem 5.8.4 in \cite{hottbook}.

\begin{thm}\label{thm:id_fundamental}
Let $A$ be a type with $a:A$, and let $B$ be be a type family over $A$ with $b:B(a)$.
Then  the following are logically equivalent for any family of maps
\begin{equation*}
  f:\prd{x:A}(a=x)\to B(x).
\end{equation*}
\begin{enumerate}
\item The family of maps $f$ is a family of equivalences.
\item The total space\index{is contractible!total space of an identity system}
\begin{equation*}
\sm{x:A}B(x)
\end{equation*}
is contractible.
\item The family $B$ is an identity system.
\end{enumerate}
In particular the canonical family of maps
\begin{equation*}
\pathind_a(b):\prd{x:A} (a=x)\to B(x)
\end{equation*}
is a family of equivalences if and only if $\sm{x:A}B(x)$ is contractible.
\end{thm}

\begin{proof}
  First we show that (i) and (ii) are equivalent.
  By \cref{thm:fib_equiv} it follows that the family of maps $f$ is a family of equivalences if and only if it induces an equivalence
  \begin{equation*}
    \eqv{\Big(\sm{x:A}a=x\Big)}{\Big(\sm{x:A}B(x)\Big)}
  \end{equation*}
  on total spaces. We have that $\sm{x:A}a=x$ is contractible. Now it follows by \cref{ex:contr_equiv}, applied in the case
  \begin{equation*}
    \begin{tikzcd}[column sep=3em]
      \sm{x:A}a=x \arrow[rr,"\tot{f}"] \arrow[dr,swap,"\eqvsym"] & & \sm{x:A}B(x) \arrow[dl] \\
      & \unit & \phantom{\sm{x:A}a=x}
    \end{tikzcd}
  \end{equation*}
  that $\tot{f}$ is an equivalence if and only if $\sm{x:A}B(x)$ is contractible.

  Now we show that (ii) and (iii) are equivalent. Note that we have the following commuting triangle
  \begin{equation*}
    \begin{tikzcd}[column sep=0]
      \prd{t:\sm{x:A}B(x)}P(t) \arrow[rr,"\evpair"] \arrow[dr,swap,"{\evpt(a,b)}"] & & \prd{x:A}\prd{y:B(x)}P(x,y) \arrow[dl,"{\lam{h}h(a,b)}"] \\
      \phantom{\prd{x:A}\prd{y:B(x)}P(x,y)} & P(a,b)
    \end{tikzcd}
  \end{equation*}
  In this diagram the top map has a section. Therefore it follows by \cref{ex:3_for_2} that the left map has a section if and only if the right map has a section. Notice that the left map has a section for all $P$ if and only if $\sm{x:A}B(x)$ satisfies singleton induction, which is by \cref{thm:contractible} equivalent to $\sm{x:A}B(x)$ being contractible.
\end{proof}
\index{identity system|)}

\subsection{Embeddings}
\index{embedding|(}
As an application of the fundamental theorem we show that equivalences are embeddings. The notion of embedding is the homotopical analogue of the set theoretic notion of injective map.

\begin{defn}
An \define{embedding} is a map $f:A\to B$\index{is an embedding} satisfying the property that\index{is an equivalence!action on paths of an embedding}
\begin{equation*}
\apfunc{f}:(\id{x}{y})\to(\id{f(x)}{f(y)})
\end{equation*}
is an equivalence for every $x,y:A$. We write $\isemb(f)$\index{is-emb(f)@{$\isemb(f)$}} for the type of witnesses that $f$ is an embedding.
\end{defn}

Another way of phrasing the following statement is that equivalent types have equivalent identity types.

\begin{thm}
\label{cor:emb_equiv} 
Any equivalence is an embedding.\index{is an embedding!equivalence}\index{equivalence!is an embedding}
\end{thm}

\begin{proof}
Let $e:\eqv{A}{B}$ be an equivalence, and let $x:A$. Our goal is to show that
\begin{equation*}
\apfunc{e} : (\id{x}{y})\to (\id{e(x)}{e(y)})
\end{equation*}
is an equivalence for every $y:A$. By \cref{thm:id_fundamental} it suffices to show that 
\begin{equation*}
\sm{y:A}e(x)=e(y)
\end{equation*}
is contractible for every $y:A$. Now observe that there is an equivalence
\begin{samepage}
\begin{align*}
\sm{y:A}e(x)=e(y) & \eqvsym \sm{y:A}e(y)=e(x) \\
& \jdeq \fib{e}{e(x)}
\end{align*}
\end{samepage}
by \cref{thm:fib_equiv}, since for each $y:A$ the map
\begin{equation*}
\invfunc : (e(x)=e(y))\to (e(y)= e(x))
\end{equation*}
is an equivalence by \cref{ex:equiv_grpd_ops}.
The fiber $\fib{e}{e(x)}$ is contractible by \cref{thm:contr_equiv}, so it follows by \cref{ex:contr_equiv} that the type $\sm{y:A}e(x)=e(y)$ is indeed contractible.
\end{proof}
\index{embedding|)}

\subsection{Disjointness of coproducts}

\index{disjointness of coproducts|(}
\index{characterization of identity type!coproduct|(}
\index{identity type!coproduct|(}
\index{coproduct!identity type|(}
\index{coproduct!disjointness|(}
To give a second application of the fundamental theorem of identity types, we characterize the identity types of coproducts. Our goal in this section is to prove the following theorem.

\begin{thm}\label{thm:id-coprod-compute}
Let $A$ and $B$ be types. Then there are equivalences
\begin{align*}
(\inl(x)=\inl(x')) & \eqvsym (x = x')\\
(\inl(x)=\inr(y')) & \eqvsym \emptyt \\
(\inr(y)=\inl(x')) & \eqvsym \emptyt \\
(\inr(y)=\inr(y')) & \eqvsym (y=y')
\end{align*}
for any $x,x':A$ and $y,y':B$.
\end{thm}

In order to prove \cref{thm:id-coprod-compute}, we first define
a binary relation $\Eqcoprod_{A,B}$ on the coproduct $A+B$.

\begin{defn}
Let $A$ and $B$ be types. We define 
\begin{equation*}
\Eqcoprod_{A,B} : (A+B)\to (A+B)\to\UU
\end{equation*}
by double induction on the coproduct, postulating
\begin{align*}
\Eqcoprod_{A,B}(\inl(x),\inl(x')) & \defeq (x=x') \\
\Eqcoprod_{A,B}(\inl(x),\inr(y')) & \defeq \emptyt \\
\Eqcoprod_{A,B}(\inr(y),\inl(x')) & \defeq \emptyt \\
\Eqcoprod_{A,B}(\inr(y),\inr(y')) & \defeq (y=y')
\end{align*}
The relation $\Eqcoprod_{A,B}$ is also called the \define{observational equality of coproducts}\index{observational equality!of coproducts}.
\end{defn}

\begin{lem}
The observational equality relation $\Eqcoprod_{A,B}$ on $A+B$ is reflexive, and therefore there is a map
\begin{equation*}
\Eqcoprodeq:\prd{s,t:A+B} (s=t)\to \Eqcoprod_{A,B}(s,t)
\end{equation*}
\end{lem}

\begin{constr}
The reflexivity term $\rho$ is constructed by induction on $t:A+B$, using
\begin{align*}
\rho(\inl(x))\defeq \refl{\inl(x)}  & : \Eqcoprod_{A,B}(\inl(x)) \\
\rho(\inr(y))\defeq \refl{\inr(y)} & : \Eqcoprod_{A,B}(\inr(y)).\qedhere
\end{align*}
\end{constr}

To show that $\Eqcoprodeq$ is a family of equivalences, we will use the fundamental theorem, \cref{thm:id_fundamental}. Moreover, we will use the functoriality of coproducts (established in \cref{ex:coproduct_functor}), and the fact that any total space over a coproduct is again a coproduct:
\begin{align*}
\sm{t:A+B}P(t) & \eqvsym \Big(\sm{x:A}P(\inl(x))\Big)+\Big(\sm{y:B}P(\inr(y))\Big)
\end{align*}
All of these equivalences are straightforward to construct, so we leave them as an exercise to the reader. 

\begin{lem}\label{lem:is-contr-total-eq-coprod}
For any $s:A+B$ the total space
\begin{equation*}
\sm{t:A+B}\Eqcoprod_{A,B}(s,t)
\end{equation*}
is contractible.
\end{lem}

\begin{proof}
We will do the proof by induction on $s$. The two cases are similar, so we only show that the total space
\begin{equation*}
\sm{t:A+B}\Eqcoprod_{A,B}(\inl(x),t)
\end{equation*}
is contractible. Note that we have equivalences
\begin{samepage}
\begin{align*}
& \sm{t:A+B}\Eqcoprod_{A,B}(\inl(x),t) \\
& \eqvsym \Big(\sm{x':A}\Eqcoprod_{A,B}(\inl(x),\inl(x'))\Big)+\Big(\sm{y':B}\Eqcoprod_{A,B}(\inl(x),\inr(y'))\Big) \\
& \eqvsym \Big(\sm{x':A}x=x'\Big)+\Big(\sm{y':B}\emptyt\Big) \\
& \eqvsym \Big(\sm{x':A}x=x'\Big)+\emptyt \\
& \eqvsym \sm{x':A}x=x'.
\end{align*}%
\end{samepage}%
In the last two equivalences we used \cref{ex:unit-laws-coprod}. This shows that the total space is contractible, since the latter type is contractible by \cref{thm:total_path}.
\end{proof}

\begin{proof}[Proof of \cref{thm:id-coprod-compute}]
The proof is now concluded with an application of \cref{thm:id_fundamental}, using \cref{lem:is-contr-total-eq-coprod}.
\end{proof}
\index{disjointness of coproducts|)}
\index{characterization of identity type!coproduct|)}
\index{identity type!coproduct|)}
\index{coproduct!identity type|)}
\index{coproduct!disjointness|)}

\begin{exercises}
  \exercise
  \begin{subexenum}
  \item \label{ex:is-emb-empty}Show that the map $\emptyt\to A$ is an embedding for every type $A$.\index{is an embedding!0 to A@{$\emptyt\to A$}}
  \item \label{ex:is-emb-inl-inr}Show that $\inl:A\to A+B$ and $\inr:B\to A+B$ are embeddings for any two types $A$ and $B$.
    \index{is an embedding!inl (for coproducts)@{$\inl$ (for coproducts)}}
    \index{is an embedding!inr (for coproducts)@{$\inr$ (for coproducts)}}
    \index{inl@{$\inl$}!is an embedding}
    \index{inr@{$\inr$}!is an embedding}
  \end{subexenum}
  \exercise Consider an equivalence $e:A\simeq B$. Construct an equivalence
  \begin{equation*}
    (e(x)=y)\simeq(x=e^{-1}(y))
  \end{equation*}
  for every $x:A$ and $y:B$.
  \exercise Show that\index{embedding!closed under homotopies}
  \begin{equation*}
    (f\htpy g)\to (\isemb(f)\leftrightarrow\isemb(g))
  \end{equation*}
  for any $f,g:A\to B$.
  \exercise \label{ex:emb_triangle}Consider a commuting triangle
  \begin{equation*}
    \begin{tikzcd}[column sep=tiny]
      A \arrow[rr,"h"] \arrow[dr,swap,"f"] & & B \arrow[dl,"g"] \\
      & X
    \end{tikzcd}
  \end{equation*}
  with $H:f\htpy g\circ h$. 
  \begin{subexenum}
  \item Suppose that $g$ is an embedding. Show that $f$ is an embedding if and only if $h$ is an embedding.\index{is an embedding!composite of embeddings}\index{is an embedding!right factor of embedding if left factor is an embedding}
  \item Suppose that $h$ is an equivalence. Show that $f$ is an embedding if and only if $g$ is an embedding.\index{is an embedding!left factor of embedding if right factor is an equivalence}
  \end{subexenum}
  \exercise \label{ex:is-equiv-is-equiv-functor-coprod}Consider two maps $f:A\to A'$ and $g:B \to B'$.
  \begin{subexenum}
  \item Show that if the map
    \begin{equation*}
      f+g:(A+B)\to (A'+B')
    \end{equation*}
    is an equivalence, then so are both $f$ and $g$ (this is the converse of \cref{ex:coproduct_functor_equivalence}).
  \item \label{ex:is-emb-coprod}Show that $f+g$ is an embedding if and only if both $f$ and $g$ are embeddings.
  \end{subexenum}
  \exercise \label{ex:htpy_total} 
  \begin{subexenum}
  \item Let $f,g:\prd{x:A}B(x)\to C(x)$ be two families of maps. Show that
    \begin{equation*}
      \Big(\prd{x:A}f(x)\htpy g(x)\Big)\to \Big(\tot{f}\htpy \tot{g}\Big). 
    \end{equation*}
  \item Let $f:\prd{x:A}B(x)\to C(x)$ and let $g:\prd{x:A}C(x)\to D(x)$. Show that
    \begin{equation*}
      \tot{\lam{x}g(x)\circ f(x)}\htpy \tot{g}\circ\tot{f}.
    \end{equation*}
  \item For any family $B$ over $A$, show that
    \begin{equation*}
      \tot{\lam{x}\idfunc[B(x)]}\htpy\idfunc.
    \end{equation*}
  \end{subexenum}
  \exercise \label{ex:id_fundamental_retr}Let $a:A$, and let $B$ be a type family over $A$. 
  \begin{subexenum}
  \item Use \cref{ex:htpy_total,ex:contr_retr} to show that if each $B(x)$ is a retract of $\id{a}{x}$, then $B(x)$ is equivalent to $\id{a}{x}$ for every $x:A$.
    \index{fundamental theorem of identity types!formulation with retractions}
  \item Conclude that for any family of maps
    \index{fundamental theorem of identity types!formulation with sections}
    \begin{equation*}
      f : \prd{x:A} (a=x) \to B(x),
    \end{equation*}
    if each $f(x)$ has a section, then $f$ is a family of equivalences.
  \end{subexenum}
  \exercise Use \cref{ex:id_fundamental_retr} to show that for any map $f:A\to B$, if
  \begin{equation*}
    \apfunc{f} : (x=y) \to (f(x)=f(y))
  \end{equation*}
  has a section for each $x,y:A$, then $f$ is an embedding.\index{is an embedding!if the action on paths have sections}
  \exercise \label{ex:path-split}We say that a map $f:A\to B$ is \define{path-split}\index{path-split} if $f$ has a section, and for each $x,y:A$ the map
  \begin{equation*}
    \apfunc{f}(x,y):(x=y)\to (f(x)=f(y))
  \end{equation*}
  also has a section. We write $\pathsplit(f)$\index{path-split(f)@{$\pathsplit(f)$}} for the type
  \begin{equation*}
    \sections(f)\times\prd{x,y:A}\sections(\apfunc{f}(x,y)).
  \end{equation*}
  Show that for any map $f:A\to B$ the following are equivalent:
  \begin{enumerate}
  \item The map $f$ is an equivalence.
  \item The map $f$ is path-split.
  \end{enumerate}
  \exercise \label{ex:fiber_trans}Consider a triangle
  \begin{equation*}
    \begin{tikzcd}[column sep=small]
      A \arrow[rr,"h"] \arrow[dr,swap,"f"] & & B \arrow[dl,"g"] \\
      & X
    \end{tikzcd}
  \end{equation*}
  with a homotopy $H:f\htpy g\circ h$ witnessing that the triangle commutes. 
  \begin{subexenum}
  \item Construct a family of maps
    \begin{equation*}
      \fibtriangle(h,H):\prd{x:X}\fib{f}{x}\to\fib{g}{x},
    \end{equation*}
    for which the square
    \begin{equation*}
      \begin{tikzcd}[column sep=8em]
        \sm{x:X}\fib{f}{x} \arrow[r,"\tot{\fibtriangle(h,H)}"] \arrow[d] & \sm{x:X}\fib{g}{x} \arrow[d] \\
        A \arrow[r,swap,"h"] & B
      \end{tikzcd}
    \end{equation*}
    commutes, where the vertical maps are as constructed in \cref{ex:fib_replacement}.
  \item Show that $h$ is an equivalence if and only if $\fibtriangle(h,H)$ is a family of equivalences.
  \end{subexenum}
\end{exercises}
\index{fundamental theorem of identity types|)}
\index{characterization of identity type!fundamental theorem of identity types|)}

\endinput

  \begin{comment}
    \exercise \label{ex:eqv_sigma_mv}Consider a map
    \begin{equation*}
      f:A \to \sm{y:B}C(y).
    \end{equation*}
    \begin{subexenum}
    \item Construct a family of maps
      \begin{equation*}
        f':\prd{y:B} \fib{\proj 1\circ f}{y}\to C(y).
      \end{equation*}
    \item Construct an equivalence
      \begin{equation*}
        \eqv{\fib{f'(b)}{c}}{\fib{f}{(b,c)}}
      \end{equation*}
      for every $(b,c):\sm{y:B}C(y)$.
    \item Conclude that the following are equivalent:
      \begin{enumerate}
      \item $f$ is an equivalence.
      \item $f'$ is a family of equivalences.
      \end{enumerate}
    \end{subexenum}
    \exercise \label{ex:coh_intro}Consider a type $A$ with base point $a:A$, and let $B$ be a type family on $A$ that implies the identity type, i.e., there is a term
    \begin{equation*}
      \alpha : \prd{x:A} B(x)\to (a=x).
    \end{equation*}
    Show that the \define{coherence reduction map}
    \begin{equation*}
      \cohreduction : \Big(\sm{y:B(a)}\alpha(a,y)=\refl{a}\Big) \to \Big(\sm{x:A}B(x)\Big)
    \end{equation*}
    defined by $\lam{(y,q)}(a,y)$ is an equivalence.
  \end{comment}


\chapter{The hierarchy of homotopical complexity}
\chaptermark{Homotopical complexity}
%Not all types have interesting higher groupoid structure. For example, we will see below that two natural numbers can only be equal in at most one way. Voevodsky articulated a useful notion to detect the homotopical complexity of types, which allows us to distinguish between contractible types (also called \emph{$(-2)$-types}), \emph{propositions} (also called \emph{$(-1)$-types}), \emph{sets} (\emph{$0$-types}), and \emph{$k$-types} for higher $k$.

%We will see [later] that there are types that are not $k$-types for any $k$.

\section{Propositions and subtypes}

\begin{defn}
A type $A$ is said to be a \define{proposition} if there is a term of type
\begin{equation*}
\isprop(A)\defeq\prd{x,y:A}\iscontr(x=y).
\end{equation*}
\end{defn}

In the following lemma we prove that in order to show that a type $A$ is a proposition, it suffices to show that any two terms of $A$ are equal.

\begin{lem}\label{lem:isprop_eq}
Let $A$ be a type. Then we have
\begin{equation*}
\isprop(A)\leftrightarrow \Big(\prd{x,y:A}x=y\Big).
\end{equation*}
\end{lem}

\begin{proof}
Suppose $A$ is a proposition. By taking the center of contraction of $\id{x}{y}$ for each $x,y:A$ we obtain a term of type $\prd{x,y:A}\id{x}{y}$.

Now suppose that $A$ is a type equipped with $H:\prd{x,y:A}\id{x}{y}$. Then we take $\ct{H(x,x)^{-1}}{H(x,y)}$ as the center of contraction of $\id{x}{y}$. To construct the contraction
\begin{equation*}
\prd{p:\id{x}{y}} \ct{H(x,x)^{-1}}{H(x,y)}=p
\end{equation*}
we proceed by path induction. Our goal is to show that
\begin{equation*}
\ct{H(x,x)^{-1}}{H(x,x)}=\refl{x}.\qedhere
\end{equation*}
\end{proof}

\begin{cor}
For any proposition $P$ we have $P\to\iscontr(P)$.
\end{cor}

\begin{thm}
Let $A$ be a type and let $P:A\to\type$ be a family of propositions, in the sense that each $P(x)$ is a proposition. Furthermore, consider $\pairr{x,p},\pairr{y,q}:\sm{x:A}P(x)$. Then the map
\begin{equation*}
\mathsf{subtype\usc{}eq}(x,y) : (\id{\pairr{x,p}}{\pairr{y,q}})\to(\id{x}{y})
\end{equation*}
defined by path induction by sending $\refl{\pairr{x,p}}$ to $\refl{x}$, is an equivalence.
\end{thm}

\begin{proof}
By \autoref{thm:id_fundamental} it suffices to show that
\begin{equation*}
\sm{y:A}P(y)\times(\id{x}{y})
\end{equation*}
is contractible, for any $x:A$ and $p:P(x)$. The center of contraction is taken to be $\pairr{x,\pairr{p,\refl{x}}}$. To construct the contraction, observe that for any $y:A$, $q:P(y)$ and $r:\id{x}{y}$, the type $P(y)$ is contractible, so we have an equivalence $\eqv{P(y)\times(\id{x}{y})}{(\id{x}{y})}$. Therefore it suffices to show that $\sm{y:A}\id{x}{y}$ is contractible, which it is.
\end{proof}

Since the equality of $\sm{x:A}P(x)$ corresponds to the equality of $A$, we call $\sm{x:A}P(x)$ a \define{subtype} of $A$.

\section{Sets}

\begin{defn}
A type $A$ is said to be a \define{set} if there is a term of type
\begin{equation*}
\isset(A)\defeq \prd{x,y:A}\isprop(\id{x}{y}).
\end{equation*}
\end{defn}

\begin{lem}
A type $A$ is a set if and only if it satisfies \define{axiom K}, which asserts that
\begin{equation*}
\prd{x:A}{p:\id{x}{x}}\id{\refl{x}}{p}.
\end{equation*}
\end{lem}

\begin{proof}
If $A$ is a set, then $\id{x}{x}$ is a proposition, so any two of its elements are equal. 
This implies axiom $K$. 

For the converse, if $A$ satisfies axiom $K$, then for any $p,q:\id{x}{y}$ we have $\id{\ct{p}{q^{-1}}}{\refl{x}}$, and hence $\id{p}{q}$. This shows that $A$ is a proposition.
\end{proof}

\begin{lem}\label{lem:prop_to_id}
Let $A$ be a type, and let $R:A\to A\to\UU$ be a binary relation on $A$ satisfying
\begin{enumerate}
\item Each $R(x,y)$ is a proposition,
\item $R$ is reflexive, as witnessed by $\rho:\prd{x:A}R(x,x)$.
\end{enumerate}
Then any fiberwise map
\begin{equation*}
\prd{x,y:A}R(x,y)\to (\id{x}{y})
\end{equation*}
is a fiberwise equivalence. Consequently, if there is such a fiberwise map, then $A$ is a set.
\end{lem}

\begin{proof}
Let $f:\prd{x,y:A}R(x,y)\to(\id{x}{y})$. 
Since $R$ is assumed to be reflexive, we also have a fiberwise transformation
\begin{equation*}
\rec{x=}(\rho(x)):\prd{y:A}(\id{x}{y})\to R(x,y).
\end{equation*}
Since each $R(x,y)$ is assumed to be a proposition, it therefore follows that each $R(x,y)$ is a retract of $\id{x}{y}$. We conclude by \autoref{ex:id_fundamental_retr} that for each $x,y:A$, the map $f(x,y):R(x,y)\to(\id{x}{y})$ must be an equivalence. 
\end{proof}

\begin{thm}
The type of natural numbers is a set.
\end{thm}

\begin{proof}
Let $E:\nat\to\nat\to\type$ be the binary relation given by
\begin{align*}
E(0,0) & \defeq \unit & E(S(m),0) & \defeq \emptyt \\
E(0,S(n)) & \defeq \emptyt & E(S(m),S(n)) & \defeq E(m,n).
\end{align*}
Note that this relation is reflexive, and that $E(m,n)$ is a proposition for all $m,n:\nat$, since $\emptyt$ and $\unit$ are propositions.

Thus, it remains to show that
\begin{equation*}
\prd{m,n:\nat}E(m,n)\to (\id{m}{n}).
\end{equation*}
We proceed by induction om $m$. For the base case, we have to show that
\begin{equation*}
\prd{n:\nat}E(0,n)\to (\id{m}{n}).
\end{equation*}
We proceed by induction on $n:\nat$. In the base case we take $\lam{t}\refl{0}$. For the inductive step we just take the recursor $\rec{\emptyt}$ on the empty type. This completes the construction in the case $m\jdeq 0$.

For the induction step, the induction hypothesis is
\begin{equation*}
e:\prd{n:\nat}E(m,n)\to (\id{m}{n}).
\end{equation*}
Our goal is to show that
\begin{equation*}
\prd{n:\nat}E(S(m),n)\to (\id{S(m)}{n}).
\end{equation*}
We proceed by induction on $n:\nat$. In the base case we take the recursor $\rec{\emptyt}$. 
For the inductive step our goal is to show that $E(S(m),S(n))\to (\id{S(m)}{S(n)})$. Let $p:E(S(m),S(n))$. Since $E(S(m),S(n))\jdeq E(m,n)$, we obtain a term of type $\id{m}{n}$. Now we simply apply $\mapfunc{S}:(\id{m}{n})\to(\id{S(m)}{S(n)})$.
\end{proof}

\begin{comment}
\begin{thm}[Hedberg]\label{thm:dec_eq}
Any type with decidable equality is a set.
\end{thm}

\begin{proof}
Let $A$ be a type, and let $d:\prd{x,y:A}(\id{x}{y})+\neg(\id{x}{y})$ be the witness that $A$ has decidable equality.
We first construct a reflexive binary relation $E:A\to A\to\type$ such that each $E(x,y)$ is a proposition.
For every $x,y:A$, we first define a type family $E'(x,y):((\id{x}{y})+\neg(\id{x}{y}))\to\type$ by
\begin{align*}
E'(x,y,\inl(p)) & \defeq \unit \\
E'(x,y,\inr(p)) & \defeq \emptyt.
\end{align*}
Note that $E'(x,y,q)$ is a proposition for each $x,y:A$ and $q:(\id{x}{y})+\neg(\id{x}{y})$. 
Now we set $E(x,y)\defeq E'(x,y,d(x,y))$. Then $E$ is clearly reflexive, and a family of propositions.
Therefore it remains to show that $E$ implies identity. 

Since $E$ is defined as an instance of $E'$, it suffices to construct a term of type
\begin{equation*}
\prd{x,y:A}{q:(\id{x}{y})+\neg(\id{x}{y})} E'(q)\to (\id{x}{y}). 
\end{equation*}
By induction of disjoint sums, it suffices to construct terms of types
\begin{align*}
& \prd{x,y:A}{p:\id{x}{y}} \unit\to (\id{x}{y}) \\
& \prd{x,y:A}{p:\neg(\id{x}{y})} \emptyt\to (\id{x}{y}).
\end{align*}
In the first case, we take $\lam{x}{y}{p}{t}p$, and the second case is by induction on the empty type.
\end{proof}
\end{comment}

\section{General truncation levels}
\begin{defn}
We define $\istrunc{} : \Z_{\geq-2}\to\UU\to\UU$ by induction on $k:\Z_{\geq -2}$, taking
\begin{align*}
\istrunc{-2}(A) & \defeq \iscontr(A) \\
\istrunc{k+1}(A) & \defeq \prd{x,y:A}\istrunc{k}(\id{x}{y}).\qedhere
\end{align*}
For any type $A$, we say that $A$ is \define{$k$-truncated}, or a \define{$k$-type}, if there is a term of type $\istrunc{k}(A)$. We say that a map $f:A\to B$ is $k$-truncated if its fibers are $k$-truncated.
\end{defn}

%For the rest of this section, let $k:\Z_{\geq-2}$.

\begin{thm}
If $A$ is a $k$-type, then $A$ is also a $(k+1)$-type.
\end{thm}

\begin{proof}
If $A$ is contractible with center of contraction $c$, and contraction $C$, then we have
\begin{equation*}
\lam{x}{y} \ct{C(x)}{C(y)^{-1}} : \prd{x,y:A}\id{x}{y}.
\end{equation*}
By \autoref{lem:isprop_eq} this shows that $A$ is a proposition. If any $k$-type is also a $(k+1)$-type, then any $(k+1)$-type is a $(k+2)$-type, since its identity types are $k$-types and therefore $(k+1)$-types.
\end{proof}

\begin{thm}\label{thm:ntype_eqv}
If $e:\eqv{A}{B}$ is an equivalence, and $B$ is a $k$-type, then so is $A$.
\end{thm}

\begin{proof}
We have seen in \autoref{ex:contr_equiv} that if $B$ is contractible and $e:\eqv{A}{B}$ is an equivalence, then $A$ is also contractible. This proves the base case.

For the inductive step, assume that the $k$-types are stable under equivalences, and consider $e:\eqv{A}{B}$ where $B$ is a $(k+1)$-type. In \autoref{ex:emb_equiv} we have seen that
\begin{equation*}
\apfunc{e}:(\id{x}{y})\to(\id{e(x)}{e(y)})
\end{equation*}
is an equivalence for any $x,y$. Note that $\id{e(x)}{e(y)}$ is a $k$-type, so by the induction hypothesis it follows that $\id{x}{y}$ is a $k$-type. This proves that $A$ is a $(k+1)$-type.
\end{proof}

\begin{comment}
\begin{proof}
By \autoref{ex:contr_retr} it follows that if $A$ is a retract of a contractible type, then $A$ is contractible.
For the inductive step, suppose that the $k$-types are closed under retracts, and consider a section-retraction pair
\begin{equation*}
\begin{tikzcd}
A \arrow[r,"i"] & B \arrow[r,"r"] & A,
\end{tikzcd}
\end{equation*}
with $H:r\circ i\htpy \idfunc$, where $B$ is a $(k+1)$-type.
By the induction hypothesis it suffices to show that for any $x,y:A$, the function $\apfunc{i}:(\id{x}{y})\to (\id{i(x)}{i(y)})$ has a retraction.
The retraction $\varphi:(\id{i(x)}{i(y)})\to(\id{x}{y})$ is defined as
\begin{equation*}
\varphi \defeq \lam{q} \ct{H(x)^{-1}}{\ap{r}{q}}{H(y)}
\end{equation*}
To see that $\varphi(\ap{i}{p})=p$, we have to show that the square
\begin{equation*}
\begin{tikzcd}
r(i(x)) \arrow[d,equals,swap,"\ap{r}{q}"] \arrow[r,equals,"H(x)"] & x \arrow[d,equals,"p"] \\
r(i(y)) \arrow[r,equals,swap,"H(y)"] & y
\end{tikzcd}
\end{equation*}
commutes. This square commutes by the naturality of homotopies, proven in \autoref{ex:htpy_nat}.
\end{proof}
\end{comment}

\begin{exercises}
\item For any proposition $P$, show that $P\to\iscontr(P)$.
\item Let $P$ and $Q$ be propositions. Show that
\begin{equation*}
\eqv{(P\leftrightarrow Q)}{(\eqv{P}{Q})}.
\end{equation*}
Conclude that any two contractible types are equivalent.
\item Let $A$ be a type, and let $\delta_A:A\to A\times A$ be given by $\lam{x}(x,x)$. 
\begin{subexenum}
\item Show that
\begin{equation*}
\eqv{\isequiv(\delta_A)}{\isprop(A)}.
\end{equation*}
\item Show that $A$ is $(k+1)$-truncated if and only if $\delta_A:A\to A\times A$ is $k$-truncated.
\end{subexenum}
\item Let $P:A\to\type$ be a type family. Show that for any $n\geq-2$, if $A$ is an $n$-type, and $P(x)$ is an $n$-type for each $x:A$, then so is $\sm{x:A}P(x)$. 
\item Show that $\bool$ is a set.
\item Show that for any two sets $A$ and $B$, the disjoint sum $A+B$ is again a set.
\item \label{ex:hedberg}(Hedberg's theorem) A type $A$ is said to have \define{decidable equality} if there is a term of type
\begin{equation*}
\prd{x,y:A} (\id{x}{y})+\neg(\id{x}{y}).
\end{equation*}
For any type $A$, and every $x,y:A$, consider the type family $D(x,y):((\id{x}{y})+\neg(\id{x}{y}))\to\type$ given by
\begin{align*}
D(x,y,\inl(p)) & \defeq \unit \\
D(x,y,\inr(p)) & \defeq \emptyt.
\end{align*}
Use $D$ to show that any type with decidable equality is a set.\footnote{By following this suggestion one avoids the use of function extensionality, which is used in Theorem 7.2.5 of \cite{hottbook} to conclude that $\neg\neg(\id{x}{y})$ is a proposition.}
\item Show that $\nat$ and $\bool$ have decidable equality, as defined in \autoref{ex:hedberg}.
\item Show that if $A$ and $B$ have decidable equality, then so do $A+B$ and $A\times B$.
\item 
\begin{subexenum}
\item Let $B:A\to\type$ be a type family over $A$. Show that iff $A$ is a $k$-type, and if $B(x)$ is a $k$-type for each $x:A$, then so is $\sm{x:A}B(x)$.
\item Show that for $k\geq -1$, any subtype of a $k$-type is again a $k$-type.
\item Show that for any map $f:A\to B$ from a $k$-type $A$ to a $(k+1)$-type $B$, the fibers of $f$ are also $k$-types.
\end{subexenum}
\item Use \autoref{ex:contr_retr,ex:retr_id} to show that if $A$ is a retract of a $k$-type $B$, then $A$ is also a $k$-type.
\end{exercises}


\chapter{The univalence axiom}

\section{Type extensionality}
\begin{defn}
We define a family of maps
\begin{equation*}
\mathsf{equiv\usc{}eq}\defeq \mathsf{rec}_{=}(\lam{A}\idfunc[A]) : \prd*{A,B:\UU} (\id{A}{B})\to(\eqv{A}{B}).
\end{equation*}
\end{defn}

\begin{defn}
The \define{univalence axiom} asserts that the family of maps $\mathsf{equiv\usc{}eq}$ is a fiberwise equivalence.
\end{defn}

The univalence axiom asserts that equivalent types are equal. It is considered to be an \emph{extensionality principle} for types.

\begin{lem}
The univalence axiom holds if and only if the type
\begin{equation*}
\sm{B:\UU}\eqv{A}{B}
\end{equation*}
is contractible for each $A:\UU$.
\end{lem}

\begin{proof}
By \autoref{thm:id_fundamental}.
\end{proof}

The following construction enables us to make construction by induction on equivalences, analogous to path induction.

\begin{defn}
Let $A:\UU$, and let $P:\prd{B:\UU} (\eqv{A}{B})\to\type$ be a type family. Using the univalence axiom we construct
\begin{equation*}
\mathsf{equiv\usc{}ind}(P,A) : P(A,\idfunc[A])\to \prd{B:\UU}{e:\eqv{A}{B}}P(B,e).
\end{equation*}
\end{defn}

\begin{constr}
Since $\sm{B:\UU}\eqv{A}{B}$ is contractible we have
\begin{equation*}
P(\idfunc[A])\to\prd{\pairr{B,e}:\sm{B:\UU}\eqv{A}{B}}P(B,e)
\end{equation*}
by \autoref{ex:contr_ind}, so we obtain the desired function by uncurrying.
\end{constr}

From now on we will assume that the univalence axiom holds.

\section{Function extensionality}

The first application of the univalence axiom was Voevodsky's proof of \emph{function extensionality}, which we introduce below.

\begin{defn}
Let $f,g:\prd{x:A}B(x)$ be two dependent functions. We define the function
\begin{equation*}
\mathsf{htpy\usc{}eq}(f,g) : (\id{f}{g})\to (f\htpy g)
\end{equation*}
by path induction, sending $\refl{f}$ to $\lam{x}\refl{f(x)}$. The \define{function extensionality principle} asserts that $\mathsf{htpy\usc{}eq}$ is a fiberwise equivalence, for any $A:\type$ and $B:A\to\type$.
\end{defn}

We first show that function extensionality follows from \emph{weak function extensionality}.

\begin{defn}
The \define{weak function extensionality principle} asserts that for any $B:A\to\type$,
\begin{equation*}
\Big(\prd{x:A}\iscontr(B(x))\Big)\to \iscontr\Big(\prd{x:A}B(x)\Big).
\end{equation*}
\end{defn}

\begin{thm}
Weak function extensionality implies function extensionality.
\end{thm}

\begin{proof}
To prove function extensionality, it suffices by \autoref{thm:id_fundamental} to show that
\begin{equation*}
\sm{g:\prd{x:A}B(x)}f\htpy g
\end{equation*}
is contractible.

Assume that products of contractible types are contractible.
Since the type $\sm{b:B(x)}f(x)=b$ is contractible for each $x:X$, it follows by our assumption of weak function extensionality that the type $\prd{x:A}\sm{b:B(x)}f(x)=b$ is contractible. By \autoref{ex:contr_retr} it therefore suffices to show that
\begin{equation*}
\sm{g:\prd{x:A}B(x)}f\htpy g
\end{equation*}
is a retract of the type $\prd{x:A}\sm{b:B(x)}f(x)=b$. We have the functions
\begin{align*}
\mathsf{pi\usc{}sigma} & \defeq \lam{\pairr{g,H}}\lam{x}\pairr{g(x),H(x)} \\
\mathsf{sigma\usc{}pi} & \defeq \lam{p}\pairr{\lam{x}\proj 1(p(x)),\lam{x}\proj 2(p(x))}.
\end{align*}
It remains to show that $\psi\circ\varphi=\idfunc$. Let $\pairr{g,H}:\sum_g f\htpy g$. 
Then we have
\begin{align*}
\mathsf{sigma\usc{}pi}(\mathsf{pi\usc{}sigma}(g,H)) & \jdeq \mathsf{sigma\usc{}pi}(\lam{x}\pairr{g(x),H(x)}) \\
& \jdeq \pairr{\lam{x}g(x),\lam{x}H(x)} \\
& \jdeq \pairr{g,H}.\qedhere
\end{align*}
\end{proof}

\begin{rmk}
Since we assumed the $\eta$-rule for $\Sigma$-types, we also have
\begin{align*}
\mathsf{pi\usc{}sigma}(\mathsf{sigma\usc{}pi}(p)) & \jdeq \mathsf{pi\usc{}sigma}(\pairr{\lam{x}\proj 1(p(x)),\lam{x}\proj 2(p(x))}) \\
& \jdeq \lam{x}\pairr{\proj 1(p(x)),\proj 2(p(x))} \\
& \jdeq \lam{x} p(x) \\
& \jdeq p.
\end{align*}
Therefore, the types $\sum_g f\htpy g$ and $\prod_x\sum_b f(x)=b$ are actually \emph{judgmentally isomorphic}. 
\end{rmk}

\begin{exercises}
\item \label{ex:tr_ap} Show that for any $P:X\to \UU$ and any $p:x=y$ in $X$, we have
\begin{equation*}
\mathsf{equiv\usc{}eq}(\ap{P}{p})=\mathsf{tr}^P(p).
\end{equation*}
\item Use the univalence axiom to show that the type $\sm{A:\UU}\iscontr(A)$ of all contractible types in $\UU$ is contractible.
\item Use the univalence axiom to show that the type $\sm{P:\prop}P$ is contractible.
\item Show that $\eqv{(\eqv{\bool}{\bool})}{\bool}$, and conclude by the univalence axiom that the universe is not a set.
\item Construct by path induction a family of maps
\begin{equation*}
\prd{A,B:\UU}{a:A}{b:B} (\id{\pairr{A,a}}{\pairr{B,b}})\to \sm{e:\eqv{A}{B}}e(a)=b,
\end{equation*}
and show that this map is an equivalence. In other words, an \emph{identification of pointed types} is a base point preserving equivalence.
\item Let $\pairr{A,a}$ and $\pairr{B,b}$ be two pointed types. Construct by path induction a family of maps
\begin{equation*}
\prd{f,g:A\to B}{p:f(a)=b}{q:g(a)=b} (\id{\pairr{f,p}}{\pairr{g,q}})\to \sm{H:f\htpy g} p = \ct{H(a)}{q},
\end{equation*}
and show that this map is an equivalence. In other words, an \emph{identification of pointed maps} is a base point preserving homotopy.
\item Let $C$ be a contractible type with center of contraction $c$. Use the function extensionality principle to show that the map $\lam{f}f(c):(C\to A)\to A$ is an equivalence, for each type $A$.
\end{exercises}


\chapter{The Yoneda embedding}
This is a good moment to review what we did so far, by recalling some basic notions and prove some more properties of them that involve function extensionality.

\section{Basic propositions}
\begin{lem}
For any type $A$, the type $\iscontr(A)$ is a proposition.
\end{lem}

\begin{lem}
For any family of propositions $P:A\to\prop$, the dependent function type $\prd{x:A}P(x)$ is a proposition.
\end{lem}

\begin{cor}
For any map $f:A\to B$, the type $\isequiv(f)$ is a proposition.
\end{cor}

\begin{cor}
For any type $A$ and any $n:\Z_{\geq -2}$, the type $\istrunc{n}(A)$ is a proposition.
\end{cor}

\begin{cor}
For any type $A$ and any proposition $P$, the type $A\to P$ is a proposition
\end{cor}

\begin{cor}
For any type $A$, the type $A\to \emptyt$ is a proposition.
\end{cor}

\begin{cor}
For any type $A$, the type $A\to \unit$ is contractible.
\end{cor}

\begin{exercises}
\item Show that for any type $A$ and any $n\geq-2$, the type $\istrunc{n}(A)$ is a proposition.
\item Show that for any map $f:A\to B$, the type $\isequiv(f)$ is a proposition, 
\item Show that for any type family $P:A\to\type$ and any $n\geq-2$, we have
\begin{equation*}
\Big(\prd{x:A}\istrunc{n}(P(x))\Big)\to\istrunc{n}\Big(\prd{x:A}P(x)\Big).
\end{equation*}
In other words, the truncation levels are closed under dependent function types.
Conclude that the univalence axiom is a proposition.
\item Use the construction of \autoref{ex:fiber_trans} to construct an equivalence
\begin{equation*}
\eqv{\Big(\sm{h:A\to B} f\htpy g\circ h\Big)}{\Big(\prd{x:X}\hfib{f}{x}\to\hfib{g}{x}\Big)}.
\end{equation*}
\end{exercises}

\chapter{The circle}
In this chapter we introduce the idea of higher inductive types, by studying the simplest non-trivial example: the circle. Moreover, we show that the loop space of the circle is equivalent to $\mathbb{Z}$, by constructing the universal cover of the circle as an application of the univalence axiom. 

\begin{exercises}
\item Use the negation equivalence $\mathsf{neg}:\eqv{\bool}{\bool}$ of \autoref{ex:neg_equiv} to construct a fiber sequence
\begin{equation*}
\bool\hookrightarrow \sphere{1}\twoheadrightarrow \sphere{1}.
\end{equation*}
\item Show that $\eqv{(\eqv{\sphere{1}}{\sphere{1}})}{\sphere{1}+\sphere{1}}$. Conclude that a univalent universe containing a circle is not a $1$-type.
\end{exercises}

\chapter{Homotopy colimits}

\section{Coequalizers}

\section{Pushouts}
\begin{enumerate}
\item General pushouts,
\item Suspensions, the spheres.
\item join, wedge, smash.
\end{enumerate}

\section{Sequential colimits}

\begin{exercises}
\item Show that $\join{P}{Q}= P\lor Q$, for any two propositions $P$ and $Q$.
\item Let $Q$ be a proposition, and let $A$ be a type. Show that $\inr:A\to \join{Q}{A}$ is an equivalence if and only if $Q\to\iscontr(A)$.
\item Consider a sequence
\begin{tikzcd}
X_0 \arrow[r] & X_1 \arrow[r] & X_2 \arrow[r] & \cdots
\end{tikzcd}
of \emph{pointed} types and \emph{pointed} maps between them. Show that the sequential colimit is contractible. 
\item Let
\begin{tikzcd}
P_0 \arrow[r] & P_1 \arrow[r] & P_2 \arrow[r] & \cdots
\end{tikzcd}
be a sequence of propositions. Show that
\begin{equation*}
\colim_n(P_n)=\exists_n P_n.
\end{equation*}
\item In this exercise we study the \define{reflexive coequalizer}. Let $R:A\to A\to\UU$ be a relation, and $\rho:\prd{x:A}R(x,x)$ be a witness of reflexivity.  
\begin{subexenum}
\item Formulate the induction principle and computation rules for the higher inductive type $\mathsf{rcoeq}(A,R,\rho)$ with constructors
\begin{align*}
\pts{\eta} &: A \to \mathsf{rcoeq}(A,R,\rho) \\
\edg{\eta} &: \prd*{x,y:A} R(x,y)\to \id{\pts{\eta}(x)}{\pts{\eta}(y)} \\
\rfx{\eta} &: \prd{x:A} \edg{\eta}(\rho(x))=\refl{\pts{\eta}(x)}.
\end{align*}
\item Show that
\begin{equation*}
\begin{tikzcd}
\sm{x,y:A} R(x,y) \arrow[r,"\pi_2"] \arrow[d,swap,"\pi_1"] & A \arrow[d,"\pts{\eta}"] \\
A \arrow[r,swap,"\pts{\eta}"] & \mathsf{rcoeq}(A,R,\rho)
\end{tikzcd}
\end{equation*}
commutes, and is a pushout square.
\item Compute $\mathsf{rcoeq}(A,\idtypevar{A},\refl{})$ and $\mathsf{rcoeq}(A,(\lam{x}{y}\unit),(\lam{x}\ttt))$. Furthermore, consider a pointed type $\pairr{X,x_0}$ as a reflexive relation on the unit type, and compute $\mathsf{rcoeq}(\unit,X,x_0)$.
\end{subexenum}
\end{exercises}

\chapter{Descent}

\section{Homotopy pullbacks}
[Tell something about why pullbacks are important.]
\begin{defn}
Given maps $f:A\to X$ and $g:B\to X$, we define \[A\times_X B\defeq \sm{x:A}{y:B}f(x)=f(y).\] The \define{canonical pullback square} of $f$ and $g$ is the commuting square
\begin{equation*}
\begin{tikzcd}
A\times_X B \arrow[r,"\pi_2"] \arrow[d,swap,"\pi_1"] & B \arrow[d,"g"] \\
A \arrow[r,swap,"f"] & X,
\end{tikzcd}
\end{equation*}
where $\pi_1(x,y,p)\defeq x$, and $\pi_2(x,y,p)\defeq y$, and the commutativity is witnessed by the homotopy $\lam{\pairr{x,y,p}}p$. 
\end{defn}

\begin{thm}
For any type $C$, the map
\begin{equation*}
(C\to A\times_X B)\to \sm{h:C\to A}{k:C\to B}f\circ h\htpy g\circ k
\end{equation*}
is an equivalence.
\end{thm}

\begin{lem}[Pullback pasting lemma]\label{lem:pb_pasting}
Consider 
\begin{equation*}
\begin{tikzcd}
A \arrow[d,swap,"f"] \arrow[r,"j"] & B \arrow[d,swap,"g"] \arrow[r,"l"] & C \arrow[d,"h"] \\
X \arrow[r,swap,"i"] & Y \arrow[r,swap,"k"] & Z
\end{tikzcd}
\end{equation*}
with homotopies $H:i\circ f\htpy g\circ j$ and $K:k\circ g\htpy h\circ l$ witnessing that the two squares commute, and suppose that the square on the right is a pullback square. Then the square on the left is a pullback square if and only if the outer rectangle is a pullback square.
\end{lem}

\section{The flattening lemma}

\begin{exercises}
\item Show that the square
\begin{equation*}
\begin{tikzcd}
(x=y) \arrow[r] \arrow[d] & \unit \arrow[d,"\mathsf{const}_y"] \\
\unit \arrow[r,swap,"\mathsf{const}_x"] & A
\end{tikzcd}
\end{equation*}
is a pullback square.
\item Let $f:A\to B$ be a map, and consider $P:A\to\type$, and $Q:B\to\type$. Furthermore, consider a fiberwise transformation
\begin{equation*}
g:\prd{x:A}P(x)\to Q(f(x)).
\end{equation*}
Show that the following are equivalent:
\begin{subexenum}
\item The fiberwise map $g$ is a fiberwise equivalence.
\item The commuting square
\begin{equation*}
\begin{tikzcd}[column sep=large]
\sm{x:A}P(x) \arrow[d,swap,"\proj 1"] \arrow[r,"\total{g}"] & \sm{y:B}Q(y) \arrow[d,"\proj 1"] \\
A \arrow[r,swap,"f"] & B
\end{tikzcd}
\end{equation*}
is a pullback square.
\end{subexenum}
\end{exercises}

\chapter{The Hopf fibration}
Our goal in this chapter is to construct the Hopf fibration, i.e.~a fiber sequence
\begin{equation*}
\sphere{1}\hookrightarrow\sphere{3}\twoheadrightarrow\sphere{2}.
\end{equation*}

\chapter{Embeddings and surjections}
Embeddings, subtypes, surjective maps, image factorization.
We give a type theoretical variant of the Yoneda lemma, using the groupoid structure of types.

\begin{exercises}
\item Let $\UU_\bool\defeq \sm{A:\UU}\brck{A=\bool}$ be the subuniverse of $2$-element types. Show that the type
\begin{equation*}
\sm{A:\UU_\bool}A
\end{equation*}
of \emph{pointed} 2-element types is contractible.
\end{exercises}

\chapter{Univalent logic}


\section{Propositional truncation}

\begin{defn}
Let $A:\UU$ be a type. We define the \define{propositional truncation} $\brck{A}:\UU$ as a higher inductive type with constructors
\begin{align*}
\eta & : A\to \brck{A} \\
\mu & : \prd{x,y:\brck{A}} x=y.
\end{align*}
The induction principle of $\brck{A}$ asserts that for any type family $P:\brck{A}\to\type$, if we have
\begin{align*}
f & : \prd{x:A}P(\eta(x)) \\
g & : \prd{x,y:\brck{A}}{p:P(x)}{q:P(y)} \trans{\mu(x,y)}{p}=q,
\end{align*}
then we obtain a section $\rec{\brck{\blank}}(f,g):\prd{x:\brck{A}}P(x)$ satisfying
\begin{align*}
\rec{\brck{\blank}}(f,g,\eta(x)) & = f(x).
\end{align*}
\end{defn}

\begin{rmk}
We will not need a computation rule corresponding to the path constructor $\mu$.
\end{rmk}

\begin{lem}
For any $A:\UU$, the type $\brck{A}$ is a proposition.
\end{lem}

\begin{proof}
The path constructor $\mu:\prd{x,y:\brck{A}}x=y$ directly provides a proof that $\brck{A}$ is a proposition.
\end{proof}

The following theorem asserts that for any map $f:A\to P$ into a proposition $P$, there is a unique map $g:\brck{A}\to P$ such that $g\circ\eta=f$, as indicated in the following diagram
\begin{equation*}
\begin{tikzcd}
A \arrow[dr,"\forall"] \arrow[d,swap,"\eta"] \\
\brck{A} \arrow[r,densely dotted,swap,"\exists!"] & P
\end{tikzcd}
\end{equation*}

\begin{thm}
Let $A:\UU$ be a type. Then for any proposition $P$, the map
\begin{equation*}
\lam{g} g\circ \eta : (\brck{A}\to P)\to (A\to P)
\end{equation*}
is an equivalence.
\end{thm}

\section{First order logic in type theory}
\begin{center}
\begin{tabular}{ll}
\toprule
\multicolumn{2}{c}{\large\textbf{Logic in type theory}} \\[2ex]
\emph{Logical connective} & \emph{Interpretation in HoTT} \\
\midrule
$\top$ & $\unit$ \\
$\bot$ & $\emptyt$ \\
$P\land Q$ & $P\times Q$ \\
$P\lor Q$ & $\brck{P+Q}$ \\
$P\to Q$ & $P\to Q$ \\
$P\leftrightarrow Q$ & $P=Q$ \\
$\neg P$ & $P\to\emptyt$ \\
$\forall x.P(x)$ & $\prd{x:A}P(x)$ \\
$\exists x.P(x)$ & $\brck{\sm{x:A}P(x)}$ \\
$\exists! x.P(x)$ & $\iscontr(\sm{x:A}P(x))$ \\
\bottomrule
\end{tabular}
\end{center}
\begin{enumerate}
\item First order logic in type theory. Stress difference between $\exists$ and $\Sigma$.
\item Propositional extensionality
\end{enumerate}

\begin{exercises}
\item \label{ex:brck_comp} Formulate the computation rule corresponding to the path constructor $\mu$. That is, compute the type of $\apd{\rec{\brck{\blank}}(f,g)}{\mu(x,y)}$, and find a canonical element in it.
\item Show that
\begin{equation*}
\eqv{\brck{A}}{\prd{P:\prop}(A\to P)\to P}.
\end{equation*}
\item \label{also}(Mart\'in Escard\'o) For any two propositions $P$ and $Q$, define
\begin{equation*}
P\boxplus Q \defeq ((P\to Q)\to Q)\times ((Q\to P)\to P).
\end{equation*}
\begin{subexenum}
\item Show that $P\lor Q\to P\boxplus Q$ and $P\boxplus Q\to\neg(\neg P\land \neg Q)$.
\end{subexenum}
\end{exercises}

\chapter{Univalent algebra}
\begin{enumerate}
\item Subsets and quotients
\item Factorization
\item The structure isomorphism theorem
\item The axiom of choice
\end{enumerate}

\chapter{Homotopy groups of types}
Introduce groups, abelian groups, Eckmann-Hilton argument, long exact sequence.

\chapter{Connectedness}

\chapter{The Blakers-Massey theorem}
The Blakers-Massey theorem is a connectivity theorem which can be used to prove the Freudenthal suspension theorem, giving rise to the field of \emph{stable homotopy theory}. It was proven in the setting of homotopy type theory by Lumsdaine et al, and their proof was the first that was given entirely in an elementary way, using only constructions that are invariant under homotopy equivalence. 

Consider a span $A \leftarrow S \rightarrow B$, consisting of an $m$-connected map $f:S\to A$ and an $n$-connected map $g:S\to B$. We take the pushout of this span, and subsequently the pullback of the resulting cospan, as indicated in the diagram
\begin{equation}\label{eq:BM}
\begin{tikzcd}
S \arrow[drr,bend left=15,"g"] \arrow[ddr,bend right=15,swap,"f"] \arrow[dr,densely dotted,"u" near end] \\
& A \times_{(A \sqcup^S B)} B \arrow[r,"\pi_2"] \arrow[d,"\pi_1"] & B \arrow[d,"\inr"] \\
& A \arrow[r,swap,"\inl"] & A \sqcup^S B.
\end{tikzcd}
\end{equation}
The universal property of the pullback determines a unique map $u:S\to A \times_{(A\sqcup^S B)} B$ as indicated.

\begin{thm}[Blakers-Massey]
The map $u:S\to A \times_{(A\sqcup^S B)} B$ of \autoref{eq:BM} is $(n+m)$-connected.
\end{thm}

\chapter{Some stable homotopy theory}
\section{The Freudenthal suspension theorem}

\chapter{Eilenberg-Mac Lane spaces}

\chapter{Cohomology}

\chapter{The real projective spaces}
...and their cohomology.

\appendix

\chapter{Formalization in the Lean HoTT mode}
\begin{enumerate}
\item Introduce universes
\item The inductive type constructor
\item identity functions, composition, constant functions
\end{enumerate}

\chapter{Literature}

\backmatter

\end{document}
