\chapter{Contractible types and contractible maps}
\chaptermark{Contractible types and maps}

\section{Contractible types}

\begin{thm}\label{thm:contractible}
Let $A$ be a type. The following are equivalent:
\begin{enumerate}
\item $A$ is \define{contractible}: there is a term of type
\begin{equation*}
\iscontr(A) \defeq \sm{c:A}\prd{x:A}c=x.
\end{equation*}
Given a term $(c,C):\iscontr(A)$, we call $c:A$ the \define{center of contraction} of $A$, and we call $C:\prd{x:A}a=x$ the \define{contraction} of $A$.
\item $A$ comes equipped with a term $a:A$, and satisfies \define{singleton induction}: for every $B:A\to\mathsf{Type}$ the map
\begin{equation*}
\Big(\prd{x:A}B(x)\Big)\to B(a)
\end{equation*}
given by $f\mapsto f(a)$ has a section.
\end{enumerate}
\end{thm}

\begin{rmk}
Suppose $A$ is a contractible type with center of contraction $c$ and contraction $C$. Then the type of $C$ is (judgmentally) equal to the type
\begin{equation*}
\mathsf{const}_c\htpy\idfunc[A].
\end{equation*}
In other words, the contraction $C$ is a \emph{homotopy} from the constant function to the identity function.
\end{rmk}

\begin{proof}[Proof of \autoref{thm:contractible}]
Suppose $A$ is contractible with center of contraction $c$ and contraction $C$. Without loss of generality we may assume that $C(c)=\refl{c}$, since for any contraction $C$ we can define a new contraction $C'$ satisfying this property by setting $C'(x)\defeq\ct{C(c)^{-1}}{C(x)}$. To show that $A$ satisfies singleton induction let $B:A\to\mathsf{Type}$ be a type family over $A$ equipped with $b:B(a)$. We define $\mathsf{sing\usc{}ind}(b):\prd{x:A}B(x)$ by $\lam{x}\mathsf{tr}^B(C(x),b)$. To see that $\mathsf{sing\usc{}ind}(c)=b$ note that we have
\begin{equation*}
\mathsf{tr}^B(C(c),b)=\mathsf{tr}^B(\refl{c},b)=b.
\end{equation*}
This completes the proof that $A$ satisfies singleton induction.

For the converse, if $A$ comes equipped with a center of contraction $a:A$ and satisfies singleton induction, then we can use singleton induction to construct the contraction using the family $B(x)\defeq a=x$ with the base point $\refl{a}:B(a)$. 
\end{proof}

\begin{eg}
By definition the unit type $\unit$ satisfies singleton induction, so it is contractible.
\end{eg}

\section{Contractible maps}
\begin{defn}
Let $f:A\to B$ be a function, and let $b:B$. The \define{fiber} of $f$ at $b$ is defined to be the type
\begin{equation*}
\fib{f}{b}\defeq\sm{a:A}f(a)=b.
\end{equation*}
\end{defn}

\begin{defn}
We say that a function $f:A\to B$ is \define{contractible} if there is a term of type
\begin{equation*}
\iscontr(f)\defeq\prd{b:B}\iscontr(\fib{f}{b}).
\end{equation*}
\end{defn}

\begin{thm}\label{thm:equiv_contr}
Any contractible map is an equivalence.
\end{thm}

\begin{proof}
Let $f:A\to B$ be a contractible map. Using the center of contraction of each $\fib{f}{y}$, we obtain a term of type
\begin{align*}
\lam{y}\pairr{g(y),G(y)}:\prd{y:B}\fib{f}{y}.
\end{align*}
Thus, we get map $g:B\to A$, and a homotopy $G:\prd{y:B} f(g(y))=y$. In other words, we get a section of $f$.

It remains to construct a retraction of $f$. Taking $g$ as our retraction, we have to show that $\prd{x:A} g(f(x))=x$. Note that we get an identification $p:f(g(f(x)))=f(x)$ since $g$ is a section of $f$. Moreover, since $\fib{f}{f(x)}$ is contractible we get an identification $q:\pairr{g(f(x)),p}=\pairr{x,\refl{f(x)}}$. The base path of this identification is an identification of type $g(f(x))=x$, as desired.
\end{proof}

\section{Equivalences are contractible maps}

In this section we will show the converse to \autoref{thm:equiv_contr}, that equivalences are contractible maps. Before we do so, we will establish some useful constructions on homotopies and section-retraction pairs.

\begin{defn}\label{defn:htpy_nat}
Let $f,g:A\to B$ be functions, and consider $H:f\htpy g$ and $p:x=y$ in $A$. We will construct an identification
\begin{align*}
\mathsf{htpy\usc{}nat}(H,p) & :\ct{H(x)}{\ap{g}{p}}=\ct{\ap{f}{p}}{H(y)}
\end{align*}
witnessing that the square
\begin{equation*}
\begin{tikzcd}
f(x) \arrow[r,equals,"H(x)"] \arrow[d,equals,swap,"\ap{f}{p}"] & g(x) \arrow[d,equals,"\ap{g}{p}"] \\
f(y) \arrow[r,equals,swap,"H(y)"] & g(y)
\end{tikzcd}
\end{equation*}
commutes.
\end{defn}

\begin{defn}\label{defn:retraction_swap}
Consider $f:A\to A$ and $H: f\htpy \idfunc[A]$. We construct an identification $H(f(x))=\ap{f}{H(x)}$, for any $x:A$.
\end{defn}

\begin{constr}
By the naturality of homotopies with respect to identifications the square
\begin{equation*}
\begin{tikzcd}[column sep=large]
ff(x) \arrow[d,swap,equals,"\ap{f}{H(x)}"] \arrow[r,equals,"H(f(x))"] & gf(x) \arrow[d,equals,"H(x)"] \\
f(x) \arrow[r,swap,equals,"H(x)"] & x
\end{tikzcd}
\end{equation*}
commutes. This gives the desired identification $H(f(x))=\ap{f}{H(x)}$.
\end{constr}

\begin{thm}\label{thm:contr_equiv}
Any equivalence is a contractible map.
\end{thm}

\begin{proof}
Since every equivalence has the structure of an invertible map by \autoref{defn:inv_equiv}, it suffices to show that any invertible map is contractible.

Let $f:A\to B$ be a map, with $g:B\to A$, $G:f\circ g\htpy\idfunc[B]$, and $H:h\circ f\htpy \idfunc[A]$.
We have for any $y:B$ the term $\pairr{g(y),G(y)}:\fib{f}{y}$. However, as our center of contraction we take
$\pairr{g(y),\epsilon(y)}$, where
\begin{equation*}
\varepsilon(y) \defeq \ct{\ap{fg}{G(y)}^{-1}}{\ap{f}{H(g(y))}}{G(y)}.
\end{equation*}
Now it remains to construct the contraction. Let $x:A$, and let $p:f(x)=y$.
Since $p:f(x)=y$ has a free endpoint, we can apply path induction on it. 
Our goal is now to construct an identification
\begin{equation*}
\pairr{g(f(x)),\varepsilon(f(x))}=\pairr{x,\refl{f(x)}}.
\end{equation*}
We will construct an identification of the form $\mathsf{eq\usc{}pair}(H(x),\nameless)$,
so it remains to construct an identification of type
\begin{equation*}
\trans{H(x)}{\varepsilon(f(x))}=\refl{f(x)}.
\end{equation*}
Using \autoref{ex:trans_ap} we see that it suffices to show that the square
\begin{equation*}
\begin{tikzcd}[column sep=8em]
fgfgf(x) \arrow[r,equals,"\ap{fg}{G(f(x))}"] \arrow[d,equals,swap,"\ap{f}{H(gf(x))}"] & fgf(x) \arrow[d,equals,"\ap{f}{H(x)}"] \\
fgf(x) \arrow[r,equals,swap,"G(f(x))"] & f(x)
\end{tikzcd}
\end{equation*}
commutes. Recall from \autoref{defn:retraction_swap} that we have $H(gf(x))=\ap{gf}{H(x)}$ and $\ap{fg}{G(y)}=G(fg(y))$. Using these two identifications and \autoref{ex:ap_ap}, we see that it suffices to show that the square
\begin{equation*}
\begin{tikzcd}[column sep=8em]
fgfgf(x) \arrow[r,equals,"G(fgf(x))"] \arrow[d,equals,swap,"\ap{fgf}{H(x)}"] & fgf(x) \arrow[d,equals,"\ap{f}{H(x)}"] \\
fgf(x) \arrow[r,equals,swap,"G(f(x))"] & f(x)
\end{tikzcd}
\end{equation*}
commutes. However, this is just a naturality square of homotopies, which commutes by \autoref{defn:htpy_nat}.
\end{proof}

\begin{cor}\label{cor:contr_path}
Let $A$ be a type, and let $a:A$. Then the type
\begin{equation*}
\sm{x:A}x=a
\end{equation*}
is contractible.
\end{cor}

\begin{proof}
By \autoref{thm:id_equiv}, the identity function is an equivalence. Therefore, the fibers of the identity function are contractible by \autoref{thm:contr_equiv}. Note that $\sm{x:A}x=a$ is exactly the fiber of $\idfunc[A]$ at $a:A$.
\end{proof}

\begin{comment}
\begin{proof}
We have the term $(a,\refl{a}):\sm{x:A}a=x$, which we take for the center of contraction. To construct the contraction, we have to show that
\begin{equation*}
\prd{p:\sm{x:A}a=x} (a,\refl{a})=p.
\end{equation*}
By the induction principle for dependent pair types it suffices to construct a term of type
\begin{equation*}
\prd{x:A}{p:a=x} (a,\refl{a})=(x,p)
\end{equation*}
Note that we may proceed here by path induction on $p$. That is, it suffices to consider the case $p\jdeq\refl{a}$, and show that $(a,\refl{a})=(a,\refl{a})$. Here we choose $\refl{(a,\refl{a})}$.
\end{proof}
\end{comment}

\begin{exercises}
\item Construct an equivalence 
\begin{equation*}
\eqv{\big(\sm{x:A}f(x)=y\big)}{\big(\sm{x:A}y=f(x)\big)}.
\end{equation*}
Conclude that $\sm{x:A}a=x$ is contractible for any $a:A$.
\item \label{ex:contr_retr}Suppose that $A$ is a retract of $B$. Show that
\begin{equation*}
\iscontr(B)\to\iscontr(A).
\end{equation*}
\item \label{ex:contr_equiv}
\begin{subexenum}
\item Show that for any type $A$, the map $\mathsf{const}_\ttt : A\to \unit$ is an equivalence if and only if $A$ is contractible. 
\item Show that for any map $f:A\to B$, if any two of the three assertions
\begin{enumerate}
\item $A$ is contractible
\item $B$ is contractible
\item $f$ is an equivalence
\end{enumerate}
hold, then so does the third.
\end{subexenum}
\item \label{ex:contr_ind} Let $C$ be a contractible type with center of contraction $c:C$. Furthermore, let $B:C\to\type$ be a type family. Show that the map $b\mapsto\pairr{c,b}:B(c)\to\sm{x:C}B(x)$ is an equivalence.
\item \label{ex:coh_intro}Consider a type $A$ with base point $a:A$, and let $B$ be a type family on $A$ that implies the identity type, i.e.~there is a term
\begin{equation*}
\alpha : \prd{x:A} B(x)\to (a=x).
\end{equation*}
Construct an equivalence
\begin{equation*}
\eqv{\Big(\sm{x:A}B(x)\Big)}{\Big(\sm{y:B(a)}\alpha(a,y)=\refl{a}\Big)}.
\end{equation*}
\end{exercises}
