\chapter{Syllabus}

\section{Essential course information}
\begin{center}
\begin{tabular}{ll}
\emph{Course title} & Introduction to Homotopy Type Theory \\
\emph{Instructor} & Egbert Rijke \\
& Department of Philosophy \\
& Carnegie Mellon University \\
\emph{Course number} & 80-518, 80-818 \\
\emph{Semester} & Spring 2018 \\
\emph{Website} & \url{http://www.andrew.cmu.edu/user/erijke/hott/} \\
\emph{Lecture room} & Baker Hall 150 \\
\emph{Meeting time} & Tue/Thu 12:00 - 1:20 \\
\emph{Email} & \href{mailto:erijke@andrew.cmu.edu}{erijke@andrew.cmu.edu} \\
\emph{Instructor's office} & Baker Hall 148 \\
\emph{Office Hours} & Mon/Wed 5:00 - 6:00, or by appointment
\end{tabular}
\end{center}

\section{Course description}
Homotopy Type Theory (HoTT) is an emerging field of mathematics and computer science that extends Martin-Löf's dependent type theory by the addition of the univalence axiom and higher inductive types. In HoTT we think of types as spaces, dependent types as fibrations, and of the identity types as path spaces. We start the course by introducing type theory as a deductive system, and once the basic ingredients of homotopy type theory are in place we will mainly focus on \emph{synthetic homotopy theory}, i.e.~the development of homotopy theory in type theory.

\section{Course material}

We will roughly follow the book \emph{Homotopy Type Theory: Univalent foundation of mathematics} \cite{hottbook}, of which a PDF is freely available.

Some of the later results of synthetic homotopy theory can only be found in recent research papers. We will also use the PhD thesis of Guillaume Brunerie \cite{BruneriePhD} as a resource.

\section{Organization}

Each session will consist of two parts: a 50 minute lecture and 30 minutes in which students present solutions to exercises provided with the previous lecture. These presentations are intended to be short (roughly 5 minutes) and focused to the problem at hand. Problem sets will be posted below with the lecture synopses.

Students are expected to:
\begin{enumerate}
\item Present a solution when they are asked to do so (usually a week in advance). Graduate students will be asked to present more often than undergraduate students.
\item Per lecture, either correct a somewhat substantial mistake made by the instructor, or hand in a written solution for one exercise of their choice. Written solutions are to be handed in at the start of the next lecture for an A, or at the start of the next lecture after that for a B. Collaborations are encouraged, but solutions must be handed in individually. Presenting students hand in a written solution for the exercise they are asked to present.
\end{enumerate}

Hints for the exercises will be presented by the instructor during office hours, a day before they have to be handed in. 
