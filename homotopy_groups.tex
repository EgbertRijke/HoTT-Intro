\chapter{Homotopy groups of types}

\section{Pointed types}
\begin{defn}
We introduce the `category' of pointed types:
\begin{enumerate}
\item A pointed type consists of a type $X$ equipped with a base point $x:X$. We will write $\UU_\ast$ for the type $\sm{X:\UU}X$ of all pointed types.
\item A pointed map $(f,p):(X,x)\to_\ast (Y,y)$ consists of a map $f:X\to Y$ and an identification $p:f(x)=y$. 
\item A pointed homotopy $(H,q):(f,p)\htpy_\ast (f',p')$ consists of a homotopy $H:f\htpy f'$ and an identification $q:p=\ct{H(x)}{p'}$ witnessing that the square
\begin{equation*}
\begin{tikzcd}
f(x) \arrow[rr,equals,"H(x)"] \arrow[dr,equals,swap,"p"] & & f'(x) \arrow[dl,equals,"{p'}"] \\
& y
\end{tikzcd}
\end{equation*}
commutes.
\end{enumerate}
\end{defn}

\begin{eg}
The circle $\sphere{1}$ is a pointed type with base point $\base:\sphere{1}$.
\end{eg}

\begin{eg}
If $X$ is a pointed type, then in the suspension of $X$ we have the canonical identification $\merid(\ast_X):\north=\south$. Therefore we do not have to worry about whether to choose $\north$ or $\south$ as the base point of $\susp{X}$. 
\end{eg} 

To consider higher pointed homotopies it is useful to first consider pointed families, and pointed $\Pi$-types.

\begin{defn}
\begin{enumerate}
\item Let $(X,\ast_X)$ be a pointed type. A \define{pointed family} over $(X,\ast_X)$ consists of a type family $P:X\to \UU_\ast$ equipped with a base point $\ast_P:P(\ast_X)$. 
\item Let $(P,\ast_P)$ be a pointed family over $(X,\ast_X)$. A \define{pointed section} of $(P,\ast_P)$ consists of a dependent function $f:\prd{x:X}P(x)$ and an identification $p:f(\ast_X)=\ast_P$. We define the \define{pointed $\Pi$-type} to be the type of pointed sections:
\begin{equation*}
\Pi^\ast_{(x:X)}P(x) \defeq \sm{f:\prd{x:X}P(x)}f(\ast_X)=\ast_P
\end{equation*}
\item Given any two pointed sections $f$ and $g$ of a pointed family $P$ over $X$, we define the type of pointed homotopies
\begin{equation*}
f\htpy_\ast g \defeq \Pi^\ast_{(x:X)} f(x)=g(x),
\end{equation*}
where the family $x\mapsto f(x)=g(x)$ is equipped with the base point $\ct{p}{q^{-1}}$. 
\end{enumerate}
\end{defn}

Since pointed homotopies are now defined as certain pointed sections, we can use the same definition of pointed homotopies again to consider pointed homotopies between pointed homotopies, and so on.

\section{Loop spaces}
\begin{defn}
Let $X$ be a pointed type with base point $x$. We define the \define{loop space} $\loopspace{X,x}$ of $X$ at $x$ to be the pointed type $x=x$ with base point $\refl{x}$. 
\end{defn}

\begin{defn}
The loop space operation $\loopspacesym$ is \emph{functorial} in the sense that
\begin{enumerate}
\item For every pointed map 
\end{enumerate}
\end{defn}

\section{Homotopy groups}
\begin{defn}
For $n\geq 1$, the \define{$n$-th homotopy group} of a type $X$ at a base point $x:X$ consists of the type
\begin{equation*}
|\pi_n(X,x)| \defeq \trunc{0}{\loopspace[n]{X,x}}
\end{equation*}
equipped with the group operations inherited from the path operations on $\loopspace[n]{X,x}$. 
Often we will simply write $\pi_n(X)$ when it is clear from the context what the base point of $X$ is.

For $n\jdeq 0$ we define $\pi_0(X,x)\defeq \trunc{0}{X}$. 
\end{defn}

\begin{eg}
In \autoref{circle_loopspace} we established that $\loopspace{\sphere{1}}=\Z$. It follows that
\begin{equation*}
\pi_1(\sphere{1})=\Z \qquad\text{ and }\qquad\pi_n(\sphere{1})=0\qquad\text{for $n\geq 2$.}
\end{equation*}
Furthermore, we have seen in \autoref{circle_conn} that $\trunc{0}{\sphere{1}}$ is contractible. 
Therefore we also have $\pi_0(\sphere{1})=0$.
\end{eg}

\begin{thm}[The Eckmann-Hilton argument]
For $n\geq 2$, the $n$-th homotopy group is abelian.
\end{thm}

\begin{exercises}
\item Show that the type of pointed families over a pointed type $(X,x)$ is equivalent to the type
\begin{equation*}
\sm{Y:\UU_\ast} Y\to_\ast X.
\end{equation*}
\item Given two pointed types $A$ and $X$, we say that $A$ is a (pointed) retract of $X$ if we have $i:A\to_\ast X$, a retraction $r:X\to_\ast A$, and a pointed homotopy $H:r\circ_\ast i\htpy_\ast \idfunc^\ast$. 
\begin{subexenum}
\item Show that if $A$ is a pointed retract of $X$, then $\loopspace{A}$ is a pointed retract of $\loopspace{X}$. 
\item Show that if $A$ is a pointed retract of $X$ and $\pi_n(X)$ is a trivial group, then $\pi_n(A)$ is a trivial group.
\end{subexenum}
\item Show that if $A\leftarrow S\rightarrow B$ is a span of pointed types, then for any pointed type $X$ the square
\begin{equation*}
\begin{tikzcd}
(A\sqcup^S B \to_\ast X) \arrow[r] \arrow[d] & (B \to_\ast X) \arrow[d] \\
(A\to_\ast X) \arrow[r] & (S\to_\ast X)
\end{tikzcd}
\end{equation*}
is a pullback square.
\item In this exercise we prove the suspension-loopspace adjunction.
\begin{subexenum}
\item Construct a pointed equivalence
\begin{equation*}
\tau_{X,Y}:(\susp(X)\to_\ast Y) \eqvsym_\ast (X\to \loopspace{Y})
\end{equation*}
for any two pointed spaces $X$ and $Y$.
\item Show that for any $f:X\to_\ast X'$ and $g:Y'\to_\ast Y$, there is a pointed homotopy witnessing that the square
\begin{equation*}
\begin{tikzcd}[column sep=large]
(\susp(X')\to_\ast Y') \arrow[r,"\tau_{X',Y'}"] \arrow[d,swap,"h\mapsto g\circ h\circ \susp(f)"] & (X'\to_\ast \loopspace{Y'}) \arrow[d,"h\mapsto\loopspace{g}\circ h\circ f"] \\
(\susp(X)\to_\ast Y) \arrow[r,swap,"\tau_{X,Y}"] & (X\to_\ast \loopspace{Y})
\end{tikzcd}
\end{equation*}
\end{subexenum}
\item Show that if
\begin{equation*}
\begin{tikzcd}
C \arrow[r] \arrow[d] & B \arrow[d] \\
A \arrow[r] & X
\end{tikzcd}
\end{equation*}
is a pullback square of pointed types, then so is
\begin{equation*}
\begin{tikzcd}
\loopspace{C} \arrow[r] \arrow[d] & \loopspace{B} \arrow[d] \\
\loopspace{A} \arrow[r] & \loopspace{X}.
\end{tikzcd}
\end{equation*}
\end{exercises}
