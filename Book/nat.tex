\section{The natural numbers}

\index{inductive type|(}
\index{natural numbers|(}

The set of natural numbers is the most important object in mathematics. We quote Bishop\index{Bishop on the positive integers}, from his Constructivist Manifesto, the first chapter in Foundations of Constructive Analysis \cite{Bishop1967}, where he gives a colorful illustration of its importance to mathematics.

\begin{quote}
  ``The primary concern of mathematics is number, and this means the positive integers. We feel about number the way Kant felt about space. The positive integers and their arithmetic are presupposed by the very nature of our intelligence and, we are tempted to believe, by the very nature of intelligence in general. The development of the theory of the positive integers from the primitive concept of the unit, the concept of adjoining a unit, and the process of mathematical induction carries complete conviction. In the words of Kronecker, the positive integers were created by God. Kronecker would have expressed it even better if he had said that the positive integers were created by God for the benefit of man (and other finite beings). Mathematics belongs to man, not to God. We are not interested in properties of the positive integers that have no descriptive meaning for finite man. When a man proves a positive integer to exist, he should show how to find it. If God has mathematics of his own that needs to be done, let him do it himself.''
\end{quote}

A bit later in the same chapter, he continues:

\begin{quote}
  ``Building on the positive integers, weaving a web of ever more sets and ever more functions, we get the basic structures of mathematics: the rational number system, the real number system, the euclidean spaces, the complex number system, the algebraic number fields, Hilbert space, the classical groups, and so forth. Within the framework of these structures, most mathematics is done. Everything attaches itself to number, and every mathematical statement ultimately expresses the fact that if we perform certain computations within the set of positive integers, we shall get certain results.''
\end{quote}

\subsection{The formal specification of the type of natural numbers}
The type $\N$\index{N@{$\N$}|see {natural numbers}} of \define{natural numbers} is the archetypal example of an inductive type\index{inductive type!natural numbers}. The rules we postulate for the type of natural numbers come in four sets, just as the rules for $\Pi$-types:
\begin{enumerate}
\item The formation rule, which asserts that the type $\N$ can be formed.
\item The introduction rules, which provide the zero element and the successor function.
\item The elimination rule. This rule is the type theoretic analogue of the induction principle for $\N$.
\item The computation rules, which assert that any application of the elimination rule behaves as expected on the constructors $\zeroN$ and $\succN$ of $\N$.
\end{enumerate}

\subsubsection{The formation rule of $\N$}

\index{natural numbers!rules for N@{rules for $\N$}!formation}
\index{rules!for N@{for $\N$}!formation}
The type $\N$ is formed by the \define{$\N$-formation} rule
\begin{prooftree}
  \AxiomC{}
  \RightLabel{$\N$-form}
  \UnaryInfC{$\vdash \N~\type$.}
\end{prooftree}
In other words, $\N$ is postulated to be a closed type.

\subsubsection{The introduction rules of $\N$}
Unlike the set of positive integers in Bishop's remarks, Peano's first axiom postulates that $0$ is a natural number. The introduction rules for $\N$ equip it with the \define{zero term} and the \define{successor function}.
\index{natural numbers!rules for N@{rules for $\N$}!introduction rules}
\index{rules!for N@{for $\N$}!introduction rules}
\index{natural numbers!operations on N@{operations on $\N$}!0 N@{$\zeroN$}}
\index{0 N@{$\zeroN$}}
\index{successor function!on N@{on $\N$}}
\index{function!succ N@{$\succN$}}
\index{natural numbers!operations on N@{operations on $\N$}!succ N@{$\succN$}}
\index{succ N@{$\succN$}}

\bigskip
\begin{minipage}{.45\textwidth}
  \begin{prooftree}
    \AxiomC{}
    \UnaryInfC{$\vdash \zeroN:\N$}
  \end{prooftree}
\end{minipage}
\begin{minipage}{.45\textwidth}
  \begin{prooftree}
    \AxiomC{}
    \UnaryInfC{$\vdash \succN:\N\to\N$}
  \end{prooftree}
\end{minipage}

\bigskip
\begin{rmk}
  We annotate the terms $\zeroN$ and $\succN$ of type $\N$ with their type in the subscript, as a reminder that $\zeroN$ and $\succN$ are declared to be terms of type $\N$, and not of any other type. In the next chapter we will introduce the type $\Z$ of the integers, on which we can also define a zero term $\zeroZ$, and a successor function $\succZ$. These should be distinguished from the terms $\zeroN$ and $\succN$. In general, we will make sure that every term is given a unique name. In libraries of mathematics formalized in a computer proof assistant it is also the case that every type must be given a unique name.
\end{rmk}

\subsubsection{The elimination rule of $\N$}

\index{natural numbers!rules for N@{rules for $\N$}!elimination|see {induction}}
\index{natural numbers!rules for N@{rules for $\N$}!induction principle|(}
\index{induction principle!of N@{for $\N$}|(}
To prove properties about the natural numbers, we postulate an \emph{induction principle} for $\N$. For a typical example, it is easy to show by induction that
\begin{equation*}
  1+\dots+n=\frac{n(n+1)}{2}.
\end{equation*}
Similarly, we can define operations by recursion on the natural numbers: the Fibonacci sequence\index{Fibonacci sequence}\index{natural numbers!operations on N@{operations on $\N$}!Fibonacci sequence} is defined by $F(0)=0$, $F(1)=1$, and
\begin{equation*}
  F(n+2)=F(n)+F(n+1).
\end{equation*}
Needless to say, we want an induction principle to hold for the natural numbers in type theory and we also want it to be possible to construct operations on the natural numbers by recursion.

In dependent type theory we may think of a type family $P$ over $\N$ as a \emph{predicate} over $\N$. Especially after we introduce a few more type-forming operations, such as $\Sigma$-types and identity types, it will become clear that the language of dependent type theory expressive enough to find definitions of all of the standard concepts and operations of elementary number theory in type theory. Many of those definitions, the ordering relations $\leq$ and $<$ for example, will make use of type dependency. Then, to prove that $P(n)$ `holds' for all $n$ we just have to construct a dependent function
\begin{equation*}
  \prd{n:\N}P(n).
\end{equation*}

The induction principle for the natural numbers in type theory exactly states what one has to do in order to construct such a dependent function, via the following inference rule:\index{ind N@{$\ind{\N}$}}\index{rules!for N@{for $\N$}!induction principle}\index{natural numbers!indN@{$\indN$}}
\begin{prooftree}
  \def\fCenter{\Gamma}
  \Axiom$\fCenter, n:\N\vdash P(n)~\type$
  \noLine
  \UnaryInf$\fCenter\ \vdash p_0:P(\zeroN)$
  \noLine
  \UnaryInf$\fCenter\ \vdash p_S:\prd{n:\N}P(n)\to P(\succN(n))$
  \RightLabel{$\N$-ind}
  \UnaryInf$\fCenter\ \vdash \ind{\N}(p_0,p_S):\prd{n:\N} P(n)$
\end{prooftree}
Just like for the usual induction principle of the natural numbers, there are two things to be constructed given a type family $P$ over $\N$: in the \define{base case}\index{base case} we need to construct a term $p_0:P(\zeroN)$, and for the \define{inductive step}\index{inductive step} we need to construct a function of type $P(n)\to P(\succN(n))$ for all $n:\N$. And this comes at one immediate advantage: induction and recursion in type theory are one and the same thing!

\begin{rmk}
  We might alternatively present the induction principle of $\N$ as the following inference rule
  \begin{prooftree}
    \AxiomC{$\Gamma,n:\N\vdash P(n)~\type$}
    \UnaryInfC{$\Gamma\vdash \indN : P(\zeroN)\to \Big(\Big(\prd{n:\N}P(n)\to P(\succN(n))\Big)\to \prd{n:\N}P(n)\Big)$}
  \end{prooftree}
  In other words, for any type family $P$ over $\N$ there is a \emph{function} $\ind{\N}$ that takes two arguments, one for the base case and one for the inductive step, and returns a section of $P$. Now it is justified to wonder: is this slightly different presentation of induction equivalent to the previous presentation?
  
  To see that indeed we get such a function from the induction principle (rule $\N$-ind above), we note that the induction principle is stated to hold in an \emph{arbitrary} context $\Gamma$. So let us wield the power of type dependency: by weakening and the variable rule we have the following well-formed terms:
  \begin{align*}
    \Gamma,~p_0:P(\zeroN),~p_S:\prd{n:\N}P(n)\to P(\succN(n)) & \vdash p_0 : P(\zeroN) \\
    \Gamma,~p_0:P(\zeroN),~p_S:\prd{n:\N}P(n)\to P(\succN(n)) & \vdash p_S : \prd{n:\N}P(n)\to P(\succN(n)).
  \end{align*}
  Therefore, the induction principle of $\N$ provides us with a term
  \begin{equation*}
    \Gamma,~p_0:P(\zeroN),~p_S:\prd{n:\N}P(n)\to P(\succN(n)) \vdash \indN(p_0,p_S) : \prd{n:\N}P(n).
  \end{equation*}
  By $\lambda$-abstraction we now obtain a function
  \begin{equation*}
    \indN : P(\zeroN)\to \Big(\Big(\prd{n:\N}P(n)\to P(\succN(n))\Big) \to \prd{n:\N}P(n)\Big)
  \end{equation*}
  in context $\Gamma$. Therefore we see that it does not really matter whether we present the induction principle of $\N$ in a more verbose way as an inference rule with the base case and the inductive step as hypotheses, or as a function taking variables for the base case and the inductive step as arguments.
\end{rmk}
\index{natural numbers!rules for N@{rules for $\N$}!induction|)}
\index{induction principle!for N@{for $\N$}|)}

\subsubsection{The computation rules of $\N$}

\index{computation rules!for N@{for $\N$}|(}
\index{natural numbers!rules for N@{rules for $\N$}!computation rules|(}
The \define{computation rules} for $\N$ postulate that the dependent function $\ind{\N}(P,p_0,p_S)$ behaves as expected when it is applied to $\zeroN$ or a successor. There is one computation rule for each step in the induction principle, covering the base case and the inductive step.

The computation rule for the base case is\index{rules!for N@{for $\N$}!computation rules|(}
\begin{prooftree}
    \def\fCenter{\Gamma}
  \Axiom$\fCenter, n:\N\vdash P(n)~\type$
  \noLine
  \UnaryInf$\fCenter\ \vdash p_0:P(\zeroN)$
  \noLine
  \UnaryInf$\fCenter\ \vdash p_S:\prd{n:\N}P(n)\to P(\succN(n))$
  \UnaryInf$\fCenter\ \vdash \ind{\N}(p_0,p_S,\zeroN)\jdeq p_0 : P(\zeroN)$
\end{prooftree}
Similarly, with the same hypotheses as for the computation rule for the base case, the computation rule for the inductive step is
\begin{prooftree}
\AxiomC{$\cdots$}
\UnaryInfC{$\Gamma, n:\N \vdash  \ind{\N}(p_0,p_S,\succN(n))\jdeq p_S(n,\ind{\N}(p_0,p_S,n)) : P(\succN(n))$}
\end{prooftree}\index{rules!for N@{for $\N$}!computation rules|)}

This completes the formal specification of $\N$.
\index{computation rules!for N@{for $\N$}|)}
\index{natural numbers!rules for N@{rules for $\N$}!computation rules|)}

\subsection{Addition on the natural numbers}

\index{addition on N@{addition on $\N$}|(}
\index{natural numbers!operations on N@{operations on $\N$}!addition|(}
\index{function!addition on N@{addition on $\N$}|(}
Using the induction principle of $\N$ we can perform many familiar constructions. 
For instance, we can define the \define{addition operation} by induction on $\N$.

\begin{defn}
  We define a function\index{add N@{$\addN$}}\index{natural numbers!operations on N@{operations on $\N$}!add N@{$\addN$}}
  \begin{equation*}
    \addN:\N\to (\N\to\N)
  \end{equation*}
  satisfying $\addN(\zeroN,n)\jdeq n$ and $\addN(\succN(m),n)\jdeq\succN(\addN(m,n))$. Usually we will write $n+m$ for $\addN(n,m)$.
\end{defn}

We first give an informal construction of the addition operation, explaining the ideas behind the construction. This is important, because there are many binary operations on the natural numbers. The correctness of a formal construction of a term
\begin{equation*}
  \vdash \N\to(\N\to\N)
\end{equation*}
only shows us that we have correctly constructed a binary operation on the natural numbers, but this doesn't tell us that the operation we've defined is deserving of the name addition. There are indeed many binary operations on the natural numbers, such as the $\minN$, $\maxN$, and multiplication operations, so we need to be careful to make sure that the binary operation we are constructing really is the addition operation.

\begin{proof}[Informal construction]
  Our goal is to construct a function of type
  \begin{equation*}
    \vdash \addN:\N\to (\N\to\N).
  \end{equation*}
  By $\lambda$-abstraction it therefore suffices to construct a term
  \begin{equation*}
    m:\N\vdash \addN(m):\N\to\N.
  \end{equation*}
  Such a term is constructed by induction. Since we are defining addition, we want our definition of $\addN$ to be such that
  \begin{align*}
    \addN(m,\zeroN) & \jdeq n \\
    \addN(m,\succN(n)) & \jdeq \succN(\addN(m,n)). 
  \end{align*}
  In other words, our definition of addition is such that $m+0\jdeq m$ and $m+\succN(n)\jdeq \succN(m+n)$.

  The inductive proof requires us to define a term
  \begin{equation*}
    n:\N\vdash \addzeroN(n)\defeq n:\N
  \end{equation*}
  in the base case, and a term
  \begin{equation*}
    n:\N \vdash \addsuccN(n):\N\to(\N\to\N)
  \end{equation*}
  in the inductive step. The result of the inductive proof will then be a function $\addN(n):\N\to\N$ satisfying
  \begin{align*}
    n:\N & \vdash \addN(n,\zeroN) \jdeq \addzeroN(n) : \N \\
    n:\N & \vdash \addN(n,\succN(m)) \jdeq \addsuccN(n,m,\addN(n,m)).
  \end{align*}
  Anticipating these computation rules, we see that the following choices result in an addition operation with the expected behavior:
  \begin{align*}
    n:\N & \vdash \addzeroN(n)\defeq n : \N \\
    n:\N & \vdash \addsuccN(n)\defeq\const_{\succN}:\N\to(\N\to\N).\qedhere
  \end{align*}
\end{proof}

\begin{proof}[Formal derivation]
The derivation for the construction of $\addsuccN$ looks as follows:
\begin{prooftree}
  \AxiomC{}
  \UnaryInfC{$\vdash\N~\type$}
  \AxiomC{}
  \UnaryInfC{$\vdash\N~\type$}
  \AxiomC{}
  \UnaryInfC{$\vdash \succN:\N\to\N$}
  \BinaryInfC{$x:\N\vdash \succN:\N\to\N$}
  \BinaryInfC{$n:\N,x:\N \vdash \succN:\N\to\N$}
  \UnaryInfC{$n:\N \vdash \addsuccN(n) \defeq \lam{x}\succN:\N\to (\N \to \N)$}
\end{prooftree}
We combine this derivation with the induction principle of $\N$ to complete the construction of addition:
\begin{prooftree}
  \AxiomC{$\vdots$}
  \UnaryInfC{$n:\N\vdash \addzeroN(n) \defeq n:\N$}
  \AxiomC{$\vdots$}
  \UnaryInfC{$n:\N\vdash \addsuccN(n):\N\to (\N \to \N)$}
  \BinaryInfC{$n:\N\vdash\addN(n)\jdeq\indN(\addzeroN,\addsuccN):\N\to \N$}
\end{prooftree}
The asserted judgmental equalities then hold by the computation rules for $\N$.
\end{proof}

\begin{rmk}
  When we define a function $f:\prd{n:\N} P(n)$, we will often do so just by indicating its definition on $\zeroN$ and its definition on $\succN(n)$, by writing
  \begin{align*}
    f(\zeroN) & \defeq p_0 \\
    f(\succN(n)) & \defeq p_S(n,f(n)).
  \end{align*}
  For example, the definition of addition on the natural numbers could be given as
  \begin{align*}
    \addN(\zeroN,n) & \defeq n \\
    \addN(\succN(m),n) & \defeq \succN(\addN(m,n)).
  \end{align*}
  This way of defining a function is called \emph{pattern matching}\index{pattern matching}. A more formal inductive argument can be obtained from a definition by pattern matching if it is possible to obtain from the expression $p_S(n,f(n))$ a general dependent function
  \begin{equation*}
    p_S : \prd{n:\N} P(n)\to P(\succN(n)).
  \end{equation*}
  In practice this is usually the case. Computer proof assistants such as Agda have sophisticated algorithms to allow for definitions by pattern matching.
\end{rmk}

\begin{rmk}
  By the computation rules for $\N$ it follows that
  \begin{equation*}
    m+\zeroN\jdeq m,\qquad\text{and}\qquad m+\succN(n)\jdeq\succN(m+n).
  \end{equation*}
  A simple consequence of this definition is that $\succN(n)\jdeq n+1$, as one would expect. However, the rules that we provided so far are not sufficient to also conclude that $\zeroN+n\jdeq n$ and $\succN(m) + n\jdeq \succN(m+n)$. In fact, we will not be able to prove such judgmental equalities. Nevertheless, once we have introduced the \emph{identity type} in \cref{chap:identity} we will be able to \emph{identify} $\zeroN+n$ with $n$, and $\succN(m)+n$ with $\succN(m+n)$. See \cref{ex:semi-ring-laws-N}. 
\end{rmk}
\index{addition on N@{addition on $\N$}|)}
\index{natural numbers!operations on N@{operations on $\N$}!addition|)}
\index{function!addition on N@{addition on $\N$}|)}

\begin{exercises}
\exercise Define the binary \define{min} and \define{max} functions
  \index{minimum function}
  \index{maximum function}
  \index{function!minN@{$\minN$}}
  \index{function!maxN@{$\maxN$}}
  \index{natural numbers!operations on N@{operations on $\N$}!minN@{$\minN$}}
  \index{natural numbers!operations on N@{operations on $\N$}!maxN@{$\maxN$}}
  \begin{equation*}
    \minN,\maxN:\N\to(\N\to\N).
  \end{equation*}
\exercise Define the \define{multiplication} operation
  \index{multiplication!on N@{on $\N$}}
  \index{function!mul N@{$\mulN$}}
  \index{natural numbers!operations on N@{operations on $\N$}!mul N@{$\mulN$}}
  \index{mul N@{$\mulN$}}
  \begin{equation*}
    \mulN :\N\to(\N\to\N).
  \end{equation*}
\exercise Define the \define{exponentiation function} $n,m\mapsto m^n$ of type $\N\to (\N\to \N)$.
  \index{exponentiation function on N@{exponentiation function on $\N$}}
  \index{function!exponentiation on N@{exponentiation on $\N$}}
  \index{natural numbers!operations on N@{operations on $\N$}!exponentiation}
\exercise Define the \define{factorial} operation $n\mapsto n!$.
  \index{factorial operation}
  \index{function!factorial operation}
  \index{natural numbers!operations on N@{operations on $\N$}!n factorial@{$n"!$}}
\exercise Define the \define{binomial coefficient} $\binom{n}{k}$ for any $n,k:\N$, making sure that $\binom{n}{k}\jdeq 0$ when $n<k$.
  \index{binomial coefficient}
  \index{function!binomial coefficient}
  \index{natural numbers!operations on N@{operations on $\N$}!binomial coefficient}
  \exercise Use the induction principle of $\N$ to define the \define{Fibonacci sequence} as a function $F:\N\to\N$ that satisfies the equations
  \begin{align*}
    F(\zeroN) & \jdeq \zeroN \\
    F(\oneN) & \jdeq \oneN \\
    F(\succN(\succN(n))) & \jdeq F(n)+F(\succN(n)).
  \end{align*}
  \index{Fibonacci sequence}
  \index{natural numbers!operations on N@{operations on $\N$}!Fibonacci sequence}
\end{exercises}
\index{natural numbers|)}
