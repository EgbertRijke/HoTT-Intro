\chapter{The type of natural numbers}

The single most important object of mathematical study is the set of natural numbers. In homotopy type theory its position of importance is only challenged by that of the identity type.

The type of natural numbers comes equipped with a \define{zero element} and a \define{successor function}
\begin{align*}
0 & : \N \\
S & : \N\to \N.
\end{align*}
Thus we first have to introduce \emph{function types}. 
It turns out however, that for many useful operations are of a more general nature, where the type of the outputs may depend on the input.  
For example, if we are given a sequence $(X_n)_{n\in\N}$ of pointed spaces, then the operation that assigns to each $n$ the base point of $X_n$ is clearly of this kind.

Therefore we will introduce \emph{dependent function types}, i.e.~function types of which the types of the outputs are allowed to vary over the input.
Dependent function types will be used to formulate the induction principle for the natural numbers.



\begin{exercises}
\item The \define{greatest common divisor} of any two natural numbers $m$ and $n$ is defined as the largest natural number $d$ such that both $d\mid m$ and $d\mid n$. Define the greatest common divisor as a function $\mathrm{gcd}:\N\to\N\to\N$ in type theory.
\end{exercises}
