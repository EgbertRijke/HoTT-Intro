\section{Finite types}

\subsection{The pigeonhole principle}

The pigeonhole principle states that if we place more than $n$ balls in $n$ bags, then at least one bag will contain more than one ball. In this section we will give a type theoretical proof of the pigeonhole principle.

First we give a definition of a function that counts the number of elements in a decidable subset of $\Fin(n)$.

\begin{defn}
  Let $P$ be a decidable subset of $\Fin(n)$. We define the number $|P|:\N$ of elements in $P$.
\end{defn}

\begin{proof}[Construction]
  We give the construction of $|P|$ by induction on $n:\N$. In the base case we note that $\Fin(\zeroN)$ has no elements, so we define $|P|\defeq\zeroN$.

  For the inductive step, we define $|P|$ by case analysis on $P(\inr(\ttt))+\neg P(\inr(\ttt))$. Let $P'$ be the family over $\Fin(n)$ given by $P'(i)\defeq P(\inl(i))$. In the case where $P(\inr(\ttt))$ holds, then we define $|P|\defeq \succN |P'|$. In the case where $P(\inr(\ttt))$ doesn't hold we define $|P|\defeq |P'|$.
\end{proof}

\begin{defn}
  For any $i:\Fin(\succN(n))$ we define a function
  \begin{equation*}
    \skipFin(i):\Fin(n)\to\Fin(\succN(n)).
  \end{equation*}
\end{defn}

\begin{proof}[Construction]
  The function $\skipFin(i)$ is defined by induction on $n:\N$. In the base case, the function
  \begin{equation*}
    \skipFin(i):\Fin(\zeroN)\to\Fin(\succN(\zeroN))
  \end{equation*}
  is defined to be the unique map out of the empty type. In the successor case we define
  \begin{equation*}
    \skipFin(i) : \Fin(\succN(n))\to\Fin(\succN(\succN(n))) 
  \end{equation*}
  by induction on $i:\Fin(\succN(\succN(n)))$. The function
  \begin{equation*}
    \skipFin(\inl(i)):\Fin(\succN(n))\to\Fin(\succN(\succN(n)))
  \end{equation*}
  is a map between coproducts, so it can be defined using the functorial action of coproducts of \cref{ex:coproduct_functor}. We take
  \begin{equation*}
    \skipFin(\inl(i))\defeq \skipFin(i)+\idfunc.
  \end{equation*}
  The function 
  \begin{equation*}
    \skipFin(\inr(i)):\Fin(\succN(n))\to\Fin(\succN(\succN(n)))
  \end{equation*}
  is just the function $\inl$.
\end{proof}

\begin{lem}
  For each $i:\Fin(\succN(n))$, the function
  \begin{equation*}
    \skipFin(i):\Fin(n)\to\Fin(\succN(n))
  \end{equation*}
  is an embedding.
\end{lem}

\begin{proof}
  This assertion is proven by induction on $n$. In the base case, we note that any map out of the empty type is an embedding, by \cref{ex:is-emb-empty}. In the inductive step we proceed by case analysis on $i:\Fin(\succN(\succN(n)))$. In the case of $\inl(i)$ we note that
  \begin{equation*}
    \skipFin(\inl(i))\jdeq \skipFin(i)+\idfunc
  \end{equation*}
  is the functorial action of coproducts on two embeddings. Therefore we conclude by \cref{ex:is-emb-coprod} that this map is an embedding. In the case of $\inr(i)$ we note that $\inl$ is an embedding by \cref{ex:is-emb-inl-inr}.
\end{proof}

\begin{lem}
  Consider a map $g:\Fin(m)\to\Fin(\succN(n))$. Furthermore, suppose that $i:\Fin(\succN(n))$ is not in the image of $g$, i.e.~that $\neg(\fib{g}{i})$. Then we can construct a commuting triangle
  \begin{equation*}
    \begin{tikzcd}
      & \Fin(n) \arrow[d,"\skipFin(i)"] \\
      \Fin(m) \arrow[r,swap,"g"] \arrow[ur,densely dotted,"f"] & \Fin(\succN(n)).
    \end{tikzcd}
  \end{equation*}
\end{lem}

Finally, we prove the pigeonhole principle.

\begin{thm}\label{thm:pigeonhole}
  For any $m,n:\N$ and any function $f:\Fin(m)\to\Fin(n)$, if $m>n$, then there is an $i:\Fin(n)$ which is in the image of more than one point in $\Fin(m)$.
\end{thm}

\begin{proof}
  The pigeonhole principle is proven by induction on $m,n:\N$. In the base case for $m$ there is nothing to show because $m>n$ is empty. For the inductive step on $m$ and the base case for $n$, we note that $\Fin(\succN(m))\jdeq \Fin(m)+\unit$ and $\Fin(\zeroN)\jdeq \emptyt$. Therefore $f:\Fin(\succN(m))\to\Fin(\zeroN)$ is a function from a pointed type to the empty type, which gives us a contradiction.

  It remains to give the inductive step for $n$. Let $i\defeq f(\inr(\ttt)):\Fin(\succN(n))$. Since the ordering relation $<$ on $\N$ is decidable, we can decide whether $i$ is in the image of more than one point in $\Fin(m)$ by deciding whether or not $1<|P|$ holds for
  \begin{equation*}
    P(j)\defeq (f(j)=i)
  \end{equation*}
  If this is the case, this completes the proof.

  Now suppose that $1\not<|P|$. Since $P(\inr(\ttt))$ holds it follows that $|P|=1$. Now we observe that $i$ is not in the image of $f\circ \inl$. Therefore we obtain a commuting square
  \begin{equation*}
    \begin{tikzcd}
      \Fin(m) \arrow[r,densely dotted,"{f'}"] \arrow[d,swap,"\inl"] & \Fin(n) \arrow[d,"{\skipFin(i)}"] \\
      \Fin(\succN(m)) \arrow[r,swap,"f"] & \Fin(\succN(n)).
    \end{tikzcd}
  \end{equation*}

  Note that the induction hypothesis the pigeonhole principle applies to the function $f':\Fin(m)\to\Fin(n)$. Since $m>n$ it follows that there is an element $j:\Fin(n)$ that is in the image of $f'$ of more than one element of $\Fin(m)$. Now we observe that there is an equivalence
  \begin{equation*}
    \fib{f'}{j}\simeq \fib{f}{\skipFin(i,j)}
  \end{equation*}
  because both the left and right maps in the commuting square are embeddings. Therefore we conclude that $\skipFin(i,j)$ is in the image of $f$ of more than one element of $\Fin(\succN(m))$. 
 \end{proof}

\begin{cor}\label{cor:pigeonhole}
  Given $m>n$, no function $\Fin(m)\to\Fin(n)$ is an embedding.
\end{cor}

It is straightforward to see that the statements of \cref{thm:pigeonhole,cor:pigeonhole} are equivalent, and one might argue that the statement of \cref{cor:pigeonhole} is the more `type theoretical way' of phrasing the pigeonhole principle. However, the relation to counting the number of points that get mapped to 

\begin{thm}\label{thm:generalized-pigeonhole}
  For any $m,n:\N$ and any function $f:\Fin(m)\to\Fin(n)$, if $m>kn$ for some $k:\N$, then there is an $i:\Fin(n)$ which is in the image of more than $k$ points in $\Fin(m)$. 
\end{thm}