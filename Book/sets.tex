\section{Set quotients}

\subsection{Equivalence relations}

\begin{defn}\label{defn:eq_rel}
Let $R:A\to (A\to\prop)$ be a binary relation valued in the propositions. We say that $R$ is an \define{($0$-)equivalence relation}\index{equivalence relation}\index{0-equivalence relation|see {equivalence relation}} if $R$ comes equipped with
\begin{align*}
\rho & : \prd{x:A}R(x,x) \\
\sigma & : \prd{x,y:A} R(x,y)\to R(y,x) \\
\tau & : \prd{x,y,z:A} R(x,y)\to (R(y,z)\to R(x,z)).
\end{align*}
Given an equivalence relation $R:A\to (A\to\prop)$, the \define{equivalence class}\index{equivalence class} $[x]_R$ of $x:A$ is defined to be
\begin{equation*}
[x]_R\defeq R(x).
\end{equation*}
\end{defn}

\begin{defn}
Let $R:A\to (A\to\prop)$ be a $0$-equivalence relation. 
We define for any $x,y:A$ a map\index{class_eq@{$\mathsf{class\usc{}eq}$}}
\begin{equation*}
\mathsf{class\usc{}eq}:R(x,y)\to ([x]_R=[y]_R).
\end{equation*}
\end{defn}

\begin{proof}[Construction.]
Let $r:R(x,y)$. By function extensionality, the identity type $R(x)=R(y)$ is equivalent to the type
\begin{equation*}
\prd{z:A}R(x,z)=R(y,z).
\end{equation*}
Let $z:A$. By the univalence axiom, the type $R(x,z)=R(y,z)$ is equivalent to the type
\begin{equation*}
\eqv{R(x,z)}{R(y,z)}.
\end{equation*}
We have the map $\tau_{y,x,z}(\sigma(r)):R(x,z)\to R(y,z)$. Since this is a map between propositions, we only have to construct a map in the converse direction to show that it is an equivalence. The map in the converse direction is just $\tau_{x,y,z}(r):R(y,z)\to R(x,z)$. 
\end{proof}

\begin{thm}\label{thm:equivalence_classes}
Let $R:A\to (A\to\prop)$ be a $0$-equivalence relation. 
Then for any $x,y:A$ the map
\begin{equation*}
\mathsf{class\usc{}eq} : R(x,y)\to ([x]_R=[y]_R)
\end{equation*}
is an equivalence.
\end{thm}

\begin{proof}
By the 3-for-2 property of equivalences, it suffices to show that the map
\begin{equation*}
\lam{r}{z}\tau_{y,x,z}(\sigma(r)) : R(x,y)\to \prd{z:A} \eqv{R(x,z)}{R(y,z)}
\end{equation*}
is an equivalence. Since this is a map between propositions, it suffices to construct a map of type
\begin{equation*}
\Big(\prd{z:A} \eqv{R(x,z)}{R(y,z)}\Big)\to R(x,y).
\end{equation*}
This map is simply $\lam{f} \sigma_{y,x}(f_x(\rho(x)))$. 
\end{proof}

\begin{rmk}
By \cref{thm:equivalence_classes} we can begin to think of the \emph{quotient}\index{quotient} $A/R$ of a type $A$ by an equivalence relation $R$. Classically, the quotient is described as the set of equivalence classes, and \cref{thm:equivalence_classes} establishes that two equivalence classes $[x]_R$ and $[y]_R$ are equal precisely when $x$ and $y$ are related by $R$.

However, the type $A\to\prop$ may contain many more terms than just the $R$-equivalence classes. Therefore we are facing the task of finding a type theoretic description of the smallest subtype of $A\to\prop$ containing the equivalence classes.
Another to think about this is as the \emph{image}\index{image} of $R$ in $A\to \prop$. 
The construction of the (homotopy) image of a map can be done with \emph{higher inductive types}\index{higher inductive type}, which we will do in \cref{chap:image}.
\end{rmk}

The notion of $0$-equivalence relation which we defined in \cref{defn:eq_rel} fits in a hierarchy of `$n$-equivalence relations'\index{n-equivalence relation@{$n$-equivalence relation}}, the study of which is a research topic on its own. However, we already know an example of a relation that should count as an `$\infty$-equivalence relation'\index{infinity-equivalence relation@{$\infty$-equivalence relation}}: the identity type. Analogous to \cref{thm:equivalence_classes}, the following theorem shows that the canonical map
\begin{equation*}
(x=y)\to (\idtypevar{A}(x)=\idtypevar{A}(y))
\end{equation*}
is an equivalence, for any $x,y:A$. In other words, $\idtypevar{A}(x)$ can be thought of as the equivalence class of $x$ with respect to the relation $\idtypevar{A}$.

\begin{thm}
Assuming the univalence axiom on $\UU$, the map
\begin{equation*}
\idtypevar{A}:A\to (A\to\UU)
\end{equation*}
is an embedding, for any type $A:\UU$.\index{identity type!is an embedding}
\end{thm}

\begin{proof}
Let $a:A$. By function extensionality it suffices to show that the canonical map
\begin{equation*}
(a=b)\to \idtypevar{A}(a)\htpy\idtypevar{A}(b)
\end{equation*}
that sends $\refl{a}$ to $\lam{x}\refl{(a=x)}$ is an equivalence for every $b:A$, and by univalence it therefore suffices to show that the canonical map
\begin{equation*}
(a=b)\to \prd{x:A}\eqv{(a=x)}{(b=x)}
\end{equation*}
that sends $\refl{a}$ to $\lam{x}\idfunc[(a=x)]$ is an equivalence for every $b:B$. To do this we employ the type theoretic Yoneda lemma, \cref{thm:yoneda}.

By the type theoretic Yoneda lemma\index{Yoneda lemma} we have an equivalence
\begin{equation*}
\Big(\prd{x:A} (b=x)\to (a=x)\Big)\to (a=b)
\end{equation*}
given by $\lam{f} f(b,\refl{b})$, for every $b:A$. Note that any family of maps $\prd{x:A}(b=x)\to (a=x)$ induces an equivalence of total spaces by \cref{ex:contr_equiv}, since their total spaces are are both contractible by \cref{cor:contr_path}. It follows that we have an equivalence
\begin{equation*}
\varphi_b:\Big(\prd{x:A} \eqv{(b=x)}{(a=x)}\Big)\to (a=b)
\end{equation*}
given by $\lam{f} f(b,\refl{b})$, for every $b:A$. 

Note that $\varphi_a(\lam{x}\idfunc[(a=x)])\jdeq\refl{a}$. Therefore it follows by another application of \cref{thm:yoneda} that the unique family of maps 
\begin{equation*}
\alpha_b:(a=b)\to \Big(\prd{x:A} \eqv{(b=x)}{(a=x)}\Big)
\end{equation*}
that satisfies $\alpha_a(\refl{a})=\lam{x}\idfunc[(a=x)]$ is a family of sections of $\varphi$. 
It follows that $\alpha$ is a family of equivalences. Now the proof is completed by reverting the direction of the family of equivalences in the codomain.
\end{proof}

\subsection{The universal property of set quotients}

\begin{defn}
Let $R:A\to (A\to \prop)$ be an equivalence relation\index{equivalence relation}, for $A:\UU$, and consider a map $q:A\to B$ where the type $B$ is a set, for which we have
\begin{equation*}
\prd{x,y:A}R(x,y)\to q(x)=q(y).
\end{equation*}
We will define a map
\begin{equation*}
\quotientrestr:(B\to X) \to \Big(\sm{f:A\to X}\prd{x,y:A}R(x,y)\to (f(x)=f(y))\Big).
\end{equation*}
\end{defn}

\begin{constr}
Let $h:B\to X$. Then we have $h\circ q : A\to X$, so it remains to show that
\begin{equation*}
\prd{x,y:A}R(x,y)\to (h(q(x))=h(q(y)))
\end{equation*}
Consider $x,y:A$ which are related by $R$. Then we have an identification $p:q(x)=q(y)$, so it follows that $\ap{h}{p}:h(q(x))=h(q(y))$.  
\end{constr}

\begin{defn}
Let $R:A\to (A\to \prop)$ be an equivalence relation\index{equivalence relation}, for $A:\UU$, and consider a map $q:A\to B$ satisfying
\begin{equation*}
\prd{x,y:A}R(x,y)\to q(x)=q(y),
\end{equation*}
where the type $B$ is a set. We say that the map $q:A\to B$ satisfies the universal property of the \define{set quotient}\index{set quotient}\index{universal property!of set quotients} $A/R$ if for any set $X$ the map
\begin{equation*}
\quotientrestr : (B\to X) \to \Big(\sm{f:A\to X}\prd{x,y:A}R(x,y)\to (f(x)=f(y))\Big)
\end{equation*}
is an equivalence.
\end{defn}

\begin{lem}
Let $R:A\to (A\to \prop)$ be an equivalence relation\index{equivalence relation}, for $A:\UU$, and consider a commuting triangle
\begin{equation*}
\begin{tikzcd}[column sep=tiny]
A \arrow[rr,"q"] \arrow[dr,swap,"R"] & & U \arrow[dl,"m"] \\
& \prop^A
\end{tikzcd}
\end{equation*}
with $H:R\htpy m\circ q$, where $m$ is an embedding. Then we have
\begin{equation*}
\prd{x,y:A}R(x,y)\to (q(x)=q(y)).
\end{equation*}
\end{lem}

\begin{thm}\label{thm:quotient_up}
Let $R:A\to (A\to \prop)$ be an equivalence relation\index{equivalence relation}, for $A:\UU$, and consider a commuting triangle
\begin{equation*}
\begin{tikzcd}[column sep=tiny]
A \arrow[rr,"q"] \arrow[dr,swap,"R"] & & U \arrow[dl,"m"] \\
& \prop^A
\end{tikzcd}
\end{equation*}
with $H:R\htpy m\circ q$, where $m$ is an embedding. Then the following are equivalent:
\begin{enumerate}
\item The embedding $m:U\to \prop^A$ satisfies the universal property of the image of $R$.
\item The map $q:A\to U$ satisfies the universal property of the set quotient $A/R$.
\end{enumerate}
\end{thm}

\begin{proof}
Suppose $m:U\to \prop^A$ satisfies the universal property of the image of $R$. Then it follows by \cref{thm:surjective} that the map $q:A\to U$ is surjective. Our goal is to prove that $U$ satisfies the universal property of the set quotient $A/R$. 
\end{proof}

\begin{rmk}
\cref{thm:quotient_up} suggests that we can define the quotient of an equivalence relation $R$ on a type $A$ as the image of a map. However, the type $\prop^A$ of which the quotient is a subtype is not a small type, even if $A$ is a small type.
Therefore it is not clear that the quotient $A/R$ is essentially small\index{essentially small}, as it should be. Luckily, our construction of the image of a map allows us to show that the image is indeed essentially small, using the fact that $\prop^A$ is locally small\index{locally small}.
\end{rmk}

\subsection{The construction of set quotients}
\begin{lem}
Consider a commuting square
\begin{equation*}
\begin{tikzcd}
A \arrow[r] \arrow[d] & B \arrow[d] \\
C \arrow[r] & D.
\end{tikzcd}
\end{equation*}
\begin{enumerate}
\item If the square is cartesian, $B$ and $C$ are essentially small, and $D$ is locally small, then $A$ is essentially small.
\item If the square is cocartesian, and $A$, $B$, and $C$ are essentially small, then $D$ is essentially small. 
\end{enumerate}
\end{lem}

\begin{cor}
Suppose $f:A\to X$ and $g:B\to X$ are maps from essentially small types $A$ and $B$, respectively, to a locally small type $X$. Then $A\times_X B$ is again essentially small. 
\end{cor}

\begin{lem}
Consider a type sequence
\begin{equation*}
\begin{tikzcd}
A_0 \arrow[r,"f_0"] & A_1 \arrow[r,"f_1"] & A_2 \arrow[r,"f_2"] & \cdots
\end{tikzcd}
\end{equation*}
where each $A_n$ is essentially small. Then its sequential colimit is again essentially small. 
\end{lem}

\begin{thm}
For any map $f:A\to X$ from a small type $A$ into a locally small type $X$, the image $\im(f)$ is an essentially small type.
\end{thm}

Recall that in set theory, the replacement axiom asserts that for any family of sets $\{X_i\}_{i\in I}$ indexed by a set $I$, there is a set $X[I]$ consisting of precisely those sets $x$ for which there exists an $i\in I$ such that $x\in X_i$. In other words: the image of a set-indexed family of sets is again a set. Without the replacement axiom, $X[I]$ would be a class. In the following corollary we establish a type-theoretic analogue of the replacement axiom: the image of a family of small types indexed by a small type is again (essentially) small.

\begin{cor}\label{cor:im_small}
For any small type family $B:A\to\UU$, where $A$ is small, the image $\im(B)$ is essentially small. We call $\im(B)$ the \define{univalent completion} of $B$. 
\end{cor}

\subsection{Connected components of types}

\subsection{Set truncation}

\begin{lem}
For each type $A$, the relation $I_{(-1)}:A\to (A\to\prop)$ given by
\begin{equation*}
I_{(-1)}(x,y)\defeq\brck{x=y}
\end{equation*}
is an equivalence relation.
\end{lem}

\begin{proof}
For every $x:A$ we have $\bproj{\refl{x}}:\brck{x=x}$, so the relation is reflexive. To see that the relation is symmetric note that by the universal property of propositional truncation there is a unique map $\brck{\invfunc}:\brck{x=y}\to\brck{y=x}$ for which the square
\begin{equation*}
\begin{tikzcd}
(x=y) \arrow[r,"\invfunc"] \arrow[d,swap,"\bproj{\blank}"] & (y=x) \arrow[d,"\bproj{\blank}"] \\
\brck{x=y} \arrow[r,densely dotted,swap,"\brck{\invfunc}"] & \brck{y=x}
\end{tikzcd}
\end{equation*}
commutes. This shows that the relation is symmetric. Similarly, we show by the universal property of propositional truncation that the relation is transitive.
\end{proof}

\begin{defn}
For each type $A$ we define the \define{set truncation}
\begin{equation*}
\trunc{0}{A}\defeq A/I_{(-1)},
\end{equation*}
and the unit of the set truncation is defined to be the quotient map.
\end{defn}

\begin{thm}
For each type $A$, the set truncation satisfies the universal property of the set truncation.
\end{thm}
