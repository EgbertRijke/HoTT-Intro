% !TEX root = hott_intro.tex

\section{Equivalences}

\subsection{Homotopies}
In homotopy type theory, a homotopy is just a pointwise equality between two functions $f$ and $g$.

\begin{defn}
Let $f,g:\prd{x:A}P(x)$ be two dependent functions. The type of \define{homotopies}\index{homotopy|textbf} from $f$ to $g$ is defined as
\begin{equation*}
f\htpy g \defeq \prd{x:A} f(x)=g(x).
\end{equation*}
\end{defn}

Since we formulated homotopies using dependent functions, we may also consider homotopies \emph{between}\index{homotopy!iterated} homotopies, and further homotopies between those higher homotopies. 
Explicitly, if $H,K:f\htpy g$, then the type $H\htpy K$ of homotopies is just the type
\begin{equation*}
\prd{x:A} H(x)=K(x).
\end{equation*}

In the following definition we define the groupoid-like structure of homotopies. Note that we implement the groupoid-laws as \emph{homotopies} rather than as identifications.

\begin{defn}\label{defn:htpy_groupoid}\index{groupoid laws!of homotopies|textbf}
For any dependent type $B:A\to\type$ there are operations
\begin{align*}
& \mathsf{htpy\usc{}refl} & & : \prd{f:\prd{x:A}B(x)}f\htpy f \\
& \mathsf{htpy\usc{}inv} & & : \prd*{f,g:\prd{x:A}B(x)} (f\htpy g)\to(g\htpy f) \\
& \mathsf{htpy\usc{}concat} & & : \prd*{f,g,h:\prd{x:A}B(x)} (f\htpy g)\to ((g\htpy h)\to (f\htpy h)).
\end{align*}
We will write $H^{-1}$ for $\mathsf{htpy\usc{}inv}(H)$, and $\ct{H}{K}$ for $\mathsf{htpy\usc{}concat}(H,K)$. 

Furthermore, we define
\begin{align*}
& \mathsf{htpy\usc{}assoc}(H,K,L) & & : \ct{(\ct{H}{K})}{L}\htpy\ct{H}{(\ct{K}{L})} \\
& \mathsf{htpy\usc{}left\usc{}unit}(H) & & : \ct{\mathsf{htpy\usc{}refl}_f}{H}\htpy H \\
& \mathsf{htpy\usc{}right\usc{}unit}(H) & & : \ct{H}{\mathsf{htpy\usc{}refl}_g}\htpy H \\
& \mathsf{htpy\usc{}left\usc{}inv}(H) & & : \ct{H^{-1}}{H} \htpy \mathsf{htpy\usc{}refl}_g \\
& \mathsf{htpy\usc{}right\usc{}inv}(H) & & : \ct{H}{H^{-1}} \htpy \mathsf{htpy\usc{}refl}_f
\end{align*}
for any $H:f\htpy g$, $K:g\htpy h$ and $L:h\htpy i$, where $f,g,h,i:\prd{x:A}B(x)$.
\end{defn}

\begin{constr}
We define
\begin{align*}
\mathsf{htpy\usc{}refl}(f) & \defeq \lam{x} \refl{f(x)} \\
\mathsf{htpy\usc{}inv}(H) & \defeq \lam{x} H(x)^{-1} \\
\mathsf{htpy\usc{}concat}(H,K) & \defeq \lam{x}\ct{H(x)}{K(x)},
\end{align*}
where $H:f\htpy g$ and $K:g\htpy h$ are homotopies. Furthermore, we define
\begin{align*}
\mathsf{htpy\usc{}assoc}(H,K,L) & \defeq \lam{x}\mathsf{assoc}(H(x),K(x),L(x)) \\
\mathsf{htpy\usc{}left\usc{}unit}(H) & \defeq \lam{x}\mathsf{left\usc{}unit}(H(x)) \\
\mathsf{htpy\usc{}right\usc{}unit}(H) & \defeq \lam{x}\mathsf{right\usc{}unit}(H(x)) \\
\mathsf{htpy\usc{}left\usc{}inv}(H) & \defeq \lam{x}\mathsf{left\usc{}inv}(H(x)) \\
\mathsf{htpy\usc{}right\usc{}inv}(H) & \defeq \lam{x}\mathsf{right\usc{}inv}(H(x)).\qedhere
\end{align*}
\end{constr}


Apart from the groupoid operations and their laws, we will occasionally need \emph{whiskering} operations.

\begin{defn}
We define the following \define{whiskering}\index{homotopy!whiskering operations|textbf}\index{whiskering operations!of homotopies|textbf} operations on homotopies:
\begin{enumerate}
\item Suppose $H:f\htpy g$ for two functions $f,g:A\to B$, and let $h:B\to C$. We define
\begin{equation*}
h\cdot H\defeq \lam{x}\ap{h}{H(x)}:h\circ f\htpy h\circ g.
\end{equation*}
\item Suppose $f:A\to B$ and $H:g\htpy h$ for two functions $g,h:B\to C$. We define
\begin{equation*}
H\cdot f\defeq\lam{x}H(f(x)):h\circ f\htpy g\circ f.
\end{equation*}
\end{enumerate}
\end{defn}

We also use homotopies to express the commutativity of diagrams. For example, we say that a triangle
\begin{equation*}
  \begin{tikzcd}[column sep=tiny]
    A \arrow[rr,"h"] \arrow[dr,swap,"f"] & & B \arrow[dl,"g"] \\
    & X
  \end{tikzcd}
\end{equation*}
commutes if it comes equipped with a homotopy $H:f\htpy g\circ h$, and we say that a square
\begin{equation*}
  \begin{tikzcd}
    A \arrow[r,"g"] \arrow[d,swap,"f"] & A' \arrow[d,"{f'}"] \\
    B \arrow[r,swap,"h"] & B'
  \end{tikzcd}
\end{equation*}
if it comes equipped with a homotopy $h \circ f~g\circ f'$.

\subsection{Bi-invertible maps}
\begin{defn}
Let $f:A\to B$ be a function. We say that $f$ has a \define{section}\index{section!of a map|textbf} if there is a term of type\index{sec(f)@{$\mathsf{sec}(f)$}|textbf}
\begin{equation*}
\mathsf{sec}(f) \defeq \sm{g:B\to A} f\circ g\htpy \idfunc[B].
\end{equation*}
Dually, we say that $f$ has a \define{retraction}\index{retraction} if there is a term of type\index{retr(f)@{$\mathsf{retr}(f)$}|textbf}
\begin{equation*}
\mathsf{retr}(f) \defeq \sm{h:B\to A} h\circ f\htpy \idfunc[A].
\end{equation*}
If a map $f:A \to B$ has a retraction, we also say that $A$ is a \define{retract}\index{retract!of a type} of $B$.
We say that a function $f:A\to B$ is an \define{equivalence}\index{equivalence|textbf}\index{bi-invertible map|see {equivalence}} if it has both a section and a retraction, i.e., if it comes equipped with a term of type\index{is_equiv@{$\isequiv$}|textbf}
\begin{equation*}
\isequiv(f)\defeq\mathsf{sec}(f)\times\mathsf{retr}(f).
\end{equation*}
We will write $\eqv{A}{B}$\index{equiv@{$\eqv{A}{B}$}|textbf} for the type $\sm{f:A\to B}\isequiv(f)$.
\end{defn}

\begin{rmk}
An equivalence, as we defined it here, can be thought of as a \emph{bi-invertible} map, since it comes equipped with a separate left and right inverse. Explicitly, if $f$ is an equivalence, then there are
\begin{align*}
g & : B\to A & h & : B\to A \\
G & : f\circ g \htpy \idfunc[B] & H & : h\circ f \htpy \idfunc[A].
\end{align*}
Clearly, if $f$ is \define{invertible}\index{invertible map} in the sense that it comes equipped with a function $g:B\to A$ such that $f\circ g\htpy\idfunc[B]$ and $g\circ f\htpy\idfunc[A]$, then $f$ is an equivalence. We write\index{is_invertible@{$\mathsf{is\usc{}invertible}$}|textbf}
\begin{equation*}
\mathsf{has\usc{}inverse}(f)\defeq\sm{g:B\to A} (f\circ g\htpy \idfunc[B])\times (g\circ f\htpy\idfunc[A]).
\end{equation*}
\end{rmk}

\begin{defn}\label{defn:inv_equiv}
Any equivalence can be given the structure of an invertible map.\index{equivalence!invertibility of}
\end{defn}

\begin{constr}
First we construct for any equivalence $f$ with right inverse $g$ and left inverse $h$ a homotopy $K:g\htpy h$. For any $y:B$, we have 
\begin{equation*}
\begin{tikzcd}[column sep=huge]
g(y) \arrow[r,equals,"H(g(y))^{-1}"] & hfg(y) \arrow[r,equals,"\ap{h}{G(y)}"] & h(y).
\end{tikzcd}
\end{equation*} 
Therefore we define a homotopy $K:g\htpy h$ by $K\defeq \ct{(H\cdot g)^{-1}}{h\cdot G}$.
Using the homotopy $K$ we are able to show that $g$ is also a left inverse of $f$. For $x:A$ we have the identification
\begin{equation*}
\begin{tikzcd}[column sep=large]
gf(x) \arrow[r,equals,"K(f(x))"] & hf(x) \arrow[r,equals,"H(x)"] & x.
\end{tikzcd}\qedhere
\end{equation*}
\end{constr}

\begin{cor}
The inverse of an equivalence is again an equivalence.
\end{cor}

\begin{proof}
Let $f:A\to B$ be an equivalence. By \cref{defn:inv_equiv} it follows that the section of $f$ is also a retraction. Therefore it follows that the section is itself an invertible map, with inverse $f$. Hence it is an equivalence.
\end{proof}

\begin{thm}\label{thm:id_equiv}
For any type $A$, the identity function $\idfunc[A]$ is an equivalence.\index{identity function!is an equivalence|textit}
\end{thm}

\begin{proof}
The identity function is trivially its own section and its own retraction.
\end{proof}

\begin{eg}
  For any type $C(x,y)$ indexed by $x:A$ and $y:B$, the function
\begin{equation*}
\sigma:\Big(\prd{x:A}{y:B}C(x,y)\Big)\to\Big(\prd{y:B}{x:A}C(x,y)\Big)
\end{equation*}
that swaps the order of the arguments $x$ and $y$ is an equivalence by \cref{ex:swap}.\index{swap function!is an equivalence|textit}
\end{eg}

\subsection{The identity type of a \texorpdfstring{$\Sigma$-}{dependent pair }type}

In this section we characterize the identity type of a $\Sigma$-type as a $\Sigma$-type of identity types. In this course we will be characterizing the identity types of many types, so we will follow the general outline of how such a characterization goes:
\begin{enumerate}
\item First we define a binary relation $R:A\to A\to \UU$ on the type $A$ that we are interested in. This binary relation is intended to be equivalent to its identity type.
\item Then we will show that this binary relation is reflexive, by constructing a term of type
  \begin{equation*}
    \prd{x:A}R(x,x)
  \end{equation*}
\item Using the reflexivity we will show that there is a canonical map
  \begin{equation*}
    (x=y)\to R(x,y)
  \end{equation*}
  for every $x,y:A$. This map is just constructed by path induction, using the reflexivity of $R$.
\item Finally, it has to be shown that the map
  \begin{equation*}
    (x=y)\to R(x,y)
  \end{equation*}
  is an equivalence for each $x,y:A$. 
\end{enumerate}
The last step is usually the most difficult, and we will refine our methods for this step in \cref{chap:fundamental}, where we establish the Fundamental Theorem of Identity Types.

In this section we consider a type family $B$ over $A$. Given two pairs
\begin{equation*}
  (x,y),(x',y'):\sm{x:A}B(x),
\end{equation*}
if we have a path $\alpha:x=x'$ then we can compare $y:B(x)$ to $y':B(x')$ by first transporting $y$ along $\alpha$, i.e., we consider the identity type
\begin{equation*}
  \mathsf{tr}_B(\alpha,y)=y'.
\end{equation*}
Thus it makes sense to think of $(x,y)$ to be identical to $(x',y')$ if there is an identification $\alpha:x=x'$ and an identification $\beta:\mathsf{tr}_B(\alpha,y)=y'$. In the following definition we turn this idea into a binary relation on the $\Sigma$-type.

\begin{defn}
  We will define a relation
  \begin{equation*}
    \mathsf{Eq}_{\Sigma} : \Big(\sm{x:A}B(x)\Big)\to\Big(\sm{x:A}B(x)\Big)\to\UU
  \end{equation*}
  by defining
  \begin{equation*}
    \mathsf{Eq}_{\Sigma}(s,t)\defeq\sm{\alpha:\proj 1(s)=\proj 1(t)}\mathsf{tr}_B(\alpha,\proj 2(s))=\proj 2 (t).
  \end{equation*}
\end{defn}

\begin{lem}
  The relation $\mathsf{Eq}_{\Sigma}$ is reflexive, i.e., there is a term
  \begin{equation*}
    \mathsf{reflexive\usc{}Eq}_{\Sigma}:\prd{s:\sm{x:A}B(x)}\mathsf{Eq}_{\Sigma}(s,s).
  \end{equation*}
\end{lem}

\begin{constr}
  This term is constructed by $\Sigma$-induction on $s:\sm{x:A}B(x)$. Thus, it suffices to construct a term of type
  \begin{equation*}
    \prd{x:A}{y:B(x)}\sm{\alpha:x=x}\mathsf{tr}_B(\alpha,y)=y.
  \end{equation*}
  Here we take $\lam{x}{y}(\refl{x},\refl{y})$.
\end{constr}

\begin{defn}
  Consider a type family $B$ over $A$. Then for any $s,t:\sm{x:A}B(x)$ we define a map\index{pair_eq@{$\mathsf{pair\usc{}eq}$}|textbf}
  \begin{equation*}
    \mathsf{pair\usc{}eq}: (s=t)\to \mathsf{Eq}_\Sigma(s,t)
  \end{equation*}
  by path induction, taking $\mathsf{pair\usc{}eq}(\refl{s})\defeq\mathsf{reflexive\usc{}Eq}_\Sigma(s)$.
\end{defn}

\begin{thm}\label{thm:eq_sigma}
  Let $B$ be a type family over $A$. Then the map
  \begin{equation*}
    \mathsf{pair\usc{}eq}: (s=t)\to \mathsf{Eq}_\Sigma(s,t)
  \end{equation*}
  is an equivalence for every $s,t:\sm{x:A}B(x)$.\index{Sigma type@{$\Sigma$-type}!identity types of|textit}\index{identity type!of a Sigma-type@{of a $\Sigma$-type}|textit}
\end{thm}

\begin{proof}
The maps in the converse direction\index{eq_pair@{$\mathsf{eq\usc{}pair}$}}
\begin{equation*}
\mathsf{eq\usc{}pair} : \mathsf{Eq}_\Sigma(s,t)\to(\id{s}{t})
\end{equation*}
are defined by repeated $\Sigma$-induction. By $\Sigma$-induction on $s$ and $t$  we see that it suffices to define a map
\begin{equation*}
\mathsf{eq\usc{}pair} : \Big(\sm{p:x=x'}\id{\mathsf{tr}_B(p,y)}{y'}\Big)\to(\id{(x,y)}{(x',y')}).
\end{equation*}
A map of this type is again defined by $\Sigma$-induction. Thus it suffices to define a dependent function of type
\begin{equation*}
\prd{p:x=x'} (\id{\mathsf{tr}_B(p,y)}{y'}) \to (\id{(x,y)}{(x',y')}).
\end{equation*}
Such a dependent function is defined by double path induction by sending $\pairr{\refl{x},\refl{y}}$ to $\refl{\pairr{x,y}}$. This completes the definition of the function $\mathsf{eq\usc{}pair}$.

Next, we must show that $\mathsf{eq\usc{}pair}$ is a section of $\mathsf{pair\usc{}eq}$. In other words, we must construct an identification
\begin{equation*}
\mathsf{pair\usc{}eq}(\mathsf{eq\usc{}pair}(\alpha,\beta))=\pairr{\alpha,\beta}
\end{equation*}
for each $\pairr{\alpha,\beta}:\sm{\alpha:x=x'}\id{\mathsf{tr}_B(\alpha,y)}{y'}$. We proceed by path induction on $\alpha$, followed by path induction on $\beta$. Then our goal becomes to construct a term of type
\begin{equation*}
\mathsf{pair\usc{}eq}(\mathsf{eq\usc{}pair}\pairr{\refl{x},\refl{y}})=\pairr{\refl{x},\refl{y}}
\end{equation*}
By the definition of $\mathsf{eq\usc{}pair}$ we have $\mathsf{eq\usc{}pair}\pairr{\refl{x},\refl{y}}\jdeq \refl{\pairr{x,y}}$, and by the definition of $\mathsf{pair\usc{}eq}$ we have $\mathsf{pair\usc{}eq}(\refl{\pairr{x,y}})\jdeq\pairr{\refl{x},\refl{y}}$. Thus we may take $\refl{\pairr{\refl{x},\refl{y}}}$ to complete the construction of the homotopy $\mathsf{pair\usc{}eq}\circ\mathsf{eq\usc{}pair}\htpy\idfunc$.

To complete the proof, we must show that $\mathsf{eq\usc{}pair}$ is a retraction of $\mathsf{pair\usc{}eq}$. In other words, we must construct an identification
\begin{equation*}
\mathsf{eq\usc{}pair}(\mathsf{pair\usc{}eq}(p))=p
\end{equation*}
for each $p:s=t$. We proceed by path induction on $p:s=t$, so it suffices to construct an identification 
\begin{equation*}
\mathsf{eq\usc{}pair}\pairr{\refl{\proj 1(s)},\refl{\proj 2(s)}}=\refl{s}.
\end{equation*}
Now we proceed by $\Sigma$-induction on $s:\sm{x:A}B(x)$, so it suffices to construct an identification
\begin{equation*}
\mathsf{eq\usc{}pair}\pairr{\refl{x},\refl{y}}=\refl{(x,y)}.
\end{equation*}
Since $\mathsf{eq\usc{}pair}\pairr{\refl{x},\refl{y}}$ computes to $\refl{(x,y)}$, we may simply take $\refl{\refl{(x,y)}}$.
\end{proof}

\begin{exercises}
%  \item Show that for any term $a:A$ the functions
%    \begin{align*}
%      \ind{\unit}(a) & : \unit \to A \\
%      \mathsf{const}_a & : \unit \to A
%    \end{align*}
%    are homotopic.
%  \item Let $A$ and $B$ be types, and consider the constant map $\mathsf{const}_b:A\to B$ for some $b:B$. Construct a homotopy
%    \begin{equation*}
%      \mathsf{ap}_{\mathsf{const}_b}(x,y)\htpy \mathsf{const}_{\refl{b}}
%    \end{equation*}
%    for any $x,y:A$.
\item \label{ex:equiv_grpd_ops}Show that the functions
  \begin{align*}
    \mathsf{inv} & :(\id{x}{y})\to(\id{y}{x}) \\
    \mathsf{concat}(p) & : (\id{y}{z})\to(\id{x}{z}) \\
    \mathsf{concat'}(q) & : (\id{x}{y}) \to (\id{x}{z}) \\
    \mathsf{tr}_B(p) & :B(x)\to B(y)
  \end{align*}
  are equivalences, where $\mathsf{concat'}(q,p)\defeq \ct{p}{q}$\index{concat'@{$\mathsf{concat'}$}}. Give their inverses explicitly.
\item Show that the maps
  \begin{align*}
    \inl & : X \to X+\emptyt &     \proj 1 & : \emptyt \times X \to \emptyt \\
    \inr & : X \to \emptyt+X &    \proj 2 & : X \times \emptyt \to \emptyt
  \end{align*}
  are equivalences.
\item
  \begin{subexenum}
  \item \label{ex:htpy_equiv}\index{equivalence!homotopic maps} Consider two functions $f,g:A\to B$ and a homotopy $H:f\htpy g$. Then
    \begin{equation*}
      \isequiv(f)\leftrightarrow\isequiv(g).
    \end{equation*}
  \item Show that for any two homotopic equivalences $e,e':\eqv{A}{B}$, their inverses are also homotopic.
  \end{subexenum}
\item \label{ex:3_for_2}\index{equivalence!three@{3-for-2 property}}\index{3-for-2 property!of equivalences}
  Consider a commuting triangle
  \begin{equation*}
    \begin{tikzcd}[column sep=tiny]
      A \arrow[rr,"h"] \arrow[dr,swap,"f"] & & B \arrow[dl,"g"] \\
      & X.
    \end{tikzcd}
  \end{equation*}
  with $H:f\htpy g\circ h$.
  \begin{subexenum}
  \item Suppose that the map $h$ has a section $s:B \to A$. Show that the triangle
    \begin{equation*}
      \begin{tikzcd}[column sep=tiny]
        B \arrow[rr,"s"] \arrow[dr,swap,"g"] & & A \arrow[dl,"f"] \\
        & X.
      \end{tikzcd}
    \end{equation*}
    commutes, and that $f$ has a section if and only if $g$ has a section.
  \item Suppose that the map $g$ has a retraction $r:X\to B$. Show that the triangle
    \begin{equation*}
      \begin{tikzcd}[column sep=tiny]
        A \arrow[rr,"f"] \arrow[dr,swap,"h"] & & X \arrow[dl,"r"] \\
        & B.
      \end{tikzcd}
    \end{equation*}
    commutes, and that $f$ has a retraction if and only if $h$ has a retraction.
  \item (The \define{3-for-2 property} for equivalences.) Show that if any two of the functions
    \begin{equation*}
      f,\qquad g,\qquad h
    \end{equation*}
    are equivalences, then so is the third.
  \end{subexenum}
\item \label{ex:neg_equiv} 
  \begin{subexenum}
  \item Define the negation function on the booleans, and show that it is an equivalence.\index{negation function!is an equivalence}
  \item Use the observational equality on the booleans, defined in \cref{ex:obs_bool}, to show that $\btrue\neq\bfalse$.
  \item Show that for any $b:\bool$, the constant function $\mathsf{const}_b$ is not an equivalence.
  \end{subexenum}
\item \label{ex:succ_equiv} Show that the successor function on the integers is an equivalence.\index{successor function!on Z@{on $\Z$}!is an equivalence}
\item \label{ex:comm_prod}Construct a equivalences $\eqv{A+B}{B+A}$ and $\eqv{A\times B}{B\times A}$.\index{coproduct!is symmetric}
\item \label{ex:retr_id} Consider a section-retraction pair
  \begin{equation*}
    \begin{tikzcd}
      A \arrow[r,"i"] & B \arrow[r,"r"] & A,
    \end{tikzcd}
  \end{equation*}
  with $H:r\circ i\htpy \idfunc$. Show that $\id{x}{y}$ is a retract of $\id{i(x)}{i(y)}$.\index{retract!identity types of}
\item \label{ex:sigma_assoc}\index{Sigma type@{$\Sigma$-type}!associativity of}Let $B$ be a family of types over $A$, and let $C$ be a family of types indexed by $x:A,y:B(x)$. Construct an equivalence
  \begin{equation*}
    \Sigma\mathsf{\usc{}assoc}:\eqv{\Big(\sm{p:\sm{x:A}B(x)}C(\proj 1(p),\proj 2(p))\Big)}{\Big(\sm{x:A}\sm{y:B(x)}C(x,y)\Big)}.
  \end{equation*}
\item \label{ex:sigma_swap}Let $A$ and $B$ be types, and let $C$ be a family over $x:A,y:B$. Construct an equivalence
  \begin{equation*}
    \Sigma\mathsf{\usc{}swap}:\eqv{\Big(\sm{x:A}{y:B}C(x,y)\Big)}{\Big(\sm{y:B}{x:A}C(x,y)\Big)}.
  \end{equation*}
  %\item \label{ex:sigma_base_equiv}Consider an equivalence $e:A'\eqv A$ and a type family $B$ over $A$. Show that the map
  %\begin{equation*}
  %\lam{(x',y)}(e(x'),y) : \Big(\sm{x':A'}B(e(x'))\Big)\to\Big(\sm{x:A}B(x)\Big)
  %\end{equation*}
  %is an equivalence.
\item \label{ex:int_group_laws}\index{Z@{$\Z$}!group laws} In this exercise we will show that the laws for abelian groups hold for addition on the integers. Note: these are obvious facts, but the proof terms that show \emph{how} the group laws hold are nevertheless fairly involved. This exercise is perfect for a formalization project. 
  \begin{subexenum}
  \item Show that addition satisfies the left and right unit laws, i.e., construct terms
    \begin{align*}
      \mathsf{left\usc{}unit\usc{}law\usc{}add\usc{}}\Z  & : \prd{x:\Z} 0 + x = x \\
      \mathsf{right\usc{}unit\usc{}law\usc{}add\usc{}}\Z  & : \prd{x : \Z} x + 0 = x.
    \end{align*}
  \item Show that addition respects predecessors and successor on both sides, i.e., construct terms
    \begin{align*}
      \mathsf{left\usc{}predecessor\usc{}law\usc{}add\usc{}}\Z & : \prd{x,y:\Z} \mathsf{pred}_{\mathbb{Z}}(x)+y = \mathsf{pred}_{\mathbb{Z}}(x+y) \\
      \mathsf{right\usc{}predecessor\usc{}law\usc{}add\usc{}}\Z & : \prd{x,y:\Z} x+\mathsf{pred}_{\mathbb{Z}}(y) = \mathsf{pred}_{\mathbb{Z}}(x+y) \\
      \mathsf{left\usc{}successor\usc{}law\usc{}add\usc{}}\Z & : \prd{x,y:\Z} \mathsf{succ}_{\mathbb{Z}}(x)+y = \mathsf{succ}_{\mathbb{Z}}(x+y) \\
      \mathsf{right\usc{}successor\usc{}law\usc{}add\usc{}}\Z & : \prd{x,y:\Z} x+\mathsf{succ}_{\mathbb{Z}}(y) = \mathsf{succ}_{\mathbb{Z}}(x+y).
    \end{align*}
    Hint: to avoid an excessive number of cases, use induction on $x$ but not on $y$. You may need to use the homotopies $\mathsf{succ}_{\mathbb{Z}}\circ \mathsf{pred}_{\mathbb{Z}}\htpy \idfunc$ and $\mathsf{pred}_{\mathbb{Z}}\circ\mathsf{succ}_{\mathbb{Z}}$ constructed in exercise \cref{ex:succ_equiv}.
  \item Use part (b) to show that addition on the integers is associative and commutative, i.e., construct terms
    \begin{align*}
      \mathsf{assoc\usc{}add\usc{}}\Z & : \prd{x,y,z:\Z} (x+y)+z = x + (y+z) \\
      \mathsf{comm\usc{}add\usc{}}\Z & : \prd{x,y:\Z} x+y = y+x.
    \end{align*}
    Hint: Especially in the construction of the associator there is a risk of running into an unwieldy amount of cases if you use $\Z$-induction on all arguments. Avoid induction on $y$ and $z$.
  \item Show that addition satisfies the left and right inverse laws:
    \begin{align*}
      \mathsf{left\usc{}inverse\usc{}law\usc{}add\usc{}}\Z & : \prd{x : \Z} (-x)+x=0 \\
      \mathsf{right\usc{}inverse\usc{}law\usc{}add\usc{}}\Z & : \prd{x : \Z} x + (-x)=0.
    \end{align*}
    Conclude that the functions $y \mapsto x + y$ and $x\mapsto x + y$ are equivalences for any $x:\Z$ and $y:\Z$, respectively.
  \end{subexenum}
\item \label{ex:coproduct_functor}In this exercise we will construct the \define{functorial action} of coproducts.
  \begin{subexenum}
  \item Construct for any two maps $f:A \to A'$ and $g:B \to B'$, a map
    \begin{equation*}
      f+g:A+B \to A'+B'.
    \end{equation*}
  \item Show that if $H:f\htpy f'$ and $K:g\htpy g'$, then there is a homotopy
    \begin{equation*}
      H+K:(f+g)\htpy (f'+g').
    \end{equation*}
  \item Show that $\idfunc[A]+\idfunc[B]\htpy \idfunc[A+B]$.
  \item Show that for any $f:A\to A'$, $f':A'\to A''$, $g:B\to B'$, and $g':B'\to B''$ there is a homotopy
    \begin{equation*}
      (f'\circ f)+(g'\circ g) \htpy (f'+g')\circ (f\circ g).
    \end{equation*}
  \item \label{ex:coproduct_functor_equivalence}Show that if $f$ and $g$ are equivalences, then so is $f+g$. (The converse of this statement also holds, see \cref{ex:is-equiv-is-equiv-functor-coprod}.)
  \end{subexenum}
\item Construct equivalences
  \begin{align*}
    \mathsf{Fin}(m+n) & \simeq \mathsf{Fin}(m)+\mathsf{Fin}(n) \\
    \mathsf{Fin}(mn) & \simeq \mathsf{Fin}(m)\times\mathsf{Fin}(n).
  \end{align*}
\end{exercises}
