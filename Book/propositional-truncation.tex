\section{Propositional truncations}\label{sec:propositional-truncation}

The propositional truncation operation is a universal way of turning type a type $A$ into a proposition $\brck{A}$. Informally, the proposition $\brck{A}$ is the proposition that $A$ is inhabited. More precisely, the propositional truncation of $A$ comes equipped with a map $A\to\brck{A}$ and it is characterized by its universal property, which asserts that any map $A\to P$ into a proposition $P$ extends uniquely to a map $\brck{A}\to P$, as indicated in the diagram
\begin{equation*}
  \begin{tikzcd}
    A \arrow[dr] \arrow[d] \\
    \brck{A} \arrow[r,densely dotted] & P.
  \end{tikzcd}
\end{equation*}
Using the propositional truncation operation we can define many important mathematical concepts, including the image of a map, surjectivity, and connected components. We will discuss those topics in \cref{chap:image}.

\subsection{The universal property of propositional truncations}\label{sec:propositional-truncation-up}

\begin{defn}
Let $A$ be a type, and let $f:A\to P$ be a map into a proposition $P$. We say that $f$ is a \define{propositional truncation of $A$} if for every proposition $Q$, the precomposition map
\begin{equation*}
\blank\circ f:(P\to Q)\to (A\to Q)
\end{equation*}
is an equivalence. This property of $f$ is also called the \define{universal property of propositional truncation of $A$}\index{universal property!of propositional truncation}
\end{defn}

In other words, a map $f:A\to P$ into a proposition $P$ is a propositional truncation of $A$ if every map $g:A\to Q$ into a proposition extends uniquely along $f$, as indicated in the diagram
\begin{equation*}
  \begin{tikzcd}
    A \arrow[d,swap,"f"] \arrow[dr,"g"] \\
    P \arrow[r,densely dotted] & Q.
  \end{tikzcd}
\end{equation*}
Indeed, this unique extension property asserts that the type
\begin{equation*}
  \sm{h:P\to Q}h\circ f=g
\end{equation*}
is contractible for every $g:A\to Q$. In other words, the unique extension property asserts that the precomposition function $\blank\circ f:(P\to Q)\to (A\to Q)$ is a contractible map, which is the case if and only if it is an equivalence.

\begin{rmk}
  Note that if $Q$ is a proposition, then the type $X\to Q$ is a proposition for any type $X$. Furthermore, recall from \cref{ex:equiv-bi-implication} that the map $(P\to Q)\to (A\to Q)$ is an equivalence as soon as there is a map in the converse direction. Therefore, in order to prove the universal property of the propositional truncation it suffices to show that
  \begin{equation*}
    (A\to Q)\to (P\to Q).
  \end{equation*}
  We also note that the universal property of the propositional truncation of a type is formulated with respect to all propositions, regardless of the universe they live in. 
\end{rmk}

\begin{eg}
  Suppose $A$ is a type that comes equipped with a point $a:A$, such as the booleans, the type of natural numbers, or the loop space $\loopspace{A}$ of a pointed type. Then the constant map
  \begin{equation*}
    \const_\ttt: A\to\unit
  \end{equation*}
  is a propositional truncation of $A$. To see this, let $Q$ be an arbitrary proposition. It suffices to show that
  \begin{equation*}
    (A\to Q)\to (\unit\to Q).
  \end{equation*}
  To see this, let $f:A\to Q$. Then we have $f(a):Q$, so we define $\const_{f(a)}:\unit\to Q$. Thus we see that we have
  \begin{equation*}
    \lam{f}\const_{f(a)}:(A\to Q)\to (\unit\to Q).
  \end{equation*}
  This proves that $\const_\ttt:A\to \unit$ satisfies the universal property of the propositional truncation of $A$. 
\end{eg}

\begin{eg}
  If the type $A$ is already a proposition, then the identity function
  \begin{equation*}
    \idfunc:A\to A
  \end{equation*}
  is a propositional truncation of $A$. To see this, simply note that the precomposittion function with the identity function
  \begin{equation*}
    \blank\circ\idfunc : (A\to Q)\to (A\to Q)
  \end{equation*}
  is itself just the identity function. In particular, it is an equivalence.

  Similarly, any equivalence $e:P\simeq P'$ between propositions satisfies the universal property of the propositional truncation of $P$, since precomposing by an equivalence is an equivalence by \cref{ex:equiv_precomp}.
\end{eg}

The universal property of the propositional truncation determines the propositional truncation up to equivalence. Such proofs of uniqueness from a universal property always follow the same pattern.

\begin{prp}\label{prp:propositional-truncation-3-for-2}
  Let $A$ be a type, and consider a commuting triangle
  \begin{equation*}
    \begin{tikzcd}[column sep=tiny]
      \phantom{P'} & A \arrow[dl,swap,"f"] \arrow[dr,"{f'}"] \\
      P \arrow[rr,swap,"h"] & & P'
    \end{tikzcd}
  \end{equation*}
  where $P$ and $P'$ are propositions. If any two of the following three assertions hold, so does the third:
  \begin{enumerate}
  \item The map $f$ satisfies the universal property of the propositional truncation of $A$.
  \item The map $f'$ satisfies the universal propertyof the propositional truncation of $A$.
  \item The map $h$ is an equivalence.
  \end{enumerate}
\end{prp}

\begin{proof}
  Note that the map $h:P\to P'$ is an equivalence if and only if for every proposition $Q$, the precomposition map
  \begin{equation*}
    \blank\circ h:(P'\to Q)\to (P\to Q)
  \end{equation*}
  is an equivalence. Thus, the claim follows by observing that for every proposition $Q$ we have the triangle
  \begin{equation*}
    \begin{tikzcd}[column sep=-1em]
      (P'\to Q) \arrow[rr,"\blank\circ h"] \arrow[dr,swap,"\blank\circ {f'}"] & & (P\to Q) \arrow[dl,"\blank\circ f"] \\
      & (A\to Q). & \phantom{(P'\to Q)}
    \end{tikzcd}
  \end{equation*}
\end{proof}

\begin{cor}\label{cor:uniquely-unique-brck}
  Consider two maps $f:A\to P$ and $f':A\to P'$ into propositions $P$ and $P'$, both satisfying the universal property of the propositional truncation of $A$. Then the type of equivalences $e:P \simeq P'$ for which the triangle
  \begin{equation*}
    \begin{tikzcd}[column sep=tiny]
      \phantom{P'} & A \arrow[dl,swap,"f"] \arrow[dr,"{f'}"] \\
      P \arrow[rr,swap,"e"] & & P'
    \end{tikzcd}
  \end{equation*}
  commutes, is contractible.
\end{cor}

\begin{rmk}
  Note that the triangles in \cref{prp:propositional-truncation-3-for-2,cor:uniquely-unique-brck} always commutes, since $P$ and $P'$ are assumed to be propositions.
\end{rmk}

Now that we have shown that propositional truncations are determined uniquely, we will assume that any universe is closed under propositional truncations.

\begin{axiom}\label{axiom:propositional-truncations}
  Any universe $\UU$ is closed under propositional truncations: for any type $A:\UU$ there is a proposition $\brck{A}:\UU$ equipped with a map $\eta:A\to\brck{A}$ that satisfies the universal property of the propositional truncation.
\end{axiom}

The propositional truncation is therefore an operation
\begin{equation*}
  \brck{\blank}:\UU\to\UU
\end{equation*}
on the universe. One simple application of the universal property of the propositional truncation is that $\brck{\blank}$ also acts on functions in a functorial way.

\begin{prp}
  There is a map
  \begin{equation*}
    \brck{\blank}:(A\to B)\to (\brck{A}\to\brck{B})
  \end{equation*}
  for any two types $A$ and $B$, such that
  \begin{align*}
    \brck{\idfunc} & \htpy \idfunc \\
    \brck{g\circ f} & \htpy \brck{g}\circ\brck{f}.
  \end{align*}
\end{prp}

\begin{proof}
  For any $f:A\to B$, the map $\brck{f}:\brck{A}\to\brck{B}$ is defined to be the unique extension
  \begin{equation*}
    \begin{tikzcd}
      A \arrow[d,swap,"\eta"] \arrow[r,"f"] & B \arrow[d,"\eta"] \\
      \brck{A} \arrow[r,densely dotted,swap,"\brck{f}"] & \brck{B}.
    \end{tikzcd}
  \end{equation*}
  To see that $\brck{\blank}$ preserves identity maps and compositions, simply note that $\idfunc[\brck{A}]$ is an extension of $\idfunc[A]$, and that $\brck{g}\circ\brck{f}$ is an extension of $g\circ f$. Hence the homotopies are obtained by uniqueness.
\end{proof}

\subsection{Propositional truncations as higher inductive types}

The idea of higher inductive types is that types can be generated inductively not only by point constructors, such as $\zeroN:\N$ and $\succN:\N\to\N$, but also by path constructors. One of the first examples of a higher inductive type was the propositional truncation of a type $A$. This is a type $\brck{A}$ equipped with one point constructor
\begin{equation*}
  \eta : A \to \brck{A},
\end{equation*}
one path constructor
\begin{equation*}
  \alpha : \prd{x,y:\brck{A}}x=y.
\end{equation*}
Note that the path constructor $\alpha$ immediately proves that $\brck{A}$ is a proposition. Now we should formulate the induction principle for the propositional truncation.

Just as we did with the universal property, we will formulate the induction principle of the propositional truncation for an arbitrary map $f:A\to P$ into a proposition $P$. When the induction principle is formulated in this way, we will be able to show that $f$ satisfies the universal property if and only if it satisfes the induction principle.

Consider a map $f:A\to P$ into a type equipped with a family of paths
\begin{equation*}
  \alpha : \prd{p,q:P}p=q
\end{equation*}
witnessing that $P$ is a proposition, and consider a type family $B$ over $P$. The induction principle of the propositional truncation of $A$ tells us what we have to do in order to construct a dependent function $h:\prd{p:P}B(p)$.

In order to figure out what the induction principle has to be, we first note that if we start with a dependent function $h:\prd{p:P}B(p)$, then we also obtain the function $h\circ f : \prd{x:A}B(f(x))$. In other words, if we think of $f:A\to P$ as the point constructor of $P$, then the function $h\circ f$ describes the action of $h$ on the points of $P$. The first requirement in the induction principle is therefore that $B$ must come equipped with a dependent function
\begin{equation*}
  g:\prd{x:A}B(f(x)).
\end{equation*}
Furthermore, the function $h$ acts on the paths in $P$ via its dependent action on paths, which we constructed in \cref{defn:apd}. The paths in $P$ are generated by $\alpha$, so we obtain a function
\begin{equation*}
  \lam{p}{q}\apd{h}{\alpha(p,q)} : \prd{p,q:P} \tr_B(\alpha(p,q),h(p))=h(q).
\end{equation*}
The induction principle must ensure that any function $h$ defined via the induction principle, satisfies this law. Therefore, the second condition in the induction principle is that we must have a family of identifications
\begin{equation*}
  \prd{p,q:P}{y:B(p)}{z:B(q)}\tr_B(\alpha(p,q),y)= z.
\end{equation*}
We now formulate the induction principle for propositional truncation.

\begin{defn}
  Consider a map $f:A\to P$ into a type $P$ equipped with a family of paths
  \begin{equation*}
    \alpha : \prd{p,q:P}p=q,
  \end{equation*}
  witnessing that $P$ is a proposition. We say that $f$ satisfies the induction principle for the propositional truncation of $A$ if for any family $B$ over $P$ that comes equipped with
  \begin{align*}
    g & : \prd{x:A}B(f(x)) \\
    \beta & : \prd{p,q:P}{y:B(p)}{z:B(q)}\tr_B(\alpha(p,q),y)=z,
  \end{align*}
  there is a dependent function $h:\prd{p:P}B(p)$ equipped with a homotopy
  \begin{equation*}
    \prd{x:A}h(f(x))= g(x).
  \end{equation*}
\end{defn}

In the following lemma we show that if a family $B$ over $\brck{A}$ comes equipped with a family of paths
\begin{equation*}
  \prd{x,y:\brck{A}}{u:B(x)}{v:B(y)}\tr_B(\alpha(x,y),u)=v,
\end{equation*}
then $B$ must be a family of propositions.

\begin{lem}\label{lem:case-paths-induction-principle-propositional-truncation}
  Let $P$ be a type equipped with a family of paths
  \begin{equation*}
    \alpha : \prd{p,q:P}p=q,
  \end{equation*}
  showing that $P$ is a proposition, and consider a type family $B$ over $P$. The following are equivalent:
  \begin{enumerate}
  \item The family $B$ comes equipped with a family of identifications
    \begin{equation*}
      \beta:\prd{p,q:P}{x:B(p)}{y:B(q)}\tr_B(\alpha(p,q),x)=y,
    \end{equation*}
  \item The family $B$ is a family of propositions.
  \end{enumerate}
\end{lem}

\begin{proof}
  Assuming that (i) holds, we will show that each $B(p)$ is a proposition by showing that
  \begin{equation*}
    B(p)\to\iscontr(B(p)).
  \end{equation*}
  Let $x:B(p)$. We have to construct a center of contraction and a contraction. Our plan is to use $\beta$ to define the contraction, so it is natural to define the center as $\tr_B(\alpha(p,p),x)$. Now we take
  \begin{equation*}
    \beta(p,p,x):\prd{y:B(p)}\tr_B(\alpha(p,p),x)=y
  \end{equation*}
  as our contraction. This completes the proof that $B$ is a family of propositions.

  The converse is immediate: if $B$ is a family of propositions, then any two terms in any $B(q)$ can be identified.
\end{proof}

\begin{defn}
  Consider a map $f:A\to P$ into a proposition $P$. We say that $f$ satisfies the dependent universal property of the propositional truncation of $A$, if for any family $Q$ of propositions over $P$, the precomposition map
  \begin{equation*}
    \blank\circ f : \Big(\prd{p:P}Q(p)\Big)\to\Big(\prd{x:A}Q(f(x))\Big)
  \end{equation*}
  is an equivalence.
\end{defn}

\begin{thm}
  Consider a map $f:A\to P$ into a proposition $P$. Then the following are equivalent:
  \begin{enumerate}
  \item The map $f$ is a propositional truncation.
  \item The map $f$ satisfies the dependent universal property of the propositional truncation.
  \item The map $f$ satisfies the induction principle of the propositional truncation.
  \end{enumerate}
\end{thm}

\begin{proof}
  We will first show that (i) and (ii) are equivalent. Of course, the universal property is a special case of the dependent universal property, so the fact that (ii) implies (i) is immediate. We now show that (i) implies (ii). Let $Q$ be a family of propositions over $P$, and consider the following commuting diagram:
  \begin{equation*}
    \begin{tikzcd}[column sep=8em]
      \Big(\sm{h:P\to P}\prd{p:P}Q(h(p))\Big) \arrow[r,"{\tot[\blank\circ f]{\blank\circ f}}"] \arrow[d,swap,"\choice^{-1}"] & \Big(\sm{g:A\to P}\prd{x:A}Q(g(x))\Big) \arrow[d,"\choice^{-1}"] \\
      \Big(P\to \sm{p:P}Q(p)\Big) \arrow[r,"\blank\circ f"] \arrow[d,swap,"\proj 1\circ\blank"]  & \Big(A \to \sm{p:P}Q(p)\Big) \arrow[d,"\proj 1\circ\blank"] \\
      \Big(P\to P\Big) \arrow[r,swap,"\blank\circ f"] & \Big(A\to P\Big)
    \end{tikzcd}
  \end{equation*}
  In this diagram the bottom map is an equivalence by the universal property of the propositional truncation of $A$. Note also that the type $\sm{p:P}Q(p)$ is a proposition by \cref{ex:istrunc_sigma}, so it follows that also the middle map is an equivalence. Furthermore, the type theoretic choice maps are equivalences by \cref{thm:choice}, so it also follows that the top map is an equivalence. Now we use \cref{thm:equiv-toto} to conclude that the family of maps
  \begin{equation*}
    \blank\circ f: \Big(\prd{p:P}Q(h(p))\Big)\to\Big(\prd{x:A}Q(h(f(x)))\Big)
  \end{equation*}
  indexed by $h:P\to P$ is a family of equivalences. The dependent universal property is now just a special case: take $h\jdeq\idfunc$. This completes the proof that (i) is equivalent to (ii).

  It remains to show that (ii) is equivalent to (iii). By \cref{lem:case-paths-induction-principle-propositional-truncation} it follows that the induction principle is equivalent to the property that for each family $Q$ of propositions over $P$, the precomposition map
  \begin{equation*}
    \blank\circ f : \Big(\prd{p:P}Q(p)\Big)\to\Big(\prd{x:A}Q(f(x))\Big)
  \end{equation*}
  has a section. Since the domain and codomain of this map are propositions by \cref{thm:trunc_pi}, we see that this precomposition map has a section if and only if it is an equivalence.
\end{proof}

\begin{exercises}
  \exercise Let $A$ be a type and let $P$ be a proposition, and suppose that $P$ is a retract of $A$. Show that the retraction $A\to P$ is a propositional truncation.
%  \exercise Show that the relation $x,y\mapsto\brck{x=y}$ is an equivalence relation, on any type.
  \exercise Consider two maps $f:A\to P$ and $g:B\to Q$ into propositions $P$ and $Q$. Recall from \cref{ex:istrunc_sigma} that the type $P\times Q$ is also a proposition. Show that if both $f$ and $g$ are propositional truncations then the map $f\times g : A\times B\to P\times Q$ is also a propositional truncation. Conclude that
  \begin{equation*}
    \brck{A\times B}\simeq \brck{A}\times\brck{B}. 
  \end{equation*}
  \exercise Consider two propositions $P$ and $Q$, and define
  \begin{align*}
    P\land Q & \defeq P\times Q\\
    P\vee Q & \defeq \brck{P+Q}.
  \end{align*}
  \begin{subexenum}
  \item Construct maps $i:P\to P\vee Q$ and $j:Q\to P\vee Q$.
  \item Prove the universal property of disjunction, i.e., show that for any proposition $R$, the map
    \begin{equation*}
      (P\vee Q\to R) \to (P\to R)\land (Q\to R) 
    \end{equation*}
    given by $h\mapsto (h\circ i,h\circ j)$ is an equivalence.
  \end{subexenum}
  \exercise Consider a family $P$ of propositions over a type $A$, and define
  \begin{align*}
    \forall_{(x:A)}P(x) & \defeq \prd{x:A}P(x) \\
    \exists_{(x:A)}P(x) & \defeq \Brck{\sm{x:A}P(x)}
  \end{align*}
  \begin{subexenum}
  \item Construct a map $i_a : P(a)\to \exists_{(x:A)}P(x)$ for each $a:A$.
  \item Prove the universal property of the existential quantification, i.e. show that for any proposition $Q$, the map
    \begin{equation*}
      \Big(\Big(\exists_{(x:A)}P(x)\Big)\to Q\Big)\to \Big(\forall_{(x:A)}(P(x)\to Q)\Big)
    \end{equation*}
    given by $h\mapsto \lam{x}h\circ i_x$, is an equivalence.
  \end{subexenum}
  \exercise Show that
  \begin{equation*}
    \eqv{\brck{A}}{\prd{P:\prop}(A\to P)\to P}
  \end{equation*}
  for any type $A:\UU$. This is called the \define{impredicative encoding} of the propositional truncation.
  % \exercise For any $B:A\to\UU$, construct an equivalence
  % \begin{equation*}
  %   \eqv{\Big(\exists_{(a:A)}\brck{B(a)}\Big)}{\brck{\sm{a:A}B(a)}}
  % \end{equation*}
\end{exercises}