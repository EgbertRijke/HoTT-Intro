\section{The universe and type-valued relations}

In this lecture we introduce type theoretic \emph{universes}. Universes are types that consist of types. In other words, a universe is a type $\UU$ that comes equipped with a type family $\mathsf{Ty}$ over $\UU$, and for any $X:\UU$ we think of $X$ as an \emph{encoding} of the type $\mathsf{Ty}(X)$. We call this type family the \emph{universal type family}.

There are several reasons to equip type theory with universes. One reason is that it enables us to define new type families over inductive types, using their induction principle. For example, since the universe is itself a type, we can use the induction principle of $\bool$ to obtain a map $P:\bool\to\UU$ from any two terms $X_0,X_1:\UU$. Then we obtain a type family over $\bool$ by substituting $P$ into the universal type family:
\begin{equation*}
  x:\bool\vdash \mathsf{Ty}(P(x))~\mathrm{type}
\end{equation*}
satisfying $\mathsf{Ty}(P(0_\bool))\jdeq \mathsf{Ty}(X_0)$ and $\mathsf{Ty}(P(1_\bool))\jdeq \mathsf{Ty}(X_1)$.

We use this way of defining type families to define many familiar relations over $\N$, such as $\leq$ and $<$. We also introduce a relation called \emph{observational equality} $\mathsf{Eq}_\N$ on $\N$, which we can think of as equality of $\N$. This relation is reflexive, symmetric, and transitive, and moreover it is the least reflexive relation. Furthermore, one of the most important aspects of observational equality $\mathsf{Eq}_\N$ on $\N$ is that $\mathsf{Eq}_\N(m,n)$ is a type for every $m,n:\N$, unlike judgmental equality. Therefore we can use type theory to reason about observational equality on $\N$. Indeed, in the exercises we show that some very elementary mathematics can already be done at this early stage in our development of type theory.

A second reason to introduce universes is that it allows us to define many types of types equipped with structure. One of the most important examples is the type of groups, which is the type of types equipped with the group operations satisfying the group laws, and for which the underlying type is a set. We won't discuss the condition for a type to be a set until \cref{chap:hierarchy}, so the definition of groups in type theory will be given much later. Therefore we illustrate this use of the universe by giving simpler examples: pointed types, graphs, and reflexive graphs.

One of the aspects that make universes useful is that they are postulated to be closed under all the type constructors. For example, if we are given $X:\UU$ and $P:\mathsf{Ty}(X)\to \UU$, then the universe is equipped with a term
\begin{equation*}
  \check{\Sigma}(X,P):\UU
\end{equation*}
satisfying the judgmental equality $\mathsf{Ty}(\check{\Sigma}(X,P)\jdeq\sm{x:\mathsf{Ty}(X)}\mathsf{Ty}(P(x))$. We will similarly assume that any universe is closed under $\Pi$-types and the other ways of forming types. However, there is an important restriction: it would be inconsistent to assume that the universe is contained in itself. One way of thinking about this is that universes are types of \emph{small} types, and it cannot be the case that the universe is small with respect to itself. We address this problem by assuming that there are many universes: enough universes so that any type family can be obtained by substituting into the universal type family of some universe.

\subsection{Specification of type theoretic universes}
\begin{defn}
  A \define{universe}\index{universe|textbf} in type theory is a closed type $\UU$\index{U@{$\UU$}|textbf} equipped with a type family $\mathsf{Ty}$\index{Ty@{$\mathsf{Ty}$}} over $\UU$ called the \define{universal family}\index{universal family}\index{family!universal family|textbf}, equipped with the following structure:
  \begin{enumerate}
  \item $\UU$ is closed under $\Pi$, in the sense that it comes equipped with a function
    \begin{equation*}
      \check{\Pi} :\prd{X:\UU}(\mathsf{Ty}(X)\to\UU)\to\UU
    \end{equation*}
    for which the judgmental equality
    \begin{equation*}
      \mathsf{Ty}\big(\check{\Pi}(X,P)\big)\jdeq \prd{x:\mathsf{Ty}(X)}\mathsf{Ty}(P(x)).
    \end{equation*}
    holds, for every $X:\UU$ and $P:\mathsf{Ty}(X)\to\UU$.
  \item $\UU$ is closed under $\Sigma$ in the sense that it comes equipped with a function
    \begin{equation*}
      \check{\Sigma} :\prd{X:\UU}(\mathsf{Ty}(X)\to\UU)\to\UU
    \end{equation*}
    for which the judgmental equality
    \begin{equation*}
      \mathsf{Ty}\big(\check{\Sigma}(X,P)\big) \jdeq \sm{x:\mathsf{Ty}(X)}\mathsf{Ty}(P(x))
    \end{equation*}
    holds, for every $X:\UU$ and $P:\mathsf{Ty}(X)\to\UU$.
  \item $\UU$ is closed under identity types, in the sense that it comes equipped with a function
    \begin{equation*}
      \check{\mathrm{I}} : \prd{X:\UU}\mathsf{Ty}(X)\to(\mathsf{Ty}(X)\to\UU)
    \end{equation*}
    for which the judgmental equality
    \begin{equation*}
      \mathsf{Ty}\big(\check{\mathrm{I}}(X,x,y)\big)\jdeq (\id{x}{y})
    \end{equation*}
    holds, for every $X:\UU$ and $x,y:\mathsf{Ty}(X)$.
  \item $\UU$ is closed under coproducts, in the sense that it comes equipped with a function
    \begin{equation*}
      \check{+}:\UU\to (\UU\to\UU)
    \end{equation*}
    that satisfies $\mathsf{Ty}\big(X\check{+}Y\big)\jdeq \mathsf{Ty}(X)+\mathsf{Ty}(Y)$.
  \item $\UU$ contains terms $\check{\emptyt},\check{\unit},\check{\N}:\UU$
    that satisfy the judgmental equalities
    \begin{align*}
      \mathsf{Ty}\big(\check{\emptyt}\big) & \jdeq \emptyt \\
      \mathsf{Ty}\big(\check{\unit}\big) & \jdeq \unit \\
      \mathsf{Ty}(\check{\N}) & \jdeq \N.
    \end{align*}
  \end{enumerate}
\end{defn}

\begin{rmk}
  Since ordinar function types are defined as a special case of dependent function types, we don't have to assume that universes are closed under ordinary function types. Similarly, it follows from the assumption that universes are closed under dependent pair types that universes are closed under cartesian product types.
\end{rmk}

\begin{rmk}
We will assume that for every finite list of types
\begin{align*}
  \Gamma_1 & \vdash A_1~\mathrm{type} \\
  & \vdots \\
  \Gamma_n & \vdash A_n~\mathrm{type},
\end{align*}
there is a universe $\UU$ that contains each $A_i$ in the sense that $\UU$ comes equipped with a term
\begin{align*}
  \Gamma_i\vdash \check{A}_i:\UU
\end{align*}
that for which the judgment
\begin{equation*}
  \Gamma_i\vdash \mathsf{Ty}\big(\check{A}_i\big)\jdeq A_i~\mathrm{type}
\end{equation*}
holds. By this assumption it is rarely necessary to work with more than one universe at the same time. Occasionally, we might need a universe $\UU$ and a second universe $\mathcal{V}$ that contains $\UU$ as well as all the types in $\UU$.
\end{rmk}

In the following definition we introduce some terminology regarding being in a universe. We think of a universe as a type of types that are \emph{small} with respect to that universe.

\begin{defn}
  Consider a universe $\UU$. We say that a type $A$ in context $\Gamma$ is \define{small}\index{small type|textbf} with respect to $\UU$ if it occurs in the universe, i.e., if it comes equipped with a term $\check{A}:\UU$ in context $\Gamma$, for which the judgment
  \begin{equation*}
    \Gamma\vdash\mathsf{Ty}\big(\check{A}\big)\jdeq A~\mathrm{type}
  \end{equation*}
  holds. If $A$ is small with respect to $\UU$, we usually write simply $A$ for $\check{A}$ and also $A$ for $\mathsf{Ty}(\check{A})$. In other words, by $A:\UU$ we mean that $A$ is a small type. 
\end{defn}

\begin{rmk}
  Another way of phrasing the condition that any universe $\UU$ is closed under $\Pi$-types, is that the dependent function type of a family of small types indexed by a small type is again small. Similarly, it follows for any universe that dependent pair types and identity types of small types are again small.

  By our assumption that any finite list of types is contained in \emph{some} universe, it follows that any type is small with respect to some universe.
\end{rmk}

\begin{defn}
  Consider a universe $\UU$, and let $B$ be a type family over an arbitrary type $A$. We say that $B$ is small with respect to $\UU$ if it comes equipped with a function
  \begin{equation*}
    \check{B}:A\to\UU
  \end{equation*}
  for which the judgmental equality $\mathsf{Ty}\big(\check{B}(x)\big)\jdeq B(x)$ holds, for every $x:A$.
\end{defn}

\begin{eg}
One important way to use a universe $\UU$ is to define types of \define{structured types}\index{structured types}. We give some examples:
\begin{enumerate}
\item The type of $\UU$-small \define{pointed types}\index{pointed types} is defined as
\begin{equation*}
\UU_\ast\defeq \sm{A:\UU}A,
\end{equation*}
\item The type of $\UU$-small \define{graphs}\index{graphs} is defined as the type
\begin{equation*}
\mathsf{Gph}_\UU \defeq \sm{A:\UU} A\to (A\to \UU),
\end{equation*}
\item The type of $\UU$-small \define{reflexive graphs}\index{reflexive graphs} is defined as the type
\begin{equation*}
\mathsf{rGph}_\UU \defeq \sm{A:\UU}{R:A\to (A\to \UU)}\prd{a:A}R(a,a).
\end{equation*}
\end{enumerate}
Once we have introduced the \emph{identity types} we will also be able to state the types of groups, rings, and many other structured types. However, when doing so one has to be cautious to make sure that the underlying type is in the level of sets, in the hierarchy of homotopical complexity of types.
\end{eg}

\subsection{Defining families and relations using a universe}
Another important way to use the universe is to \emph{define} new type families by induction. For example, we can define the finite types as family over the natural numbers.

\begin{defn}\label{defn:fin}
We define the type family $\mathsf{Fin}:\N\to\UU$ of finite types\index{Fin@{$\mathsf{Fin}$}|textbf}\index{finite types|textbf} by induction on $\N$\index{family!of finite types}, taking
\begin{align*}
\mathsf{Fin}(\zeroN) & \defeq \emptyt \\
\mathsf{Fin}(\succN(n)) & \defeq \mathsf{Fin}(n)+\unit
\end{align*}
\end{defn}

A second example of this kind is the notion of \emph{observational equality} on the natural numbers.

\begin{defn}\label{defn:obs_nat}
We define the \define{observational equality}\index{observational equality!on N@{on $\N$}} on $\N$ as binary relation $\mathsf{Eq}_\N:\N\to(\N\to\UU)$\index{Eq_N@{$\mathsf{Eq}_\N$}|textbf} satisfying
\begin{align*}
\mathsf{Eq}_\N(\zeroN,\zeroN) & \jdeq \unit & \mathsf{Eq}_\N(\succN(n),\zeroN) & \jdeq \emptyt \\
\mathsf{Eq}_\N(\zeroN,\succN(n)) & \jdeq \emptyt & \mathsf{Eq}_\N(\succN(n),\succN(m)) & \jdeq \mathsf{Eq}_\N(n,m).
\end{align*}
\end{defn}

\begin{constr}
We define $\mathsf{Eq}_\N$ by double induction on $\N$. By the first application of induction it suffices to provide
\begin{align*}
E_0 & : \N\to\UU \\
E_S & : \N\to (\N\to\UU)\to(\N\to\UU)
\end{align*}
We define $E_0$ by induction, taking $E_{00}\defeq \unit$ and $E_{0S}(n,X,m)\defeq \emptyt$. The resulting family $E_0$ satisfies
\begin{align*}
E_0(\zeroN) & \jdeq \unit \\
E_0(\succN(n)) & \jdeq \emptyt.
\end{align*} 
We define $E_S$ by induction, taking $E_{S0}\defeq \emptyt$ and $E_{S0}(n,X,m)\defeq X(m)$. The resulting family $E_S$ satisfies
\begin{align*}
E_S(n,X,\zeroN) & \jdeq \emptyt \\
E_S(n,X,\succN(m)) & \jdeq X(m) 
\end{align*}
Therefore we have by the computation rule for the first induction that the judgmental equality
\begin{align*}
\mathsf{Eq}_\N(\zeroN,m) & \jdeq E_0(m) \\
\mathsf{Eq}_\N(\succN(n),m) & \jdeq E_S(n,\mathsf{Eq}_\N(n),m)
\end{align*}
holds, from which the judgmental equalities in the statement of the definition follow.
\end{constr}

We can also define observational equality for many other kinds of types, such as $\bool$ or $\Z$. In each of these cases, what sets the observational equality apart from other relations is that it is the \emph{least} reflexive relation. However, the definitions of observational equality in the various situations are always very specific to the type they are defined on. In the next chapter we introduce the \emph{identity type}, which is an equality type defined uniformly for all types. In \cref{chap:hierarchy} we will show that the observational equality on $\N$ is equivalent to the identity type on $\N$.

\begin{exercises}
\item Let $A$ be a type.
  \begin{subexenum}
  \item Show that $(A+\neg A)\to(\neg\neg A\to A)$.
  \item Show that $\neg\neg\neg A \to \neg A$.
  \end{subexenum}
\item Construct a function
  \begin{equation*}
    \check{\Pi}:\prd{A:\UU} (\mathsf{Ty}(A)\to\UU)\to \UU
  \end{equation*}
  such that
  \begin{equation*}
    \mathsf{Ty}(\check{\Pi}(A,B))\jdeq \prd{x:\mathsf{Ty}(A)}\mathsf{Ty}(B(x))
  \end{equation*}
  holds for every $A:\UU$ and $B:\mathsf{Ty}(A)\to\UU$. 
  
  \emph{A similar exercise can be posed for $\Sigma$ and $+$ (and for $\to$ and $\times$ as special cases of $\Pi$ and $\Sigma$).}
\item \label{ex:obs_nat_eqrel}Show that observational equality on $\N$\index{observational equality!on N@{on $\N$}!is an equivalence relation} is an equivalence relation\index{equivalence relation!observational equality on N@{observational equality on $\N$}}, i.e., construct terms of the following types:
  \begin{align*}
    & \prd{n:\N} \EqN(n,n) \\
    & \prd{n,m:\N} \EqN(n,m)\to \EqN(m,n) \\
    & \prd{n,m,l:\N} \EqN(n,m)\to (\EqN(m,l)\to \EqN(n,l)).
  \end{align*}
\item \label{ex:obs_nat_least}\index{observational equality!on N@{on $\N$}!is least reflexive relation}Let $R$ be a reflexive binary relation\index{reflexive relation}\index{relation!reflexive} on $\N$, i.e., $R$ is of type $\N\to (\N\to\UU)$ and comes equipped with a term $\rho:\prd{n:\N}R(n,n)$. Show that
  \begin{equation*}
    \prd{n,m:\N} \EqN(n,m)\to R(n,m).
  \end{equation*}
\item \index{observational equality!on N@{on $\N$}!is preserved by functions}Show that every function $f:\N\to \N$ preserves observational equality in the sense that
  \begin{equation*}
    \prd{n,m:\N} \EqN(n,m)\to \EqN(f(n),f(m)).
  \end{equation*}
  \emph{Hint: to get the inductive step going the induction hypothesis has to be strong enough. Construct by double induction a term of type}
  \begin{equation*}
    \prd{n,m:\N}{f:\N\to\N} \EqN(n,m)\to \EqN(f(n),f(m)),
  \end{equation*}
  \emph{and pull out the universal quantification over $f:\N\to\N$ by \cref{ex:swap}.}
\item 
  \begin{subexenum}
  \item Define the \define{order relations}\index{relation!order}\index{order relation} $\leq$ and $<$ on $\N$.
  \item Show that $\leq$ is reflexive and that $<$ is \define{anti-reflexive}\index{anti-reflexive}\index{relation!anti-reflexive}, i.e., that $\neg(n<n)$. 
  \item Show that both $\leq$ and $<$ are transitive, and that $n<S(n)$.
  \item Show that $k\leq \min(m,n)$ holds if and only if both $k\leq m$ and $k\leq n$ hold, and show that $\max(m,n)\leq k$ holds if and only if both $m\leq k$ and $n\leq k$ hold.
  \end{subexenum}
\item \label{ex:obs_bool}\index{observational equality!on 2@{on $\bool$}}
  \begin{subexenum}
  \item Define observational equality $\mathsf{Eq}_\bool$\index{Eq_bool@{$\mathsf{Eq}_\bool$}|textbf} on the booleans.
  \item Show that $\mathsf{Eq}_\bool$ is reflexive.\index{observational equality!on 2@{on $\bool$}!is reflexive}
  \item Show that for any reflexive relation $R:\bool\to(\bool\to \UU)$ one has\index{observational equality!on 2@{on $\bool$}!is least reflexive relation}
    \begin{equation*}
      \prd{x,y:\bool} \mathsf{Eq}_\bool(x,y)\to R(x,y).
    \end{equation*}
  \end{subexenum}
\item \label{ex:int_order}
  \begin{subexenum}
  \item Define the order relations\index{relation!order}\index{order relation} $\leq$ and $<$ on and $\Z$.
  \item For $k:\Z$, consider the type $\Z_{\geq k}\defeq \sm{n:\Z}n\geq k$. Construct
    \begin{align*}
      b_k & : \Z_{\geq k} \\
      s_k & : \Z_{\geq k}\to\Z_{\geq k},
    \end{align*}
    and show that $\Z_{\geq k}$ satisfies the induction principle of the natural numbers\index{induction principle!of N@{of $\N$}}:
    \begin{equation*}
      \ind{\Z_{\geq k}} : P(b_k)\to \Big(\prd{n:\Z_{\geq k}} P(n)\to P(s_k(n))\Big)\to \Big(\prd{n:\Z_{\geq k}} P(n)\Big)
    \end{equation*}
  \end{subexenum}
\item
  \begin{subexenum}
  \item Show that $\N$ satisfies \define{strong induction}, i.e., construct for any type family $P$ over $\N$ a function of type
    \begin{equation*}
      P(\zeroN) \to \Big(\prd{k:\N}\Big(\prd{m:\N} (m\leq k) \to P(m)\Big)\to P(\succN(k))\Big) \to \prd{n:\N}P(n).
    \end{equation*}
  \item Show that $\N$ satisfies \define{ordinal induction}, i.e., construct for any type family $P$ over $\N$ a function of type
    \begin{equation*}
      \Big(\prd{k:\N} \Big(\prd{m:\N} (m< k) \to P(m)\Big)\to P(k)\Big) \to \prd{n:\N}P(n).
    \end{equation*}
  \end{subexenum}
\end{exercises}
