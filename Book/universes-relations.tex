\section{Universes}

\index{universe|(}
\index{universal family|(}
To complete our specification of dependent type theory, we introduce type theoretic \emph{universes}. Universes are types that consist of types. In other words, a universe is a type $\UU$ that comes equipped with a type family $\Ty$ over $\UU$, and for any $X:\UU$ we think of $X$ as an \emph{encoding}\index{encoding of a type in a universe} of the type $\Ty(X)$. We call this type family the \emph{universal type family}.

There are several reasons to equip type theory with universes. One reason is that it enables us to define new type families over inductive types, using their induction principle. For example, since the universe is itself a type, we can use the induction principle of $\bool$ to obtain a map $P:\bool\to\UU$ from any two terms $X_0,X_1:\UU$. Then we obtain a type family over $\bool$ by substituting $P$ into the universal type family:
\begin{equation*}
  x:\bool\vdash \Ty(P(x))~\type
\end{equation*}
satisfying $\Ty(P(0_\bool))\jdeq \Ty(X_0)$ and $\Ty(P(1_\bool))\jdeq \Ty(X_1)$.

We use this way of defining type families to define many familiar relations over $\N$, such as $\leq$ and $<$. We also introduce a relation called \emph{observational equality} $\EqN$ on $\N$, which we can think of as equality of $\N$. This relation is reflexive, symmetric, and transitive, and moreover it is the least reflexive relation. Furthermore, one of the most important aspects of observational equality $\EqN$ on $\N$ is that $\EqN(m,n)$ is a type for every $m,n:\N$, unlike judgmental equality. Therefore we can use type theory to reason about observational equality on $\N$. Indeed, in the exercises we show that some very elementary mathematics can already be done at this early stage in our development of type theory.

A second reason to introduce universes is that it allows us to define many types of types equipped with structure. One of the most important examples is the type of groups\index{group}, which is the type of types equipped with the group operations satisfying the group laws, and for which the underlying type is a set\index{set}. We won't discuss the condition for a type to be a set until \cref{chap:hierarchy}, so the definition of groups in type theory will be given much later. Therefore we illustrate this use of the universe by giving simpler examples: pointed types, graphs, and reflexive graphs.

One of the aspects that make universes useful is that they are postulated to be closed under all the type constructors. For example, if we are given $X:\UU$ and $P:\Ty(X)\to \UU$, then the universe is equipped with a term
\begin{equation*}
  \check{\Sigma}(X,P):\UU
\end{equation*}
satisfying the judgmental equality $\Ty(\check{\Sigma}(X,P)\jdeq\sm{x:\Ty(X)}\Ty(P(x))$. We will similarly assume that any universe is closed under $\Pi$-types and the other ways of forming types. However, there is an important restriction: it would be inconsistent to assume that the universe is contained in itself. One way of thinking about this is that universes are types of \emph{small} types, and it cannot be the case that the universe is small with respect to itself. We address this problem by assuming that there are many universes: enough universes so that any type family can be obtained by substituting into the universal type family of some universe.

\subsection{Specification of type theoretic universes}

In the following definition we already state that universes are closed under identity types. Identity types will be introduced in \cref{chap:identity}.

\begin{defn}
  A \define{universe} in type theory is a closed type $\UU$\index{U, V, W@{$\UU$, $\mathcal{V}$, $\mathcal{W}$}|see {universe}} equipped with a type family $\Ty$\index{T@{$\Ty$}|see {universal family}} over $\UU$ called the \define{universal family}\index{family!universal family}, equipped with the following structure:
  \begin{enumerate}
  \item $\UU$ is closed under $\Pi$, in the sense that it comes equipped with a function
    \begin{equation*}
      \check{\Pi} :\prd{X:\UU}(\Ty(X)\to\UU)\to\UU
    \end{equation*}
    for which the judgmental equality
    \begin{equation*}
      \Ty\big(\check{\Pi}(X,P)\big)\jdeq \prd{x:\Ty(X)}\Ty(P(x)).
    \end{equation*}
    holds, for every $X:\UU$ and $P:\Ty(X)\to\UU$.
  \item $\UU$ is closed under $\Sigma$ in the sense that it comes equipped with a function
    \begin{equation*}
      \check{\Sigma} :\prd{X:\UU}(\Ty(X)\to\UU)\to\UU
    \end{equation*}
    for which the judgmental equality
    \begin{equation*}
      \Ty\big(\check{\Sigma}(X,P)\big) \jdeq \sm{x:\Ty(X)}\Ty(P(x))
    \end{equation*}
    holds, for every $X:\UU$ and $P:\Ty(X)\to\UU$.
  \item $\UU$ is closed under identity types, in the sense that it comes equipped with a function
    \begin{equation*}
      \check{\mathrm{I}} : \prd{X:\UU}\Ty(X)\to(\Ty(X)\to\UU)
    \end{equation*}
    for which the judgmental equality
    \begin{equation*}
      \Ty\big(\check{\mathrm{I}}(X,x,y)\big)\jdeq (\id{x}{y})
    \end{equation*}
    holds, for every $X:\UU$ and $x,y:\Ty(X)$.
  \item $\UU$ is closed under coproducts, in the sense that it comes equipped with a function
    \begin{equation*}
      \check{+}:\UU\to (\UU\to\UU)
    \end{equation*}
    that satisfies $\Ty\big(X\check{+}Y\big)\jdeq \Ty(X)+\Ty(Y)$.
  \item $\UU$ contains terms $\check{\emptyt},\check{\unit},\check{\N}:\UU$
    that satisfy the judgmental equalities
    \begin{align*}
      \Ty\big(\check{\emptyt}\big) & \jdeq \emptyt \\
      \Ty\big(\check{\unit}\big) & \jdeq \unit \\
      \Ty(\check{\N}) & \jdeq \N.
    \end{align*}
  \end{enumerate}
  Given a universe $\UU$, we say that a type $A$ in context $\Gamma$ is \define{small}\index{small type}\index{universe!small types} with respect to $\UU$ if it occurs in the universe, i.e., if it comes equipped with a term $\check{A}:\UU$ in context $\Gamma$, for which the judgment
  \begin{equation*}
    \Gamma\vdash\Ty\big(\check{A}\big)\jdeq A~\type
  \end{equation*}
  holds. If $A$ is small with respect to $\UU$, we usually write simply $A$ for $\check{A}$ and also $A$ for $\Ty(\check{A})$. In other words, by $A:\UU$ we mean that $A$ is a small type. 
\end{defn}

\begin{rmk}
  Since ordinary function types are defined as a special case of dependent function types, we don't have to assume that universes are closed under ordinary function types. Similarly, it follows from the assumption that universes are closed under dependent pair types that universes are closed under cartesian product types.
\end{rmk}

\subsection{Assuming enough universes}

\index{enough universes|(}
\index{universe!enough universes|(}
  Most of the time we will get by with assuming one universe $\UU$, and indeed we recommend on a first reading of this text to simply assume that there is one universe $\UU$. However, sometimes we might need a second universe $\mathcal{V}$ that contains $\UU$ as well as all the types in $\UU$. In such situations we cannot get by with a single universe, because the assumption that $\UU$ is a term of itself would lead to inconsistencies like the Russel's paradox.

  Russel's paradox is the famous argument that there cannot be a set of all sets. If there were such a set $S$, then we could consider Russel's subset
  \begin{equation*}
    R:=\{x\in S\mid x\notin x\}.
  \end{equation*}
  Russell then observed that $R\in R$ if and only if $R\notin R$, so we reach a contradiction. A variant of this argument reaches a similar contradiction when we assume that $\UU$ is a universe that contains a term $\check{\UU}:\UU$ such that $\mathcal{T}\big(\check{\UU}\big)\jdeq \UU$. In order to avoid such paradoxes, Russell and Whitehead formulated the \emph{ramified theory of types} in their book \emph{Principia Mathematica}. The ramified theory of types is a precursor of Martin L\"of's type theory that we are studying in this course.  

  Even though the universe is not a term of itself, it is still convenient if every type, including any universe, is small with respect to \emph{some} universe. Therefore we will assume that there are sufficiently many universes: we will assume that for every finite list of types
\begin{align*}
  \Gamma_1 & \vdash A_1~\type \\
  & ~\vdots \\
  \Gamma_n & \vdash A_n~\type,
\end{align*}
there is a universe $\UU$ that contains each $A_i$ in the sense that $\UU$ comes equipped with a term
\begin{align*}
  \Gamma_i\vdash \check{A}_i:\UU
\end{align*}
for which the judgment
\begin{equation*}
  \Gamma_i\vdash \Ty\big(\check{A}_i\big)\jdeq A_i~\type
\end{equation*}
holds. With this assumption it will rarely be necessary to work with more than one universe at the same time.

\begin{rmk}\label{rmk:universe-constructions}
  Using the assumption that for any finite list of types in context there is a universe that contains those types, we obtain many specific universes:
  \begin{enumerate}
  \item There is a \emph{base universe} $\UU_0$ that we obtain using the empty list of types in context. This is a universe, but it isn't specified to contain any further types.
  \item Given a finite list
    \begin{align*}
      \Gamma_1 & \vdash A_1~\type \\
      & ~\vdots \\
      \Gamma_n & \vdash A_n~\type,
    \end{align*}
    of types in context, and a universe $\UU$ that contains them, there is a universe $\UU^+$ that contains all the types in $\UU$ as well as $\UU$. More precisely, it is specified by the finite list
    \begin{align*}
      & \vdash \UU~\type \\
      X:\UU & \vdash \mathcal{T}(X)~\type.
    \end{align*}
    The universe $\UU^+$ therefore contains the type $\UU$ as well as every type in $\UU$, in the following sense
    \begin{align*}
      & \vdash \check{\UU}:\UU^+ & & \vdash \mathcal{T}^+(\check{\UU})\jdeq\UU~\type \\
      X:\UU & \vdash \check{\mathcal{T}}(X) :\UU^+ & X:\UU & \vdash \mathcal{T}^+(\check{\mathcal{T}}(X))\jdeq \mathcal{T}(X)~\type.
    \end{align*}
    In particular, we obtain a function $i:\UU\to\UU^+$ that includes the $\UU$-small types into $\UU^+$.
    
    Note that since the universe $\UU^+$ contains all the types in $\UU$, it also contains the types $A_1,\ldots,A_n$. To see this, we derive that there is a code for $A_i$ in $\UU^+$.
    \begin{prooftree}
      \AxiomC{$\Gamma_i\vdash\check{A}_i:\UU$}
      \AxiomC{$X:\UU\vdash\check{\mathcal{T}}(X):\UU^+$}
      \UnaryInfC{$\Gamma_i,X:\UU\vdash\check{\mathcal{T}}(X):\UU^+$}
      \BinaryInfC{$\Gamma_i\vdash\check{\mathcal{T}}(\check{A}_i):\UU^+$}
    \end{prooftree}
    We leave it as an exercise to derive the judgmental equality
    \begin{equation*}
      \mathcal{T}^+(\check{\mathcal{T}}(\check{A}_i))\jdeq A_i.
    \end{equation*}
  \item Given two finite lists
    \begin{align*}
      \Gamma_1 & \vdash A_1~\type & \Delta_1 & \vdash B_1~\type \\
      & ~\vdots & & ~\vdots \\
      \Gamma_n & \vdash A_n~\type & \Delta_m & \vdash B_m~\type
    \end{align*}
    of types in context, and two universes $\mathcal{U}$ and $\mathcal{V}$ that contain $A_1,\ldots,A_n$ and $B_1,\ldots,B_m$ respectively, there is a universe $\UU\sqcup\mathcal{V}$ that contains the types of both $\UU$ and $\mathcal{V}$. The universe $\UU\sqcup\mathcal{V}$ is specified by the finite list
    \begin{align*}
      X:\UU & \vdash \mathcal{T}_{\mathcal{U}}(X)~\type \\
      Y:\mathcal{V} & \vdash \mathcal{T}_{\mathcal{V}}(Y) ~\type.
    \end{align*}
    With an argument similar to the previous construction of a universe, we see that the universe $\UU\sqcup\mathcal{V}$ contains the types $A_1,\ldots,A_n$ as well as the types $B_1,\ldots,B_m$.

    Note that we could also directly obtain a universe $\mathcal{W}$ that contains the types $A_1,\ldots,A_n$ and $B_1,\ldots,B_m$. However, this universe might not contain all the types in $\UU$ or all the types in $\mathcal{V}$.
  \end{enumerate}
  Since we don't postulate any relations between the universes, there are indeed very few of them. For example, the base universe $\UU_0$ might contain many more types than it is postulated to contain. Nevertheless, there are some relations between the universes. For instance, there is a function $\UU\to\UU^+$, since we can simply derive
  \begin{prooftree}
    \AxiomC{$X:\UU\vdash \check{\mathcal{T}}(X):\UU^+$}
    \UnaryInfC{$\vdash \lam{X}\check{\mathcal{T}}(X) : \UU\to\UU^+$}
  \end{prooftree}
  Similarly, there are functions $\UU\to \UU\sqcup\mathcal{V}$ and $\mathcal{V}\to \UU\sqcup\mathcal{V}$ for any two universes $\UU$ and $\mathcal{V}$.
\end{rmk}
\index{enough universes|)}
\index{universe!enough universes|)}

\subsection{Pointed types}

\index{pointed type|(}
\begin{defn}
  A \define{pointed type} is a pair $(A,a)$ consisting of a type $A$ and a term $a:A$. The type of all pointed types in a universe $\UU$ is defined to be\index{UU*@{$\UU_\ast$}}
  \begin{equation*}
    \UU_\ast \defeq \sm{X:\UU}X.
  \end{equation*}
\end{defn}

\begin{defn}
  Consider two pointed types $(A,a)$ and $(B,b)$. A \define{pointed map}\index{pointed map} from $(A,a)$ to $(B,b)$ is a pair $(f,p)$ consisting of a function $f:A\to B$ and an identification $p:f(a)=b$. We write
  \begin{equation*}
    A\to_\ast B \defeq \sm{f:A\to B}f(a)=b
  \end{equation*}
  for the type of all pointed maps from $(A,a)$ to $(B,b)$, leaving the base point implicit.
\end{defn}

Since we have a type $\UU_\ast$ of \emph{all} pointed types in a universe $\UU$, we can start defining operations on $\UU_\ast$. An important example of such an operation is to take the loop space of a pointed type.

\begin{defn}
  We define the \define{loop space}\index{loop space}\index{Omega@{$\Omega$}|see {loop space}} operation $\Omega : \UU_\ast \to \UU_\ast$
  \begin{equation*}
    \Omega(A,a)\defeq \big((a=a),\refl{a}\big).
  \end{equation*}
\end{defn}

We can even go further and define the \emph{iterated loop space} of a pointed type. Note that this definition could not be given in type theory if we didn't have universes.

\begin{defn}
  Given a pointed type $(A,a)$ and a natural number $n$, we define the $n$-th loop space $\Omega^n(A,a)$ by induction on $n:\N$, taking\index{iterated loop space}\index{Omega^n@{$\Omega^n$}|see {iterated loop space}}
  \begin{align*}
    \Omega^0(A,a) & \defeq (A,a) \\
    \Omega^{n+1}(A,a) & \defeq \Omega(\Omega^n(A,a)).
  \end{align*}
\end{defn}
\index{pointed type|)}

\subsection{Families and relations on the natural numbers}

As we have already seen in the case of the iterated loop space, we can use the universe to define a type family over $\N$ by induction on $\N$. For example, we can define the finite types in this way.

\begin{defn}\label{defn:fin}
We define the type family $\Fin:\N\to\UU$ of finite types\index{Fin@{$\Fin$}}\index{finite types} by induction on $\N$\index{family!of finite types}, taking
\begin{align*}
\Fin(\zeroN) & \defeq \emptyt \\
\Fin(\succN(n)) & \defeq \Fin(n)+\unit
\end{align*}
\end{defn}

Similarly, we can define many relations on the natural numbers using a universe. We give here the example of \emph{observational equality} on $\N$. This inductively defined equivalence relation is very important, as it can be used to show that equality on the natural numbers is \emph{decidable}, i.e., there is a program that decides for any two natural numbers $m$ and $n$ whether they are equal or not.

\begin{defn}\label{defn:obs_nat}
We define the \define{observational equality}\index{observational equality!on N@{on $\N$}}\index{natural numbers!observational equality} on $\N$ as binary relation $\EqN:\N\to(\N\to\UU)$\index{Eq N@{$\EqN$}} satisfying
\begin{align*}
\EqN(\zeroN,\zeroN) & \jdeq \unit & \EqN(\succN(n),\zeroN) & \jdeq \emptyt \\
\EqN(\zeroN,\succN(n)) & \jdeq \emptyt & \EqN(\succN(n),\succN(m)) & \jdeq \EqN(n,m).
\end{align*}
\end{defn}

\begin{constr}
We define $\EqN$ by double induction on $\N$. By the first application of induction it suffices to provide
\begin{align*}
E_0 & : \N\to\UU \\
E_S & : \N\to (\N\to\UU)\to(\N\to\UU)
\end{align*}
We define $E_0$ by induction, taking $E_{00}\defeq \unit$ and $E_{0S}(n,X,m)\defeq \emptyt$. The resulting family $E_0$ satisfies
\begin{align*}
E_0(\zeroN) & \jdeq \unit \\
E_0(\succN(n)) & \jdeq \emptyt.
\end{align*} 
We define $E_S$ by induction, taking $E_{S0}\defeq \emptyt$ and $E_{S0}(n,X,m)\defeq X(m)$. The resulting family $E_S$ satisfies
\begin{align*}
E_S(n,X,\zeroN) & \jdeq \emptyt \\
E_S(n,X,\succN(m)) & \jdeq X(m) 
\end{align*}
Therefore we have by the computation rule for the first induction that the judgmental equality
\begin{align*}
\EqN(\zeroN,m) & \jdeq E_0(m) \\
\EqN(\succN(n),m) & \jdeq E_S(n,\EqN(n),m)
\end{align*}
holds, from which the judgmental equalities in the statement of the definition follow.
\end{constr}

\begin{lem}\label{ex:obs_nat_least}\index{observational equality!on N@{on $\N$}!is least reflexive relation}
  Suppose $R:\N\to(\N\to\UU)$ is a reflexive relation on $\N$\index{reflexive relation}\index{relation!reflexive}, i.e., $R$ comes equipped with
  \begin{equation*}
    \rho : \prd{n:\N}R(n,n).
  \end{equation*}
  Then there is a family of maps
  \begin{equation*}
    \prd{m,n:\N}\EqN(m,n)\to R(m,n).
  \end{equation*}
\end{lem}

\begin{proof}
  We will prove by induction on $m,n:\N$ that there is a term of type
  \begin{equation*}
    f_{m,n}:\prd{e:\EqN(m,n)}\prd{R:\N\to(\N\to\UU)}\Big(\prd{x:\N}R(x,x)\Big)\to R(m,n)
  \end{equation*}
  The dependent function $f_{m,n}$ is defined by
  \begin{align*}
    f_{\zeroN,\zeroN} & \defeq \lam{\ttt}\lam{r}\lam{\rho}\rho(\zeroN) \\
    f_{\zeroN,\succN(n)} & \defeq \indempty \\
    f_{\succN(m),\zeroN} & \defeq \indempty \\
    f_{\succN(m),\succN(n)} & \defeq \lam{e}\lam{R}\lam{\rho}f_{m,n}(e,R',\rho'),
  \end{align*}
  where $R'$ and $\rho'$ are given by
  \begin{align*}
    R'(m,n) & \defeq R(\succN(m),\succN(n)) \\
    \rho'(n) & \defeq \rho(\succN(n)).\qedhere
  \end{align*}
\end{proof}

We can also define observational equality for many other kinds of types, such as $\bool$ or $\Z$. In each of these cases, what sets the observational equality apart from other relations is that it is the \emph{least} reflexive relation. 

\begin{exercises}
\exercise \label{ex:obs_nat_eqrel}Show that observational equality on $\N$\index{observational equality!on N@{on $\N$}!is an equivalence relation} is an equivalence relation\index{equivalence relation!observational equality on N@{observational equality on $\N$}}, i.e., construct terms of the following types:
  \begin{align*}
    & \prd{n:\N} \EqN(n,n) \\
    & \prd{n,m:\N} \EqN(n,m)\to \EqN(m,n) \\
    & \prd{n,m,l:\N} \EqN(n,m)\to (\EqN(m,l)\to \EqN(n,l)).
  \end{align*}
\exercise \index{observational equality!on N@{on $\N$}!is preserved by functions}Show that every function $f:\N\to \N$ preserves observational equality in the sense that
  \begin{equation*}
    \prd{n,m:\N} \EqN(n,m)\to \EqN(f(n),f(m)).
  \end{equation*}
\exercise \label{ex:order_N}
  \begin{subexenum}
  \item Define the \define{order relations}\index{relation!order}\index{order relation!leq on N@{$\leq$ on $\N$}}\index{order relation!le on N@{$<$ on $\N$}}\index{leq@{$\leq$}!on N@{on $\N$}}\index{le@{$<$}!on N@{on $\N$}} $\leq$ and $<$ on $\N$.
  \item Show that $\leq$ is reflexive\index{reflexive!leq on N@{$\leq$ on $\N$}} and that $<$ is \define{irreflexive}\index{irreflexive}\index{relation!irreflexive}, i.e., show that
    \begin{equation*}
      (n\leq n)\qquad\text{and}\qquad (n\nless n).
    \end{equation*}
  \item Show that $n\leq\succN(n)$ and $n<\succN(n)$.
  \item Show that both $\leq$ and $<$ are transitive.
  \item Show that $\leq$ is \define{antisymmetric}, i.e., show that $m=n$ whenever both $m\leq n$ and $n\leq m$ hold. Also show that if $m<n$, then $n\nless m$.
  \item Show that
    \begin{equation*}
      (m\leq n)\leftrightarrow (m=n)+(m<n)
    \end{equation*}
    holds for all $m,n$.
  \item Show that
    \begin{equation*}
      (m\leq n)+(n<m)
    \end{equation*}
    holds for all $m,n$.
  \item Show that $k\leq \min(m,n)$ holds if and only if both $k\leq m$ and $k\leq n$ hold, and show that $\max(m,n)\leq k$ holds if and only if both $m\leq k$ and $n\leq k$ hold.
  \end{subexenum}
\exercise \label{ex:obs_bool}\index{observational equality!on 2@{on $\bool$}}
  \begin{subexenum}
  \item Define observational equality $\EqBool$\index{Eq 2@{$\EqBool$}} on the booleans.
  \item Show that $\EqBool$ is reflexive.\index{observational equality!on 2@{on $\bool$}!is reflexive}
  \item Show that for any reflexive relation $R:\bool\to(\bool\to \UU)$ one has\index{observational equality!on 2@{on $\bool$}!is least reflexive relation}
    \begin{equation*}
      \prd{x,y:\bool} \EqBool(x,y)\to R(x,y).
    \end{equation*}
  \end{subexenum}
\exercise \label{ex:int_order}
  \begin{subexenum}
  \item Define the order relations\index{relation!order}\index{order relation} $\leq$ and $<$ on and $\Z$.
  \item Show that $\leq$ is reflexive, transitive, and antisymmetric.
  \item Show that $<$ is irreflexive and transitive.
  \end{subexenum}
\begin{comment}
\item
  \begin{subexenum}
  \item For each $i:\Fin(\succN(n))$, define a function
    \begin{equation*}
      \skipFin_i : \Fin(n)\to\Fin(\succN(n))
    \end{equation*}
    that includes $\Fin(n)$ in $\Fin(\succN(n))$ by skipping $i$.
  \item For each $i:\Fin(n)$, define a function
    \begin{equation*}
      \doubleFin_i : \Fin(\succN(n))\to\Fin(n)
    \end{equation*}
    that projects $\Fin(\succN(n))$ onto $\Fin(n)$ by doubling at $i$. 
  \end{subexenum}
\end{comment}
\end{exercises}
\index{universe|)}
\index{universal family|)}

