\chapter{The univalence axiom}

\section{Type extensionality}
\begin{defn}
We define a family of maps
\begin{equation*}
\mathsf{equiv\usc{}eq}\defeq \mathsf{rec}_{=}(\lam{A}\idfunc[A]) : \prd*{A,B:\UU} (\id{A}{B})\to(\eqv{A}{B}).
\end{equation*}
\end{defn}

\begin{defn}
The \define{univalence axiom} asserts that the family of maps $\mathsf{equiv\usc{}eq}$ is a fiberwise equivalence.
\end{defn}

The univalence axiom asserts that equivalent types are equal. It is considered to be an \emph{extensionality principle} for types.

\begin{lem}
The univalence axiom holds if and only if the type
\begin{equation*}
\sm{B:\UU}\eqv{A}{B}
\end{equation*}
is contractible for each $A:\UU$.
\end{lem}

\begin{proof}
By \autoref{thm:id_fundamental}.
\end{proof}

The following construction enables us to make construction by induction on equivalences, analogous to path induction.

\begin{defn}
Let $A:\UU$, and let $P:\prd{B:\UU} (\eqv{A}{B})\to\type$ be a type family. Using the univalence axiom we construct
\begin{equation*}
\mathsf{equiv\usc{}ind}(P,A) : P(A,\idfunc[A])\to \prd{B:\UU}{e:\eqv{A}{B}}P(B,e).
\end{equation*}
\end{defn}

\begin{constr}
Since $\sm{B:\UU}\eqv{A}{B}$ is contractible we have
\begin{equation*}
P(\idfunc[A])\to\prd{\pairr{B,e}:\sm{B:\UU}\eqv{A}{B}}P(B,e)
\end{equation*}
by \autoref{ex:contr_ind}, so we obtain the desired function by uncurrying.
\end{constr}

From now on we will assume that the univalence axiom holds.

\section{Function extensionality}

The first application of the univalence axiom was Voevodsky's proof of function extensionality.

\begin{thm}
Univalence implies the weak principle of function extensionality.
\end{thm}

\begin{exercises}
\item \label{ex:tr_ap} Show that for any $P:X\to \UU$ and any $p:x=y$ in $X$, we have
\begin{equation*}
\mathsf{equiv\usc{}eq}(\ap{P}{p})=\mathsf{tr}^P(p).
\end{equation*}
\item Use the univalence axiom to show that the type $\sm{A:\UU}\iscontr(A)$ of all contractible types in $\UU$ is contractible.
\item Use the univalence axiom to show that the type $\sm{P:\prop}P$ is contractible.
\item Show that $\eqv{(\eqv{\bool}{\bool})}{\bool}$, and conclude by the univalence axiom that the universe is not a set.
\item Construct by path induction a family of maps
\begin{equation*}
\prd{A,B:\UU}{a:A}{b:B} (\id{\pairr{A,a}}{\pairr{B,b}})\to \sm{e:\eqv{A}{B}}e(a)=b,
\end{equation*}
and show that this map is an equivalence. In other words, an \emph{identification of pointed types} is a base point preserving equivalence.
\item Let $\pairr{A,a}$ and $\pairr{B,b}$ be two pointed types. Construct by path induction a family of maps
\begin{equation*}
\prd{f,g:A\to B}{p:f(a)=b}{q:g(a)=b} (\id{\pairr{f,p}}{\pairr{g,q}})\to \sm{H:f\htpy g} p = \ct{H(a)}{q},
\end{equation*}
and show that this map is an equivalence. In other words, an \emph{identification of pointed maps} is a base point preserving homotopy.
\end{exercises}
