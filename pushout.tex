\chapter{Homotopy pushouts}

Homotopy pushouts are dual to homotopy pullbacks. However, unlike pullbacks we have to \emph{assume} that pushouts exist by postulating rules for higher inductive types. For the purpose of this course, the only higher inductive types that we add are the pushouts. Some of the more exotic higher inductive types are described in \cite{hottbook}.

\section{Cocartesian squares}

In \cref{defn:cospan} we broke down a commuting square into a cospan, a type, and a cone with that type as its vertex. 
\begin{defn}
A \define{span} from $A$ to $B$ consists of a type $S$ and maps $f:S\to A$ and $g:S\to B$, as indicated in the diagram
\begin{equation*}
\begin{tikzcd}
A & S \arrow[l,swap,"f"] \arrow[r,"g"] & B.
\end{tikzcd}
\end{equation*}
\end{defn}

\begin{defn}
Consider a span $\mathcal{S}\jdeq (S,f,g)$ from $A$ to $B$, and let $X$ be a type.
A \define{cocone} with vertex $X$ on $\mathcal{S}$ is a triple $(i,j,H)$ consisting of maps $i:A\to X$ and $j:B\to X$, and a homotopy $H:i\circ f\htpy j\circ g$ witnessing that the square
\begin{equation*}
\begin{tikzcd}
S \arrow[r,"g"] \arrow[d,swap,"f"] & B \arrow[d,"j"] \\
A \arrow[r,swap,"i"] & X
\end{tikzcd}
\end{equation*}
commutes.
We write $\mathsf{cocone}_{\mathcal{S}}(X)$ for the type of cocones with vertex $X$.
\end{defn}

\begin{defn}
Consider a commuting square
\begin{equation*}
\begin{tikzcd}
S \arrow[r,"g"] \arrow[d,swap,"f"] & B \arrow[d,"j"] \\
A \arrow[r,swap,"i"] & X,
\end{tikzcd}
\end{equation*}
where $(i,j,H)$ is a cocone with vertex $X$ on the span $A \leftarrow S \rightarrow B$.
We say that the square is a \define{homotopy pushout}, or that it is \define{cocartesian}, if it satisfies the \define{universal property of pushouts}, which asserts that for any type $Y$ the map
\begin{equation*}
\mathsf{cocone\usc{}map}(i,j,H):(X\to Y)\to \mathsf{cocone}(Y)
\end{equation*}
given by $f\mapsto (f\circ i,f\circ j,f\cdot H)$ is an equivalence. Sometimes we simply say that $X$ is a pushout if the square is a pushout.
\end{defn}

\begin{rmk}
Given a pushout square
\begin{equation*}
\begin{tikzcd}
S \arrow[r,"g"] \arrow[d,swap,"f"] & B \arrow[d,"j"] \\
A \arrow[r,swap,"i"] & X,
\end{tikzcd}
\end{equation*}
we can view the cocone $(i,j,H)$ as \emph{structure} on $X$, in the sense that $X$ comes equipped with
\begin{align*}
i & : A\to X \\
j & : B\to X \\
H & : \prd{s:S} i(f(s))=j(g(s)).
\end{align*}
As we will see in \cref{thm:pushout_up}, the type $X$ is a pushout precisely when it satisfies an \emph{induction principle} formulated in terms of $(i,j,H)$. However, the homotopy $H$ provides \emph{path constructors} of $X$. 

The induction principle of pushouts is formulated with respect to families $P$ over $X$, and provides a way to construct sections of $P$. Note that from any section $s:\prd{x:X}P(x)$ we obtain
\begin{align*}
s\circ i & : \prd{a:A}P(i(a)) \\
s\circ j & : \prd{b:B}P(j(b)) \\
s\cdot H & : \prd{x:S}\mathsf{tr}_P(H(x),s(i(x)))=s(j(x)).
\end{align*}
It will be useful to write
\begin{equation*}
i' \htpy_H j' \defeq \prd{s:S} \mathsf{tr}_P(H(s),i'(f(s)))=j'(g(s))
\end{equation*}
for the type of $s\cdot H$. Thus we see that there is a map
\begin{equation*}
\Big(\prd{x:X}P(x)\Big)\to \sm{i':\prd{a:A}P(i(a))}{j':\prd{b:B}P(j(b))} i'\htpy_H j'
\end{equation*}
given by $s\mapsto (s\circ i,s\circ j,s\cdot H)$.
\end{rmk}

\begin{thm}\label{thm:pushout_up}
Consider a commuting square
\begin{equation*}
\begin{tikzcd}
S \arrow[r,"g"] \arrow[d,swap,"f"] & B \arrow[d,"j"] \\
A \arrow[r,swap,"i"] & X,
\end{tikzcd}
\end{equation*}
where $(i,j,H)$ is a cocone with vertex $X$ on the span $A \leftarrow S \rightarrow B$.
The following are equivalent:
\begin{enumerate}
\item The square is a pushout square.
\item The commuting square
\begin{equation*}
\begin{tikzcd}
Y^X \arrow[r,"\blank\circ j"] \arrow[d,swap,"\blank\circ i"] & Y^B \arrow[d,"\blank\circ g"] \\
Y^A \arrow[r,swap,"\blank\circ f"] & Y^S
\end{tikzcd}
\end{equation*}
is a pullback square, for every type $Y$.
\item The type $X$ satisfies \define{span induction} for the span $A\leftarrow S \rightarrow B$, in the sense that for any type family $P$ over $X$, the map
\begin{equation*}
\Big(\prd{x:X}P(x)\Big)\to \Big(\sm{i':\prd{a:A}P(i(a))}{j':\prd{b:B}P(j(b))} i'\htpy_H j'\Big)
\end{equation*}
given by $s\mapsto (s\circ i,s\circ j,s\cdot H)$ has a section.
\end{enumerate}
\end{thm}

\begin{eg}
By \autoref{ex:circle_up_pushout} and the second characterization of pushouts in \autoref{thm:pushout_up} it follows that the circle is a pushout
\begin{equation*}
\begin{tikzcd}
\bool \arrow[r] \arrow[d] & \unit \arrow[d] \\
\unit \arrow[r] & \sphere{1}.
\end{tikzcd}
\end{equation*}
\end{eg}

\begin{proof}[Proof of \cref{thm:pushout_up}]
The type $\mathsf{cocone}(X)$ is a pullback in the following way:
\begin{equation*}
\begin{tikzcd}
\mathsf{cocone}(X) \arrow[r,"\pi_2"] \arrow[d,swap,"\pi_1"] & X^B \arrow[d,"\blank\circ g"] \\
X^A \arrow[r,swap,"\blank\circ f"] & X^S
\end{tikzcd}
\end{equation*}
Moreover, $\mathsf{cocone\usc{}map}(\mathcal{X}):(X\to Y)\to \mathsf{cocone}(Y)$ is the unique map making the diagram
\begin{equation*}
\begin{tikzcd}
Y^X \arrow[ddr,bend right=15] \arrow[drr,bend left=15] \arrow[dr,densely dotted,"\mathsf{cocone\usc{}map}(\mathcal{X})" {description,xshift=1em}] &[2em] \\
& \mathsf{cocone}(Y) \arrow[r,"\pi_2"] \arrow[d,swap,"\pi_1"] & Y^B \arrow[d,"\blank\circ g"] \\
& Y^A \arrow[r,swap,"\blank\circ f"] & Y^S
\end{tikzcd}
\end{equation*}
Therefore it follows that the cocone $\mathcal{X}$ is colimiting precisely when the outer commuting square is a pullback.
\end{proof}

\begin{cor}
Consider two commuting squares
\begin{equation*}
\begin{tikzcd}
S \arrow[r,"g"] \arrow[d,swap,"f"] & B \arrow[d,"j"] & S \arrow[r,"g"] \arrow[d,swap,"f"] & B \arrow[d,"{j'}"] \\
A \arrow[r,swap,"i"] & X & A \arrow[r,swap,"{i'}"] & {X'}
\end{tikzcd}
\end{equation*}
with homotopies $H:i\circ f\htpy j\circ g$ and $H':i'\circ f\htpy j'\circ g$. Furthermore, consider a map
\begin{equation*}
h:X\to X'
\end{equation*}
equipped with
\begin{align*}
K & : h\circ i\htpy i' \\
L & : h\circ j\htpy j' \\
M & : \ct{(h\cdot H)}{(L\cdot g)} \htpy \ct{(K\cdot f)}{H'}.
\end{align*}
If any two of the following three properties hold, then so does the third:
\begin{enumerate}
\item $X$ is a pushout.
\item $X'$ is a pushout.
\item $h$ is an equivalence.
\end{enumerate}
\end{cor}

\section{Rules for pushouts}

The \define{formation rule} for pushouts simply states that for any span $\mathcal{S}\defeq (S,f,g)$ from $A$ to $B$, a type $A\sqcup^{\mathcal{S}} B$ can be formed. We call $A\sqcup^{\mathcal{S}} B$ the \define{canonical pushout} of $\mathcal{S}$. 

\begin{prooftree}
\AxiomC{$\Gamma\vdash f:S\to A$}
\AxiomC{$\Gamma\vdash g:S\to B$}
\BinaryInfC{$\Gamma\vdash A\sqcup^{\mathcal{S}} B~\mathrm{type}$}
\end{prooftree}

The \define{introduction rules} for pushouts provide ways to construct terms of the type $A\sqcup^{\mathcal{S}} B$, and ways to identify some of those.
\begin{prooftree}
\AxiomC{$\Gamma\vdash f:S\to A$}
\AxiomC{$\Gamma\vdash g:S\to B$}
\BinaryInfC{$\Gamma\vdash \inl : A \to A\sqcup^{\mathcal{S}} B$}
\end{prooftree}

\begin{prooftree}
\AxiomC{$\Gamma\vdash f:S\to A$}
\AxiomC{$\Gamma\vdash g:S\to B$}
\BinaryInfC{$\Gamma\vdash \inr : B \to A\sqcup^{\mathcal{S}} B$}
\end{prooftree}

\begin{prooftree}
\AxiomC{$\Gamma\vdash f:S\to A$}
\AxiomC{$\Gamma\vdash g:S\to B$}
\BinaryInfC{$\Gamma\vdash \glue : \inl\circ f \htpy \inr\circ g$}
\end{prooftree}
We assume that $A\sqcup^{\mathcal{S}} B$ is span inductive in the sense of \autoref{thm:pushout_up}. Moreover, if $A$, $B$, and $S$ are types in $\UU$, then we assume that also $A\sqcup^{\mathcal{S}} B$ is in $\UU$. In other words, we assume that the universe is \emph{closed under pushouts}.

\section{Suspension and join}
Many interesting types can be defined as homotopy pushouts. 

\begin{defn}
\begin{enumerate}
\item Let $X$ be a type. We define the \define{suspension} $\susp X$ of $X$ to be the pushout of the span
\begin{equation*}
\begin{tikzcd}
X \arrow[r] \arrow[d] & \unit \arrow[d,"\inr"] \\
\unit \arrow[r,swap,"\inl"] & \susp X 
\end{tikzcd}
\end{equation*}
\item We define the \define{$n$-sphere} $\sphere{n}$ for any $n:\N$ by induction on $n$, by taking
\begin{align*}
\sphere{0} & \defeq \bool \\
\sphere{n+1} & \defeq \susp{\sphere{n}}.
\end{align*}
\item Given a map $f:A\to B$, we define the \define{mapping cone} $M_f$ of $f$ as the pushout
\begin{equation*}
\begin{tikzcd}
A \arrow[r,"f"] \arrow[d] & B \arrow[d,"\inr"] \\
\unit \arrow[r,swap,"\inl"] & M_f 
\end{tikzcd}
\end{equation*}
\item We define the \define{join} $\join{X}{Y}$ of $X$ and $Y$ to be the pushout 
\begin{equation*}
\begin{tikzcd}
X\times Y \arrow[r,"\proj 2"] \arrow[d,swap,"\proj 1"] & Y \arrow[d,"\inr"] \\
X \arrow[r,swap,"\inl"] & X \ast Y 
\end{tikzcd}
\end{equation*}
\end{enumerate}
\end{defn}

\begin{thm}
Consider the following configuration of commuting squares:
\begin{equation*}
\begin{tikzcd}
A \arrow[r,"i"] \arrow[d,swap,"f"] & B \arrow[r,"k"] \arrow[d,swap,"g"] & C \arrow[d,"h"] \\
X \arrow[r,swap,"j"] & Y \arrow[r,swap,"l"] & Z
\end{tikzcd}
\end{equation*}
with homotopies $H:j\circ f\htpy g\circ i$ and $K:l\circ g\htpy h\circ k$, and suppose that the square on the left is a pushout square. 
Then the square on the right is a pushout square if and only if the outer rectangle is a pushout square.
\end{thm}

\begin{proof}
Let $T$ be a type. Taking the exponent $T^{(\blank)}$ of the entire diagram of the statement of the theorem, we obtain the following commuting diagram
\begin{equation*}
\begin{tikzcd}
T^Z \arrow[r,"\blank\circ l"] \arrow[d,swap,"\blank\circ h"] & T^Y \arrow[d,swap,"\blank\circ g"] \arrow[r,"\blank\circ j"] & T^X \arrow[d,"\blank\circ f"] \\
T^C \arrow[r,swap,"\blank\circ k"] & T^B \arrow[r,swap,"\blank\circ i"] & T^A.
\end{tikzcd}
\end{equation*}
By the assumption that $Y$ is the pushout of $B\leftarrow A \rightarrow X$, it follows that the square on the right is a pullback square. It follows by \autoref{lem:pb_pasting} that the rectangle on the left is a pullback if and only if the outer rectangle is a pullback. Thus the statement follows by the second characterization in \autoref{thm:pushout_up}.
\end{proof}

\begin{comment}
\begin{defn}
Let $X$ and $Y$ be types with base points $x_0$ and $y_0$, respectively.
We define the \define{wedge} $X\lor Y$ of $X$ and $Y$ to be the pushout
\begin{equation*}
\begin{tikzcd}[column sep=8em]
\bool \arrow[r,"{\ind{\bool}(\inl(x_0),\inr(y_0))}"] \arrow[d,swap,"\mathsf{const}_\ttt"] & X+Y \arrow[d,"\inr"] \\
\unit \arrow[r,swap,"\inl"] & X\lor Y
\end{tikzcd}
\end{equation*}
\end{defn}

\begin{defn}
Let $X$ and $Y$ be types with base points $x_0$ and $y_0$, respectively.
We define a map
\begin{equation*}
\mathsf{wedge\usc{}incl} : X \lor Y \to X\times Y.
\end{equation*}
as the unique map obtained from the commutative square
\begin{equation*}
\begin{tikzcd}[column sep=8em]
\bool \arrow[r,"{\ind{\bool}(\inl(x_0),\inr(y_0))}"] \arrow[d,swap,"\mathsf{const}_\ttt"] & X+Y \arrow[d,"{\ind{X+Y}(\lam{x}\pairr{x,y_0},\lam{y}\pairr{x_0,y})}"] \\
\unit \arrow[r,swap,"\lam{t}\pairr{x_0,y_0}"] & X\times Y.
\end{tikzcd}
\end{equation*}
\end{defn}

\begin{defn}
We define the \define{smash product} $X\wedge Y$ of $X$ and $Y$ to be the pushout
\begin{equation*}
\begin{tikzcd}[column sep=huge]
X\lor Y \arrow[r,"\mathsf{wedge\usc{}incl}"] \arrow[d,swap,"\mathsf{const}_\ttt"] & X\times Y \arrow[d,"\inr"] \\
\unit \arrow[r,swap,"\inl"] & X\wedge Y.
\end{tikzcd}
\end{equation*}
\end{defn}
\end{comment}

\begin{exercises}
\item Use the univalence axiom to show that the type $\mathsf{span}(A,B)$ of spans from $A$ to $B$ is equivalent to the type $A\to B\to\UU$.
\item Show that $\join{P}{Q}= P\lor Q$, for any two propositions $P$ and $Q$.
\item Show that $\sphere{1}$ is equivalent to $\susp\bool$. 
\item Show that $\eqv{A\sqcup^{\mathcal{S}} B}{B\sqcup^{\mathcal{S}'} A}$, where $\mathcal{S'}\defeq (S,g,f)$ is the \define{opposite span} of $\mathcal{S}$. 
\item Let $Q$ be a proposition, and let $A$ be a type. Show that $\inr:A\to \join{Q}{A}$ is an equivalence if and only if $Q\to\iscontr(A)$.
\item Show that if $\mathcal{S}\jdeq(S,f,g)$ is a span from $A$ to $B$ and $f:S\to A$ is an equivalence, then so is $\inr:B\to A\sqcup^\mathcal{S} B$. Use this observation to conclude the following:
\begin{subexenum}
\item If $X$ is contractible, then $\susp X$ is contractible.
\item There is an equivalence $\eqv{X}{\join{\emptyt}{X}}$.
\item If $X$ is contractible, then $\join{X}{Y}$ is contractible. 
\end{subexenum}
\item Show that if
\begin{equation*}
\begin{tikzcd}
S \arrow[r] \arrow[d] & Y \arrow[d] \\
X \arrow[r] & Z
\end{tikzcd}
\end{equation*}
is a pushout square, then so is
\begin{equation*}
\begin{tikzcd}
A\times S \arrow[r] \arrow[d] & A\times Y \arrow[d] \\
A\times X \arrow[r] & A\times Z
\end{tikzcd}
\end{equation*}
\item Show that if
\begin{equation*}
\begin{tikzcd}
S_1 \arrow[r] \arrow[d] & Y_1 \arrow[d] & S_2 \arrow[r] \arrow[d] & Y_2 \arrow[d] \\
X_1 \arrow[r] & Z_1 & X_2 \arrow[r] & Z_2
\end{tikzcd}
\end{equation*}
are pushout squares, then so is
\begin{equation*}
\begin{tikzcd}
S_1+S_2 \arrow[r] \arrow[d] & Y_1+ Y_2 \arrow[d] \\
X_1 +X_2 \arrow[r] & Z_1+Z_2. 
\end{tikzcd}
\end{equation*}
\item Consider a span $(S,f,g)$ from $A$ to $B$. Show that the pushout of the span
\begin{equation*}
\begin{tikzcd}
& S+S \arrow[dl] \arrow[dr] \\
A + B & & S
\end{tikzcd}
\end{equation*}
is equivalent to the pushout of $(S,f,g)$.
\item Consider maps $f:A\to X$, $f':A'\to X$, and $h:A\to A'$ such that $H:f\htpy f'\circ h$. 
\begin{subexenum}
\item Construct a map $M_{(h,H)}: M_{f'}\to M_f$.
\item Show that $\eqv{M_{M(h,h)}}{M_h}$.
\end{subexenum}
\end{exercises}
